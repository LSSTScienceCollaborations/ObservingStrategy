\chapter[Solar System]{A Solar System Census}
\def\chpname{solarsystem}\label{chp:\chpname}

\noindent {\it
Lynne Jones, David Trilling, Mike Brown, Eric Christensen
}

\section{Discovering Solar System Objects}
\def\secname{discovery}\label{sec:\secname}

Discovering, rather than simply detecting, small objects throughout
the Solar System requires unambiguously linking a series of detections
together into an orbit. The orbit provides the information necessary
to scientifically characterize the object itself and to understand the
population as a whole. Without orbits, the detections of Solar System
Objects (SSOs) by LSST will be of limited use; objects discovered with
other facilities could be followed up by LSST, but almost the entire
science benefit to planetary astronomy would be lost.

Therefore, the primary concern regarding the Solar System is related
to the question ``Can we link detections of moving objects into
orbits?''.  This requirement poses varying levels of difficulty as we
move from Near Earth Objects (NEOs) through the Main Belt Asteroids
(MBAs) and to TransNeptunian Objects (TNOs) and Scattered Disk Objects
(SDOs), as well as for comets and for other unusual but very
interesting populations such as Earth minimoons.

{\it discuss specific challenges for each population; TNOs and SDOs are
  relatively easy, MBAs are very numerous, NEOs are hard because of
  speed, comets and minimoons are hard because of nongravitational forces}

Much of the answer to this question comes down to the performance of
various pieces of LSST Data Management software. In particular, the
false positive rate resulting from difference imaging, the compute
limitations of the Moving Object Processing System (MOPS) to extend to high
apparent velocities, and the capability to unambiguously determine if
a linkage is `real' or not via orbit determination (done as part of
MOPS). Additional concerns are related to how well observations
widely separated in time can be linked into the `discovery' orbits
(i.e. if we have a discovery in year 1, but do not detect the object
again until year 3, could these observations be linked?). The answers
to these questions range beyond the limits of the OpSim simulated
surveys, but bear on the observing strategy requirements for
discovering Solar System Objects.

{\it describe current minimum observation requirements for existing
  surveys, describe current expected requirements for MOPS, describe
  current effort to understand if MOPS requirements are realistic in
  LSST context}

If we assume various detection requirements, ranging from XXX to the
minimum MOPS requirements, we can characterize the performance of
available simulated surveys in terms of their expected detection rates
for various known populations.

{\it describe completeness metrics for NEOs/MBAs/TNOs/etc - known
  populations. what do we do about unknown populations?}

Beyond this basic but absolutely critical requirement to actually
discover SSOs across the Solar System, we can start to look at other
science goals: detecting activity, determining colors for moving
objects, and measuring shapes and spin states for objects.

{\it describe requirements and challenges for these; why colors are
  hard, how many objects will we actually be able to determine
  shape/spin for, how lightcurves may differ from shape/spin}

Note: take a look at
\texttt{https://github.com/rhiannonlynne/MafSSO/blob/master/SSO\_Analysis.ipynb}
(an extremely messy ipython notebook, but starting to point at some of
the ideas I have for metrics -- let's expand on this)

\navigationbar
