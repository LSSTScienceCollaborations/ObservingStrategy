\chapter[Solar System]{Discovering and Characterizing Small Bodies in
  the Solar System}
\def\chpname{solarsystem}\label{chp:\chpname}

Chapter editors:
\credit{rhiannonlynne},
\credit{davidtrilling}.

% ====================================================================

\section{Introduction}
\label{sec:\chpname:intro}

LSST has tremendous potential as a discovery and characterization tool
for small bodies in the Solar System. With LSST, we have the
opportunity to increase our sample sizes of Potentially Hazardous
Asteroids (PHAs), Near Earth Objects (NEOs), Main Belt Asteroids
(MBAs), Jupiter Trojans, Centaurs, TransNeptunian Objects (TNOs),
Scattered Disk Objects (SDOs), comets and other small body populations
such as Earth mini-moons, irregular satellites, and other planetary
Trojan populations, by at least an order of
magnitude, often two orders of magnitude or more. In addition to
hundreds of astrometric measurements for most objects, LSST will also
provide precisely calibrated multiband photometry. With this
information, we can also characterize these populations -- deriving
colors, light curves, rotation periods, spin states, and even shape
models where possible.

The motivation behind studying these small body populations is
fundamentally to understand planet formation and evolution. The
orbital parameters of these populations record traces of the orbital
evolution of the giant planets. The migration of Jupiter, Saturn and
Neptune in particular have left marks on the orbital distribution of
MBAs, Jupiter Trojans, TNOs and SDOs. Rapid migration of
Jupiter and Saturn may have emplaced a large number of planetesimals
in the Scattered Disk; later slow migration of Neptune will affect the
number of TNOs in resonance and the details of their orbital parameters
within the resonance. Adding color information provides further
insights; colors roughly track composition, indicating formation
location and temperature or space weathering history. For example, the color
gradient of main belt asteroids, combined with their orbital
distribution, suggests that perhaps Jupiter migrated inwards,
mixing planetesimals from the outer Solar System into the outer parts
of the main belt, before eventually migrating outwards. Studying the
size distribution of each of the small body populations themselves
provides more constraints on planetesimal formation; this is
complicated by the effects of dynamical stirring from the giant
planets, which can increase the rate of erosion vs. growth during
collisions, and by the existence of the remnants of collisions such as
collisional families in the main belt. The presence of binaries and range
of spin states and shapes provides further constraints on the history
of each population. The location
of the planets before migration, the amount of migration, and the size
distribution of the small bodies themselves (after detangling the
dynamical evolution) all tell a deeper story about how the planets in
the Solar System formed, and how our formation history fits into the
range of observed extrasolar planetary systems.

These Solar System populations are unique when compared to other
objects which will be investigated by LSST, due to the simple fact
that they move across the sky. Metrics to evaluate
LSST's performance for moving objects need to be based on `per object'
measurements, rather than at a series of points on the sky or per
field pointing. For all metrics discussed in this chapter, the orbit
of each object is integrated over the time of the simulated opsim
survey and the times when each object is visible are recorded; these
series of observations per object are then the basis for metric
evaluations.

% Introduce, with a very broad brush, this chapter's science projects,
% and why it makes sense for them to be considered together.

% ====================================================================

\section{Discovering and linking Solar System Objects}
\def\secname{\chpname:discovery}\label{sec:\secname}

Discovering, rather than simply detecting, small objects throughout
the Solar System requires unambiguously linking a series of detections
together into an orbit. The orbit provides the information necessary
to scientifically characterize the object itself and to understand the
population as a whole. Without orbits, the detections of Solar System
Objects (SSOs) by LSST will be of limited use; objects discovered with
other facilities could be followed up by LSST, but almost the entire
science benefit to planetary astronomy would be lost. Linking and
orbit determination for Solar System objects is similar to source
association for non-moving objects; it provides the means to identify
multiple detections as coming from a single object.

Therefore, the first concern regarding the Solar System is related
to the question ``Can we accurately link individual detections of moving objects into
orbits?''.  This requirement poses varying levels of difficulty as we
move from Near Earth Objects (NEOs) through the Main Belt Asteroids
(MBAs) and to TransNeptunian Objects (TNOs) and Scattered Disk Objects
(SDOs), as well as for comets and for other unusual but very
interesting populations such as Earth minimoons. Due to their small
heliocentric and geocentric distances, NEOs appear move with
relatively high velocities and are distributed over a large fraction
of the sky, far from the ecliptic plane. MBAs are densely distributed,
primarily within about 30 degrees of the ecliptic. TNOs and SDOs move
slowly, however short time intervals between repeat visits in each night may make these difficult
to link. Comets and Earth mini-moons may require more complicated
orbit fitting to allow for non-gravitational or geocentric
orbits. It also implies that we do not create false objects by
incorrectly linking detections and/or noise.

Much of the answer to this question comes down to the performance of
various pieces of LSST Data Management software. In particular,
important questions are the
rate of false positive detections resulting from difference imaging, the compute
limitations of the Moving Object Processing System (MOPS) to extend to high
apparent velocities, and the capability to unambiguously determine if
a linkage is `real' or not via orbit determination (done as part of
MOPS). Thus this question ranges beyond the limits of the OpSim simulated
surveys, but bears on the observing strategy requirements for
discovering Solar System Objects. An in-depth study of the performance
of difference imaging and MOPS is currently ongoing. However, we can
make a range of assumptions on how MOPS will perform and evaluate how
many and which objects can be linked under observational cadence, given those assumptions.


% --------------------------------------------------------------------

\subsection{Target measurements and discoveries}
\label{sec:\secname:targets}

The criteria for `discovery' with MOPS depends on the number
of observations of an object acquired per night, within some time
window (creating `tracklets'), repeated over a number of nights within window of some
days (creating `tracks'), linked into an orbit with a threshhold on
astrometric residuals. The current assumptions are that we can link
detections into orbits with 2 detections per night within 90 minutes,
repeated for 3 nights within 15 days. The additional assumptions are
that with these 6 observations, we will be able to create low-accuracy orbits that will suffice to link
additional observations obtained at later (or earlier, in the LSST
archive) times, and that that the orbit fitting will enable rejection
of mislinkages.

We can also set other requirements for discovery. Requiring 4
detections within 90 minutes is a fairly common discovery criteria for
NEO surveys, as it reduces the number of mislinked tracklets to almost
zero. We could also require 4 nights of pairs, in order to improve the
initial orbit fitting and mislinkage rejection.

With these discovery criteria, we can then evaluate the completeness
of an LSST simulated survey, for a given population. We can look at
this as a function of H magnitude and as a function of orbital
parameters.

For PHAs and NEOs there are special considerations in terms of
completion that arise from planetary defense concerns. For most other
populations, the general desire is simply to have a high level of
completeness, with no gaps in completeness that depend strongly on
orbital parameters. In particular, the desire is to be able to
calibrate any selection effects in discovery so that the survey completeness can
be used to debias the underlying population models.

Discovery opportunity, and thus the completeness of the underlying
population, is very sensitive to the time interval between
observations. For most solar system objects, with a 90 minute window
within a night, gathering two or more repeat observations within a night requires
that the field pointing is revisited two or more times. Gathering
observations over multiple nights for a wide variety of Solar System
objects (moving at a wide variety of apparent velocities) generally requires covering a large
neighboring area of sky; thus the internight revisit rate for large contiguous
blocks of sky is important. An optimal discovery strategy for moving
objects could be ensuring a minimum (default: two) number of revisits
within a night within a short time window (default: 90 minutes), and
covering large contiguous amounts of sky several (default: 3) times within a
longer time window (default: 15 days).

% --------------------------------------------------------------------

\subsection{Metrics}
\label{sec:\secname:metrics}

The {\tt Discovery Metric} can be used to identify sets of detections
of a particular object that meet the defined criteria for discovery: X
detections within TTI minutes in a night, Y nights within a W day
window; this describes the number of discovery opportunities for each object. The results from the Discovery Metric can be fed to the {\tt
  MO\_Completeness} summary metric, where if an object achieves a
user-defined requirement for the minimum number of discovery
opportunities (typically 1), then it is counted as `discovered'; then
the total number of objects discovered at each H magnitude is compared
to the total number of objects in the population at that H magnitude,
in order to evaluate `completeness' as a function of H. Discovery
opportunities can be evaluated as a function of orbital parameters, to
look for areas of orbital space that may be missed in a particular
survey strategy; completeness, since it marginalizes over the entire
population at a particular H value, loses this
capability. Completeness can be evaluated as a differential value
(completeness @ H=X) or integrated over the size distribution
(completeness @ H <= X).
 
A further simplification of the completeness can be achieved if the
completeness at a particular H magnitude is the desired value. For
example, completeness for PHAs at H=22 is an important summary value.

% --------------------------------------------------------------------

\subsection{OpSim Analysis}
\label{sec:\secname:analysis}

Put figures here ... 

OpSim analysis: how good would the default observing strategy be, at
the time of writing for this science project?


% --------------------------------------------------------------------

\subsection{Discussion}
\label{sec:\secname:discussion}

Discussion: what risks have been identified? What suggestions could be
made to improve this science project's figure of merit, and mitigate
the identified risks?



% ====================================================================

\section{Orbital Accuracy}
\def\secname{\chpname:orbits}\label{sec:\secname}


A short preamble goes here. What's the context for this science
project? Where does it fit in the big picture?

% --------------------------------------------------------------------

\subsection{Target measurements and discoveries}
\label{sec:\secname:targets}

Describe the discoveries and measurements you want to make.

Now, describe their response to the observing strategy. Qualitatively,
how will the science project be affected by the observing schedule and
conditions? In broad terms, how would we expect the observing strategy



% --------------------------------------------------------------------

\subsection{Metrics}
\label{sec:\secname:metrics}

Quantifying the response via MAF metrics: definition of the metrics,
and any derived overall figure of merit.


% --------------------------------------------------------------------

\subsection{OpSim Analysis}
\label{sec:\secname:analysis}

OpSim analysis: how good would the default observing strategy be, at
the time of writing for this science project?


% --------------------------------------------------------------------

\subsection{Discussion}
\label{sec:\secname:discussion}

Discussion: what risks have been identified? What suggestions could be
made to improve this science project's figure of merit, and mitigate
the identified risks?


% ====================================================================

\section{Detecting Activity}
\def\secname{\chpname:activity}\label{sec:\secname}


A short preamble goes here. What's the context for this science
project? Where does it fit in the big picture?

How secure is the orbit - is it going to hit us?
Libration amplitude distribution for TNOs?
Can we find it after X years for further study?
Can we identify the source region for NEOs within the main belt?

% --------------------------------------------------------------------

\subsection{Target measurements and discoveries}
\label{sec:\secname:targets}

Describe the discoveries and measurements you want to make.

Now, describe their response to the observing strategy. Qualitatively,
how will the science project be affected by the observing schedule and
conditions? In broad terms, how would we expect the observing strategy
to be optimized for this science?


% --------------------------------------------------------------------

\subsection{Metrics}
\label{sec:\secname:metrics}

Quantifying the response via MAF metrics: definition of the metrics,
and any derived overall figure of merit.


% --------------------------------------------------------------------

\subsection{OpSim Analysis}
\label{sec:\secname:analysis}

OpSim analysis: how good would the default observing strategy be, at
the time of writing for this science project?


% --------------------------------------------------------------------

\subsection{Discussion}
\label{sec:\secname:discussion}

Discussion: what risks have been identified? What suggestions could be
made to improve this science project's figure of merit, and mitigate
the identified risks?

Different discussion / risks for each science case within this general metric?

% ====================================================================

\section{Measuring colors}
\def\secname{\chpname:colors}\label{sec:\secname}


A short preamble goes here. What's the context for this science
project? Where does it fit in the big picture?

% --------------------------------------------------------------------

\subsection{Target measurements and discoveries}
\label{sec:\secname:targets}

Describe the discoveries and measurements you want to make.

Now, describe their response to the observing strategy. Qualitatively,
how will the science project be affected by the observing schedule and
conditions? In broad terms, how would we expect the observing strategy
to be optimized for this science?


% --------------------------------------------------------------------

\subsection{Metrics}
\label{sec:\secname:metrics}

Quantifying the response via MAF metrics: definition of the metrics,
and any derived overall figure of merit.


% --------------------------------------------------------------------

\subsection{OpSim Analysis}
\label{sec:\secname:analysis}

OpSim analysis: how good would the default observing strategy be, at
the time of writing for this science project?


% --------------------------------------------------------------------

\subsection{Discussion}
\label{sec:\secname:discussion}

Discussion: what risks have been identified? What suggestions could be
made to improve this science project's figure of merit, and mitigate
the identified risks?


% ====================================================================

\section{Measuring lightcurves/rotation periods}
\def\secname{\chpname:lightcurves}\label{sec:\secname}


A short preamble goes here. What's the context for this science
project? Where does it fit in the big picture?

% --------------------------------------------------------------------

\subsection{Target measurements and discoveries}
\label{sec:\secname:targets}

Describe the discoveries and measurements you want to make.

Now, describe their response to the observing strategy. Qualitatively,
how will the science project be affected by the observing schedule and
conditions? In broad terms, how would we expect the observing strategy
to be optimized for this science?


% --------------------------------------------------------------------

\subsection{Metrics}
\label{sec:\secname:metrics}

Quantifying the response via MAF metrics: definition of the metrics,
and any derived overall figure of merit.


% --------------------------------------------------------------------

\subsection{OpSim Analysis}
\label{sec:\secname:analysis}

OpSim analysis: how good would the default observing strategy be, at
the time of writing for this science project?


% --------------------------------------------------------------------

\subsection{Discussion}
\label{sec:\secname:discussion}

Discussion: what risks have been identified? What suggestions could be
made to improve this science project's figure of merit, and mitigate
the identified risks?


% ====================================================================


\navigationbar

% ====================================================================
