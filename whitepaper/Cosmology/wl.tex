% ====================================================================
%+
% SECTION NAME:
%    wl.tex
%
% CHAPTER:
%    cosmology.tex
%
% ELEVATOR PITCH:
%-
% ====================================================================

\section{Weak Lensing}
\def\secname{wl}\label{sec:\secname}

\credit{jmeyers314},
\credit{tonytyson},
\credit{StephenRidgway}.

Much of LSST cosmology may be limited by systematic errors rather than
signal-to-noise. This is especially true of weak gravitational lensing,  which
relies on very accurate (\ie low bias), but very low signal-to-noise,
measurements of the shapes of galaxies, and high signal-to-noise measurements of
PSF calibration stars. As outlined in the SRD, uniformity of seeing in the bands
used for WL and special observing strategies are required in order to reduce
additive and multiplicative shear systematics.

Achieving the ultimate sensitivity of the LSST to weak lensing science places
stringent requirements on our ability to accurately measure galaxy shapes and
redshifts, which in turn demands precise and accurate knowledge of the point
spread function, astrometry, and photometry. These measurements are influenced
by the interaction of light with the Earth's atmosphere, the telescope optics,
and the CCD sensors. Sysematics in the shear are introduced in each case.
Observing strategies have been developed for suppressing these systematics in
current lensing surveys, such as the DLS. These and new methods will be applied
to the LSST survey.  The observed shear is a convolution of the PSF with the
intrinsic galaxy image.  The observed shear thus has additive systematics (some
of which arise in the CCD due to the anisotropic brighter-fatter effect) as well
as multiplicative systematics (due to PSF convolution).

$$ \gamma_{obs} = (1+m) \gamma + c  $$ where $m$ is the multiplicative and $c$
is the additive systematic.

From the LSST SRD requirements on residual systematics in the shear-shear
correlation function one can specify the level of residual shear systematics.
Over the sample of 3-4 billion galaxies, the shear systematics must be below 3
parts in 10,000 for additive shear $|c|$, and 3 parts in 1000 for multiplicative
shear $|m|$. Each visit to a sky patch encounters these systematics. In
particular, each re-visit to a given field generates the same CCD-based additive
shear systematic.  Some observing strategies can effectively randomize these
over all visits to a field.  It is important to note that the full survey
shear-shear correlation error due to these systematics is expected to be no
better than the corresponding systematic in any given field after all re-visits
to that field.  This is because the  useful angular scales in cosmic shear are
less than a field radius, and the systematics floor  in shear-shear correlation
is set therefore by the floor in any one typical field.  Below we discuss the
observing strategies for suppressing shear systematics and metrics for their
success.

\subsection{Target Selection}

Image quality must be uniformly good in the bands used for weak lens shear.
These will be mainly the $r$ and $i$ bands. It is possible that the $z$ band
will also be used for shear measurement.  The decision on which field to observe
next must be based mainly on its WL priority.  Depending on the current weather
and seeing, the scheduler will have a list of priorities for next-field, based
on prior history of coverage.  The relevant parameters are seeing, depth, and
coverage at some CCD orientation relative to the sky.  Nearby fields in need of
coverage in these bands will be given high priority if the seeing is better than
some specified value, likely 0.7 arcsec FWHM. An optimum strategy would also
consider the needs for coverage at an easily achievable CCD orientation relative
to sky.


\subsection{Target Measurements}

It is expected that even after maximal optimization of camera optics and
electronics, that systematic image shape errors will be associated with the
orientation of the camera focal plane.  Using data from vendor CCDs, simulations
of LSST observing have shown that a combination of x-y dithering on the sky and
pipeline processing with pixel re-map (to cancel much of the CCD frame fixed
distortions) can get well within a factor of ten of the goal for shear
systematics residuals.  Simulations which add camera angle dithering show that
the goal can be achieved in fields with relatively uniform seeing history (Jee
and Tyson 2011).

Thus shear systematics will be partially reduced by randomization of the
orientation of the camera with respect to the sky.  This is represented by the
parameter RotSkyPos: we can construct diagnostic metrics that quantify the
uniformity of its distribution at each sky position.   Given the spin 2 symmetry
of shear, the optimal strategy for shear systematics will be to aim for
uniformity of RotSkyPos mod $\pi$, since angles separated by $\pi$ radians are
degenerate.

Similarly, the telescope optics may harbor systematic aberrations, and these
also could be mitigated by recording images with varying parallactic angle.
Also important is the effect of atmospheric differential chromatic refraction
which acts along the parallactic angle.  Re-visits to a given field should be
distributed over parallactic angles (or equivalently, hour angles), consistent
with airmass and seeing limits.  Note that wide HA coverage for a given field
will also  be helpful in order to efficiently achieve full 180 deg coverage in
CCD sky angle.

As argued below, initial survey depth is more important than survey area.
Uniformity of depth is important, but less so than uniformity in camera rotator
shear suppression.  Simulations have shown that for the Gold sample of galaxies,
uniformity at the 0.2 mag level in limiting magnitude produces little shear
bias. The largest effect comes from bias in weak lens magnification tomography
(Morrison et al. 2012).


\subsection{Metrics}

For characterizing the isotropy of rotational sampling, both for rotSkyPos and
the parallactic angle, we investigate two metrics: the AngularSpreadMetric and
the KuiperMetric.  The AngularSpreadMetric characterizes the balance of a set of
angular values, in the sense that opposing angles, those separated by $\pi$
radians, have zero contribution to the AngularSpread.  The Kuiper statistic,
which is related to the well known Kolmogorov-Smirnov statistic, characterizes
the departure of a distribution from uniform, but with the added quality of
being invariant under cyclic transformations of the input set of angles.

The AngularSpread metric is computed as follows:  Given a set of angles
$\{\theta\}_{i=1, ..., N}$, map these angles onto a unit circle: $(x_i, y_i) =
(\cos \theta_i, \sin \theta_i)$, and find the 2D centroid: $(\bar{x}, \bar{y}) =
\frac{1}{N} (\sum_i x_i, \sum_i y_i)$.  The AngularSpread is the distance of the
2D centroid from the unit circle: $\mathrm{AngularSpread} = 1 - \sqrt{\bar{x}^2 +
\bar{y}^2}$.  An AngularSpread of 1 therefore corresponds to a perfectly
balanced distribution, in which the averages of both $\cos \theta$ and $\sin
\theta$ are zero, while an AngularSpread of 0 indicates a maximally anisotropic
distribution in which every angle is identical: $\theta_i = \mathrm{const}$.  As
mentioned above, weak lensing shear systematics cancel to first order when those
systematics are separated not by an angle of $\pi$ radians, but by an angle of
$\pi/2$ radians.  To incorporate this spin-2 nature of shear systematics is
simple, we just multiply each angle $\theta_i$ by two before applying the
AngularSpread metric, so that, for example, pairs of angles initially separated
by $\pi/2$ radians become separated by $\pi$ radians and correctly cancel.

While the AngularSpread metric does a good job at characterizing the balance of
a distribution defined on a circle, it does not directly address the {\emph
{uniformity}} of said distribution.  For instance, the AngularSpread of the
angles $\{0, 0, 0, 0, \pi, \pi, \pi, \pi\}$ is zero, but the distribution is far
from uniform.  The Kolmogorov-Smirnov (KS) test is well known for investigating
whether a set of data are consistent with a given distribution.  The KS
statistic, off which the test is based, is defined as the maximum absolute
difference in the empirical cumulative distribution function (CDF) of the data
and the CDF of the distribution being tested.  The Kuiper statistic is a slight
modification of the KS statistic, defined as the sum of the maximum difference
and absolute minimum (maximally negative) difference between the empirical and
test CDFs.  This modification is convenient for characterizing distributions
defined on a circle, since it makes the statistic invariant under rotations of
the data.  The larger the test statistic (which ranges between 0 and 1), the
larger the difference between the empirical distribution and the test
distribution.  To incorporate the spin-2 nature of shear systematics in the
Kuiper statistic, we map the values $\theta_i \rightarrow \theta_i \mod \pi$ and
compare to the uniform distribution between 0 and $\pi$.


\subsection{OpSim Analysis}

The distribution of AngularSpread for $2 \times$ rotSkyPos is shown in
\autoref{fig:WL_AngularSpread_rotSkyPos} for the lastest baseline OpSim run,
minion\_1016.  The left panel shows a sky map for the i-band (in this and the
following figures, the sky maps vary only minimally between the two principal
lensing filters, $r$ and $i$), while the right panel shows a histogram of values
for each LSST filter.  The distribution of the Kuiper statistic for rotSkyPos
mod $\pi$ is similarly shown in \autoref{fig:WL_Kuiper_rotSkyPos}. Jee and Tyson
did a study of the shear residual systematics due to known LSST CCD
brighter-fatter anisotropy in 100 revisits to a single field with random angular
orientations and sampling seeing from the expected distribution.  The needed
factor of 10 suppression of the CCD-based shear systematic was obtained,
reaching the SRD floor on cosmic shear systematics (WL systematics workshop, Dec
2015).    While we do not currently have a method to quantitatively connect the
distribution of rotSkyPos to cosmological systematics, these figures appear to
indicate that rotSkyPos is already being well sampled in current simulations.

The distribution of parallactic angles is similarly shown in figures
\autoref{fig:WL_AngularSpread_ParallacticAngle} and
\autoref{fig:WL_Kuiper_ParallacticAngle}.  These figures show significantly less
isotropy and significantly more structure across the survey footprint than those
for rotSkyPos, likely due to the fact that, unlike rotSkyPos, the parallactic
angle is independent of the camera rotator position.  Hence the parallactic
angle is more tightly constrained by geometry than rotSkyPos.  In fact, the only
mechanism by which the parallactic angle varies for a given field is through
variations in the hour angle at which that field is observed.

% \begin{figure}
% \centering\includegraphics[width=\linewidth]{figs/enigma1189RmsAnglerotSkyPosugrizybandallpropsOPSIComboHistogram.png}
% \caption{The relative angle of the detector plane with respect to the sky, RotSkyPos, as a histogram showing the number of fields vs. rms of the parameter.}
% \label{RotSkyPos}
% \end{figure}

% The distribution of rms values by filter is shown in
% \autoref{RotSkyPos} for the current candidate baseline simulation,
% enigma\_1189.  As shown, the rms values cluster around the value 1
% radian,  with typical values 1 +- 0.3 radian.  This compares to a
% completely uniform distribution over the half circle with an rms of
% 1.14.  As mentioned above, uniformity in cosine squared is the goal.
% Simulated observing of 100 visits to a field show this will produce
% a factor of 10 decrease in CCD-based shear systematics such as edge
% effects and the brighter-fatter x-y anisotropy.




\newcommand\plottwo[2]{{%
\typeout{Plottwo included the files #1 #2}
\centering
\leavevmode
\includegraphics*[width=0.45\columnwidth]{#1}%
\hfil
\includegraphics*[width=0.45\columnwidth]{#2}%
}}%


%  rotSkyPos metrics

\begin{figure}[tbh!]
\plottwo{figs/WL/minion_1016_AngularSpread_rotSkyPos_propID_54_and_i_HEAL_SkyMap.pdf}
        {figs/WL/minion_1016_AngularSpread_rotSkyPos_u_g_r_i_z_y_propID_54_HEAL_ComboHistogram.pdf}
\caption{\textbf{Left:} Sky map showing the distribution of the AngularSpread
    metric applied to the angle $2 \times$ rotSkyPos, where rotSkyPos is the
    angle between the $+y$ camera direction and North, and the factor of two
    takes into account the degeneracy of angles separated by $\pi$ radians for
    spin-2 shear systematics.  An AngularSpread of 0 indicates a maximally
    anisotropic distribution (all visits have the same angle), while an
    AngularSpread of 1 indicates that visits are maximally balanced (the mean of
    $\cos \theta$ and $\sin \theta$ are both 0.) For the complete definition of
    the AngularSpread metric, please see the text.  To leading order, shear
    systematics permanently imprinted on the camera cancel when AngularSpread =
    1.  \textbf{Right:} Distribution of the AngularSpread metric applied to
    $2 \times$ rotSkyPos for all LSST filters.}
\label{fig:WL_AngularSpread_rotSkyPos}
\end{figure}

\begin{figure}[tbh!]
\plottwo{figs/WL/minion_1016_Kuiper_rotSkyPos_propID_54_and_i_HEAL_SkyMap.pdf}
        {figs/WL/minion_1016_Kuiper_rotSkyPos_u_g_r_i_z_y_propID_54_HEAL_ComboHistogram.pdf}
\caption{\textbf{Left:} Sky map showing the distribution of the Kuiper metric
    (see text for definition) applied to the angle rotSkyPos mod $\pi$.  A
    Kuiper value of 0 indicates an isotropic distribution of angles (mod $\pi$),
    while a Kuiper value of 1 indicates a maximally anisotropic distribution.
    \textbf{Right:} Distribution of the Kuiper metric applied to (rotSkyPos mod
    $\pi$) for all LSST filters.}
\label{fig:WL_Kuiper_rotSkyPos}
\end{figure}

%  ParallacticAngle metrics

\begin{figure}[tbh!]
\plottwo{figs/WL/minion_1016_AngularSpread_ParallacticAngle_propID_54_and_i_HEAL_SkyMap.pdf}
        {figs/WL/minion_1016_AngularSpread_ParallacticAngle_u_g_r_i_z_y_propID_54_HEAL_ComboHistogram.pdf}
\caption{Same as Fig. \ref{fig:WL_AngularSpread_rotSkyPos}, but for the
    parallactic angle (the angle between North and zenith) instead of rotSkyPos.
    The isotropy of the parallactic angle affects the impact of shear
    systematics due to telescope aberrations and differential chromatic
    refraction.}
\label{fig:WL_AngularSpread_ParallacticAngle}
\end{figure}

\begin{figure}[tbh!]
\plottwo{figs/WL/minion_1016_Kuiper_ParallacticAngle_propID_54_and_i_HEAL_SkyMap.pdf}
        {figs/WL/minion_1016_Kuiper_ParallacticAngle_u_g_r_i_z_y_propID_54_HEAL_ComboHistogram.pdf}
\caption{Same as Fig. \ref{fig:WL_Kuiper_rotSkyPos}, but for the parallactic
    angle instead of rotSkyPos.}
\label{fig:WL_Kuiper_ParallacticAngle}
\end{figure}


\subsection{Ancilliary data}

Optimal observing strategy relies on the use of all relevant data, with current
and historical coverage per field.  Ancilliary data, and auxilliary uses of the
science exposures, play an important role informing next-field strategy.  We can
use large scale patterns of distortions of the PSF over the 20,000 stars per
exposure for PSF regularization in the per-CCD PSF fitting.  In the per CCD fits,
there is a benefit to setting aside some stars for validation tests of PSF
extrapolation.  These data may be used as a metric for image quality and thus
the ranked value of an exposure for shear systematics removal.  Fields with poor
image quality rise to the top of the priority list for re-observing at that
camera angle.  In addition to using all the stars in a given visit, there is
useful information in the wavefront sensors and the guide CCDs that may be used
to regularize the PSF reconstruction in a visit.  We might read out guider CCDs
in different ways to better monitor the atmosphere.

\subsection{Deep vs Wide}

As outlined above, many revisits to each field spanning many RotSkyPos angles
aids the suppression  of shear systematics.  For a given vist exposure time this
leads to a deep survey.  There are several advantages to a deep survey over a
shallow-wide survey for WL science.  A stragegy question for LSST is whether to
go wide first and then deep, or the reverse.

There are actually several drivers for depth over area, given fixed observing
time and camera+telescope etendue.  This observing strategy maximizes the
cosmological S/N either by maximizing the signal or minimizing the noise.

First, it boosts the amplitude of the cosmic shear signals due to the increased
amplitude of the lensing kernel.  Given the same lens mass, the amplitude of
lensing signals is a simple function of the two distances: observer-to-lens
versus lens-to-source. For example, when a lens is at z = 0.5, the shear at z =
1.1 is nearly twice that at z = 0.7, leading to a factor of 4 ratio in
correlation function amplitudes.  Thus, a deep survey can take advantage of the
geometric effect of gravitational lensing more efficiently.

Second, it enables a longer redshift baseline to break parameter degeneracies.
For example, shallower surveys with wider area are not efficient in shrinking
the length of the "banana" in the $\Omega_M$-$\sigma_8$ plane because of the $\sigma_8 \Omega_M = \mathrm{constant}$ 
degeneracy.  The deep strategy compensates for the loss of
volume due to a reduced area by extending the volume along the line of sight.

Third, deep surveys provide more useful galaxies per area.  This merit does not
entirely overlap with the second point.  In addition to new sources at higher
redshift, the fainter limiting magnitude enables detection of fainter galaxies
at a given redshift and better S/N.  Needless to say, the increased source
density reduces shape noise caused by intrinsic ellipticity dispersion.  Shape
noise is reduced by larger number of source galaxies per square arcminute 
$n_\mathrm{eff}$.

Fourth, it mitigates the effects of intrinsic alignments (IA), an important 
theoretical systematic in precision cosmic
shear.  Current studies (Heymans et al. 2013) indicate that their effects are
dominated by LRGs.  Because a deep survey can access fainter galaxies, the net IA
systematic decreases because the fraction of the LRGs decreases at fainter
limiting magnitude. Using the approach of Joachimi et al. (2011),  at z = 0.5
the amplitude of the IA power spectrum is reduced by a factor of two for an
increase the limiting r magnitude from 24 to 27 mag.

For WL cosmic shear the volume at 5-60 arcmin scales and over a wide range of
redshift is most important. Once a fair sampling of “cosmic variance” is
achieved the depth matters most, because of the z-width of the lensing kernel.
With fixed survey duration on a camera+telescope of fixed etendue, it is better
to prioritize depth in order to maximize the WL cosmological S/N.  In
particular, for the LSST survey it would be helpful to have perhaps 5 widely
spaced deep drilling fields done early and deeper than the main survey, in order
to explore detailed observing strategies -- particularly the angle dither
scheme.  By the same reasoning, it would be important to cover  a siginifant
area [perhaps 2000 sq.deg] to full depth during the first year of the survey.
This would allow full assment of asystematics, and could be chosen to overlap
the WFIRST footprint.



\subsection{Discussion}

The RotSkyPos metrics show that the majority of fields have good randomization
of detector angles projected on the sky.  The randomization of parallactic
angles is less successful, though this is to be expected due to fewer knobs
available to adjust the parallactic angle of observations of a given field.  In
both cases, however, a significant fraction of fields show metric values lower
than expected for a uniform distribution.  Regardless of the {\emph per field}
criterion adopted, it is desirable to avoid the incidence of individual
discrepant fields.  The recommended criterion for randomization of RotSkyPos and
parallactic angle is not the behavior of the majority of the fields, but of the
minority with the least random behavior -- the number of non-random fields
should be minimized.

It is certain that actively controlling the statistics of RotSkyPos will require
additional slewing of the camera rotator.  At present, the operations plan is to
only slew (beyond that required to track the sky during exposures) when
necessary to prepare for a filter change - that could be estimated at the
equivalent of $\simeq 3$ complete rotations per night.  To engage the rotator by
up to $\simeq 30$ degrees per visit would require $\simeq 300$ complete
rotations per night, though it may be possible to increase isotropy with far
less additional slewing.  The impact on survey efficiency and hardware wear and
tear has not been considered.  Whether or not this uniformity could be achieved
with less slew time if implemented in scheduling remains to be demonstrated.

Increasing the isotropy of the parallactic angle is trickier, since the
parallactic angle is only affected by the hour angle of observations (for a
given field).  It may be possible, however, to adjust the scheduler cost
function to better favor parallactic angle isotropy.

In summary, image quality weighted randomization of RotSkyPos is required at
10-20\% accuracy.   Randomization of parallactic angle is also desired, but the
accuracy is TBD.  Interestingly there is an excellent  opportunity to combine
these two randomizations in an efficient observing strategy which actually
minimizes the number of camera rotations.   Re-visits to a field over a range of
HA and with camera rotations assures full 180 deg range coverage for CCD
coordinates relative to sky.  A metric for this can be written and run with
OpSim to explore optimization, including minimizing camera rotations. 


% The RotSkyPos metric analysis shows that the majority of fields have a
% good randomization of detector angles projected on the sky.
%
% There are some limitations to this observation.
%
% %First, we do not have at present a quantitative requirement for
% %randomization of this parameter.  In future development of weak
% %lensing analysis, a criterion should be developed.
%
% A significant fraction of fields  have median values that are
% lower or higher than expected for a random distribution, with some far
% from uniformly distributed.  Regardless of the $per field$ criterion,
% it is desirable to avoid the incidence of individual discrepant
% fields.
%
% The recommended criterion for randomization of RotSkyPos is not the
% behavior of the majority of the fields, but of the minority with the
% least random behavior.  The number of non-random fields should be
% minimized.  A recommended metric is the count of fields with median
% RMS less then 0.8 or greater than 1.5 radians (these values to be
% reviewed again as additional experience is gained with additional
% OpSim schedule simulations and weak lensing analysis.)
%
% It is certain that actively controlling the statistics of RotSkyPos
% will require additional slewing of the camera rotator.  At present,
% the operations plan is to only slew when necessary to prepare for a
% filter change - that could be estimated at the equivalent of $\simeq
% 3$ complete rotations per day.  \autoref{RotSkyPos} shows that to
% render the distribution completely uniform would require moving all
% observing angles an average of $\simeq 30$ degrees, or 300 complete
% rotations per night.  The timing of this has not been considered.
% Whether or not this uniformity could be achieved with less slew time
% if implemented in scheduling remains to be demonstrated.
%
% A similar metric for RotTelPos should be developed.
