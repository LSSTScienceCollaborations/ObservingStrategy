% ====================================================================
%+
% SECTION:
%    AGN_Clustering.tex
%
% CHAPTER:
%    agn.tex
%
% ELEVATOR PITCH:
%-
% ====================================================================

% \section{AGN Clustering}
\subsection{AGN Clustering}\lebel{sec:AGNClustering}
\def\secname{\chpname:clustering}\label{sec:\secname}

\credit{ohadshemmer}

Measurements of the spatial clustering of AGNs with respect
to those of quiescent galaxies can provide clues as to how galaxies
form inside their dark-matter halos, and what causes the growth of
their supermassive black holes (SMBHs). The impressive inventory of
LSST AGNs will enable the clustering, and thus the host galaxy halo
mass, to be determined over the widest ranges of cosmic epoch
and accretion power.


% % --------------------------------------------------------------------
%
% \subsection{Target measurements and discoveries}
% \label{sec:\secname:targets}

% Describe the discoveries and measurements you want to make.

% Now, describe their response to the observing strategy. Qualitatively,
% how will the science project be affected by the observing schedule and
% conditions? In broad terms, how would we expect the observing strategy
% to be optimized for this science?

We would consider the 2-point angular correlation function of AGN as our
target measurement.
The LSST cadence will not only affect the overall AGN census and its
$L-z$ distribution, but also the depth in each band as a function of
sky position that can directly affect the clustering signal.

% CROSS REFERENCE TO THE COSMOLOGY CHAPTER'S LSS SECTION! WHAT'S
% DIFFERENT ABOUT AGN CLUSTERING?

% --------------------------------------------------------------------
%
% \subsection{Metrics}
% \label{sec:\secname:metrics}
%
% Quantifying the response via MAF metrics: definition of the metrics,
% and any derived overall figure of merit.
%
%
% % --------------------------------------------------------------------
%
% \subsection{OpSim Analysis}
% \label{sec:\secname:analysis}
%
% OpSim analysis: how good would the default observing strategy be, at
% the time of writing for this science project?
%
%
% % --------------------------------------------------------------------
%
% \subsection{Discussion}
% \label{sec:\secname:discussion}
%
% Discussion: what risks have been identified? What suggestions could be
% made to improve this science project's figure of merit, and mitigate
% the identified risks?
%
%
% ====================================================================
%
% \subsection{Conclusions}
%
% Here we answer the ten questions posed in
% \autoref{sec:intro:evaluation:caseConclusions}:
%
% \begin{description}
%
% \item[Q1:] {\it Does the science case place any constraints on the
% tradeoff between the sky coverage and coadded depth? For example, should
% the sky coverage be maximized (to $\sim$30,000 deg$^2$, as e.g., in
% Pan-STARRS) or the number of detected galaxies (the current baseline but
% with 18,000 deg$^2$)?}
%
% \item[A1:] ...
%
% \item[Q2:] {\it Does the science case place any constraints on the
% tradeoff between uniformity of sampling and frequency of  sampling? For
% example, a rolling cadence can provide enhanced sample rates over a part
% of the survey or the entire survey for a designated time at the cost of
% reduced sample rate the rest of the time (while maintaining the nominal
% total visit counts).}
%
% \item[A2:] ...
%
% \item[Q3:] {\it Does the science case place any constraints on the
% tradeoff between the single-visit depth and the number of visits
% (especially in the $u$-band where longer exposures would minimize the
% impact of the readout noise)?}
%
% \item[A3:] ...
%
% \item[Q4:] {\it Does the science case place any constraints on the
% Galactic plane coverage (spatial coverage, temporal sampling, visits per
% band)?}
%
% \item[A4:] ...
%
% \item[Q5:] {\it Does the science case place any constraints on the
% fraction of observing time allocated to each band?}
%
% \item[A5:] ...
%
% \item[Q6:] {\it Does the science case place any constraints on the
% cadence for deep drilling fields?}
%
% \item[A6:] ...
%
% \item[Q7:] {\it Assuming two visits per night, would the science case
% benefit if they are obtained in the same band or not?}
%
% \item[A7:] ...
%
% \item[Q8:] {\it Will the case science benefit from a special cadence
% prescription during commissioning or early in the survey, such as:
% acquiring a full 10-year count of visits for a small area (either in all
% the bands or in a  selected set); a greatly enhanced cadence for a small
% area?}
%
% \item[A8:] ...
%
% \item[Q9:] {\it Does the science case place any constraints on the
% sampling of observing conditions (e.g., seeing, dark sky, airmass),
% possibly as a function of band, etc.?}
%
% \item[A9:] ...
%
% \item[Q10:] {\it Does the case have science drivers that would require
% real-time exposure time optimization to obtain nearly constant
% single-visit limiting depth?}
%
% \item[A10:] ...
%
% \end{description}

% ====================================================================

\navigationbar
