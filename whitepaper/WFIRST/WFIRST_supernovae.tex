% ====================================================================
%+
% SECTION:
%    WFIRST_supernovae.tex
%
% CHAPTER:
%    wfirst.tex
%
% ELEVATOR PITCH:
%-
% ====================================================================

\section{Supernova Cosmology with WFIRST and LSST}
\def\secname{\chpname:supernovae}\label{sec:\secname}

\credit{rubind}

% This individual section will need to describe the particular
% discoveries and measurements that are being targeted in this section's
% science case. It will be helpful to think of a ``science case" as a
% ``science project" that the authors {\it actually plan to do}. Then,
% the sections can follow the tried and tested format of an observing
% proposal: a brief description of the investigation, with references,
% followed by a technical feasibility piece. This latter part will need
% to be quantified using the MAF framework, via a set of metrics that
% need to be computed for any given observing strategy to quantify its
% impact on the described science case. Ideally, these metrics would be
% combined in a well-motivated figure of merit. The section can conclude
% with a discussion of any risks that have been identified, and how
% these could be mitigated.
%
% A short preamble goes here. What's the context for this science
% project? Where does it fit in the big picture?

% --------------------------------------------------------------------

\subsection{Target measurements and discoveries}
\label{sec:\secname:targets}

% Describe the discoveries and measurements you want to make.
%
% Now, describe their response to the observing strategy. Qualitatively,
% how will the science project be affected by the observing schedule and
% conditions? In broad terms, how would we expect the observing strategy
% to be optimized for this science?

WFIRST seeks to measure thousands of SNe Ia to $z \sim 1.7$. 

All these goals can be met with $\sim 3$ day rest-frame cadence ($\sim 5$ observer-frame days). We should target NUV to rest-frame $V$-band (with WFIRST providing redder filter coverage), or observer-frame $grizY$ for most of the SNe.

% --------------------------------------------------------------------

\subsection{Metrics}
\label{sec:\secname:metrics}

% Quantifying the response via MAF metrics: definition of the metrics,
% and any derived overall figure of merit.

The primary metrics are based on constraining cosmological parameters; the DETF FoM is the primary example (although other FoMs can be constructed using eigenmode constraints). For the joint observations proposed here, we anticipate an increase in the FoM of about 20\% (DR is still working to optimize the WFIRST side of the joint survey for the best possible constraints). The cosmological metric will be composed of several (related) metrics: the fraction of non SNe Ia mistakenly sent to WFIRST for followup, the number of SNe Ia sent to WFIRST, but lost due to weather gaps (with insufficent weather recovery), and the fraction of SNe sent to WFIRST, but lacking the light-curve sampling to constrain key light-curve parameters (date of maximum, rise time and decline time, etc.). 

% --------------------------------------------------------------------

\subsection{OpSim Analysis}
\label{sec:\secname:analysis}

% OpSim analysis: how good would the default observing strategy be, at
% the time of writing for this science project?


% --------------------------------------------------------------------

\subsection{Discussion}
\label{sec:\secname:discussion}

% Discussion: what risks have been identified? What suggestions could be
% made to improve this science project's figure of merit, and mitigate
% the identified risks?


% ====================================================================

\navigationbar
