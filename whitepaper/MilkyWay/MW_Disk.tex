% ====================================================================
%+
% SECTION:
%    section-name.tex  % eg lenstimedelays.tex
%
% CHAPTER:
%    chapter.tex  % eg cosmology.tex
%
% ELEVATOR PITCH:
%    Explain in a few sentences what the relevant discovery or
%    measurement is going to be discussed, and what will be important
%    about it. This is for the browsing reader to get a quick feel
%    for what this section is about.
%
% COMMENTS:
%
%
% BUGS:
%
%
% AUTHORS:
%    Phil Marshall (@drphilmarshall)  - put your name and GitHub username here!
%-
% ====================================================================

\section{Mapping the Milky Way Disk}
\def\secname{MW_Disk}\label{sec:\secname} % For example, replace "keyword" with "lenstimedelays"

\noindent{\it Will Clarkson, Peregrine McGehee}  % (Writing team)

% This individual section will need to describe the particular
% discoveries and measurements that are being targeted in this section's
% science case. It will be helpful to think of a ``science case" as a
% ``science project" that the authors {\it actually plan to do}. Then,
% the sections can follow the tried and tested format of an observing
% proposal: a brief description of the investigation, with references,
% followed by a technical feasibility piece. This latter part will need
% to be quantified using the MAF framework, via a set of metrics that
% need to be computed for any given observing strategy to quantify its
% impact on the described science case. Ideally, these metrics would be
% combined in a well-motivated figure of merit. The section can conclude
% with a discussion of any risks that have been identified, and how
% these could be mitigated.

A short preamble goes here. What's the context for this science
project? Where does it fit in the big picture?

% --------------------------------------------------------------------

\subsection{Target measurements and discoveries}
\label{sec:keyword:MW_Disk_targets}

Describe the discoveries and measurements you want to make.

Now, describe their response to the observing strategy. Qualitatively,
how will the science project be affected by the observing schedule and
conditions? In broad terms, how would we expect the observing strategy
to be optimized for this science?


% --------------------------------------------------------------------

\subsection{Metrics}
\label{sec:keyword:MW_Disk_metrics}

Quantifying the response via MAF metrics: definition of the metrics,
and any derived overall figure of merit.


% --------------------------------------------------------------------

\subsection{OpSim Analysis}
\label{sec:keyword:MW_Disk_analysis}

OpSim analysis: how good would the default observing strategy be, at
the time of writing for this science project?


% --------------------------------------------------------------------

\subsection{Discussion}
\label{sec:keyword:MW_Disk_discussion}

Discussion: what risks have been identified? What suggestions could be
made to improve this science project's figure of merit, and mitigate
the identified risks?


% ====================================================================

\navigationbar
