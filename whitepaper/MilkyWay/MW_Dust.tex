% ====================================================================
%+
% SECTION:
%    section-name.tex  % eg lenstimedelays.tex
%
% CHAPTER:
%    chapter.tex  % eg cosmology.tex
%
% ELEVATOR PITCH:
%    Explain in a few sentences what the relevant discovery or
%    measurement is going to be discussed, and what will be important
%    about it. This is for the browsing reader to get a quick feel
%    for what this section is about.
%
% COMMENTS:
%
%
% BUGS:
%
%
% AUTHORS:
%    Phil Marshall (@drphilmarshall)  - put your name and GitHub username here!
%-
% ====================================================================

\section{Dust in the Milky Way}
\def\secname{MW_Dust}\label{sec:\secname} % For example, replace "keyword" with "lenstimedelays"

\noindent{\it Peregrine M. McGehee} % (Writing team)

% This individual section will need to describe the particular
% discoveries and measurements that are being targeted in this section's
% science case. It will be helpful to think of a ``science case" as a
% ``science project" that the authors {\it actually plan to do}. Then,
% the sections can follow the tried and tested format of an observing
% proposal: a brief description of the investigation, with references,
% followed by a technical feasibility piece. This latter part will need
% to be quantified using the MAF framework, via a set of metrics that
% need to be computed for any given observing strategy to quantify its
% impact on the described science case. Ideally, these metrics would be
% combined in a well-motivated figure of merit. The section can conclude
% with a discussion of any risks that have been identified, and how
% these could be mitigated.

Interstellar dust is a significant constituent of the Galaxy. Its composition and associated extinction
properties tell us about the material and environments in which stars and their planets are formed.
Dust also presents an obstacle for a wide-range of astronomical observations, causing light from
stars in the plane of the Milky Way to be severely dimmed and causing the apparent colors of
objects observed in any direction to be shifted from their intrinsic values. These color shifts
are dependent upon the dust column density along the line of sight and the radiative transport
properties of the dust grains.

To first order, i.e. neglecting the effects of heterochromatic extinction, the absorption of light
in each band due to dust is dependent upon the column density, related to $E(B-V)$, and the
nature of the dust grains, as parameterized by the ratio of general to selection extinction 
in the Johnson $B$ and $V$ bands, defined as $R_V = A_V /E(B − V)$. 
In the low-density diffuse ISM, $R_V$ has a value $\sim 3.1$, while in
dense molecular clouds $R_V$ can be higher with values $4 < R_V < 6$.

In general, however, the use of broad band photometry requires attention to the intrinsic
SEDs of the background stars in order to correct for heterochromatic variations in the
effective reddening law. As discussed in the LSST Science Book, possession of an accurate
dust map is important to many astrophysical studies. The two most significant all-sky maps generated
in the past two decades are the SFD98 maps based on IRAS observations, and the recent thermal
dust maps derived from Planck submillimeter data. The angular resolutions of both maps are similar - 
between 4 to 6 arcminutes.

Both of the aforementioned maps are strictly two-dimensional and conntain no information about the 
distribution of dust along the line of sight. A third dimension can be obtained by analysis of
accurate stellar photometry which constrain both the reddening $E(B-V)$ and $R_V$ towards 
individual stars. This approach requires determination of the intrinsic stellar colors and the photometric
parallax of each star in the presence of an unknown amount and law of extinction.
Recent work on 3-D maps include the Bayesian analysis method based on Pan-STARRS 1 data 
(Green et al. 2015, ApJ, 810, 25) and an alternative technique using SDSS photometry of 
M dwarfs (McGehee et al. 2016, in preparation). 

% --------------------------------------------------------------------

\subsection{Target measurements and discoveries}
\label{sec:\secname:targets}

The use of stellar samples to create three-dimensional extinction maps has an established history
beginning with the work of Neckel \& Klare (1980); however these, including studies based on SDSS and
PS1 photometry, are typically limited to heliocentric distances of $\sim$4 kpc. In the full co-added survey,
LSST will be able to map dust structures out to distances exceeding 40 kpc, thus revealing a
detailed picture of this component of the Milky Way Galaxy.

The Pan-STARRS1 survey (PS1) has
produced a three-dimensional dust map of the region of the sky covered
in their 3$\pi$ survey (which excludes a large part of the Galactic
Plane toward the south). Such maps are necessary to accurately measure
the intrinsic luminosities and colors of both Galactic and
extragalactic sources. 
Green et al. (2015) estimated $R_V$ along sightlines having higher 
reddening values as well as reddening values.
The PS1 map saturates at
extinctions $E(B-V) > 1.5$ as their tracer stars fall out of the
survey catalogs fainter than $g\sim 22$, meaning that this
high-fidelity map does not extend uniformly to within a few degrees of
the midplane. In addition, it only extends to a distance of about 4.5
kpc. Deep LSST data will allow this map to be extended to much higher
extinctions and larger distances. Owing to the high extinction and the
use of blue filters, this project is less affected by crowding than
other projects requiring photometry in the Plane. 

In comparison, the SDSS survey makes use of M dwarf locus in $(g-r,r-i)$ being
nearly perpendicular to the reddening vector in that color-color space. This 
allows mapping of a reddening-invariant index to the intrinsic stellar $g-i$ color
and subsequent deterimation of the light-of-sight reddening. This approach assumes
a set extinction law, i.e $R_V = 3.1$, in order compute the reddening-invariant 
index from the observed $g-r$ and $r-i$ colors. Given the relative faintness of M dwarfs,
this technique is distance limited to $\sim$1 kpc when based on SDSS data.

The LSST will be in a unique position to measure the changes in the observed reddening vector
due to $R_V$ variations due to its superb photometric accuracy. 
Both of the dust survey techniques mentioned here can be used on LSST data, and perhaps other 
methods will be developed before the start of survey operations. T

% --------------------------------------------------------------------

\subsection{Metrics}

\label{sec:\secname:metrics}

Production of a 3-D map of the dust component of the ISM based on LSST photometry will tell us 
how much dust is present, what type it is, and where it is along the line of sight. 
The latter concern brings in issues of how to determine stellar photometric parallaxes ($\mu = m-M$) under
an unknown reddening.

The dust maps that are created will consist of the median and variance of $E(B-V)$ and $R_V$ expressed as functions of 
$\mu$ under a suitable binning scheme. We can create simple Figure of Merit maps that lose the 
$\mu$ dependency by computing the mean and variance of the measured variances in $E(B-V)$ and $R_V$ 
over the $\mu$ bins.

With the possible exception of sightlines towards star formation regions, the spacing in time of the visits 
doesn't matter for dust studies. In the case of active star formation regions it is possible that changes in the
ISM could be apparent over the lifetime of the survey.
Pushing to fainter magnitudes (which means both better seeing and longer exposures) matters, 
both because we want more stars, and in particular, we want more stars behind the dust.  


{\bf Metric 1: Uncertainty and bias in $E(B-V)$~estimates as a
  function of location on-sky.} Dependencies:

\begin{itemize}
  \item Stellar population throughout the survey (e.g. Knut / Peter developments; TRILEGAL?);
    \item Dust map throughout the survey region;
    \item Scale photometric error predictions for each band from program requirements per exposure;
      \item Produce formal estimate on the error in extinction and reddening as a function of position on-sky within the survey.
\end{itemize}


% --------------------------------------------------------------------

\subsection{OpSim Analysis}
\label{sec:\secname:analysis}

OpSim analysis: how good would the default observing strategy be, at
the time of writing for this science project?


% --------------------------------------------------------------------

\subsection{Discussion}
\label{sec:\secname:discussion}

Discussion: what risks have been identified? What suggestions could be
made to improve this science project's figure of merit, and mitigate
the identified risks?


% ====================================================================

\navigationbar
