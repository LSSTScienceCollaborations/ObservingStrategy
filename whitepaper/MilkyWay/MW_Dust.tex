% ====================================================================
%+
% SECTION:
%    section-name.tex  % eg lenstimedelays.tex
%
% CHAPTER:
%    chapter.tex  % eg cosmology.tex
%
% ELEVATOR PITCH:
%    Explain in a few sentences what the relevant discovery or
%    measurement is going to be discussed, and what will be important
%    about it. This is for the browsing reader to get a quick feel
%    for what this section is about.
%
% COMMENTS:
%
%
% BUGS:
%
%
% AUTHORS:
%    Phil Marshall (@drphilmarshall)  - put your name and GitHub username here!
%-
% ====================================================================

\section{Dust in the Milky Way}
\def\secname{MW_Dust}\label{sec:\secname} % For example, replace "keyword" with "lenstimedelays"

\noindent{\it Peregrine M. McGehee} % (Writing team)

% This individual section will need to describe the particular
% discoveries and measurements that are being targeted in this section's
% science case. It will be helpful to think of a ``science case" as a
% ``science project" that the authors {\it actually plan to do}. Then,
% the sections can follow the tried and tested format of an observing
% proposal: a brief description of the investigation, with references,
% followed by a technical feasibility piece. This latter part will need
% to be quantified using the MAF framework, via a set of metrics that
% need to be computed for any given observing strategy to quantify its
% impact on the described science case. Ideally, these metrics would be
% combined in a well-motivated figure of merit. The section can conclude
% with a discussion of any risks that have been identified, and how
% these could be mitigated.

Interstellar dust is a significant constituent of the Galaxy. Its composition and associated extinction
properties tell us about the material and environments in which stars and their planets are formed.
Dust also presents an obstacle for a wide-range of astronomical observations, causing light from
stars in the plane of the Milky Way to be severely dimmed and causing the apparent colors of
objects observed in any direction to be shifted from their intrinsic values. These color shifts
are dependent upon the dust column density along the line of sight and the radiative transport
properties of the dust grains.

The wavelength dependence of the absorption due to dust is parametrized in the widely used model
of Cardelli et al. (1989) by the ratio of general to selection extinction in the Johnson B and V
bands, defined as $R_V = A_V /E(B − V)$. The value of RV depends on the dust composition and
grain size along the line of sight. In the low-density diffuse ISM, $R_V$ has a value $\sim 3.1$, while in
dense molecular clouds $R_V$ can be higher with values $4 < R_V < 6$.

The fundamental importance of a well-characterized dust map to astronomy is underscored by the
$>$ 5,000 citations to the dust and extinction maps by Schlegel et al. (1998), henceforth SFD98.
The SFD98 maps are based on far-infrared observations and predict reddening in specific bands by
assuming a dust model and $R_V$ = 3.1 as appropriate for sky areas away from the Galactic plane.
Despite the great contribution that the SFD98 extinction map has made to the field, these maps
suffer from several issues that limit their utility in some regimes of study. 1) While the SFD98
map seems to be well calibrated at low column density, various tests using galaxy counts, star
counts and colors, and stellar spectrophotometry indicate that SFD98 overpredicts dust by $\sim30\%$
above E(B − V ) $\sim$ 1 mag. Because this overcorrection appears especially in cold clouds, it is
likely related to the temperature correction adopted in the SFD98 model. 2) In some cases,
especially at low Galactic latitudes, $R_V$ variation is important and is not tracked by SFD98. 3)
For study of low-redshift, large-scale structure, contamination by unresolved point sources can be
important (see Yahata et al. 2007). 4) Finally, the resolution of the SFD98 map is $\sim$6 arcmin, which
is larger than the angular scales subtended by nearby, resolved, galaxies for which a carefully
characterized foreground dust distribution is particularly important. For all these reasons, LSST
stellar photometry, which can constrain the temperature correction, overall calibration, and point
source contamination of SFD98, is valuable. The dust map based on {\it Planck} observations in the submillimeter, 
which provides a comparison against SFD98, 
has a comparable resolution, in this case limited by the $\sim$4.5 arcmin beam size.

For the study of stellar populations and objects within the Galactic disk it is also important to
determine both the line of sight extinction and the value of $R_V$ at a specific distance, neither of
which is dealt with by SFD98. By analysis of the observed reddening of stellar colors, we will verify
both the dust column density and RV values predicted by these maps and can also determine the
local spatial distribution of the dust.

The reddening of stellar colors due to the presence of interstellar dust along the line of sight can,
in principle, be used to map the three-dimensional distribution of that dust. This requires that
two important parameters are determined - the amount the observed stellar color is reddened and
the distance to the star. By comparison of the color excess measured in stars at varying distances
we can infer the location of the extincting medium. However, given lack of an a priori knowledge
of the light of sight extinction, which is the very quantity we wish to measure, it can be difficult
to accurately assign intrinsic stellar colors and luminosities in order to determine the amount of
color excess and the distance. 

Recent work on 3-D maps include the Bayesian analysis utilizing Pan-STARSS 1 data 
(Green et al. 2015 ApJ 2015, 810, 25) and an alternative technique using SDSS photometry of 
M dwarfs (McGehee et al. 2016, in preparation). 

% --------------------------------------------------------------------

\subsection{Target measurements and discoveries}
\label{sec:\secname:targets}

The use of stellar samples to create three-dimensional extinction maps has an established history
beginning with the work of Neckel \& Klare (1980); however these, including studies based on SDSS and
PS1 photometry, are typically limited to heliocentric distances of $\sim$4 kpc. In the full co-added survey,
LSST will be able to map dust structures out to distances exceeding 40 kpc, thus revealing a
detailed picture of this component of the Milky Way Galaxy.

Mapping of the dust component of the Galactic ISM requires detection of the reddening in the
colors of stars at known distances. The reddening is determined from the color excess deduced
by comparison of the observed colors with those expected based on the stellar spectral type. In
the absence of identifying spectra, the spectral type can be inferred by dereddening the observed
colors (assuming a specific extinction law, i.e., a particular value of $R_V$) back to the unreddened
stellar locus in a color-color diagram. This dereddening is equivalent to assignment of reddeningfree
colors along the stellar locus, which measure the location in the color-color diagram along
the direction perpendicular to the reddening vector. Once the effective line of sight reddening has
been computed, the distance to each star can be determined using dereddened photometry and
well-calibrated color-absolute magnitude relations.

The Pan-STARRS1 survey (PS1) has
produced a three-dimensional dust map of the region of the sky covered
in their 3$\pi$ survey (which excludes a large part of the Galactic
Plane toward the south). Such maps are necessary to accurately measure
the intrinsic luminosities and colors of both Galactic and
extragalactic sources. The PS1 map saturates at
extinctions $E(B-V) > 1.5$ as their tracer stars fall out of the
survey catalogs fainter than $g\sim 22$, meaning that this
high-fidelity map does not extend uniformly to within a few degrees of
the midplane. In addition, it only extends to a distance of about 4.5
kpc. Deep LSST data will allow this map to be extended to much higher
extinctions and larger distances. Owing to the high extinction and the
use of blue filters, this project is less affected by crowding than
other projects requiring photometry in the Plane. 

Changes in the absorption properties of dust grains, as parametrized by $R_V$ , result in a shift in
both the direction and length (for a specific dust column density) of the reddening vector in a
color-color diagram.
By analysis of the observed color shifts due to reddening it is possible to constrain the value of $R_V$
along the line of sight and gain insight into the nature and composition of the interstellar dust in
that region of the Galaxy. Green et al. (2015) have estimated $R_V$ along sightlines having higher 
reddening values.

The LSST will be in a unique position to measure the changes in the observed reddening vector
due to $R_V$ variations due to its superb photometric accuracy. The specifications for
LSST are a factor of two more stringent than typically achieved in previous surveys, including the
SDSS (except for limited photometric conditions).


% --------------------------------------------------------------------

\subsection{Metrics}

\label{sec:\secname:metrics}

Production of a 3-D map of the dust component of the ISM based on LSST photometry will tell us 
how much dust is present, what type it is, and where it is along the line of sight. 
The latter concern brings in issues of how to determine stellar photometric parallaxes ($\mu = m-M$) under
an unknown reddening.

The dust maps that are created consist of the median and variance of $E(B-V)$ and $R_V$ expressed as functions of 
$\mu$ under a suitable binning scheme. We can create simple Figure of Merit maps that lose the 
$\mu$ dependency by computing the mean and variance of the measured variances in $E(B-V)$ and $R_V$ 
over the $\mu$ bins.

With the possible exception of sightlines towards star formation regions, the spacing in time of the visits 
doesn't matter for dust studies. In the case of active star formation regions it is possible that changes in the
ISM could be apparent over the lifetime of the survey.
Pushing to fainter magnitudes (which means both better seeing and longer exposures) matters, 
both because we want more stars, and in particular, we want more stars behind the dust.  


{\bf Metric 1: Uncertainty and bias in $E(B-V)$~estimates as a
  function of location on-sky.} Dependencies:

\begin{itemize}
  \item Stellar population throughout the survey (e.g. Knut / Peter developments; TRILEGAL?);
    \item Dust map throughout the survey region;
    \item Scale photometric error predictions for each band from program requirements per exposure;
      \item Produce formal estimate on the error in extinction and reddening as a function of position on-sky within the survey.
\end{itemize}


% --------------------------------------------------------------------

\subsection{OpSim Analysis}
\label{sec:\secname:analysis}

OpSim analysis: how good would the default observing strategy be, at
the time of writing for this science project?


% --------------------------------------------------------------------

\subsection{Discussion}
\label{sec:\secname:discussion}

Discussion: what risks have been identified? What suggestions could be
made to improve this science project's figure of merit, and mitigate
the identified risks?


% ====================================================================

\navigationbar
