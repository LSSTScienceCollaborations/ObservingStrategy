% ====================================================================
%+
% SECTION:
%    MW_FutureWork.tex
%
% CHAPTER:
%    galaxy.tex
%
% ELEVATOR PITCH:
%    Ideas for future metric investigation, with quantitaive analysis
%    still pending.
%-
% ====================================================================

\section{Future Work}
\def\secname{MW_future}\label{sec:\secname}

% ====================================================================

% % ====================================================================
%+
% SECTION:
%    section-name.tex  % eg lenstimedelays.tex
%
% CHAPTER:
%    chapter.tex  % eg cosmology.tex
%
% ELEVATOR PITCH:
%    Explain in a few sentences what the relevant discovery or
%    measurement is going to be discussed, and what will be important
%    about it. This is for the browsing reader to get a quick feel
%    for what this section is about.
%
% COMMENTS:
%
%
% BUGS:
%
%
% AUTHORS:
%    Phil Marshall (@drphilmarshall)  - put your name and GitHub username here!
%-
% ====================================================================

\section{Dust in the Milky Way}
\def\secname{MW_Dust}\label{sec:\secname} % For example, replace "keyword" with "lenstimedelays"

\noindent{\it Peregrine M. McGehee} % (Writing team)

% This individual section will need to describe the particular
% discoveries and measurements that are being targeted in this section's
% science case. It will be helpful to think of a ``science case" as a
% ``science project" that the authors {\it actually plan to do}. Then,
% the sections can follow the tried and tested format of an observing
% proposal: a brief description of the investigation, with references,
% followed by a technical feasibility piece. This latter part will need
% to be quantified using the MAF framework, via a set of metrics that
% need to be computed for any given observing strategy to quantify its
% impact on the described science case. Ideally, these metrics would be
% combined in a well-motivated figure of merit. The section can conclude
% with a discussion of any risks that have been identified, and how
% these could be mitigated.

Interstellar dust is a significant constituent of the Galaxy. Its composition and associated extinction
properties tell us about the material and environments in which stars and their planets are formed.
Dust also presents an obstacle for a wide-range of astronomical observations, causing light from
stars in the plane of the Milky Way to be severely dimmed and causing the apparent colors of
objects observed in any direction to be shifted from their intrinsic values. These color shifts
are dependent upon the dust column density along the line of sight and the radiative transport
properties of the dust grains.

To first order, i.e. neglecting the effects of heterochromatic extinction, the absorption of light
in each band due to dust is dependent upon the column density, related to $E(B-V)$, and the
nature of the dust grains, as parameterized by the ratio of general to selection extinction 
in the Johnson $B$ and $V$ bands, defined as $R_V = A_V /E(B − V)$. 
In the low-density diffuse ISM, $R_V$ has a value $\sim 3.1$, while in
dense molecular clouds $R_V$ can be higher with values $4 < R_V < 6$.

In general, however, the use of broad band photometry requires attention to the intrinsic
SEDs of the background stars in order to correct for heterochromatic variations in the
effective reddening law. As discussed in the LSST Science Book, possession of an accurate
dust map is important to many astrophysical studies. The two most significant all-sky maps generated
in the past two decades are the SFD98 maps based on IRAS observations, and the recent thermal
dust maps derived from Planck submillimeter data. The angular resolutions of both maps are similar - 
between 4 to 6 arcminutes.

Both of the aforementioned maps are strictly two-dimensional and conntain no information about the 
distribution of dust along the line of sight. A third dimension can be obtained by analysis of
accurate stellar photometry which constrain both the reddening $E(B-V)$ and $R_V$ towards 
individual stars. This approach requires determination of the intrinsic stellar colors and the photometric
parallax of each star in the presence of an unknown amount and law of extinction.
Recent work on 3-D maps include the Bayesian analysis method based on Pan-STARRS 1 data 
(Green et al. 2015, ApJ, 810, 25) and an alternative technique using SDSS photometry of 
M dwarfs (McGehee et al. 2016, in preparation). 

% --------------------------------------------------------------------

\subsection{Target measurements and discoveries}
\label{sec:\secname:targets}

The use of stellar samples to create three-dimensional extinction maps has an established history
beginning with the work of Neckel \& Klare (1980); however these, including studies based on SDSS and
PS1 photometry, are typically limited to heliocentric distances of $\sim$4 kpc. In the full co-added survey,
LSST will be able to map dust structures out to distances exceeding 40 kpc, thus revealing a
detailed picture of this component of the Milky Way Galaxy.

The Pan-STARRS1 survey (PS1) has
produced a three-dimensional dust map of the region of the sky covered
in their 3$\pi$ survey (which excludes a large part of the Galactic
Plane toward the south). Such maps are necessary to accurately measure
the intrinsic luminosities and colors of both Galactic and
extragalactic sources. 
Green et al. (2015) estimated $R_V$ along sightlines having higher 
reddening values as well as reddening values.
The PS1 map saturates at
extinctions $E(B-V) > 1.5$ as their tracer stars fall out of the
survey catalogs fainter than $g\sim 22$, meaning that this
high-fidelity map does not extend uniformly to within a few degrees of
the midplane. In addition, it only extends to a distance of about 4.5
kpc. Deep LSST data will allow this map to be extended to much higher
extinctions and larger distances. Owing to the high extinction and the
use of blue filters, this project is less affected by crowding than
other projects requiring photometry in the Plane. 

In comparison, the SDSS survey makes use of M dwarf locus in $(g-r,r-i)$ being
nearly perpendicular to the reddening vector in that color-color space. This 
allows mapping of a reddening-invariant index to the intrinsic stellar $g-i$ color
and subsequent deterimation of the light-of-sight reddening. This approach assumes
a set extinction law, i.e $R_V = 3.1$, in order compute the reddening-invariant 
index from the observed $g-r$ and $r-i$ colors. Given the relative faintness of M dwarfs,
this technique is distance limited to $\sim$1 kpc when based on SDSS data.

The LSST will be in a unique position to measure the changes in the observed reddening vector
due to $R_V$ variations due to its superb photometric accuracy. 
Both of the dust survey techniques mentioned here can be used on LSST data, and perhaps other 
methods will be developed before the start of survey operations. T

% --------------------------------------------------------------------

\subsection{Metrics}

\label{sec:\secname:metrics}

Production of a 3-D map of the dust component of the ISM based on LSST photometry will tell us 
how much dust is present, what type it is, and where it is along the line of sight. 
The latter concern brings in issues of how to determine stellar photometric parallaxes ($\mu = m-M$) under
an unknown reddening.

The dust maps that are created will consist of the median and variance of $E(B-V)$ and $R_V$ expressed as functions of 
$\mu$ under a suitable binning scheme. We can create simple Figure of Merit maps that lose the 
$\mu$ dependency by computing the mean and variance of the measured variances in $E(B-V)$ and $R_V$ 
over the $\mu$ bins.

With the possible exception of sightlines towards star formation regions, the spacing in time of the visits 
doesn't matter for dust studies. In the case of active star formation regions it is possible that changes in the
ISM could be apparent over the lifetime of the survey.
Pushing to fainter magnitudes (which means both better seeing and longer exposures) matters, 
both because we want more stars, and in particular, we want more stars behind the dust.  


{\bf Metric 1: Uncertainty and bias in $E(B-V)$~estimates as a
  function of location on-sky.} Dependencies:

\begin{itemize}
  \item Stellar population throughout the survey (e.g. Knut / Peter developments; TRILEGAL?);
    \item Dust map throughout the survey region;
    \item Scale photometric error predictions for each band from program requirements per exposure;
      \item Produce formal estimate on the error in extinction and reddening as a function of position on-sky within the survey.
\end{itemize}


% --------------------------------------------------------------------

\subsection{OpSim Analysis}
\label{sec:\secname:analysis}

OpSim analysis: how good would the default observing strategy be, at
the time of writing for this science project?


% --------------------------------------------------------------------

\subsection{Discussion}
\label{sec:\secname:discussion}

Discussion: what risks have been identified? What suggestions could be
made to improve this science project's figure of merit, and mitigate
the identified risks?


% ====================================================================

\navigationbar


% ====================================================================

% WIC - promoted this back to MW Halo section

% % ====================================================================
%+
% SECTION:
%    section-name.tex  % eg lenstimedelays.tex
%
% CHAPTER:
%    chapter.tex  % eg cosmology.tex
%
% ELEVATOR PITCH:
%    Explain in a few sentences what the relevant discovery or
%    measurement is going to be discussed, and what will be important
%    about it. This is for the browsing reader to get a quick feel
%    for what this section is about.
%
% COMMENTS:
%
%
% BUGS:
%
%
% AUTHORS:
%    Phil Marshall (@drphilmarshall)  - put your name and GitHub username here!
%-
% ====================================================================

\section{Mapping the Milky Way Halo}
\def\secname{MW_Halo}\label{sec:\secname} % For example, replace "keyword" with "lenstimedelays"

\noindent{\it Kathy Vivas, Colin Slater, David Nidever}  % (Writing team)

The study of the Halo of the Milky Way is of the most importance not only to understand
the formation and early evolution of our own galaxy, but also to test 
test current models of hierarchical galaxy formation. 
LSST will provide an unprecedented combination of
area, depth, multi-band, multi-epoch information for pursuing detail studies
of the structure of this old Galactic component. We focus here in three
specific projects that can be pursued with LSST. We define metrics that can
be calculated in order to quantify the feasibility of the projects under different
observational strategies. We expect more projects will join later.

RR Lyrae stars have been known for several decades as excellent tracers
of the halo population. They are not only old stars ($>10$ Gyrs) but they are
also excellent standard candles that allow to build 3-dimensional maps. 
The halo of the Milky Way has been now surveyed in a very large extension up to 
$\sim 60-80$ kpc from the Galactic center (refs). Beyond that, the halo is
mostly uncharted territory.
From these RR Lyrae surveys, we have learned that the halo is filled with substructures
which are usually interpreted as debris from destroyed satellite galaxies. The smooth 
component of the RR Lyrae distribution is well described
with a power-law of the mean number density of RR Lyrae stars as a function of
galactocentric distance, which gets steeper after $\sim 30$ kpc (refs). 
Thus, beyond $\sim 60$ kpc, few field RR Lyrae stars are expected. However, we expect that 
any RR Lyrae star beyond this distance may be part of either debris material or distant
satellite galaxies of low luminosity that have been escaped detection until now (refs). 
LCDM models predict debris as far as $0.5$~Mpc from the galactic center
This is the territory that will be explored by LSST.

Similarly, red giant stars can be used to trace the structure of the halo up to large
distances. They have the advantage 
of being bright and numerous stars. 
%%COLIN, PLEASE STEP IN HERE.

Fainter than these two tracers, main sequence stars stand up as a tool for studying
the Halo. They are the most numerous type of stars available and statistical studies 
are possible. Using the technique of photometric metallicities (Ivezic et al 2008), 
SDSS provided unprecedented maps of the metallicity distribution up to  $\sim 10$ 
kpc from the Galactic center, unveiling not only the mean metallicity distribution 
of the halo but also, sub-structures within the halo. This kind of works will be extended
to the outermost parts of the Galaxy with LSST data.


% --------------------------------------------------------------------

\subsection{Target measurements and discoveries}
\label{sec:keyword:MW_Halo_targets}

The three projects just described require the discovery and/or measurement of the following 
type of objects:

\begin{itemize}

\item RR Lyrae stars: These are bright horizontal-branch variable stars with
periods between 0.2 to 1.0 days and large amplitudes, particularly in the bluer 
bandpasses (g amplitudes $0.5 - 1.5$~mag). Optimal use of LSST for discovering
RR Lyraes involves the simultaneous use of multi-band time series (refs).
Chapter \ref{chp:variables} discusses with details the discovery metrics for RR Lyrae stars.
A particularly valuable measurement for studies in the halo is the infrared mean
magnitudes z and y since they provide the most accurate way to obtain 
distances.

\item Main sequence stars: lacking any distinguishable variability, the
challenge in selecting a large and clean sample of main sequence stars comes
from tremendous number of small and nearly-unresolved galaxies present at
faint magnitudes. Precise star/galaxy separation is thus the limiting factor
on the useful depth of the main sequence sample. In addition to identifying
dwarfs, using dwarfs to map the metallicity distribution of the halo requires
precise u-band data, since it exhibits the strongest metallicity dependence of
the LSST filters.

\item Red Giants: due to their intrinsic luminosity red giants will be sample
a far deeper volume than main sequence stars at similar apparent magnitudes,
but they must first be identified and separated from the very numerous main
sequence stars present in the field. A gravity-sensitive photometric index can
be used for separating efficiently giants from dwarfs (refs). The u magnitude
is an essential ingredient in this process and it is necessary to follow-up
the behavior of the u limiting magnitude under different observational
strategies.

\end{itemize}

% --------------------------------------------------------------------

\subsection{Metrics}
\label{sec:keyword:MW_Halo_metrics}

\textbf{Star-Galaxy Separation:} For main sequence stars, the useful depth of
the survey will likely not be the photometric detection limit but will instead
be set by the ability to differentiate stars from unresolved background
galaxies. Towards faint magnitudes the contamination by galaxies worsens
significantly for several reasons: the number of galaxies is rising
substantially, the angular size of galaxies is shrinking, and our ability to
distinguish stars from marginally resolved galaxies diminishes for faint
sources simply due to photon statistics. While the fundamental properties of
the contaminant sources is beyond our control, our ability to reject these
sources depends on survey parameters such as the distribution of seeing across
visits and the depth of these visits.

Our star galaxy separation metric accounts for these factors, using a modeling
framework described in (s/g paper ref). This model uses the distribution of
galaxies in size and number, derived from HST COSMOS observations, along with
a fully Bayesian model decision formalism to compute the expected completeness
and contamination in star-galaxy separation. Computationally for each position
in the survey footprint we interpolate the results from that work on a grid in
seeing, galaxy size, and coadd depth, then integrate over the distribution of
galaxy sizes.

[… will have more to say about the actual metric once it is fully implemented]


% --------------------------------------------------------------------

\subsection{OpSim Analysis}
\label{sec:keyword:MW_Halo_analysis}

\begin{itemize}

\item Comment on the north ecliptic spur in enigma\_1189. Is it close to the
WFD S/G limit?

\item Pan-STARRS-like cadence ops2\_1092, observing up to dec +15. How much
volume do we gain, and how much do we lose in the WFD survey?

\end{itemize}


% --------------------------------------------------------------------

\subsection{Discussion}
\label{sec:keyword:MW_Halo_discussion}

Discussion: what risks have been identified? What suggestions could be
made to improve this science project's figure of merit, and mitigate
the identified risks?


% ====================================================================

\navigationbar


% ====================================================================

% \subsection{Other Ideas}

\credit{willclarkson}, \credit{akvivas}, \credit{vpdebattista}

In this final section we provide an extremely brief list of important science
cases that are still in an early stage of development, but that are
deserving of quantitative MAF analysis in the future.

\subsection{Further considerations for Milky Way static science}

\new{One important area of Milky Way science on which further
  community input is still sorely needed, is {\it static science} (a
  category that includes population disentanglement through deep,
  multicolor photometry), particularly in regions outside the main
  ``Wide-Fast-Deep'' (WFD) survey (Sections \ref{sec:MW_Astrometry} \&
  \ref{sec:MW_Halo} include discussion of static science in WFD regions). Since
  static science depends on depth (for, e.g., precise colors near the
  main sequence turn-off of some population) and uniformity over the
  survey (to aid characterization of strong selection functions),
  static science observing requirements may be in tension with (or at
  least not explicitly addressed by) requirements communicated
  elsewhere in this chapter.}

\new{For example, probing deep within spatially crowded
  populations may lead to a sharper requirement on the selection of
  observations in good seeing conditions towards crowded regions, than
  has been apparent to-date. This needs quantification.} 

\new{To pick another example, while we have indicated that co-added
  depth is a lower priority than temporal coverage for
  variability-driven studies in the Galactic Disk (conclusion A.1 in
  Section \ref{sec:MW_Disk}), co-added depth will likely be crucial for
  population disentanglement through photometry. While a
  judiciously-chosen observing strategy should be able to support both
  static and variable science, at this date quantitative trade-offs have not yet been specified.}

\new{In many cases the implementation of figures of merit for static
  science in the Milky Way is complicated by the requirement to
  interface custom population simulations with the observational
  characterizations produced by the \MAF framework. For many
  investigators, the preferred method may be to use \MAF to produce
  parameterizations of the observational quantities of interest - for
  example, the run of photometric uncertainty against apparent
  magnitude, for each location on the sky, and including spatial
  confusion (all of which \MAF can currently produce) - and then use
  these characteristics as input to their own population simulations,
  on which the investigator may have invested substantial time and
  effort.}

\new{To provoke progress, we specify in Table
  \ref{table:strawmanMWstaticScience} a possible Figure of Merit for
  static science in terms of capabilities mostly already provided by
  the \MAF framework, which does not require custom simulation. This
  Figure of Merit - which asks what fraction of fields in a spatial
  region of interest, are sufficiently well-observed to permit
  population disentanglement to some desired level of precision -
  could form the basis of several science FoM's (for example, the
  fraction of fields in which photometric age determinations of
  bulge/bar populations might be attempted). We encourage community
  development and implementation of this and other FoMs for Milky Way
  static science.}

\begin{table}[h]
  \small
  \begin{tabular}{c p{12cm}}
    & {\it FoM innerMW-Static: fraction of fields in Inner Plane adequately covered for population discrimination} \\
    \hline
    1. & Produce absolute magnitudes $M_{u,g,r,i,z,y}$~of the population of interest, using \MAF's spectral libraries; \\
    2. & For each HEALPIX (i.e. pointing):\\
       & 2.1. Place the fiducial star at appropriate line-of-sight distance, produce apparent magnitudes $m_{u,g,r,i,z,y}$;\\
       & 2.2. Modify $m_{u,g,r,i,z,y}$~for extinction using \MAF's extinction model;\\
    & 2.3. Compute the photometric and astrometric uncertainties due to sampling and random error (from \MAF's ``m52snr'' method);\\
    & 2.4. Convert exposure-by-exposure estimates of photometric and astrometric uncertainty due to spatial confusion, into co-added uncertainties;\\
    & 2.5. Combine the random and confusion uncertainties into final measurement uncertainties on the photometry and astrometry;\\
    & 2.6. Use uncertainty propagation to estimate color uncertainties $u-g, g-r, r-i, i-z$;\\
    3. & Count the fraction of sight-lines for which the color below the threshold needed for "sufficient" accuracy in parameter determination \citep{ivezic08}. {\bf This is the figure of merit.} \\
\hline
    \end{tabular}
 \caption{Description of Figures of Merit 2.1. \& 2.2.}
  \label{table:strawmanMWstaticScience}
\end{table}

\new{Finally, we provide an extremely brief list of important science
cases that are still in an early stage of development, which fall into the ``Static science'' category of Milky Way science:}
\begin{itemize}
  \item {\it Formation history of the Bulge and present-day balance of
  populations:} Sensitivity to metallicity and age distribution of Bulge
  objects near the Main Sequence Turn-off;
  \item {\it Migration and heating in the Milky Way disk:} Error and
  bias in the determination of components in the (velocity dispersion vs
  metallicity) diagram, for disk populations along various lines of
  sight \citep[e.g.][]{2016ApJ...818L...6L}.

% WIC 2017-05-24: commented this out in favor of a dedicated local volume subsection.
%
%\item Fraction of Local-Volume objects discovered as a function of
%  survey strategy.
\end{itemize}

\subsection{Short exposures}

\new{Populations near, or brighter than, LSST's nominal saturation
  limit ($r \sim 16$~with 15s exposures) are likely to be crucial to a
  number of investigations for Milky Way investigations, whether as
  science tracers in their own right, or as contaminants that might
  interfere with measurements of fainter program objects (due, for
  example, to charge bleeds of bright, foreground disk objects).}

\new{Quantitative exploration of these issues now requires involvement from
  the community (e.g. to determine LSST's discovery space for bright
  tracers in context with other facilities and surveys like ZTF, Gaia
  and VVV), and the project (e.g. to determine the parameters of short
  exposures that might be supported by the facility). To provoke
  development, here we list a few example questions regarding short
  exposures that still need resolution:}

\new{
\begin{itemize}
  \item{What level of bright-object charge-bleeding can be tolerated? For example, is some minimum distribution of position-angles required in order to spread the bleeds azimuthally over the set of observations of a particular line of sight (so that different pixels fall under bleeds in different exposures)?\footnote{Note that with $\sim 30$~exposures per filter per field over ten years towards some regions, charge-bleed directions might not be highly randomized.} What is this minimum?}
    \item{What is the science impact of restricting targets to $r \gtrsim 16$?}
      \item{Will the proposed twilight survey of short exposures be adequate for science cases requiring short exposures? Are there enough observations from other surveys (e.g. DES or other DECam surveys) to cover the bright end of the entire LSST footprint, and are {\it new} observations of very bright objects required scientifically in any case?}
        \item{What is the bright limit required for adequate astrometric cross-calibration against Gaia?}
          \item{To what extent would a given bright cutoff hamper the combination of LSST photometry or astrometry with that from other surveys (like VVV)?}
          \item{How short an exposure time can the facility support? Does the OpSim framework already include all the operational limitations on scheduling short exposures?}
\end{itemize}
}

\subsection{The Local Volume}

\new{Finally, we remind the reader that substantial opportunity
  remains to develop science figures of merit for {\it Local Volume}
  science cases. Figures of Merit for Local Volume science likely
  share much common ground with those for the Halo (discussed in
  Section \ref{sec:MW_Halo}), and substantial prior expertise exists
  \citep[e.g.][]{2014ApJ...795L..13H}. One straightforward Figure of
  Merit (FoM) might be the fraction of dwarf galaxies in the Local
  Volume that are correctly identified as a function of survey
  strategy.}

% ====================================================================

\navigationbar
