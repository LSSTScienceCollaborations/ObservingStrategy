% ====================================================================
%+
% SECTION:
%    MW_FutureWork.tex
%
% CHAPTER:
%    galaxy.tex
%
% ELEVATOR PITCH:
%    Ideas for future metric investigation, with quantitaive analysis
%    still pending.
%-
% ====================================================================

\section{Future Work}
\def\secname{MW_future}\label{sec:\secname}

% ====================================================================

% % ====================================================================
%+
% SECTION:
%    section-name.tex  % eg lenstimedelays.tex
%
% CHAPTER:
%    chapter.tex  % eg cosmology.tex
%
% ELEVATOR PITCH:
%    Explain in a few sentences what the relevant discovery or
%    measurement is going to be discussed, and what will be important
%    about it. This is for the browsing reader to get a quick feel
%    for what this section is about.
%
% COMMENTS:
%
%
% BUGS:
%
%
% AUTHORS:
%    Phil Marshall (@drphilmarshall)  - put your name and GitHub username here!
%-
% ====================================================================

\section{Dust in the Plane of the Milky Way}
\def\secname{MW_Dust}\label{sec:\secname} % For example, replace "keyword" with "lenstimedelays"

\noindent{\it Author Name(s)} % (Writing team)

% This individual section will need to describe the particular
% discoveries and measurements that are being targeted in this section's
% science case. It will be helpful to think of a ``science case" as a
% ``science project" that the authors {\it actually plan to do}. Then,
% the sections can follow the tried and tested format of an observing
% proposal: a brief description of the investigation, with references,
% followed by a technical feasibility piece. This latter part will need
% to be quantified using the MAF framework, via a set of metrics that
% need to be computed for any given observing strategy to quantify its
% impact on the described science case. Ideally, these metrics would be
% combined in a well-motivated figure of merit. The section can conclude
% with a discussion of any risks that have been identified, and how
% these could be mitigated.

A short preamble goes here. What's the context for this science
project? Where does it fit in the big picture?

% --------------------------------------------------------------------

\subsection{Target measurements and discoveries}
\label{sec:keyword:targets}

Describe the discoveries and measurements you want to make.

Now, describe their response to the observing strategy. Qualitatively,
how will the science project be affected by the observing schedule and
conditions? In broad terms, how would we expect the observing strategy
to be optimized for this science?


% --------------------------------------------------------------------

\subsection{Metrics}
\label{sec:keyword:metrics}

Quantifying the response via MAF metrics: definition of the metrics,
and any derived overall figure of merit.


% --------------------------------------------------------------------

\subsection{OpSim Analysis}
\label{sec:keyword:analysis}

OpSim analysis: how good would the default observing strategy be, at
the time of writing for this science project?


% --------------------------------------------------------------------

\subsection{Discussion}
\label{sec:keyword:discussion}

Discussion: what risks have been identified? What suggestions could be
made to improve this science project's figure of merit, and mitigate
the identified risks?


% ====================================================================

\navigationbar


% ====================================================================

% WIC - promoted this back to MW Halo section

% % ====================================================================
%+
% SECTION:
%    section-name.tex  % eg lenstimedelays.tex
%
% CHAPTER:
%    chapter.tex  % eg cosmology.tex
%
% ELEVATOR PITCH:
%    Explain in a few sentences what the relevant discovery or
%    measurement is going to be discussed, and what will be important
%    about it. This is for the browsing reader to get a quick feel
%    for what this section is about.
%
% COMMENTS:
%
%
% BUGS:
%
%
% AUTHORS:
%    Phil Marshall (@drphilmarshall)  - put your name and GitHub username here!
%-
% ====================================================================

\section{Mapping the Milky Way Halo}
\def\secname{MW_Halo}\label{sec:\secname} % For example, replace "keyword" with "lenstimedelays"

\noindent{\it Kathy Vivas, David Nidever, Colin Slater}  % (Writing team)

% This individual section will need to describe the particular
% discoveries and measurements that are being targeted in this section's
% science case. It will be helpful to think of a ``science case" as a
% ``science project" that the authors {\it actually plan to do}. Then,
% the sections can follow the tried and tested format of an observing
% proposal: a brief description of the investigation, with references,
% followed by a technical feasibility piece. This latter part will need
% to be quantified using the MAF framework, via a set of metrics that
% need to be computed for any given observing strategy to quantify its
% impact on the described science case. Ideally, these metrics would be
% combined in a well-motivated figure of merit. The section can conclude
% with a discussion of any risks that have been identified, and how
% these could be mitigated.

A short preamble goes here. What's the context for this science
project? Where does it fit in the big picture?

% --------------------------------------------------------------------

\subsection{Target measurements and discoveries}
\label{sec:keyword:MW_Halo_targets}

Describe the discoveries and measurements you want to make.

Now, describe their response to the observing strategy. Qualitatively,
how will the science project be affected by the observing schedule and
conditions? In broad terms, how would we expect the observing strategy
to be optimized for this science?


% --------------------------------------------------------------------

\subsection{Metrics}
\label{sec:keyword:MW_Halo_metrics}

Quantifying the response via MAF metrics: definition of the metrics,
and any derived overall figure of merit.


% --------------------------------------------------------------------

\subsection{OpSim Analysis}
\label{sec:keyword:MW_Halo_analysis}

OpSim analysis: how good would the default observing strategy be, at
the time of writing for this science project?


% --------------------------------------------------------------------

\subsection{Discussion}
\label{sec:keyword:MW_Halo_discussion}

Discussion: what risks have been identified? What suggestions could be
made to improve this science project's figure of merit, and mitigate
the identified risks?


% ====================================================================

\navigationbar


% ====================================================================

% \subsection{Other Ideas}

\credit{willclarkson}, \credit{akvivas}, \credit{vpdebattista}

In this final section we provide an extremely brief list of important science
cases that are still in an early stage of development, but that are
deserving of quantitative MAF analysis in the future.

\subsection{Further considerations for Milky Way static science}

  One important area of Milky Way science on which further
  community input is still sorely needed, is {\it static science} (a
  category that includes population disentanglement through deep,
  multicolor photometry), particularly in regions outside the main
  ``Wide-Fast-Deep'' (WFD) survey (Sections \ref{sec:MW_Astrometry} \&
  \ref{sec:MW_Halo} include discussion of static science in WFD regions). Since
  static science depends on depth (for, e.g., precise colors near the
  main sequence turn-off of some population) and uniformity over the
  survey (to aid characterization of strong selection functions),
  static science observing requirements may be in tension with (or at
  least not explicitly addressed by) requirements communicated
  elsewhere in this chapter.

  For example, probing deep within spatially crowded
  populations may lead to a sharper requirement on the selection of
  observations in good seeing conditions towards crowded regions, than
  has been apparent to-date. This needs quantification. 

  To pick another example, while we have indicated that co-added
  depth is a lower priority than temporal coverage for
  variability-driven studies in the Galactic Disk (conclusion A.1 in
  Section \ref{sec:MW_Disk}), co-added depth will likely be crucial for
  population disentanglement through photometry. While a
  judiciously-chosen observing strategy should be able to support both
  static and variable science, at this date quantitative trade-offs have not yet been specified.

  In many cases the implementation of figures of merit for static
  science in the Milky Way is complicated by the requirement to
  interface custom population simulations with the observational
  characterizations produced by the \MAF framework. For many
  investigators, the preferred method may be to use \MAF to produce
  parameterizations of the observational quantities of interest - for
  example, the run of photometric uncertainty against apparent
  magnitude, for each location on the sky, and including spatial
  confusion (all of which \MAF can currently produce) - and then use
  these characteristics as input to their own population simulations,
  on which the investigator may have invested substantial time and
  effort.

  To provoke progress, we specify in Table
  \ref{table:strawmanMWstaticScience} a possible Figure of Merit for
  static science in terms of capabilities mostly already provided by
  the \MAF framework, which does not require custom simulation. This
  Figure of Merit - which asks what fraction of fields in a spatial
  region of interest, are sufficiently well-observed to permit
  population disentanglement to some desired level of precision -
  could form the basis of several science FoM's (for example, the
  fraction of fields in which photometric age determinations of
  bulge/bar populations might be attempted). We encourage community
  development and implementation of this and other FoMs for Milky Way
  static science.

\begin{table}[h]
  \small
  \begin{tabular}{c p{12cm}}
    & {\it FoM innerMW-Static: fraction of fields in inner Galactic plane adequately covered for population discrimination} \\
    \hline
    1. & Produce absolute magnitudes $M_{u,g,r,i,z,y}$~of the population of interest, using \MAF's spectral libraries; \\
    2. & For each HEALPIX (i.e. pointing):\\
       & 2.1. Place the fiducial star at appropriate line-of-sight distance, produce apparent magnitudes $m_{u,g,r,i,z,y}$;\\
       & 2.2. Modify $m_{u,g,r,i,z,y}$~for extinction using \MAF's extinction model;\\
    & 2.3. Compute the photometric and astrometric uncertainties due to sampling and random error (from \MAF's ``m52snr'' method);\\
    & 2.4. Convert exposure-by-exposure estimates of photometric and astrometric uncertainty due to spatial confusion, into co-added uncertainties;\\
    & 2.5. Combine the random and confusion uncertainties into final measurement uncertainties on the photometry and astrometry;\\
    & 2.6. Use uncertainty propagation to estimate color uncertainties $u-g, g-r, r-i, i-z$;\\
    3. & Count the fraction of sight-lines for which the color below the threshold needed for "sufficient" accuracy in parameter determination \citep{ivezic08}. {\bf This is the figure of merit.} \\
\hline
    \end{tabular}
 \caption{Description of Figure of merit ``innerMW-Static''}
  \label{table:strawmanMWstaticScience}
\end{table}

Finally, we provide an extremely brief list of important science
cases that are still in an early stage of development, which fall into the ``Static science'' category of Milky Way science:
\begin{itemize}
  \item {\it Formation history of the Bulge and present-day balance of
  populations:} Sensitivity to metallicity and age distribution of Bulge
  objects near the Main Sequence Turn-off;
  \item {\it Migration and heating in the Milky Way disk:} Error and
  bias in the determination of components in the (velocity dispersion vs
  metallicity) diagram, for disk populations along various lines of
  sight \citep[e.g.][]{2016ApJ...818L...6L}.

% WIC 2017-05-24: commented this out in favor of a dedicated local volume subsection.
%
%\item Fraction of Local-Volume objects discovered as a function of
%  survey strategy.
\end{itemize}

\subsection{Short exposures}

  Populations near, or brighter than, LSST's nominal saturation
  limit ($r \sim 16$~with 15s exposures) are likely to be crucial to a
  number of investigations for Milky Way investigations, whether as
  science tracers in their own right, or as contaminants that might
  interfere with measurements of fainter program objects (due, for
  example, to charge bleeds of bright, foreground disk objects).

  Quantitative exploration of these issues now requires involvement from
  the community (e.g. to determine LSST's discovery space for bright
  tracers in context with other facilities and surveys like ZTF, Gaia
  and VVV), and the project (e.g. to determine the parameters of short
  exposures that might be supported by the facility). To provoke
  development, here we list a few example questions regarding short
  exposures that still need resolution:

\begin{itemize}
  \item{What level of bright-object charge-bleeding can be tolerated? For example, is some minimum distribution of position-angles required in order to spread the bleeds azimuthally over the set of observations of a particular line of sight (so that different pixels fall under bleeds in different exposures)?\footnote{Note that with $\sim 30$~exposures per filter per field over ten years towards some regions, charge-bleed directions might not be highly randomized.} What is this minimum?}
    \item{What is the science impact of restricting targets to $r \gtrsim 16$?}
      \item{Will the proposed twilight survey of short exposures be adequate for science cases requiring short exposures? Are there enough observations from other surveys (e.g. DES or other DECam surveys) to cover the bright end of the entire LSST footprint, and are {\it new} observations of very bright objects required scientifically in any case?}
        \item{What is the bright limit required for adequate astrometric cross-calibration against Gaia?}
          \item{To what extent would a given bright cutoff hamper the combination of LSST photometry or astrometry with that from other surveys (like VVV)?}
          \item{How short an exposure time can the facility support? Does the OpSim framework already include all the operational limitations on scheduling short exposures?}
\end{itemize}


\subsection{The Local Volume}

  Finally, we remind the reader that substantial opportunity
  remains to develop science figures of merit for {\it Local Volume}
  science cases. Figures of Merit for Local Volume science likely
  share much common ground with those for the Halo (discussed in
  Section \ref{sec:MW_Halo}), and substantial prior expertise exists
  \citep[e.g.][]{2014ApJ...795L..13H}. One straightforward Figure of
  Merit (FoM) might be the fraction of dwarf galaxies in the Local
  Volume that are correctly identified as a function of survey
  strategy.

% ====================================================================

\navigationbar
