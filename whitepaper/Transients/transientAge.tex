% ====================================================================
%+
% SECTION:
%    transientAge.tex
%
% CHAPTER:
%    transients.tex
%
% ELEVATOR PITCH:
%
%-
% ====================================================================

% \section{Young Transients Discrimination Power}
\subsection{Young Transients Discrimination Power}
\def\secname{transientsAge}\label{sec:\secname}

\credit{svalenti}

In this section we investigate the possibility to identify young
transients using the intra-night visits. The Baseline Cadence predicts
that on average, fields in the main survey get revisited about every 3
days using all filters, and every 15 days when using only r band visits
(see section 2.2).  This means that we are likely to discover transients
that are between 0 and 3 days old. As pointed out in section 6.1.2, the
first hours after the explosion reveal fundamental information on the
nature of transients. It is then important to be able to select, among
the large number of transients discovered by LSST, the youngest objects.
The Baseline Cadence predicts that the second intra-night visit will
occur between 30 minutes to 2 hours after the first visit.  The question
we are trying to answer here is: Which intra-night gap will maximize the
identification of young objects? To answer this question, we have
selected a set of transients with good photometric coverage in the first
week after the the outburst/explosion (see left panel of Figure 1) and
compute the light curve slope (mag/day) as a function of time (see right
panel of Figure 1). In Figure 2,  we report the change in brightness
between the first and the second visit for a set of different transients
as function of phase from explosion. Despite the heterogeneity in light
curves shapes most of the transients show a similar change in brightens
on a short time scale. This confirms that early classification and
identification of interesting transients in a short time scale is a
major challenge. However, independently on the type of transient, young
transients may be easier to identify with large time gap between visits
(2 hours). In general, most of the transients have a large increase in
brightness at early phase. If the second visit occurs only 30 minutes
after the first visit, the change in brightness will be of the order of
1$\%$ or less independently on the type of transient or the time from
the start of the outburst/explosion, (see left panel of Figure 2). If
the second visit occurs 2 hours after the first visit, the change of
brightness will be large enough to be detected for young transients
($\sim$ 5$\%$). It is also worth to notice that a larger gap (24 hours),
while could help in identify young transients, does not help in identify
the type of transient.  The identification of interesting transients, at
early stage, can be achieved using supplementary information like
historical information from previous visits or color information of the
transients. Finally, we want to stress that the quality of early
multiwavelengh data available today is still limited; the sample of
astronomical transients used here is not comprehensive and an uniform
set of homogeneous data of different transients is still needed in order
to further investigate the need of color information.

\begin{figure}[hbt]
\centerline{
\includegraphics[width=0.6\textwidth]{figs/transients/earlyslope.pdf}
}
\caption{light curve slope [mag/days] of different type of transients as
function of the phase from transient outburst/explosion.}
\label{fig:earlyslope}
\end{figure}

\begin{figure}[hbt]
\centerline{
\includegraphics[width=0.6\textwidth]{figs/transients/earlyrise.pdf}
}
\caption{Expected difference magnitudes between two consecutive
observations for a set of astronomical transients as a function of the
phase of the transient. We consider the cases of the second observation
occurring 30 minutes (left panel), 2 hours (central panel) and 24 hours
(right panel) after the first observation. Data from: SN~Ia,  Olling et
al 2015; SNII, Rubin et al 2016; SN~.Ia, 2010ApJ...715..767S; SN~Ib,
Valenti et al 2011, Chao et al 2013; SN~Ic, Mazzali et al 2002; cv,
Sokoloski et al. 2013, Finzell, et al in prep .
}
\label{fig:earlyrise}
\end{figure}
%%%%%%%%%%%%%%%%%%%%%%%%
%    Ineed to add the references
%
%Olling et al 2015, Nature, 521, 332O
%Rubin et al 2016, 2016ApJ, 820, 33R
%Valenti et al 2011, MNRAS, 416, 3138V
%Chao et al 2013,  ApJ, 775, 7C
%Mazzali et al, 2002, ApJ, 572L, 61M
%Sokoloski et al. 2013, 2013ApJ...770L..33S
%Finzell, et al in prep
