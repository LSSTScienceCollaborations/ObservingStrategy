% ====================================================================
%+
% SECTION:
%    eruptive.tex
%
% CHAPTER:
%    transients.tex
%
% ELEVATOR PITCH:
%-
% ====================================================================

% \section{LBVs and related non-supernova transients}
\subsection{LBVs and related non-supernova transients}
\def\secname{\chpname:LBVs}\label{sec:\secname}

\credit{nathansmith}

There is a large and diverse class of visible-wavelength transient
sources recognized in nearby galaxies that appear to be distinct from
traditional novae and from SNe, and have often been associated with
the giant eruptions of luminous blue varibles (LBV), such as the 19th
century outburst of $\eta$ Carinae.  Broadly speaking, members of this
class of transients share the common properties that they have peak
luminosities below those of most core-collapse SNe and more luminous
than novae and CVs (absolute magnitudes of roughly $-$9 to $-$15 mag).
They also have H-rich spectra (usually) with relatively narrow lines
that indicate modest bulk outflow velocities of 10$^2$ to 10$^3$ km
s$^{-1}$ (although some have exhibited small amounts of material at
faster speeds).  They tend to evolve on fairly long timescales of
weeks to years (although sometimes they exhibit a quick rise to peak
similar to SNe II-P). This group of transients has gone by many names,
such as LBV eruptions, SN impostors, Type V supernovae,
intermediate-luminosity optical (or red) transients, as well as others
that often include a physical interpretation.
%For brevity, these are
%often collectively referred to as ``LBVs'', although many of them may
%not actually be LBVs.
%This may be largely for historical reasons,
%since LBVs were the first of these to be recognized as a class.  Some
%of the subgroups may be very different from objects like $\eta$
%Carinae, however.

Observationally, these eruptions are understood to represent important
and dramatic mass-loss episodes in the lives of massive stars, based
on empirical estimates of the amount of ejected matter.
%Guided
%largely by nearby LBVs with resolved shells, t
These eruptions are
expected to instigate mass loss that is comparable to or more
important than metallicity-dependent winds of massive stars.  This
mode of mass loss, regardless of the mechanism, may be a very
important ingredient in the evolution of massive stars that is
currently not included in stellar evolution models.  Correcting this
is one of the key science drivers in trying to understand the physics
these eruptions.

%The degeneracy
%arises because when the objects are
%fully obscured by dust, one cannot actually meaure the star's
%temperature, and the bolometric luminosities of super-AGB and red and
%blue supergiants overlap.  Unfortunately,
%cases when we have strong constraints on the quiescent progenitor are
%rare, and once they reach their peak luminosity, there is a great deal
%of overlap in observed properties.

Theoretically, these eruptions are not understood.  %There are many
%ideas, but few if any confirmed mechanisms tied to observed
%objects.
Some
%previously discussed
theoretical ideas involve (1)
winds driven by super-Eddington instabilities (although the root cause
for suddenly exceeding the Eddington limit remains unexplained), (2)
hydrodynamic explosions caused by deep-seated energy deposition, such
as unsteady nuclear burning, (3) accretion onto companion stars in
binary systems (degenerate or not), (4) mergers in binary and triple
systems, (5) electron-capture SNe, and (6) ``failed SNe'' associated
with a weak explosion and envelope ejection that results from black
hole formation during core collapse.
Because of the relatively low total energy indicated by
radiative luminosities and outflow speeds, these are usually discussed
as non-terminal eruptions, however, the last two are terminal events
that are less luminous and lower energy than normal SNe, and the last
3 should only occur once for a given source.
%Together with several well-studied examples that indicate
%repeating eruptions, there are indeed many
%cases where only one such transient has been seen at the same
%position, and some cases where late-time observations suggest that no
%source has survived with a luminosity comparable to its progenitor.
%However, there are also several well-studied examples that indicate
%repeating eruptions (multiple repeating transients, multiple nebular
%shells with different ages, etc).
All these theoretical mechanisms
may lead to similar observed phenomena: weak explosions, moderate
luminosities, slow expansion, dusty aftermath, but this class of objects
may represent a mixed-bag of different mechanisms that get lumped
together by default as ``other'' because they are not traditional SNe.

Rates for these LBV-like eruptions are still very poorly constrained;
%largely because most previous SN and transient searches with small
%telescopes have been optimized for finding more luminous SNe in a
%larger volume.  This field begun to change with recent surveys, and will
%be revolutionized with LSST.  F
%from discovered examples we have,
numbers are very roughly consistent with a volumetric rate comparable
to that of core-collapse SNe or larger.  %, but with a large error bar.
Limited information often makes classification into various
subgroups difficult or highly subjective, thus subclass rates are even
less well constrained.  %The ``rate'' also depends on how
%faint the lower limit of inclusions is; e
Evidence suggests that the brightest events occurr less frequently and that
numbers increase as one moves to lower luminosity.  At the faint end,
it becomes difficult to distinguish between eruptions and regular
variability of LBVs, or between massive star eruptions and CVs.  \emph{With
deep LSST stacks identifying faint CV in quiescent states this will
 change dramatically in the LSST age, with the unvailing of
detailed progenitor information}.  Having deep, pre-eruption
characterization of sources at the positions of these eruptive
transients (as well as SN precursors) will likely be a major
contribution of LSST.
An important empirical discriminant of subgroups in this class comes
from their progenitor stars.  Some are indeed seen to be very
luminous, blue supergiant stars consistent with traditional LBVs.
Some, however, have somewhat less luminous, heavily dust-obscured
progenitor stars that have been associated with either dust-enshrouded
blue or red supergiants, or alternatively, with super-AGB stars of
8-10 M$_{\odot}$%, with uncertainty.


An area of recent interest is that eruptive non-terminal transients
have been observed, in some cases, to precede much more powerful
explosions that are seen as Type IIn supernovae.  \emph{Detectability of SN
precursors eruptions is discussed in ~\autoref{chp:galaxy}.
LSST can provide a large enough sample of these events to enable the study
or rates.} %There may also be
SN
precursors have observed or inferred properties that are very similar
to LBVs and related transients,
%.  This may suggest some link between them,
but then again, most of the LBVs and other SN impostors have been
observed for decades and have not gone SN (yet).  \emph{Being able to
  distinguish which of these optical transients are SN precursors and
  which are not is a major science driver.}  The amount of mass lost
in a precursor eruption may dramatically alter the type of SN that is
observed.  Even if the pre-SN transients are not observed directly,
pre-SN eruptive mass loss can be inferred and constrained with
persistent observations of the detected eruptions' lightcurves (and
spectra) through circumstellar interaction diagnostocs of
the bright eruptions and explosions.
%a continuum of energies in pre-SN outbursts, extending down to more
%normal classes of core-collapse SNe, but these may often go
%unrecognized unless the SN is caught very early after explosion.

In terms of timescales, many of the eruptive transients exhibit rise
and decline timescales similar to normal SNe~II-P or II-L, but with
fainter peak luminosity.  For these, observational cadence
requirements will be the same as SNe.  For some eruptive transients,
however, the rise timescales can be very long (rising a few magnitudes
in years).  While LSST's cadence will certainly be fast enough, being
able to discover slowly rising transients that do not change much from
night to night will be an important metric, and the edge effects
should be investigated.
For the faster-rising transients, just like for SNe, spectroscopic
follow-up is needed to discriminate these from normal SNe, and also
contextual information about the host galaxy (and hence, the absolute
magnitude) is needed to differentiate these non-terminal eruptions
from Type IIn supernovae (their spectra look similar, although LBVs do
tend to have narrower lines).  %Spectral and color evolution, as well
%as information about the progenitor, is needed to distinguish among
%subgroups within the class.
Multiwavelength follow-up is often extremely valuable or even
essential; i.e. mid-IR tells us if an optically invisible source is
cloaked in a dust shell but still quite luminous; Xrays and radio tell
us if an expanding shock wave is the likely source of persistent
luminosity.  For these reasons, nearby cases will continue to be the
most valuable in deciphering the physics of subclasses, whereas the
increased volume in which LSST discovers these fainter transients will
drastically improve our understanding of their rates.  Armed with both
a better understanding of their underlying physics and
characterization, as well as their rates and duty cycles, these
eruptive events can then be incorporated into stellar evolution models
and population synthesis/feedback models.

% % --------------------------------------------------------------------
%
% \subsection{Metrics}
% \label{sec:\secname:metrics}
%
% % --------------------------------------------------------------------
%
% \subsection{OpSim Analysis}
% \label{sec:\secname:analysis}
%
% % --------------------------------------------------------------------
%
% \subsection{Discussion}
% \label{sec:\secname:discussion}
%
% ====================================================================
%
% \subsection{Conclusions}
%
% Here we answer the ten questions posed in
% \autoref{sec:intro:evaluation:caseConclusions}:
%
% \begin{description}
%
% \item[Q1:] {\it Does the science case place any constraints on the
% tradeoff between the sky coverage and coadded depth? For example, should
% the sky coverage be maximized (to $\sim$30,000 deg$^2$, as e.g., in
% Pan-STARRS) or the number of detected galaxies (the current baseline
% of 18,000 deg$^2$)?}
%
% \item[A1:] ...
%
% \item[Q2:] {\it Does the science case place any constraints on the
% tradeoff between uniformity of sampling and frequency of  sampling? For
% example, a rolling cadence can provide enhanced sample rates over a part
% of the survey or the entire survey for a designated time at the cost of
% reduced sample rate the rest of the time (while maintaining the nominal
% total visit counts).}
%
% \item[A2:] ...
%
% \item[Q3:] {\it Does the science case place any constraints on the
% tradeoff between the single-visit depth and the number of visits
% (especially in the $u$-band where longer exposures would minimize the
% impact of the readout noise)?}
%
% \item[A3:] ...
%
% \item[Q4:] {\it Does the science case place any constraints on the
% Galactic plane coverage (spatial coverage, temporal sampling, visits per
% band)?}
%
% \item[A4:] ...
%
% \item[Q5:] {\it Does the science case place any constraints on the
% fraction of observing time allocated to each band?}
%
% \item[A5:] ...
%
% \item[Q6:] {\it Does the science case place any constraints on the
% cadence for deep drilling fields?}
%
% \item[A6:] ...
%
% \item[Q7:] {\it Assuming two visits per night, would the science case
% benefit if they are obtained in the same band or not?}
%
% \item[A7:] ...
%
% \item[Q8:] {\it Will the case science benefit from a special cadence
% prescription during commissioning or early in the survey, such as:
% acquiring a full 10-year count of visits for a small area (either in all
% the bands or in a  selected set); a greatly enhanced cadence for a small
% area?}
%
% \item[A8:] ...
%
% \item[Q9:] {\it Does the science case place any constraints on the
% sampling of observing conditions (e.g., seeing, dark sky, airmass),
% possibly as a function of band, etc.?}
%
% \item[A9:] ...
%
% \item[Q10:] {\it Does the case have science drivers that would require
% real-time exposure time optimization to obtain nearly constant
% single-visit limiting depth?}
%
% \item[A10:] ...
%
% \end{description}
%
% ====================================================================
%
\navigationbar
