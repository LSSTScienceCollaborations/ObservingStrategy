% ====================================================================

% LSST Observing Strategy White Paper

% Copyright 2015 The LSST Science Collaborations

% ====================================================================

\documentclass[11pt,headsepline,cleardoubleempty,twoside,openright]{scrbook}
% 11pt font
% Draw a line under the header
% Don't draw a line or print page numbers when a page is empty
% Pages are two-sided - so margins alternate
% Chapters start on the righthand side of the page

\usepackage{LSST_Observing_Strategy_White_Paper}

% ====================================================================

\begin{document}

\begin{titlepage}
\begin{center}

\vspace*{\stretch{3}}

{\Huge\bfseries\scshape
 Science Driven Optimization \\
 of the LSST Observing Strategy}

\vspace*{\stretch{2}}

% Version number, to be taken out for published draft

% \include{VersionDate}

% \begin{center}
% {\Large\bf Version 1.0\\
%
% \vspace*{\stretch{0.2}}
%
% November 2009}
% \end{center}

\vspace*{\stretch{2.5}}

{\Large  Prepared by the LSST Science Collaborations,}\\
\vspace*{\stretch{0.15}}
{\Large with contributions from the LSST Project. }\\
\vspace*{\stretch{1}}

\end{center}
\end{titlepage}

% --------------------------------------------------------------------

\tableofcontents

% --------------------------------------------------------------------
\clearemptydoublepage

\chapter*{Preface}
\addcontentsline{toc}{section}{Preface}
\markboth{}{}

\noindent This is a community white paper outlining various science
cases and the impacts that observing strategy will have on them,
quantified using the Metric Analysis Framework. We will describe
various strategies and tradeoffs that impact the observing cadence
(visit sequence), the current cadence baseline, and future directions
for the optimization of the  survey strategy. We aim to publish this
white paper on arXiv, and invite community feedback.

The timescale for producing this white paper, started before and
finished after the Observing Strategy workshop at the  August 2015
LSST Project and Community workshop, is many months.

The main points we will aim to convey in this white paper are as follows:

\begin{itemize}

    \item We have a pretty good idea of how we would deploy LSST:
    there is a baseline strategy and example cadences, with which it
    can be demonstrated that the data required for the promised
    science can be delivered.

    \item The baseline strategy can and will be optimized -- even small
    improvements can be significant. Most importantly, the strategy is
    not set in stone and it will evolve.

    \item The cadence optimization process will be as open and
    inclusive as technically possible. All stakeholders will
    participate in this process.

\end{itemize}

\raggedright{Project start: July 2015.}


% --------------------------------------------------------------------

\chapter[Introduction]{Introduction}
\def\chpname{intro}\label{chp:\chpname}

Chapter editors:
\credit{bethwillman},
\credit{connolly}.

The Large Synoptic Survey Telescope (LSST) is a dedicated optical
telescope with an effective aperture of 6.7 meters, currently under
construction on Cerro Pach\'on in the Chilean Andes.  The telescope
and camera will have a huge field of view, 9.6 deg$^2$, and the
\'etendue, i.e., the product of collecting area and field of view will
be significantly larger than any other optical telescope.  Thus this telescope
is designed for wide-field deep imaging of the sky; its mantra is
``Wide-Deep-Fast'', i.e., the ability to cover large swaths of sky
(``Wide'') to faint magnitudes (``Deep'') in a short amount of time
(``Fast''), allowing it to scan the sky repeatedly.  LSST will image
in six broad filters, $ugrizy$, spanning the optical band from the
atmospheric cutoff in the ultraviolet to the limit of CCD sensitivity
in the near-infrared.  

  The science case for the LSST is based broadly on four science
  themes:
\begin{itemize}
\item Dark energy and dark matter (via measurements of strong and weak lensing,
  large-scale structure, clusters of galaxies, and supernovae);
\item Exploring the transient and variable universe;
\item Studying the structure of the Milky Way galaxy and its neighbors
  via resolved stellar populations;
\item An inventory of the Solar System, including Near Earth Asteroids
  and Potential Hazardous Objects, Main Belt Asteroids, and
  Kuiper Belt Objects.
\end{itemize}

These themes, together with {\em many} other science applications, are
described in detail in the
\href{http://lsst.org/scientists/scibook}{LSST Science Book}, produced
by the LSST Project Team and Science collaborations in 2009.  The
present white paper represents an important next step in science
planning beyond the Science Book.  In particular, we now need to
quantify how well the LSST (for a given realization of its observing
strategy, or ``{\em cadence}'') will be able to carry out its science
goals; indeed, we will use this quantification to refine and optimize
the cadence itself.  To zeroth order, the large \'etendue of LSST
allows it to meet all its science goals with a single dataset with a
universal cadence.  This document describes the design of the LSST
cadence and the various ways in which can be further refined to
optimize the science output of the survey.  As we describe in detail
below, we quantify the effectiveness of a given cadence realization to
meet science goals by defining a series of quantitative {\em metrics}.
Any given realization will be more favorable for some science areas,
and less so for others; the metrics allow us to quantify this, and
optimize the overall cadence for the broadest range of LSST science
areas.

In the six years since the Science Book was written, some of the
science themes described in there have evolved or become obselete,
while new science opportunities and ideas have arisen.  Moreover, our
understanding of the capabilities (such as system response and
therefore depth, telescope optics, and so on) have matured
considerably.  The present document endeavors to explore the principal
science themes as described in the Science Book, but is not slaved to
them, and where appropriate, we will point out relevant updates to the
Science Book.



% --------------------------------------------------------------------

\section{Synoptic Sky Surveying at Universal Cadence}
\def\secname{intro:baseline}\label{sec:\secname}

  The LSST defined a so-called ``baseline cadence'', described in the
  \href{http://adsabs.harvard.edu/abs/2008arXiv0805.2366I}{LSST
    overview paper} and Chapter 3 of the Science Book.  This was used
  to demonstrate that LSST could meet its basic science goals, and
  indeed the formal
  \href{https://www.lsstcorp.org/docushare/dsweb/Get/LPM-17}{science
    requirements}.    As 
  described in these references, the default LSST exposure is 15
  seconds, and all exposures are taken in pairs, called a ``{\em
    visit}'', before the telescope is slewed to a neighboring field.
   Any given field is observed twice on a given night, which allows
   preliminary trajectories of asteroids to be determined.  

   The baseline cadence optimizes the amount of sky covered in any
   given night (subject to the constraint of observing at airmass less
   than 1.4 throughout), and allows the entire sky visible at any time
   of the year to be covered in about three nights.  The cadence is
   designed to give uniform coverage at any given time, and reaches
   survey goals for measuring stellar parallax and proper motion over
   the ten-year survey.  The survey requirements on depth lead to
   roughly 825 visits (summing over the six filters) in the 10-year
   LSST survey to any given point on the sky.  The resulting
   Deep-Wide-Fast component of the survey covers roughly 18,000
   deg$^2$ of high Galactic latitude sky, and requires about 85\% of
   the available observing time. 

   There are obvious science cases that the Deep-Wide-Fast survey does
   not address, and thus the remaining 15\% of the telescope time in
   the baseline cadence is devoted to a series of specialized
   surveys.  They are as follows: 
\begin{itemize} 
\item Imaging at low Galactic latitudes.  This is currently defined as
  a wedge which is broader closer to the Galactic Center,
  corresponding roughly to a locus of constant stellar density.  In
  this region, the number of repeat observations is reduced, given the
  confusion limit in the stacked LSST data. 
\item Imaging in the South Celestial Cap.  The airmass limit of 1.4
  restricts observations to declination $> -70^\circ$ ({\bf get
    precise value}), thus missing both the Magellanic Clouds.
  Observations are done in the Cap to cover this region of sky, again
  to shallower depth. 
\item Imaging in a series of four {\em Deep Drilling Fields}, single
  pointings in which 
  we will obtain roughly four {\bf check!} times more exposures in all
  filters in order to go about a magnitude fainter in the stacked
  data, as well as to get better sampled light curves of variable
  objects. 
\item Imaging in the Northern portions of the Ecliptic Plane.  The
  airmass limit of 1.4 restricts us to declination $< +2^\circ$ ({\bf
    get precise value}), which means that much of the ecliptic plane
  is uncovered.  By observing with a reduced cadence close to the
  Ecliptic Plane north of this limit, we will be able to significantly
  increase the fraction of Near-Earth Asteroids and Main Belt
  Asteroids for which LSST obtains orbits.
\end{itemize}

  The LSST Project has developed an ``Operations Simulator'' (\OpSim),
  described in detail in Chapter~\ref{chp:cadexp} of this paper, which includes a
  realistic model of telescope operations, including time required for
  camera readout, slew time, filter exchange, as well as time loss due
  to clouds.  Given a series of ``proposals'' {\bf Do we still want to
    use this vocabulary?} that set priorities of which fields to
  observe at any given time, \OpSim has developed a series of
  realizations of the series of observations that make up the ten-year
  LSST survey.  Given such a realization, \OpSim outputs detailed
  metrics describing, for example, the expected depth of the LSST
  survey in each filter as a function of time.  The baseline cadence
  is a specific realization of the \OpSim output, which meets the LSST
  survey requirements, following the rules briefly outlined above.  
   
  Again, while the baseline cadence demonstrates that the LSST is
  capable of meeting its stated science goals, it is not optimized for
  all science, and Chapter~\ref{chp:cadexp} of this document describes a series of
  experiments varying the assumptions in \OpSim.  In
  Chapter~\ref{chp:specialsurveys}, we explore additional ideas for
  future experiments to be done in \OpSim.  

%Synoptic Surveying with LSST - the basic observing strategy determined
%by key projects described in the LSST Science Requirements Document,
%and constrained by the LSST's design \citep{IvezicEtal2008}.

% --------------------------------------------------------------------
\iffalse
THIS SECTION HAS BEEN ABSORBED INTO WHAT IS ABOVE.
\section{Beyond the Baseline Observing Strategy}
\def\secname{intro:baseline}\label{sec:\secname}

Optimizing the Observing Strategy: what perturbations are we
permitted to introduce, to maximize the system's science capabilities?
What are our constraints? And our opportunities?
\fi

% --------------------------------------------------------------------

\section{Evaluation and Optimization of the LSST Observing Strategy}
\def\secname{intro:evaluation}\label{sec:\secname}

The next step is to quantify how well any given realization of the
LSST observing strategy (i.e., an output of \OpSim) supports the (many)
science projects that LSST will enable.  As the algorithms controlling
\OpSim are varied, some projects will benefit, while others may
suffer.  By quantifying this for each projects, we can determine which cadence
maximizes the science potential overall of the project. 

Thus we need 
%e first step towards a science-based optimization of the LSST
%observing strategy is 
a {\it science-based evaluation of the baseline
  LSST observing strategy and its variants}. After simulating a sample
observing schedule consistent with this strategy (see
\autoref{chp:cadexp}), we then need to quantify its value to each
science team.  This is what the LSST DM Sims team's ``Metric Analysis
Framework'' was designed to enable. Once the fiducial strategy has
been evaluated in this way, then any other strategy can be evaluated
in the same terms, using the same code, and we will be able to %start
optimize the strategy through iterations between \OpSim and MAF.

With this program in mind, it makes sense to define {\it one ``Figure
of Merit'' (FoM) per science project}, that captures the value of  the
observing strategy under consideration to that science team. This FoM
will probably be a function of several ``metrics'' that quantify
lower-level features of the observing sequence.  For Figures of Merit
to be directly comparable between disparate science projects,  they
need to be dimensional, and have the same units. One natural
choice could be the {\it information gained} by the science team, in
bits. This is a well-defined statistical quantity, albeit not yet one
in common use. A given observing schedule's value would then depend on
both this information gain, but also {\it how much that information is
worth to the whole community}. It is at this point that the debate
could become heated: probably the best we can do in Cadence Diplomacy
is to quantify all the information gains implied by each proposed
change to the baseline  observing strategy, combine them to see
whether it makes everyone happy, and iterate. In this way we might
hope to minimize the debates about the less quantifiable worth of each
piece of information.

We are some way from being able to define information-based Figures of
Merit for most science cases -- but the metrics that they will depend
on will be easier to derive. Writing this white paper is an
opportunity to think through the Figure of Merit for each science
project that we as a community want to carry out, and how that measure
of success is likely (or even known) to depend on metrics that
summarize the observing sequence presented to us. Thinking about the
problem in terms of science projects, each with a  Figure of Merit,
encourages us to design modular document sections, with one science
project and one Figure of Merit per section.

{\bf I am not sure the following paragraph belongs in the white paper;
  it is more a description of the writing process.}

This will have the happy side-effect of allowing the chapters to be
straightforwardly re-arranged as we go, to make the white paper easier
to read. It will also naturally lead to the definition of a suite of
MAF  super-metrics, can be evaluated on any future \OpSim output
database.  A table in each section showing the values of the metrics
and the FoM, for different schedules, for that science project, will
be very helpful. The metric names in these tables should match the
metric class names in the
\href{https://github.com/LSST-nonproject/sims_maf_contrib/wiki}{\simsMafContrib}
module. In principle these tables could be auto-generated by the MAF
framework, and extended as \OpSim is repeatedly reconfigured and run.

For an example of how all this could look, please see the
\hyperref[sec:lenstimedelays]{lens
time delays section}. The MAF subsections are still under development
there, but keep checking back to see it come together during the
August 2015 workshop week. Templates for the chapters and sections can
be found in \autoref{chp:example}.


% --------------------------------------------------------------------

\section{Outline of This Paper}
\def\secname{intro:outline}\label{sec:\secname}

The rest of this white paper is structured as follows. In
\autoref{chp:cadexp} we describe a number of \OpSim simulated observing
schedules (``cadences'') explored by the LSST Sims team in summer 2015
in preparation for this paper: they include a ``baseline cadence'', and
then some small but interesting perturbations to it. Then, we present
the science cases considered so far, organised into the following
chapters:

\begin{itemize}
    \item \autoref{chp:solarsystem}: \nameref{chp:solarsystem}
    \item \autoref{chp:galaxy}: \nameref{chp:galaxy}
    % \item \autoref{chp:astrometry}: \nameref{chp:astrometry}
    \item \autoref{chp:variables}: \nameref{chp:variables}
    \item \autoref{chp:transients}: \nameref{chp:transients}
    \item \autoref{chp:MCs}: \nameref{chp:MCs}
    \item \autoref{chp:agn}: \nameref{chp:agn}
    \item \autoref{chp:cosmo}: \nameref{chp:cosmo}
    % \item \autoref{chp:deepdrilling}: \nameref{chp:deepdrilling}
    \item \autoref{chp:specialsurveys}: \nameref{chp:specialsurveys}
    \item \autoref{chp:wfirst}: \nameref{chp:wfirst}
\end{itemize}

Finally, in \autoref{chp:tradeoffs} we bring the results of all the
science metric analyses  together and discuss the tensions between
them, and the trade-offs that we can anticipate having to make.

\navigationbar

% --------------------------------------------------------------------


% --------------------------------------------------------------------

\chapter[Solar System]{A Solar System Census}
\def\chpname{solarsystem}\label{chp:\chpname}

Chapter editors:
\credit{rhiannonlynne},
\credit{davidtrilling}.
% Mike Brown, Eric Christensen

% ====================================================================

\section{Introduction}
\label{sec:\chpname:intro}

% Introduce, with a very broad brush, this chapter's science projects,
% and why it makes sense for them to be considered together.

% ====================================================================

\section{Discovering Solar System Objects}
\def\secname{\chpname:discovery}\label{sec:\secname}

Discovering, rather than simply detecting, small objects throughout
the Solar System requires unambiguously linking a series of detections
together into an orbit. The orbit provides the information necessary
to scientifically characterize the object itself and to understand the
population as a whole. Without orbits, the detections of Solar System
Objects (SSOs) by LSST will be of limited use; objects discovered with
other facilities could be followed up by LSST, but almost the entire
science benefit to planetary astronomy would be lost.

Therefore, the primary concern regarding the Solar System is related
to the question ``Can we link detections of moving objects into
orbits?''.  This requirement poses varying levels of difficulty as we
move from Near Earth Objects (NEOs) through the Main Belt Asteroids
(MBAs) and to TransNeptunian Objects (TNOs) and Scattered Disk Objects
(SDOs), as well as for comets and for other unusual but very
interesting populations such as Earth minimoons.

{\it discuss specific challenges for each population; TNOs and SDOs are
  relatively easy, MBAs are very numerous, NEOs are hard because of
  speed, comets and minimoons are hard because of nongravitational forces}

Much of the answer to this question comes down to the performance of
various pieces of LSST Data Management software. In particular, the
false positive rate resulting from difference imaging, the compute
limitations of the Moving Object Processing System (MOPS) to extend to high
apparent velocities, and the capability to unambiguously determine if
a linkage is `real' or not via orbit determination (done as part of
MOPS). Additional concerns are related to how well observations
widely separated in time can be linked into the `discovery' orbits
(i.e. if we have a discovery in year 1, but do not detect the object
again until year 3, could these observations be linked?). The answers
to these questions range beyond the limits of the OpSim simulated
surveys, but bear on the observing strategy requirements for
discovering Solar System Objects.

{\it describe current minimum observation requirements for existing
  surveys, describe current expected requirements for MOPS, describe
  current effort to understand if MOPS requirements are realistic in
  LSST context}

If we assume various detection requirements, ranging from XXX to the
minimum MOPS requirements, we can characterize the performance of
available simulated surveys in terms of their expected detection rates
for various known populations.

{\it describe completeness metrics for NEOs/MBAs/TNOs/etc - known
  populations. what do we do about unknown populations?}

Beyond this basic but absolutely critical requirement to actually
discover SSOs across the Solar System, we can start to look at other
science goals: detecting activity, determining colors for moving
objects, and measuring shapes and spin states for objects.

{\it describe requirements and challenges for these; why colors are
  hard, how many objects will we actually be able to determine
  shape/spin for, how lightcurves may differ from shape/spin}

Note: take a look at
\texttt{https://github.com/rhiannonlynne/MafSSO/blob/master/SSO\_Analysis.ipynb}
(an extremely messy ipython notebook, but starting to point at some of
the ideas I have for metrics -- let's expand on this)


% --------------------------------------------------------------------

\subsection{Target measurements and discoveries}
\label{sec:\secname:targets}

Describe the discoveries and measurements you want to make.

Now, describe their response to the observing strategy. Qualitatively,
how will the science project be affected by the observing schedule and
conditions? In broad terms, how would we expect the observing strategy
to be optimized for this science?


% --------------------------------------------------------------------

\subsection{Metrics}
\label{sec:\secname:metrics}

Quantifying the response via MAF metrics: definition of the metrics,
and any derived overall figure of merit.


% --------------------------------------------------------------------

\subsection{OpSim Analysis}
\label{sec:\secname:analysis}

OpSim analysis: how good would the default observing strategy be, at
the time of writing for this science project?


% --------------------------------------------------------------------

\subsection{Discussion}
\label{sec:\secname:discussion}

Discussion: what risks have been identified? What suggestions could be
made to improve this science project's figure of merit, and mitigate
the identified risks?


% ====================================================================

\navigationbar

% ====================================================================
% PLACEHOLDER ADDITIONAL SECTION PUT IN BY PHIL

\section{Some Other Solar System Science Case}
\def\secname{\chpname:discovery}\label{sec:\secname}

\credit{authorgithubname} % (Writing team)

% This individual section will need to describe the particular
% discoveries and measurements that are being targeted in this section's
% science case. It will be helpful to think of a ``science case" as a
% ``science project" that the authors {\it actually plan to do}. Then,
% the sections can follow the tried and tested format of an observing
% proposal: a brief description of the investigation, with references,
% followed by a technical feasibility piece. This latter part will need
% to be quantified using the MAF framework, via a set of metrics that
% need to be computed for any given observing strategy to quantify its
% impact on the described science case. Ideally, these metrics would be
% combined in a well-motivated figure of merit. The section can conclude
% with a discussion of any risks that have been identified, and how
% these could be mitigated.

A short preamble goes here. What's the context for this science
project? Where does it fit in the big picture?

% --------------------------------------------------------------------

\subsection{Target measurements and discoveries}
\label{sec:\secname:targets}

Describe the discoveries and measurements you want to make.

Now, describe their response to the observing strategy. Qualitatively,
how will the science project be affected by the observing schedule and
conditions? In broad terms, how would we expect the observing strategy
to be optimized for this science?


% --------------------------------------------------------------------

\subsection{Metrics}
\label{sec:\secname:metrics}

Quantifying the response via MAF metrics: definition of the metrics,
and any derived overall figure of merit.


% --------------------------------------------------------------------

\subsection{OpSim Analysis}
\label{sec:\secname:analysis}

OpSim analysis: how good would the default observing strategy be, at
the time of writing for this science project?


% --------------------------------------------------------------------

\subsection{Discussion}
\label{sec:\secname:discussion}

Discussion: what risks have been identified? What suggestions could be
made to improve this science project's figure of merit, and mitigate
the identified risks?


% ====================================================================

\navigationbar

% ====================================================================


% --------------------------------------------------------------------

\chapter[The Galaxy]{The Galaxy}
\def\chpname{galaxy}\label{chp:\chpname}

\noindent {\it
Will Clarkson, Kathy Vivas, ...
}

Includes: stellar populations; halo, bulge and disk; variables as
disgnostics of structure; photometry in crowded regions; preferred
distribution of visits in time

Confirmed leads for galactic plane science: Will Clarkson, Kathy Vivas

\navigationbar


% --------------------------------------------------------------------

\chapter[Astrometry]{Mapping Our Galaxy: Positions, Proper Motions and Parallax}
\label{chp:astrometry}

\noindent {\it
David Monet, ...
}

Includes: time baseline and parallax factor distribution needed,
distribution of and filter choice for superior seeing conditions for
star galaxy separation, survey uniformity issues.

Confirmed leads for astrometry: David Monet


% --------------------------------------------------------------------

% --------------------------------------------------------------------

\chapter[Variables and Transients]{Variable and Transient Sources in the Galaxy and Beyond}
\label{chp:vartrans}

\noindent {\it
Mike Lund, Ashish Mahabal, Stephen Ridgway, Lucianne Walkowicz, Rahul Biswas, Michelle Lochner,
Jeonghee Rho...
}

% --------------------------------------------------------------------

Includes: periodic, near-periodic and transient targets, periods that
drift, time sensitivity of color information, rapid classification of
transients, and temporal sampling strategies, particularly including
rolling cadences

Confirmed leads for Stellar variability and Fast Stellar Transients:
Mike Lund, Ashish Mahabal, Stephen Ridgway, Lucianne Walkowicz

Confirmed leads for cosmological SNe: Rahul Biswas, Michelle Lochner,
Jeonghee Rho

Confirmed leads for extragalactic transients: Ashish Mahabal

Confirmed leads on rolling cadence: Stephen Ridgway

% --------------------------------------------------------------------


% ====================================================================
%+
% NAME:
%    section-name.tex
%
% ELEVATOR PITCH:
%    Explain in a few sentences what the relevant discovery or
%    measurement is going to be discussed, and what will be important
%    about it. This is for the browsing reader to get a quick feel
%    for what this section is about.
%
% COMMENTS:
%
%
% BUGS:
%
%
% AUTHORS:
%    Phil Marshall (@drphilmarshall)  - put your name and GitHub username here!
%-
% ====================================================================

\section{Periodic Variable Stars}
\def\secname{periodicvariables}\label{sec:\secname}

\noindent{\it Author Name(s)} % (Writing team)

% This individual section will need to describe the particular
% discoveries and measurements that are being targeted in this section's
% science case. It will be helpful to think of a ``science case" as a
% ``science project" that the authors {\it actually plan to do}. Then,
% the sections can follow the tried and tested format of an observing
% proposal: a brief description of the investigation, with references,
% followed by a technical feasibility piece. This latter part will need
% to be quantified using the MAF framework, via a set of metrics that
% need to be computed for any given observing strategy to quantify its
% impact on the described science case. Ideally, these metrics would be
% combined in a well-motivated figure of merit. The section can conclude
% with a discussion of any risks that have been identified, and how
% these could be mitigated.

Some stars may be strictly periodic, or sufficiently so to be treated as such for some purposes, in which case data from different cycles can be combined according to phase to provide a more fully sampled light curve.  It is unreasonable to suppose that LSST visits will be synchronized with variable stars, and visits will occur effectively at random phases. In a 10-year survey, most periodic stars of almost any period will benefit from  excellent phase coverage in all filters. Only a very small period range close to the sidereal day will be poorly observed.  There is no reason to believe that any likely LSST observing strategy could seriously disturb good sampling of periodic variables.

Eclipsing binaries are discussed here with variable stars, as detection of eclipses is dependent on adequate sampling of the phase curve.  However, study of the  features of an eclipse, particularly one of short duration in phase, may require sampling more appropriate to the discussion of transients.

\subsection{Nearly Periodic Variables}

Stars with a drifting  period will be served well with sampling which constrains period variations frequently through the survey.  For targets with a wide range of periods, this will be most effectively accomplished with sampling that is rather uniform through the survey.  A considerable degree of uniformity is needed for many science objectives, and distribution of visits over the full survey is more important than the exact timing.

Some variable stars do not exhibit a strictly repeating light curve, and show variations in light curve structure from period to period.  For observational purposes, these targets are better described as periodic transients, discussed in a later section.


% --------------------------------------------------------------------

\subsection{Targets and Measurements}
\label{sec:keyword:targets}

\begin{center}
\begin{tabular}{| l | p{8cm} |l | l |}
\hline Periodic Variable Type & Examples of target science & Amplitude & Timescale\\
\hline
Periodic binaries & Eclipses, physical properties of stars, distances, ages, evolution, apsidal precession, mass transfer induced period changes, Applegate effect &  small &  hr-day \\
RR Lyrae & Galactic structure, distance ladder, RR Lyrae properties&  large &  day \\
Cepheids & Distance ladder, cepheid properties&  large &  day \\
Long Period Variables & Distance ladder, LPV properties& large  &  weeks \\
Short period pulsators & Instability strip, white dwarf interior properties, evolution&  small & min  \\
Rotational Modulation & Gyrochronology, stellar activity& small  &  days \\
 \hline \end{tabular}
 \end{center}

These targets share the requirement for good sampling over the variation phase curve.

For each target, the coverage of the phase curve sampling will accumulate randomly, and particular measurements or discoveries will become possible at a rate that is somewhat linear with number of acquired visits (hence linearly with time in a uniform survey).

With millions of different periods, it is difficult to imagine designing the survey to optimize this sampling, but the sampling achieved can be predicted with appropriate metrics.



% --------------------------------------------------------------------

\subsection{Metrics}
\label{sec:keyword:metrics}

\begin{center}
\begin{tabular}{| p{5cm} |p{10cm} |}
\hline Metric & Description\\
\hline
Eclipsing binary discovery & Fraction of discoveries vs fractional duration of eclipse\\
Transiting exoplanets (depth dependent) & Fraction of discoveries vs fractional duration of eclipse\\
Phase gap & Histogram vs period of the median and maximum phase gaps achieved in all fields\\
Period determination (period dependent) & Fraction of targets vs survey duration, for which the period can be determined to 5-sigma confidence\\
Period variability (period dependent) & Fraction of targets vs survey duration, for which a period change of 1\% can be determined with 5-sigma confidence\\
  \hline \end{tabular}
 \end{center}

The period metrics can be based on a standard variable curve (e.g.sinusoid) of fiducial amplitude and brightness, and/or a realistic model population of a particular variable type. These metrics can be informative for science programs.  However, it is not clear that the survey strategy can or should attempt to control these metrics, as the requirements are specific to each target, and all targets benefit from a generally uniform distribution of visits.



% --------------------------------------------------------------------

\subsection{OpSim Analysis}
\label{sec:keyword:analysis}

Current simulations show for the main survey a broad uniformity of visits, with thorough randomization of visit phase per period, giving very good phase coverage with minimum phase gaps.


% --------------------------------------------------------------------

\subsection{Discussion}
\label{sec:keyword:discussion}

For periodic variable science, two cadence characteristics should be avoided:
\begin{itemize}
\item an exactly uniform spacing of visits (which is anyway virtually impossible); \
\item a very non-uniform distribution, such as most visits concentrated in a few survey years.
 \end{itemize}

A metric for maximum phase gap will guard against the possibility that a very unusual cadence might compromise the random sampling of periodic variables.

In each case, it would help to jump-start science programs if some fraction of targets had more complete measurements early in the survey.


% ====================================================================

\navigationbar



% ====================================================================
%+
% NAME:
%    section-name.tex
%
% ELEVATOR PITCH:
%    Explain in a few sentences what the relevant discovery or
%    measurement is going to be discussed, and what will be important
%    about it. This is for the browsing reader to get a quick feel
%    for what this section is about.
%
% COMMENTS:
%
%
% BUGS:
%
%
% AUTHORS:
%    Phil Marshall (@drphilmarshall)  - put your name and GitHub username here!
%-
% ====================================================================

\section{Non-periodic Variable Stars}
\label{sec:keyword} % For example, replace "keyword" with "lenstimedelays"

\noindent{\it Author Name(s)} % (Writing team)

% This individual section will need to describe the particular
% discoveries and measurements that are being targeted in this section's
% science case. It will be helpful to think of a ``science case" as a
% ``science project" that the authors {\it actually plan to do}. Then,
% the sections can follow the tried and tested format of an observing
% proposal: a brief description of the investigation, with references,
% followed by a technical feasibility piece. This latter part will need
% to be quantified using the MAF framework, via a set of metrics that
% need to be computed for any given observing strategy to quantify its
% impact on the described science case. Ideally, these metrics would be
% combined in a well-motivated figure of merit. The section can conclude
% with a discussion of any risks that have been identified, and how
% these could be mitigated.

Non-periodic variable stars may benefit from complete phase coverage over a single cycle, or other time interval  of of interest, repeated in consecutive intervals, or in intervals distributed over the survey.
% --------------------------------------------------------------------

\subsection{Targets and Measurements}
\label{sec:keyword:targets}

Describe the discoveries and measurements you want to make for a generic  transient,
with additional comment on specific variable types which have any special requirements.

Example events:  multi-mode pulsators, Applegate effect, etc

Now, describe their response to the observing strategy. Qualitatively,
how will the science project be affected by the observing schedule and
conditions? In broad terms, how would we expect the observing strategy
to be optimized for this science?


% --------------------------------------------------------------------

\subsection{Metrics}
\label{sec:keyword:metrics}

Quantifying the response via MAF metrics: definition of the metrics,
and any derived overall figure of merit.


% --------------------------------------------------------------------

\subsection{OpSim Analysis}
\label{sec:keyword:analysis}

OpSim analysis: how good would the default observing strategy be, at
the time of writing for this science project?


% --------------------------------------------------------------------

\subsection{Discussion}
\label{sec:keyword:discussion}

Discussion: what risks have been identified? What suggestions could be
made to improve this science project's figure of merit, and mitigate
the identified risks?


% ====================================================================




% ====================================================================
%+
% NAME:
%    section-name.tex
%
% ELEVATOR PITCH:
%    Explain in a few sentences what the relevant discovery or
%    measurement is going to be discussed, and what will be important
%    about it. This is for the browsing reader to get a quick feel
%    for what this section is about.
%
% COMMENTS:
%
%
% BUGS:
%
%
% AUTHORS:
%    Phil Marshall (@drphilmarshall)  - put your name and GitHub username here!
%-
% ====================================================================

\section{Periodic Transient Events}
\label{sec:keyword} % For example, replace "keyword" with "lenstimedelays"

\noindent{\it Author Name(s)} % (Writing team)

% This individual section will need to describe the particular
% discoveries and measurements that are being targeted in this section's
% science case. It will be helpful to think of a ``science case" as a
% ``science project" that the authors {\it actually plan to do}. Then,
% the sections can follow the tried and tested format of an observing
% proposal: a brief description of the investigation, with references,
% followed by a technical feasibility piece. This latter part will need
% to be quantified using the MAF framework, via a set of metrics that
% need to be computed for any given observing strategy to quantify its
% impact on the described science case. Ideally, these metrics would be
% combined in a well-motivated figure of merit. The section can conclude
% with a discussion of any risks that have been identified, and how
% these could be mitigated.

Periodic transient events may benefit from dense phase coverage in order to discover events and to 
determine their period.
% --------------------------------------------------------------------

\subsection{Targets and Measurements}
\label{sec:keyword:targets}

Describe the discoveries and measurements you want to make for a generic  transient,
with additional comment on specific variable types which have any special requirements.

Example events:  eclipsing binary stars, exoplanet eclipses 

Now, describe their response to the observing strategy. Qualitatively,
how will the science project be affected by the observing schedule and
conditions? In broad terms, how would we expect the observing strategy
to be optimized for this science?


% --------------------------------------------------------------------

\subsection{Metrics}
\label{sec:keyword:metrics}

Quantifying the response via MAF metrics: definition of the metrics,
and any derived overall figure of merit.


% --------------------------------------------------------------------

\subsection{OpSim Analysis}
\label{sec:keyword:analysis}

OpSim analysis: how good would the default observing strategy be, at
the time of writing for this science project?


% --------------------------------------------------------------------

\subsection{Discussion}
\label{sec:keyword:discussion}

Discussion: what risks have been identified? What suggestions could be
made to improve this science project's figure of merit, and mitigate
the identified risks?


% ====================================================================




% ====================================================================
%+
% NAME:
%    section-name.tex
%
% ELEVATOR PITCH:
%    Explain in a few sentences what the relevant discovery or
%    measurement is going to be discussed, and what will be important
%    about it. This is for the browsing reader to get a quick feel
%    for what this section is about.
%
% COMMENTS:
%
%
% BUGS:
%
%
% AUTHORS:
%    Phil Marshall (@drphilmarshall)  - put your name and GitHub username here!
%-
% ====================================================================

\section{Transient Events}
\def\secname{transients}\label{sec:\secname}

\noindent{\it Author Name(s)} % (Writing team)

% This individual section will need to describe the particular
% discoveries and measurements that are being targeted in this section's
% science case. It will be helpful to think of a ``science case" as a
% ``science project" that the authors {\it actually plan to do}. Then,
% the sections can follow the tried and tested format of an observing
% proposal: a brief description of the investigation, with references,
% followed by a technical feasibility piece. This latter part will need
% to be quantified using the MAF framework, via a set of metrics that
% need to be computed for any given observing strategy to quantify its
% impact on the described science case. Ideally, these metrics would be
% combined in a well-motivated figure of merit. The section can conclude
% with a discussion of any risks that have been identified, and how
% these could be mitigated.

Transient events may benefit from substantial temporal sampling
(matched to the time constant of the event) with color information
(perhaps contemporaneous) to support characterization and
classification, obtained over the limited duration of interest.
Transient events slower than $\sim$ weeks may be adequately sampled by
a uniform LSST cadence.  Faster events may require special scheduling
strategies.  For some event types, LSST can only be expected to
provide a discovery service, and followup will necessarily be
performed elsewhere.

% --------------------------------------------------------------------

\subsection{Targets and Measurements}
\label{sec:\secname:targets}

The class of transients includes a heterogeneous assortment of objects and phenomena.

\begin{center}
\begin{tabular}{| p{5cm} | p{10cm} |}
\hline Transient Type & Examples of target science\\
\hline
Flare stars & Flare frequency, energy, stellar age\\
Cataclysmic variables  \& novae & Interacting binaries, stellar evolution, compact objects, explosive events\\
Supernovae & SN physics, mass loss, distance scale, cosmology\\
Active galactic nuclei & Galaxy evolution, reverberation mapping, black hole physics\\
Stellar microlensing & Exoplanet statistics\\
Gamma ray bursts & Optical discovery and characterization\\
LIGO detections & Source position and characterization\\
Serendipity & Discovery and characterization\\
Tidal Disruption Events & Discovery and characterization\\
 \hline \end{tabular}
 \end{center}

Among the targets in this list, only AGN are likely to be sampled with sufficient resolution by a uniform LSST cadence - in fact for AGN, a challenge may be to spread visits sufficiently in time to avoid excessive seasonal gaps.

For very short lived phenomena (stellar flares, CV outbursts, GRBs, LIGO events) it appears that the function of LSST will be to provide discoveries and/or simple characterization.  Followup to discovery/identification, if required, will surely take place elsewhere.

For events requiring intensive monitoring (stellar microlensing, exoplanet transits), the followup will certainly take place elsewhere.

Supernovae fall in an intermediate time range.  LSST will provide multiple visits in multiple filters during the typical SN duration.  This sampling may be insufficient for many (including key) science objectives.  However, a moderate, and feasible, change to LSST observing strategy, may enhance the sampling for part of the sky part of the time, greatly enhancing the usefulness of SN observations.

Serendipitous discoveries are of course harder to plan for.  An ideal transient discovery survey would include heavy coverage of all time scales. LSST will cover longer time periods well, but will have to make some choices of emphasis in coverage of shorter time-scales.



% --------------------------------------------------------------------

\subsection{Metrics}
\label{sec:\secname:metrics}

\begin{center}
\begin{tabular}{| p{5cm} |p{10cm} |}
\hline Metric & Description\\
\hline
SNe & Number of events adequately sampled\\
Serendipity & Histogram of median visit series length vs maximum visit spacing within the series\\
  \hline \end{tabular}
 \end{center}

The metrics for SNe will be highly specialized and based on the best available understanding of SN light curve analysis and the expected event population.

The suggested metric set for serendipity is based on the simple-minded idea that a novel transient will be characterized by a band-limited, finite waveform, and that a useful observation series will consist of a series of samples extending over the duration of the event, with at least critical sampling of the fastest variations.  Since for some event durations the number of useful time series will be small, it may be useful to look not at the median length, but the median length of a subset size preselected as possibly useful (e.g. the$10^3$ longest series).

% --------------------------------------------------------------------

\subsection{OpSim Analysis}
\label{sec:\secname:analysis}

Analysis shows that current simulations provide  poor coverage in any one filter for transient events longer than a deep drilling session ($\sim$30 minutes) and shorter than $\sim$ weeks.

Simulated performance for SN observations must be analyzed for both main survey and mini-survey (deep drilling) productivity.  It is considered that current simulated schedules give inadequate performance for SN science.



% --------------------------------------------------------------------

\subsection{Discussion}
\label{sec:\secname:discussion}

Community studies are providing improving SNe metrics, and continuing communication between the SN and LSST communities is essential to tuning the observing strategy to deliver the SN time series that are needed and possible.

Improving LSST science return for SNe will also improve sampling of all transients with similar or somewhat shorter characteristic times.  Non-uniform survey strategies (rolling cadence) can significantly improve the LSST performance for faster transients.  Interpretation of multiple filters for novel events may be powerful, or problematic, since color may be uncertain.

Some insight into fast transients may be available from image pairs  or triples (as opposed to more complete series).  These include the pair of images in a visit - which could be useful in studying the rise time of an extremely fast event.  This includes the characteristic grouping of visits (typically 0.5 to 1.0 hour separation) planned for purposes of identifying asteroids.  It also includes fortuitous multiple sampling due to field overlap, providing additional sampling, which may be random or systematic, depending on the scheduling, on a time scale of minutes to hours.  The sampling benefits of this fortuitous overlap have not yet been investigated.




% ====================================================================

\navigationbar


% --------------------------------------------------------------------

% \input{cosmologicalSNe}

% --------------------------------------------------------------------

% \input{extragalactictransients}

% --------------------------------------------------------------------



% ====================================================================
% commands for stand-alone printing
%\documentclass[11pt,headsepline,cleardoubleempty,twoside,openright]{scrbook}
%\usepackage{SciBook}
%\begin{document}
% ====================================================================

% ====================================================================
%+
% NAME:
%    rollingcadence.tex
%
% ELEVATOR PITCH:
%    TODO: Explain in a few sentences what the relevant discovery or
%    measurement is going to be discussed, and what will be important
%    about it. This is for the browsing reader to get a quick feel
%    for what this section is about.
%
% COMMENTS:
%
%
% BUGS:
%
%
% AUTHORS:
%    Steve Ridgway (@StephenRidgway)
%-
% ====================================================================

\section{ Rolling Cadence }
\def\secname{rolling}\label{sec:\secname}

\noindent{\it Stephen Ridgway, \ldots} % (Writing team)

% This individual section will need to describe the particular
% discoveries and measurements that are being targeted in this section's
% science case. It will be helpful to think of a ``science case" as a
% ``science project" that the authors {\it actually plan to do}. Then,
% the sections can follow the tried and tested format of an observing
% proposal: a brief description of the investigation, with references,
% followed by a technical feasibility piece. This latter part will need
% to be quantified using the MAF framework, via a set of metrics that
% need to be computed for any given observing strategy to quantify its
% impact on the described science case. Ideally, these metrics would be
% combined in a well-motivated figure of merit. The section can conclude
% with a discussion of any risks that have been identified, and how
% these could be mitigated.

With a total of ~800 visits spaced approximately uniformly over 10 years, and distributed among 6 filters,
it is not clear that LSST can offer the sufficiently dense sampling in time for study of transients with typical durations less than or $\simeq 1$week.
This is particularly a concern for key science requiring well-sampled SNIa light curves.  Rolling cadences stand out as a
general solution that can potentially enhance sampling rates by 2$\times$ or more, on some of the sky all of the time and all of the sky some of the time, while maintaining a sufficient uniformity for survey objectives that require it.

\subsection{The Uniform Cadence}

Current schedule simulations allocate visits as pairs separated by 30-60 minutes, for the purposes of identifying asteroids.  For most science purposes, the 30-60 minute spacing is too small to reveal temporal information, and a pair will constitute effectively a single epoch of measurement.  If the expected 824 (design value) LSST visits are realized as 412 pairs, and distributed uniformly over 10 observing seasons of 6 months each, the typical separation between epochs will be 4 days.   The most numerous visits will be in the {\it r} and {\it i} filters, and the repeat visit rate in either of these will be $\simeq$ 20 days.

The possibility is still open that, for asteroid identification, visits might be required as triples or quadrupoles, in which case the universal temporal sampling will be further slowed by 1.5 or 2$\times$.

Under a strict universal cadence it is not possible to satisfy a need for more frequent sample epochs.  This leads the simulations group to investigate the options opened up by reinterpreting the concept of a universal cadence.  Instead of aiming for a strategy which attempts to observe all fields ``equally'' all the time, it would allow significant deviations from equal coverage during the survey, returning to balance at the end of the survey.

Stronger divergence from a universal cadence, allowing significant inhomogeneities to remain at the end of the survey, is of course possible, but is not under investigation or discussed here.

There is currently considerable interest in the community in strategies that provide enhanced sampling over a selected area of the sky, and rotating the selected area in order to exercise enhanced sampling over all of the survey area part of the time.  The class of cadences that provides such intervals of enhanced visits, with the focus region shifting from time to time, is termed here a rolling cadence.  As a point of terminology, observing a single sky area with enhanced cadence for a period of time will be described as a ``roll''.

\subsection{Rolling Cadence Basics}

Assume a fixed number of observing epochs for each point on the sky, nominally distributed uniformly over the survey duration.  A subset of these can be reallocated to provide improved sampling of a sky region.  This will have the inevitable effects of: (1) reducing the number of epochs available for that sky region during the rest of the survey, and (2) displace observations of other sky regions during the time of the improved temporal sampling.  In short, the cadence outside the enhanced interval will be degraded.

The essential parameters of rolling cadence are: (1) the number of samples taken from the uniform cadence, and (2) the enhancement factor for the observing rate.  The LSST document 16370, ``A Rolling Cadence Strategy for the Operations Simulator'', by K. Cook and S. Ridgway,  contains more detailed discussion and analysis.

\begin{figure}
  \includegraphics[width=2.3in]{figs/ops2_1098_Count_expMJD_r_and_night_lt_365_HEAL_SkyMap.pdf}\includegraphics[width=2.3in]{figs/ops2_1098_Count_expMJD_r_and_night_lt_730_HEAL_SkyMap.pdf}\includegraphics[width=2.3in]{figs/ops2_1098_Count_expMJD_r_and_night_lt_3653_HEAL_SkyMap.pdf} \\
  \includegraphics[width=2.3in]{figs/enigma_1260_Count_expMJD_r_and_night_lt_365_HEAL_SkyMap.pdf}\includegraphics[width=2.3in]{figs/enigma_1260_Count_expMJD_r_and_night_lt_730_HEAL_SkyMap.pdf}\includegraphics[width=2.3in]{figs/enigma_1260_Count_expMJD_r_and_night_lt_3653_HEAL_SkyMap.pdf}
  \caption{Example of a regular uniform survey (top) and a rolling cadence survey (bottom) after 1, 2, and 10 years in the $r$ filter.  For the regular survey, the number of visits for any part of the sky is relatively constant throughout the survey.  For the rolling cadence simulation, there are regions with many more exposures in year one which then fade in year two as other parts of the sky are emphasized.\label{fig:rollingcadence}}
\end{figure}

% --------------------------------------------------------------------
% --------------------------------------------------------------------
\subsection{ Supernovae and Rolling Cadence}
\label{sec:rolling:supernovae}

\noindent{\it Author Name(s)} % (Writing team)

Supernovae as a science topic are addressed elsewhere.
In this section, the demands of SN are used to directly constrain or
orient the rolling cadence development.

Pending more quantitative guidance, the SN objective for rolling cadence is to obtain multicolor time series significantly longer than the typical SN duration, with a cadence significantly faster than uniform.  As an example we discuss the option of a rolling cadence with the regular distribution of filters.

As a simple example, consider improving the cadence by a factor of 2 or 3.  Is we accept that some regions of the sky will be enhanced every year, and that uniform sky coverage will only arrive at the end of 10 years, then we could use, e.g., 10\% of the total epochs in a single roll.  If the enhancement is 2$\times$, each roll would last for $\simeq$ 6 months, with high efficiency for capture of complete SN events.  If the enhancement is 4$\times$, each roll would last for 2 months, with lower efficiency.

If it is important to achieve survey uniformity after 3 years, the available visits for each roll would be reduced also.  With a 2$\times$ enhancement of epoch frequency, a roll would last 2 months.

Some leverage would be gained by using more than 10\% of the available visits for a single roll.  However, this begins to impact the sampling of slow variables reduce schedule flexibility and robustness, and should be approached with caution.

From these examples, it appears that a 2$\times$ enhancement with uniformity closure after 10 years is relatively feasible and promising.  Much higher gains, or more rapid closure, require additional compromises.

% --------------------------------------------------------------------

\subsection{ Fast Transients and Rolling Cadence}
\label{sec:rolling:transients}

\noindent{\it Author Name(s)} % (Writing team)

Fast transients as a science topic are addressed elsewhere. In this section, the demands of fast transients are used to directly constrain or
orient the rolling cadence development.

By ``fast transients'', we are referring to events that are sufficiently fast that they are not addressed by the rolling cadence designed for SN observations, and slow enough that they are not covered in ``deep drilling'' type mini-surveys.  For higher tempo rolls, it is quite difficult to obtain full color data, because of the constraints on filter selection.  For this example, we will examine a rolling cadence utilizing only the {\it r} and {\it i} filters, as they are used for most visits. They are close in wavelength, and we assume that sufficient color information will be obtained by the ``background'' uniform survey that continues during a roll.

Again using 10\% of the available visits from the full 10 year survey for a single roll, we find that there would be enough epochs for each roll to acquire 1 visit per day for 21 consecutive days, giving an enhancement of 10$\times$.

Alternatively, the same epochs could be used to observe a target every 20 minutes for 12 hours during a single night (here it is assumed that visit pairs are not required, doubling the available epochs) for an enhancement of 300$\times$.

Several different possible redeployments of portions of a uniform survey have been described, each using 10\% of available time.  Of course it is possible in principal to implement multiple options, sequentially or maybe in parallel in some cases. This may pose considerable challenges to the scheduling strategy design by introducing incompatible boundary conditions.

While rolling cadences are powerful, they have limitations.  For example, sampling events that last longer than $\simeq$1 day and less than $\simeq$ 1 week have the obvious problem of diurnal availability.  In this example, intermediate cadences could be implemented in the circumpolar region, where diurnal access is much extended.  This is an example of a case in which a mini-survey of a limited number of regions could be considered as an alternative to a rolling cadence applied to the entire main survey.

% --------------------------------------------------------------------

\subsection{ Constraints, Trades and Compromises for Rolling Cadences}
\label{sec:rolling:trades}

While rolling cadences offer some attractive benefits, it is important to realize that rolling cadences are very highly constrained, and that they do bring disadvantages and compromises.

There are strong arguments against beginning a rolling cadence in the first, or even the second year of the survey.  Early in the survey, it is important to obtain for each field/filter combination, an adequate number of good quality photometric images, and at least one image in excellent seeing, to support closure of photometry reductions and to support generation of template images.

Since major science goals require a significant degree of survey homogeneity, it may be advisable to implement a strategy that brings the survey to nominal uniform depth at several times, e.g. after 3 or 5 years.  This would strongly constrain rolling cadences.

Some science objectives favor certain distributions of visits.  For astrometry, visits early and late in the survey and at large parallax factors, are beneficial.  Slow variables may benefit from uniform spacing.  Rolling cadences might impact these constraints either favorably or unfavorably.

Many objectives are served by randomization of observing conditions for each field.  Some rolling cadences could tend to reduce this randomization, for example by acquiring a large number of observations during a meteorologically favorable or unfavorable season, or during a period of instrument performance variance.

Dithering does not work gracefully with a rolling cadence, reducing temporal coverage at the boundaries of the selected sky region.  This is negligible for small dithers, but important for large dithers, which are under consideration.

These cautions illustrate that evaluation of rolling cadences must be based on the full range of schedule performance metrics, and not just those targeted by rolling cadence development.

% ====================================================================

\navigationbar

% ====================================================================
% commands for stand-alone printing
% \bibliographystyle{apj}
% \bibliography{references}
% \end{document}
% ====================================================================


% --------------------------------------------------------------------


% --------------------------------------------------------------------

% --------------------------------------------------------------------

\chapter[Cosmology]{Keeping It Even: Accurate Cosmological Measurements on the Largest Scales}
\label{chp:cosmology}

\noindent {\it
Eric Gawiser, Peter Kurczynski, Phil Marshall, Ohad Shemmer, Timo Anguita, ...
}

% --------------------------------------------------------------------


\section{Introduction}
\label{sec:cosmology:intro}

% Introduce, with a very broad brush, this chapter's science projects,
% and why it makes sense for them to be considered together.

Cosmology is one of the key science themes for which LSST was designed. Our goal is to measure cosmological parameters, such as the equation of state of dark energy, or departures from General Relativity, with sufficient accuracy to distinguish one model from another, and hence drive our theoretical understanding of how the universe works, as a whole. To do this will necessarily involve a variety of different measurements, that can act as cross-checks of each other, and break parameter degeneracies in any single one.

The  Dark Energy Science Collaboration (DESC) has identified five
different cosmological probes enabled by the LSST: weak lensing (WL),
large scale structure (LSS), type Ia supernovae (SN), strong lensing
(SL), and clusters of galaxies (CL). In all cases, the primary concern
is residual systematic error: the shapes and photometric redshifts of
galaxies, and the properties of supernova and lensed quasar light
curves, will all need to be measured with extraordinary accuracy in order for LSST's high statistical power to be properly harnessed. This accuracy will come from the abundance and heterogeneity of the individual measurements made, and the degree to which they can be modeled and understood. This latter point implies a need for uniformity in the survey, which enables powerful simplifying assumptions to be made when calibrating on the largest, cosmologically most important scales. The need for heterogeneity also implies  uniformity, in the sense that the nuisance parameters that describe the systematic effects need to be sampled over as wide a range as possible (examples include the need to sample a wide range of roll angles to minimize shape error, and observing conditions to understand photometric errors due to the changing atmosphere).

In this chapter we look at some of the key measurements planned by the Dark Energy Science Collaboration, and how they depend on the Observing Strategy.

% Anticipate the results of the chapter: summarize the results of a
% number of investigative sections, where there will be one on each
% science case.


% --------------------------------------------------------------------

% ====================================================================
%+
% SECTION NAME:
%    \secname.tex
%
% CHAPTER:
%    cosmology.tex
%
% ELEVATOR PITCH:
%    Lensed quasars and supernovae provide distance measurements for
%    cosmology. They are a few days to a few weeks in length. To
%    measure them well we need long campaigns (>~3 years) with high
%    night-to-night cadence (better than the standard 5 days if
%    possible, especially as combining all filters might be difficult.)
%
% COMMENTS:
%
%
% BUGS:
%
%
% AUTHORS:
%   Phil Marshall (@drphilmarshall)
%-
% ====================================================================
\clearpage
\section{ Strong Gravitational Lens Time Delays }
\def\secname{lenstimedelays}\label{sec:\secname}

\noindent{\it Phil Marshall} % (Writing team)

% This individual section will need to describe the particular
% discoveries and measurements that are being targeted in this section's
% science case. It will be helpful to think of a ``science case" as a
% ``science project" that the authors {\it actually plan to do}. Then,
% the sections can follow the tried and tested format of an observing
% proposal: a brief description of the investigation, with references,
% followed by a technical feasibility piece. This latter part will need
% to be quantified using the MAF framework, via a set of metrics that
% need to be computed for any given observing strategy to quantify its
% impact on the described science case. Ideally, these metrics would be
% combined in a well-motivated figure of merit. The section can conclude
% with a discussion of any risks that have been identified, and how
% these could be mitigated.

% A short preamble goes here. What's the context for this science
% project? Where does it fit in the big picture?

The multiple images of strongly lensed quasars and supernovae have
delayed arrival times: variability in the first image will be observed
in the second image some time later, as the photons take different
paths around the deflector galaxy, and through different depths of
gravitational potential. If the lens mass distribution can be modeled
independently, using a combination of high resolution imaging of the
distorted quasar/SN host galaxy and stellar dynamics in the lens
galaxy, the measured time delays can be used to infer the``time delay
distance'' in the system. This distance enables a direct measurement
of the Hubble constant, independent of the distance ladder.

% --------------------------------------------------------------------

\subsection{Target measurements and discoveries}
\label{sec:\secname:targets}

% Describe the discoveries and measurements you want to make.

For this cosmological probe to be competitive with LSST's others, the
time delays of several hundred systems (which will be distributed
uniformly over the extragalactic sky) will need to be measured with
bias below the sub-percent level, while the precision required is a
few percent per lens.  In galaxy-scale lenses, the kind that are most
accurately modeled, these time delays are typically between several
days and several weeks long, and so are measurable in monitoring
campaigns having night-to-night cadence of between one and a few days,
and seasons lasting several months or more.

% Now, describe their response to the observing strategy.
% Qualitatively, how will the science project be affected by the
% observing schedule and conditions? In broad terms, how would we
% expect the observing strategy to be optimized for this science?

To obtain accurate as well as precise lensed quasar time delays, several monitoring seasons are required. Lensed supernova time delays have not yet been measured, but their transient nature means that their time delay measurements may be more sensitive to cadence than season or campaign length.

% --------------------------------------------------------------------

\subsection{Metrics}
\label{sec:\secname:metrics}

% Quantifying the response via MAF metrics: definition of the metrics,
% and any derived overall figure of merit.

Anticipating that the time delay accuracy would depend on night-to-night cadence, season length, and campaign length, we carried out a large scale simulation and measurement program that coarsely sampled these schedule properties. In \citet{LiaoEtal2015}, we simulated 5 different light curve datasets, each containing 1000 lenses, and presented them to the strong lensing community in a ``Time Delay Challenge.'' These 5 challenge ``rungs'' differed by their schedule properties, in the ways shown in \autoref{tab:tdcrungs}. Focusing on the best challenge submissions made by the community, we derived a simple power law model for the variation of each of the time delay accuracy, time delay precision, and useable sample fraction, with the schedule properties cadence, season length and campaign length. These models are shown in \autoref{fig:tdcresults}, reproduced from \citet{LiaoEtal2015}, and are given by the following equations:
\begin{align}
|A|_{\rm model} &\approx 0.06\% \left(\frac{\rm cad} {\rm 3 days}  \right)^{0.0}
                          \left(\frac{\rm sea}  {\rm 4 months}\right)^{-1.0}
                          \left(\frac{\rm camp}{\rm 5 years} \right)^{-1.1} \notag \\
  P_{\rm model} &\approx 4.0\% \left(\frac{\rm cad} {\rm 3 days}  \right)^{ 0.7}
                         \left(\frac{\rm sea}  {\rm 4 months}\right)^{-0.3}
                         \left(\frac{\rm camp}{\rm 5 years} \right)^{-0.6} \notag \\
  f_{\rm model} &\approx 30\% \left(\frac{\rm cad} {\rm 3 days}  \right)^{-0.4}
                        \left(\frac{\rm sea}  {\rm 4 months}\right)^{ 0.8}
                        \left(\frac{\rm camp}{\rm 5 years} \right)^{-0.2} \notag
\end{align}

%%%%%%%%%%%%%%%%%%%%%%%%%%%%%%%%%%%%
\begin{table*}
\begin{center}
\capstart
\begin{tabular}{cccccc} \hline\hline
  Rung &  Mean Cadence & Cadence Dispersion & Season   & Campaign & Length   \\
       &  (days)       & (days)             & (months) & (years)  & (epochs) \\ \hline
  0    &    3.0        &   1.0              &   8.0    &    5     & 400      \\
  1    &    3.0        &   1.0              &   4.0    &    10    & 400      \\
  2    &    3.0        &   0.0              &   4.0    &    5     & 200      \\
  3    &    3.0        &   1.0              &   4.0    &    5     & 200      \\
  4    &    6.0        &   1.0              &   4.0    &    10    & 200      \\
\hline\hline
\end{tabular}
\end{center}
\caption{The observing parameters for the five rungs of the Time Delay
Challenge. Reproduced from \citet{LiaoEtal2015}.\label{tab:tdcrungs}}
\end{table*}
%%%%%%%%%%%%%%%%%%%%%%%%%%%%%%%%%%%%

%%%%%%%%%%%%%%%%%%%%%%%%%%%%%%%%%%%
\begin{figure*}[!ht]
  \capstart
  \begin{minipage}[b]{\linewidth}
    \begin{minipage}[b]{0.32\linewidth}
      \centering\includegraphics[width=\linewidth]{figs/Accuracy_season_nca.pdf}
    \end{minipage} \hfill
    \begin{minipage}[b]{0.32\linewidth}
      \centering\includegraphics[width=\linewidth]{figs/Precision_cadence_nca.pdf}
    \end{minipage} \hfill
    \begin{minipage}[b]{0.32\linewidth}
      \centering\includegraphics[width=\linewidth]{figs/Fraction_season_nca.pdf}
    \end{minipage}
  \end{minipage}
\caption{Examples of changes in accuracy $A$ (left), precision $P$ (center) and success fraction $f$ (right) with schedule properties, as seen in the different TDC submissions. The gray
approximate power law model was derived by visual inspection of the
pyCS-SPL results; the signs of the indices were pre-determined according to our expectations. Reproduced from \citet{LiaoEtal2015}.}
\label{fig:tdcresults}
\end{figure*}
%%%%%%%%%%%%%%%%%%%%%%%%%%%%%%%%%%%

All three of these metrics would, in an ideal world, be optimized:
this could be achieved by decreasing the night-to-night cadence (to
better sample the light curves), extending the observing season length
(to maximize the chances of capturing a strong variation and its
echo), and extending the campaign length (to increase the number of
effective time delay measurements). A combined figure of merit should
therefore be readily available. The quantity of greatest scientific
interest is the accuracy in cosmological parameters: efforts to derive
such a figure of merit in terms of the Hubble constant are underway.

% --------------------------------------------------------------------

\subsection{OpSim Analysis}
\label{sec:\secname:analysis}

% OpSim analysis: how good would the default observing strategy be, at
% the time of writing for this science project?

In this section we will present the results of our OpSim analysis,
answering the question ``how good would the current default observing
strategy be for time delay lens cosmography?''

% --------------------------------------------------------------------

\subsection{Discussion}
\label{sec:\secname:discussion}

Discussion: what risks have been identified? What suggestions could be
made to improve this science project's figure of merit, and mitigate
the identified risks?


\navigationbar

% ====================================================================


% --------------------------------------------------------------------

% % ====================================================================
%+
% SECTION NAME:
%    dithering.tex
%
% CHAPTER:
%    cosmology.tex
%
% ELEVATOR PITCH:
%    Large Scale Structure, Weak Lensing, and Clusters all require
% survey uniformity in the static 10-year survey.  A key contributor to 
%this is the pattern of dithers adopted.  
%
% COMMENTS:
%
%
% BUGS:
%
%
% AUTHORS:
%   Eric Gawiser (@egawiser)
%-
% ====================================================================
\clearpage
\section{Dithering Patterns and Timescales}
\def\secname{dithering}\label{sec:\secname}

\noindent{\it Humna Awan, Eric Gawiser, Peter Kurczynski, Lynne Jones} % (Writing team)

% This individual section will need to describe the particular
% discoveries and measurements that are being targeted in this section's
% science case. It will be helpful to think of a ``science case" as a
% ``science project" that the authors {\it actually plan to do}. Then,
% the sections can follow the tried and tested format of an observing
% proposal: a brief description of the investigation, with references,
% followed by a technical feasibility piece. This latter part will need
% to be quantified using the MAF framework, via a set of metrics that
% need to be computed for any given observing strategy to quantify its
% impact on the described science case. Ideally, these metrics would be
% combined in a well-motivated figure of merit. The section can conclude
% with a discussion of any risks that have been identified, and how
% these could be mitigated.

% A short preamble goes here. What's the context for this science
% project? Where does it fit in the big picture?

Three of the key cosmology probes available with LSST represent ``static science'' insensitive to time-domain concerns.  These are Weak Lensing, Large-Scale Structure, and Galaxy Clusters.  Nonetheless, due to the need to track and correct for the survey ``window function'' in all of these probes, cosmology with LSST will benefit greatly from achieving survey depth as uniform as possible over the WFD area.  OpSim tiles the sky in hexagons inscribed within the nearly-circular LSST field-of-view.  It has been shown in \citet{CarrollEtal2014} that the default LSST survey strategy implemented in OpSim runs leads to a strongly non-uniform ``honeycomb'' pattern due to overlapping regions on the edges of these hexagons receiving double the observing time.  A pattern of large dithers proves sufficient to greatly reduce these overlaps, leading to an increase in median survey depth in each filter of 0.08 magnitudes.  

In this section, we report results from an investigation by Awan et al. (in preparation) of several geometrical patterns for dithers performed on timescales varying from once per observing season to once per night to every visit.  

\todo{EG}{Flesh out WL, LSS, and Clusters dependence on survey uniformity to make this section more clearly science-driven.}  

% --------------------------------------------------------------------

\subsection{Dithering Patterns and Timescales}
\label{sec:\secname:strategies}


% --------------------------------------------------------------------

\subsection{Metrics}
\label{sec:\secname:metrics}

% Quantifying the response via MAF metrics: definition of the metrics,
% and any derived overall figure of merit.

Our primary metric is total uncertainty in the derived window function over relevant angular scales, modeled via variations in the angular power spectrum of fake galaxy fluctuations between $gri$ bands.  
Intermediate metrics include the number of galaxies in 
each pixel, fluctuations in this number, total power in the angular power spectrum of a skymap of those fluctuations, and residual power that angular power spectrum after subtracting a smooth fit to it.  



% --------------------------------------------------------------------

\subsection{OpSim Analysis}
\label{sec:\secname:analysis}

% OpSim analysis: how good would the default observing strategy be, at
% the time of writing for this science project?

In this section we present our ongoing \OpSim / MAF
analysis, as we try to
answer the question ``what dithering strategies produce acceptable variations in survey uniformity, and which appears optimal?''

%We used the
%\simsMAFcontrib{SeasonStacker}{mafContrib/seasonStacker.py} to work
%with seasons.

%We used \texttt{ops2\_1075} for most of our tests, but we need to now
%re-run on \opsimdbref{db:enigma}, and others from \autoref{chp:cadence2015}.


%\citeauthor{LiaoEtal2015}). These sky maps show that, over the main

%\autoref{tab:lenstimedelays:results} shows the global (i.e. al-sky)


%--------------------------------------------------------------------

\subsection{Results}
\label{sec:\secname:results}

%%%%%%%%%%%%%%%%%%%%%%%%%%%%%%%%%%%
\begin{figure*}[!ht]
  \capstart
  \begin{minipage}[b]{\linewidth}
    \begin{minipage}[b]{0.32\linewidth}
      \centering\includegraphics[width=\linewidth]{figs/Accuracy_season_nca.pdf}
    \end{minipage} \hfill
    \begin{minipage}[b]{0.32\linewidth}
      \centering\includegraphics[width=\linewidth]{figs/Precision_cadence_nca.pdf}
    \end{minipage} \hfill
    \begin{minipage}[b]{0.32\linewidth}
      \centering\includegraphics[width=\linewidth]{figs/Fraction_season_nca.pdf}
    \end{minipage}
  \end{minipage}
\caption{Examples of changes in accuracy $A$ (left), precision $P$ (center) and success fraction $f$ (right) with schedule properties, as seen in the different TDC submissions. The gray
approximate power law model was derived by visual inspection of the
pyCS-SPL results; the signs of the indices were pre-determined according to our expectations. Reproduced from \citet{LiaoEtal2015}.}
\label{fig:tdcresults}
\end{figure*}
%%%%%%%%%%%%%%%%%%%%%%%%%%%%%%%%%%%


\todo{EG}{Improve figures to originals rather than screen-captures.}

\todo{EG}{Input fuller results and text from Awan et al. draft.}  

%---------------------------------------------------------------------

\subsection{Discussion}
\label{sec:\secname:discussion}



\navigationbar

% ====================================================================


% --------------------------------------------------------------------

% % ====================================================================
%+
%
% SECTION NAME:
%    \secname.tex
%
% CHAPTER:
%    ???.tex
%
%
% COMMENTS:
%
%
% BUGS:
%
%
% AUTHORS:
%   Ohad Shemmer (@ohadshemmer), Timo Anguita (@tanguita), Niel Brandt, Gordon Richards, Scott Anderson(?),
%   Phil Marshall(?) (@drphilmarshall)
%-
% ====================================================================
\clearpage
\section{AGN Science}
\def\secname{agn}\label{sec:\secname}

\noindent{\it Ohad Shemmer, Timo Anguita, Niel Brandt, Gordon Richards, Scott Anderson(?), Phil Marshall(?)}

% This section discusses the potential effects of the LSST observing strategy on AGN science. In short, there appears to be
% a consensus among the AGN and galaxies communities that AGN science will benefit from the most uniform cadence in
% terms of even sampling for each band and uniform sky coverage. It is also expected that any reasonable
% perturbation to the nominal LSST observing strategy will have mostly minor effects on AGN science. This section attempts
% to identify all the areas of AGN science that may be affected by the observing strategy and to point out the metrics that
% can be used to quantify any potential effect. Since the total number of metrics that must be quantified is quite large, and
% the effects are likely small in most cases, the goal of this section is to identify potential ``killers'' that may undermine
% key AGN research areas. For example, certain perturbations may reduce significantly the number of ``interesting'' AGNs,
% such as $z>6$ quasars, lensed quasars, or transient AGNs. Another example is photometric reverberation mapping
% which is one of LSST's greatest advantages for AGN research but is also very sensitive to the cadence; care must
% be taken to ensure that the observing strategy does not undermine the ability to make the best use of this method.

\subsection{AGN Selection and Census}
\label{sec:\secname:selection}

\noindent About $10^7 - 10^8$ AGNs will be selected in the main LSST survey using a combination of criteria, split
broadly into four categories: colors, astrometry, variability, and multiwavelength matching with other surveys.
The LSST observing strategy will affect mostly the first three of these categories.

{\bf Colors:}~The LSST observing strategy will determine the depth in each band, as a function of position on the sky, and will thus affect
the color selection of AGNs. This will eventually determine the AGN $L-z$ distribution and, in particular, may affect the identification
of quasars at $z\gtsim 6$ if, for example, $Y$-band exposures will not be sufficiently deep.

{\bf Variability:} AGNs can be effectively distinguished from (variable) stars, and from quiescent galaxies, by exhibiting certain characteristic variability patterns (e.g., \citet{ButlerandBloom2011}). Non-uniform sampling may ``contaminate'' the variability signal of AGN candidates.

{\bf Astrometry:} AGNs will be selected among sources having zero proper motion, within the uncertainties. The LSST cadence
may affect the level of this uncertainty in each band, and may therefore affect the ability to identify (mostly fainter) AGNs.
%
Differential chromatic refraction (DCR), making use of the astrometric offset a source with emission lines has with respect to
a source with a featureless power-law spectrum, can help in the selection of AGNs and in confirming their photometric redshifts \citep{KaczmarczikEtal2009}. The DCR effect is more pronounced at higher airmasses. AGN selection and photometric redshift confirmation may be affected since the LSST cadence will affect the airmass distribution, in each band, for each AGN candidate.

\subsection{AGN Clustering}
\label{sec:\secname:clustering}

\noindent Measurements of the spatial clustering of AGNs with respect to those of quiescent galaxies can provide clues as to how galaxies
form inside their dark-matter halos and what causes the growth of their supermassive black holes (SMBHs). The impressive inventory
of LSST AGNs will enable the clustering, and thus the host galaxy halo mass, to be determined over the widest range ranges of cosmic
epoch and accretion power.
%
The LSST cadence will not only affect the overall AGN census and its $L-z$ distribution, but also the
depth in each band as a function of sky position that can directly affect the clustering signal.

\subsection{AGNs and the Time Domain}
\label{sec:\secname:time}

{\bf AGN Variability:} A variety of AGN variability studies will be enabled by LSST. These are intended to probe the physical properties of the unresolved inner regions of the central engine. Relations will be sought between variability amplitude and timescale vs. $L$, $z$, $\lambda_{\rm eff}$, color, multiwavelength and spectroscopic properties, if available. The LSST sampling is expected to provide high-quality power spectral density functions for a large number of AGNs; these can be used to constrain the SMBH mass and accretion rate/mode. Furthermore, LSST AGNs exhibiting excess variability over that expected from their luminosities will be further scrutinized as candidates for lensed systems having unresolved images with the excess (extrinsic) variability being attributed mainly to microlensing.

Photometric reverberation mapping (PRM), measuring the time-delayed response of either the flux of the broad emission line region (BELR) lines to the flux of the AGN continuum or between the continuum flux in one (longer wavelength) band to the continuum flux in another (band with shorter wavelength), will be one of the cornerstones of AGN research in the LSST era
(e.g., \citet{Chelouche2013}; \citet{CheloucheandZucker2013}; \citet{CheloucheEtal2014}). For example, LSST is expected to deliver BELR line-continuum time delays in $\sim10^5-10^6$ sources, which is unprecedented when compared to $\sim50-100$ such measurements conducted via the traditional, yet much more expensive (per source) spectroscopic method. Sources in the deep-drilling fields (DDFs) will benefit from the highest quality PRM
time-delay measurements given the factor of $\sim10$ denser sampling. The PRM measurements will probe the size and structure of the accretion disk and BELR, in a statistical sense, and may provide improved SMBH mass estimates for sources that have at least single-epoch spectra.

The PRM method is very sensitive to the sampling in each band, therefore the ability to derive reliable time delays can be affected significantly
by the LSST cadence. The best results will be obtained by having the most uniform sampling equally for each band. Additionally, there is
a trade-off between the number of DDFs and the number of time delays that PRM can obtain \citep{CheloucheEtal2014}. For example,
an increase in the number of DDFs, with similarly dense sampling in each field, can yield a proportionately larger number of high-quality time delays,
down to lower luminosities, but at the expense of far fewer time delays (of relatively high luminosity sources) in the main survey.

{\bf Time Delays in Gravitationally Lensed Quasars:} This aspect is discussed in detail in the
lens time delays section (\autoref{sec:lenstimedelays}).

{\bf AGN Size and Structure with Microlensing:} Microlensing due to stars projected on top of individual lensed quasar images produce additional magnification. Using the fact that the Einstein radii of stars in lensing galaxies closely match the scales of different emission regions in high-redshift AGNs (micro-arcseconds), analyzing microlensing induced flux variations statistically on individual systems allows us to measure ``sizes'' of AGN regions.
%
Assuming a thermal profile for accretion disks, sizes in different emission wavelengths will be probed and as such, constraints on the slope of this thermal profile. Given the sheer number of lensed systems that LSST is expected to discover ($\sim8000$), this will allow us to stack systems for better constraints and hopefully determine the evolution of the size and profile. Due to the typical relative velocities of lenses, microlenses, observers (Earth) and source AGN, the microlensing variation timescales are between months to a few decades.

The quasar microlensing optical depth is $\sim1$, so every lensed quasar should be affected by microlensing at any given point in time. However, measurable variability can occur on longer timescales. \citet{MosqueraandKochanek2011} did a study using all known lensed quasars. They found the median timescale between high magnification events (Einstein crossing time scales) in the observed $I$-band is of the order of $\sim20$~yr (with a distribution between 10 and 40~yr). However, the source crossing time (duration of a high magnification event) is $\sim7.3$~months (with a distribution tail up to 3~yr). This basically means that out of all the lensed quasar {\em images} (microlensing between images is completely uncorrelated) about half of them will be quiescent during the 10~yr baseline of LSST. However, since the typical number of lensed images is either two or four, it means that, statistically, in every system, one (for doubles) or two (for quads) high magnification events should be observed in 10~yr of LSST monitoring.

Note that, the important cadence parameter is the source crossing time, as it is the length of the event to be as uniformly sampled as possible. The 7.3 months crossing time is the median for the observed $i$-band, but this time would be significantly shorter for bluer bands: for a thermal profile with slope $\alpha: R_\lambda \propto \lambda^\alpha$ implies source crossing time $t_{\rm s} \propto \lambda^{1/\alpha} \rightarrow t_u=t_i \times (\lambda_{\rm u} / \lambda_{\rm i})^{1/\alpha}$. For a Shakura-Sunyaev slope of $\alpha=0.75$ this would correspond to $7.3 \times (3600/8140)^{4/3}$ months $\approx 2.5$ months in the $u$-band.

In terms of the cadence, at least three evenly sampled data points per band within 2 to 3 months would be preferred to be able to map the constraining high magnification event(?). Hopefully uniformly spaced. Very tight cadence (e.g., DDFs) would increase the constraints significantly. However, since lensed quasars are not that common, this smaller area would mean only a few ($\sim80$?) suitable systems monitored in the DDFs.
%
Regarding the season length, the ``months'' timescale of high magnification events very likely means that we can/will miss high magnification events in the season gaps, at least in the bluer bands.
%
Killer: observations spread on timescales larger than 3 months(??). This would likely miss the high magnification events. In those cases we could perhaps consider close consecutive photometric bands as equivalent accretion disk regions, however this would mean weaker constraints on the thermal profile.
%
Important Note: all this science needs to be done on lensed quasars with measured or very short time delays to remove the intrinsic variability signal, which might significantly reduce the sample.

{\bf Microlensing Aided Reverberation Mapping:} Given that microlensing mostly affects continuum emission rather than BELR line emission, microlensing may enable disentangling the BELR line $+$ continuum emission in single photometric bands, allowing the use of single broad band PRM measurements \citep{SluseandTewes2014}. As with the two-band PRM method discussed above, the denser (and the longer) the sampling, the more accurate are the constraints that can be obtained for the time delays.

{\bf Transient AGN and TDEs:} This aspect is discussed in detail in the non-periodic variables section (\autoref{sec:variables}).

% --------------------------------------------------------------------

\subsection{Metrics}
\label{sec:\secname:metrics}

% Quantifying the main impacts on AGN science via MAF metrics, including the effects
% of additional cadence facto,rs such as the number of DDFs
% and MC fields, or different dithering patterns,: definition of the metrics,
% and any derived overall figure of merit.

% --------------------------------------------------------------------

\subsection{Discussion}
\label{sec:\secname:discussion}

% Discussion: what risks have been identified? What suggestions could be
% made to improve the figures of merit, and mitigate the identified risks?
% What ``tweaks'', if any, can be proposed to the nominal LSST observing strategy
% in order to help achieve key AGN science goals?

\navigationbar


% --------------------------------------------------------------------


% --------------------------------------------------------------------

% ====================================================================
%+
% SECTION:
%    section-name.tex  % eg lenstimedelays.tex
%
% CHAPTER:
%    chapter.tex  % eg cosmology.tex
%
% ELEVATOR PITCH:
%    Explain in a few sentences what the relevant discovery or
%    measurement is going to be discussed, and what will be important
%    about it. This is for the browsing reader to get a quick feel
%    for what this section is about.
%
% COMMENTS:
%
%
% BUGS:
%
%
% AUTHORS:
%    Phil Marshall (@drphilmarshall)  - put your name and GitHub username here!
%-
% ====================================================================

\section{Supernova Cosmology and Physics}
\def\secname{supernovae}\label{sec:\secname}
% \label{sec:cosmology, supernovae, classification, lenstimedelays, deepdrillingfields }

\noindent{\it Jeonghee Rho, Michelle Lochner, Rahul Biswas} % (Writing team)

% This individual section will need to describe the particular
% discoveries and measurements that are being targeted in this section's
% science case. It will be helpful to think of a ``science case" as a
% ``science project" that the authors {\it actually plan to do}. Then,
% the sections can follow the tried and tested format of an observing
% proposal: a brief description of the investigation, with references,
% followed by a technical feasibility piece. This latter part will need
% to be quantified using the MAF framework, via a set of metrics that
% need to be computed for any given observing strategy to quantify its
% impact on the described science case. Ideally, these metrics would be
% combined in a well-motivated figure of merit. The section can conclude
% with a discussion of any risks that have been identified, and how
% these could be mitigated.

This section is concerned with the detection, characterization of supernovae 
over time using the Large Synoptic Sky Telescope (LSST) and the use of these
supernovae for a number of science applications. The most important science 
application is the use of supernovae Type Ia (SNIa) and potentially some core-colapse SN (like Type IIP) to trace the recent expansion history of the universe,
and confront models of the physics driving the late time accelerated expansion
of the universe. 

This objective of supernova cosmology follows (at least for SNIa) several
highly successful surveys; improvement in that knowledge could come from
substantially larger numbers of well-characterized supernovae and potentially
useful redshift distributions of such detected supernovae. In this sense, this
goal is not directly tied to the large survey area that is an unprecedented
characteristic of the LSST. However, we shall argue that in practice, even this
goal would be largely helped by the spatial scale offered by the Wide Fast Deep
(WFD) component of the LSST. 

On the other hand, the WFD component of the LSST survey is potentially the 
first single survey to scan supernovae across the very large area of the
entire Southern sky. Therefore, supernovae detected and well characterized
(a) probing the isotropy of the universe, or (b) using peculiar velocities of 
supernovae to probe the growth of structure and finally (c) this may be the
best avenue for a highly complete sample of supernovae that will enable further
sharpening of our understanding of the properties of the supernova population 
of different types. 
This last point is extremely important for supernova cosmology goals: The success of supernova cosmology has always been based on the emperical model of intrinsic peak brightnesses being related to the certain observable characteristics of
supernovae. While the spatial location of the supernovae is not important, the 
WFD has the potential to dramatically increase the size of the sample 
available to train such an emperical model, as well understand the probability of deviations and scatter from this model. Aside from issues like calibration 
which need to be addressed differently, a larger sample size of such well measured supernovae is probably the only way to address deviations from the emperical
model usually discussed as `systematics'. This can be thought of in two 
components: the low redshift sample of supernovae which is more likely to be complete, and the higher redshift sample that might be able to constrain evolution. 
% --------------------------------------------------------------------

\subsection{Target measurements and discoveries}
\label{sec:keyword:targets}

% Describe the discoveries and measurements you want to make.

Supernovae of different types are visible over a time scale of about a few 
weeks (eg. Type Ia) to close to a year (Type IIP). During the full ten year
 survey of LSST, the telescope will scan the entire Southern Sky repeatedly
 with a universal Wide Fast Deep (WFD) Candence, and certain specific locations
of the sky called the Deep Drilling Fields (DDF) with special enhanced cadence. 

This spatio-temporal window should contain millions (RB: remember to check) of supernovae, that will have apparent magnitudes brighter than the single exposure limiting magnitude of LSST, for at least some time.  However, the actual
 sequence of observations in LSST defined by series of field pointings as a
 function of time in filter bands (along with weather conditions) will
 determine the extent to which each of such supernovae can be detected and
 characterized well.  Characterization of these supernovae is at the core of a
 number of science programs that use supernovae as bright, abundant objects with empirically determined intrinsic brightness. For LSST, this goal entails (a) detection of supernovae (b) photometric typing of supernovae, (c) estimating photometric redshifts of supernovae (or identifying host galaxies,
 and obtaining their redshifts from photometry or follow-up spectroscopy)
(c) estimation of intrinsic brightnesses of the supernovae, and finally use these data in addressing our science goals of cosmological inference, etc.
The efficacy of photometric typing, redshifts and estimation of intrinsic brightnesses are all
dependent on the amount of information available in the observed light curves of supernovae. While these steps are not necessarily independent, it is useful to think of the requirements on some of these steps separately; it is not unlikely  that combining some of these steps would still be affected by similar requirements. 

{\emph{Our first objective is to detect such supernovae}}. By `detection of supernovae', we mean a process
that detects transients from the subsets of LSST detection, and classify them as supernovae (as opposed to an AGN, or an asteroid). In brief, this process 
consists the identification of a set of image subtractions between high 
resolution `template` image of a sky section, and a set of single exposures at
different times (usually of lower resolution) of the same region, after 
attempting to correctly account for the different resolutions of images, and alignments. These sets of image subtractions associated
 with a single object will be used to detect the object as a transient and then
classify the transient as a supernova . Clearly, this step of detecting a supernova depends on the number of such images recorded per object, the number of filters and the signal to noise ratio of these images. One might expect that the efficiency of this step may be summarized as a threshold on the joint properties 
of an astrophysical candidate (apparent brightness, light curve characteristics, background) as well as observing conditions (Seeing etc.).  

{\emph{Our second objective is to photometrically classify different kinds of supernovae}} 
{\bfseries Photometric supernova classification}\\
In the past, only spectroscopically typed supernovae have been used for cosmology. Photometric 
typing from the light curve alone has only been used to select candidates for spectroscopic 
follow-up (see for example \citet{Sako2008}). However, LSST will simply produce far too many 
candidates for any chance of following up even a significant fraction of them. In order to avoid 
throwing away the majority of the supernova dataset, we need to use techniques capable of 
determining cosmological parameters from a potentially contaminated photometric supernova dataset.

There have been several techniques proposed in recent literature to solve this problem. One 
approach proposes applying stringent cuts to the photometric dataset to obtain a nearly pure sample 
of type Ia supernovae \citep{Bernstein2012,Campbell2013} and to run the standard supernova analysis 
with this sample. Another approach, BEAMS \citep{Kunz2007,Newling2011,Hlozek2012,Knights2013}, 
makes use of the full dataset, coping with contamination by using a mixture model for the 
likelihood, thus allowing for multiple populations. Whatever the technique ultimately used to for 
cosmological analysis, it will rely on accurate initial classifications of supernova type and 
unbiased estimates for the probability of each type.

The current state-of-the-art photometric classification techniques rely on fitting empirically 
determined templates of supernovae to light curves \citep{Jha2007,Guy2007,Sako2011}. However in 
recent years, new approaches have been published in response to the 2010 `Supernova 
Photometric Classification Challenge' \citep{Kessler2010a}. Many of these use novel light curve 
parameterisation and employ machine learning algorithms to perform the classification (see \citet{Kessler2010b} and references therein).

While many of these methods have been tested on standard sets of simulated data and (in some cases) 
on SDSS data, it is still not clear which technique (if any) is superior in all situations. For 
example, some techniques rely heavily on reliable redshift information being available, while others 
are less reliant on it. Some techniques may be more robust to non-representative datasets than 
others and it is not clear how the techniques will respond to changes in cadence, filter sets, SNR 
etc. With this in mind, we propose the use of a multifaceted classification system which employs 
several different methods of extracting features from the light curves (e.g. fitting parametric 
functions or templates) and several different classification algorithms. This system is highly 
modular, allowing the easy addition of new approaches for direct comparison with existing  techniques. This also allows direct analysis of different observing strategies, without having to 
make an initial choice of classification technique. 


{\emph{Our third obvective is to characterize supernovae in terms of emperical
    light curve models}}

The ultimate goal of using supernovae for a cosmology analaysis (either SNIa or SNIIP) requires an estimate of the intrinsic brightness of the supernova. The
first (and sometimes only step depending on the light curve model) to this, is
to fit the calibrated fluxes to a light curve model with a set of parameters.
According to the ansatz used in supernova cosmology, the intrinsic brightness of
 supernovae is largely determined by the parameters of the light curve model; 
 hence the uncertainties on the inferred parameters largely determine the
 uncertainties on the inferred peak intrinsic brightness or distance moduli of the supernovae.

% Now, describe their response to the observing strategy. Qualitatively,
% how will the science project be affected by the observing schedule and
% conditions? 

% In broad terms, how would we expect the observing strategy
% to be optimized for this science?





% --------------------------------------------------------------------

\subsection{Metrics}
\label{sec:keyword:metrics}

Quantifying the response via MAF metrics: definition of the metrics,
and any derived overall figure of merit.

\emph{To be added: discussion of the ROC curve as a useful metric for photometric supernova 
classification}




% --------------------------------------------------------------------

\subsection{OpSim Analysis}
\label{sec:keyword:analysis}

OpSim analysis: how good would the default observing strategy be, at
the time of writing for this science project?

As noted above the science goal of trying to characterize supernovae is largely
dependent on how well the light curves of individual supernovae are sampled in
time and filters. To study this, we reindex the opsim output on spatial
locations rather than use the temporal index. There are different methods (which will be merged), and here we will first illustrate this in terms the cadence in an example LSST field.

\begin{figure}
\includegraphics[width=\textwidth]{figs/supernova/fig_firstSeason_0}
\includegraphics[width=\textwidth]{figs/supernova/fig_firstSeason_1}
\includegraphics[width=\textwidth]{figs/supernova/fig_firstSeason_2}
\includegraphics[width=\textwidth]{figs/supernova/fig_firstSeason_3}
\includegraphics[width=\textwidth]{figs/supernova/fig_firstSeason_4}
\label{fig:opsimSummary}
\caption{Cadence in different filters for a few LSST deep drilling fields in
    the the ouptut of OpSim version Enigma 1189. This ignores issues of chip 
gaps or overlaps between LSST. These issues have been addressed in \citep{CarrollEtal2014} and Awan et.al. 2015, in preparation. We will add these to this analysis.}
\end{figure}



% --------------------------------------------------------------------

\subsection{Discussion}
\label{sec:keyword:discussion}

Discussion: what risks have been identified? What suggestions could be
made to improve this science project's figure of merit, and mitigate
the identified risks?


\begin{itemize}
\item Intinsic Dispersion, environmental effects, newer analysis methods
\item Follow-up procedures: What is feasible? Where will our training samples for classification and light curve models come from (other experiments, our own 
sub-samples with spectroscopic follow-up), spectroscopic follow-up of host galaxies. Can hosts be identified?
\item `Systematics': In what ways will the real data not match the assumptions made in analysis. Having a large sample of SN, to understand the astrophysics would be useful for this. 
\end{itemize}


% ====================================================================

\navigationbar


% --------------------------------------------------------------------

\chapter[Deep Drilling Fields]{Drilling Deep: Options for a Small Number of Enhanced Observation
Fields}
\def\chpname{deepdrilling}\label{chp:\chpname}

Chapter editors:
\credit{nielbrandt},
\credit{rhiannonlynne}.

% --------------------------------------------------------------------

\section{Introduction}
\label{sec:\chpname:intro}

% Introduce, with a very broad brush, this chapter's science projects,
% and why it makes sense for them to be considered together.

% Individual sections go below, one science project per section, one
% section per file.

% --------------------------------------------------------------------

% \input{DDsection1}

% --------------------------------------------------------------------

% \input{DDsection2}

% --------------------------------------------------------------------


% --------------------------------------------------------------------

\chapter[Special Surveys]{Special Surveys}
\def\chpname{specialsurveys}\label{chp:\chpname}

Chapter editors:
\credit{dnidever},
\credit{knutago}.

% Confirmed leads for LMC/SMC: Knut Olsen, David Nidever

% Confirmed leads for special surveys:

% --------------------------------------------------------------------

\section{Introduction}
\label{sec:specials:intro}

The four main LSST science themes, as defined by the Science Book,
drive the design of LSST's main Wide-Fast-Deep survey.  However, it
has always been recognized that many important scientific projects,
including some that are highly relevant to LSST's main science themes,
are not well served by the areal coverage and/or cadence constraints
placed on the WFD survey.  To this end, the LSST Project set aside a
nominal 10\% of the observing time to serve what are collectively
called ``special surveys''.

Projects that
will certainly make use of this 10\% time (that is not dedicated to the WFD
survey) include the Deep Drilling fields and the Galactic Plane surveys,
as well as any survey wishing to
observe at declinations below $-60^\circ$, such as the Magellanic
Clouds.  These special programs have the potential to
heavily oversubscribe the nominal 10\%
of time assigned to them.  It is of thus critical importance for these
programs to define compelling science cases, clearly justify their
observing requirements, and derive metrics to quantify the performance
of a given schedule for the program. This chapter provides a venue for
such investigations.

% A minimal set of 4--5 ``extragalactic'' Deep Drilling Fields have been
% included in the `OpSim' runs to date (\autoref{chp:cadexp}), and have
% been evaluated in various science sections throughout this paper.
% Here, we aim to explore some other proposals for Deep Drilling Fields, and
% make some suggestions for \OpSim runs based on them. As well as these
% DDF's, we also describe
% a special survey designed to serve scientific
% goals related to the Magellanic Clouds, and
% several special surveys
% aimed at particular Solar System science cases.  Descriptions of
% further proposed special surveys are welcome here.


% Add sections below, one science investigation per section, one
% section per file.

% --------------------------------------------------------------------

% PJM: commented out for now, for lack of content:
% \chapter[Deep Drilling Fields]{Drilling Deep: Options for a Small Number of Enhanced Observation
Fields}
\def\chpname{deepdrilling}\label{chp:\chpname}

Chapter editors:
\credit{nielbrandt},
\credit{rhiannonlynne}.

% --------------------------------------------------------------------

\section{Introduction}
\label{sec:\chpname:intro}

% Introduce, with a very broad brush, this chapter's science projects,
% and why it makes sense for them to be considered together.

% Individual sections go below, one science project per section, one
% section per file.

% --------------------------------------------------------------------

% \input{DDsection1}

% --------------------------------------------------------------------

% \input{DDsection2}

% --------------------------------------------------------------------


% --------------------------------------------------------------------

% PJM: This is currently in the Magellanic Clouds Chapter, but
% could be moved back here soon...
% % ====================================================================
%+
% NAME:
%    section-name.tex
%
% ELEVATOR PITCH:
%    Explain in a few sentences what the relevant discovery or
%    measurement is going to be discussed, and what will be important
%    about it. This is for the browsing reader to get a quick feel
%    for what this section is about.
%
% COMMENTS:
%
%
% BUGS:
%
%
% AUTHORS:
%    David Nidever (@dnidever)
%    Knut Olsen (@knutago)
%-
% ====================================================================

\section{The Magellanic Clouds Special Survey}
\def\secname{mc}\label{sec:\secname}

\credit{dnidever},
\credit{knutago}.


An LSST survey that did not include coverage of the Magellanic Clouds
and their periphery would be tragically incomplete.  LSST has a unique
role to play in surveys of the Clouds.  First, its large $A\Omega$
will allow us to probe the thousands of square degrees that comprise
the extended periphery of the Magellanic Clouds with unprecedented
completeness and depth, allowing us to detect and map their extended
disks, stellar halos, and debris from interactions that we already
have strong evidence must exist (REFS).  Second, the ability of LSST
to map the entire main bodies in only a few pointings will allow us to
identify and classify their extensive variable source populations with
unprecedented time and areal coverage, discovering, for example,
extragalactic planets, rare variables and transients, and light echoes
from explosive events that occurred thousands of years ago (REFS).
Finally, the large number of observing opportunities that the LSST
10-year survey will provide will enable us to produce a static imaging
mosaic of the main bodies of the Clouds with extraordinary image
quality, an invaluable legacy product of LSST.

% --------------------------------------------------------------------

\subsection{A Proposed Magellanic Clouds Mini-survey}
\label{sec:\secname:proposal}

We propose two distinct mini-surveys to meet the goals of LSST
Magellanic Clouds science:
\begin{itemize}
\item A mini-survey covering the 2700$\deg^2$ with $\delta < -60$ to
the standard LSST single-exposure depth and to stacked depths of XXX,
with cadence sufficient to detect and measure light curves of RR Lyrae
stars.
\item A mini-survey covering $\sim$250$\deg^2$ of the main bodies of
the Clouds with cadence sufficient to detect exoplanet transits and
other variable objects; a subset of these images should be taken with
seeing of $0.5\arcsec$, with stacked depth reaching the confusion
limits in the Clouds.
\end{itemize}

Figure X shows a rough map of the proposed mini-surveys.
% Need the figure and caption


% --------------------------------------------------------------------

\subsection{Mini-survey Impact on the Magellanic Cloud Science Projects}
\label{sec:\secname:revisit}

\new{Here we revisit the metric analysis of the Magellanic  Clouds'
science cases (\autoref{chp:mc}), and make some predictions about how
they are likely to improve given  the above proposal.}


% --------------------------------------------------------------------

\subsection{Discussion}
\label{sec:\secname:discussion}


% ====================================================================

\navigationbar


% --------------------------------------------------------------------

% ====================================================================
%+
% NAME:
%    section-name.tex
%
% ELEVATOR PITCH:
%    Explain in a few sentences what the relevant discovery or
%    measurement is going to be discussed, and what will be important
%    about it. This is for the browsing reader to get a quick feel
%    for what this section is about.
%
% COMMENTS:
%
%
% BUGS:
%
%
% AUTHORS:
%    David Nidever (@dnidever)
%    Knut Olsen (@knutago)
%-
% ====================================================================

\section{Solar System mini-surveys}
\def\secname{solar_system_specials}\label{sec:\secname}

\credit{davidtrilling},
\credit{rhiannonlynne}.

% This individual section will need to describe the particular
% discoveries and measurements that are being targeted in this section's
% science case. It will be helpful to think of a ``science case" as a
% ``science project" that the authors {\it actually plan to do}. Then,
% the sections can follow the tried and tested format of an observing
% proposal: a brief description of the investigation, with references,
% followed by a technical feasibility piece. This latter part will need
% to be quantified using the MAF framework, via a set of metrics that
% need to be computed for any given observing strategy to quantify its
% impact on the described science case. Ideally, these metrics would be
% combined in a well-motivated figure of merit. The section can conclude
% with a discussion of any risks that have been identified, and how
% these could be mitigated.

%A short preamble goes here. What's the context for this science
%project? Where does it fit in the big picture?

There are several populations of Near Earth Objects (Solar System bodies
whose orbits bring them close to the Earth's orbit) that, because of
their orbital properties, would not be easily detected in the
wide-fast-deep survey. These populations are very interesting for both
scientific and sociological purposes, though, due to their close
proximity to the Earth, and in fact their potential for impacting the
Earth. LSST will have the capability to carry out surveys for these
populations by using a small amount of time in ``mini-surveys.'' Two of
these mini-surveys have pointings that fall within the nominal
wide-fast-deep plan, and simply require a modification of the cadence.
The third program is a twilight program, with a special cadence (though
all twilight programs are likely to  have special cadences). These three
programs are listed here and described below. The three mini-surveys are
the following:

\begin{itemize}
\item A mini-survey to look for mini-moons, which are temporarily captured
satellites of the Earth;
\item A mini-survey to find meter-sized impactors up to two weeks prior to impact.
This would allow telescopic characterization of these impactors, which could
be compared to laboratory measurements of the meteorites derived from
the impactor. Advanced warning of an impactor also allows detailed
study of impact physics by being on location when the impact
occurs;
\item A mini-survey to observe the ``sweetspot'' in twilight fields
to look for NEOs in very Earth-like orbits that would otherwise not
be found in opposition fields.
\end{itemize}

% Need the figure and caption
These surveys will support three important scientific investigations:
\begin{enumerate}
\item What are the properties of the population of objects that is
nearest to the Earth?
\item What is the impact risk from NEOs in populations that
have not yet been well characterized (mini-moons, sweetspot objects)?
\item How do the telescopic properties of an impactor relate to the
laboratory-measured properties of the ensuing meteorites?
\end{enumerate}

%Many different types of objects and measurements with their own cadence
%``requirements'' will fall into these two broad categories (with some
%overlap).  These will be outlined in the next section.
Some details of the special cadence requirements for these 
science investigations are described in the following section.

% --------------------------------------------------------------------

\subsection{Target measurements and discoveries}
\label{sec:\secname:targets}

\subsubsection{Special cadences}

Each of the three Solar System mini-surveys requires a special
cadence. These cadences are described here.

\begin{itemize}

\item{{\bf Mini-moons}}
Mini-moons are objects that are temporary satellites of the Earth
\citep{2014Icar..241..280B, 2017Icar..285...83F}
% bolin et al, icarus, 241, 280 http://adsabs.harvard.edu/abs/2014Icar..241..280B
% fedorets et al. icarus 285 83 http://adsabs.harvard.edu/abs/2017Icar..285...83F
Therefore, they have orbital motions similar to the Earth's moon,
and much faster than other Solar System populations. Therefore,
a special cadence is required to detect these objects enough
times to link objects, create tracklets, and determine orbits.
A suggested cadence for a mini-moon survey is a series
of 3~second exposures, with each pointing visited at least
twice per night. Such a survey would cover essentially
all of the opposition sky each night. The opposition sky should
be re-observed several nights in a row in order to
link objects from night to night and determine their orbits.
While the details of this special cadence are not yet
fully refined, this mini-survey would likely have little impact
on the overall LSST program since this small-scale
program, which extends over a small number of nights,
is effectively a compressed rolling cadence in which
the aggregate field coverage is unchanged.

\item{{\bf Impactors}}
The Earth is struck by meter-sized impactors about
once a month \citep[\eg][Trilling \etal 2017 submitted]{Boslough2015, 2017Icar..284..416T}. 
% boslough et al. 2015 in Aerospace Conference, 2015 IEEE, 1-12
% tricarico 2017 icarus 284 416
% trilling et al 2017 AAS journals submitted
On two occasions, impacting asteroids have
been discovered some hours before impact, but
there are no existing surveys that are dedicated to finding
impactors.
% xxx ATLAS xxx.
Impactors generally have small apparent motions
on the sky (because their orbits are not too different
than the Earth's). The single exposure depth of LSST
images suggests that a meter-sized NEO could be
discovered perhaps a week before impact, given
the typical Earth-relative velocity of such a body
\citep[\eg][]{2017arXiv170506209C}.
% chesley & veres 2017 https://arxiv.org/abs/1705.06209
A suggested cadence for an impactor survey would be
to survey the opposition patch four times per night.
This is more visits than in the nominal cadence, and
would allow high fidelity linking of observations to
find orbits. The nominal wide-fast-deep cadence
(twice per night, three times during a lunation) has
a latency of orbit determination of up to two weeks,
which is not acceptable for the impactor survey, as an
impact would occur on a timescale of just a few days
from discovery.
The cadence of four observations/night should be repeated
roughly every three days, so that an object on an
inbound trajectory could be observed at least once,
and possibly twice, before impact.
Note that this cadence is compatible with
the wide-fast-deep survey, in that the fields and
exposure times are nominal; the only difference is that
each field is visited four times in a night, and that
the fields are revisited every few nights. The overall
impact of this mini-survey on the wide-fast-deep
survey is likely to be small, and possibly negligible.
Given the importance of this small but
significant investigation, it is critical that the
survey simulators be capable of including such
a mini-survey in planning for LSST operations.

\item{{\bf Twilight/sweetspot survey}}

NEOs on very Earth-like orbits are relatively
unlikely to come to opposition, and therefore
are relatively unlikely to appear in data
obtained in the wide-fast-deep survey.
These objects are particularly interesting
since, having very Earth-like orbits, they
are the most likely objects to be Earth
impactors.
These objects are most likely to be detected
in a twilight survey that looks at the ``sweetspot'' ---
a location at around 60~degrees Solar
elongation that is only visible at twilight.
Because these sweetspot fields are only visible
for 30--60~twilight minutes each night,
a special
cadence is required to find and link these objects
to determine their orbits.
These observations would be best carried out
in the z filter (because the observations are
made in twilight, when the sky is still relatively
bright). Fields should be revisited at 15~minute
intervals, and each field should be revisited
every other night during this experiment, so that
observations can be linked.
(A long interval
between observations prohibits linking.)
The total experiment
should last roughly one week, so that each
object would have a tracklet on four nights
(nights 1,3,5,7).
During twilight, some 25~pointings could be visited
after the sky is sufficiently dark but
before the fields have set.
Because these observations are made during twilight,
there may be no significant impact on the
nominal wide-fast-deep survey.
\end{itemize}

\subsubsection{Measurements}

For each of these three programs, the most important measurement
to be made is the position of any object as a function of time.
In other words, the usual measurements of moving
objects from LSST images is also the requirement for
the source detections for these mini-surveys. As usual
for Solar System surveys, there is a trade-off of
sensitivity (Solar System objects are most easily
detected in r band) against characterization (observing
a given object in multiple filters yields an estimate
of composition). For these three cases, discovery and
good orbit determination is probably more important than
immediate characterization from LSST measurements,
so the nominal expectation is that 
the nighttime mini-surveys would be carried out in 
r band and the twilight program in z band.


% --------------------------------------------------------------------

\subsection{Metrics}
\label{sec:\secname:metrics}

The metrics to be used to determine the efficacy of LSST
at scientific success of these mini-surveys are identical
to those employed in \autoref{chp:solarsystem}.
The most important of these metrics
include the completeness as a function of size; the
number of detections over a given length of time (for instance,
the one week approach timescale of impactors); and
the quality of the derived orbit. These metrics are defined
in more detail in \autoref{chp:solarsystem}. The important question is:
how much value do the mini-surveys add?


% --------------------------------------------------------------------

\subsection{OpSim Analysis}
\label{sec:\secname:analysis}

The current default observing strategy does not include
any of these mini-surveys. Therefore, the scientific yield,
at this default, is zero. Both the mini-moons and impactor
surveys are relatively small experiments, on the scale of
the LSST project, at something like 10--20~hours total
per instance of the experiment. (The impactor experiment,
for example, might be carried out one or several times a year,
both to build up statistics and to identify further potential
impactors.) Furthermore, the impactors survey cadence
is different from the nominal wide-fast-deep survey,
but could be a simple modification of the nominal wide-fast-deep survey
cadence.

The twilight/sweetspot survey is also not included in
the nominal OpSim strategy, and the overall discussion
of twilight observations is deferred to a later discussion.

It is critical to ensure that OpSim can handle the kind
of dedicated cadences described above in order to assess
the global impact of these small-scale but highly
important Solar System mini-surveys.



%
% % --------------------------------------------------------------------
%
% \subsection{Discussion}
% \label{sec:\secname:discussion}
%
% Discussion: what risks have been identified? What suggestions could be
% made to improve this science project's figure of merit, and mitigate
% the identified risks?
%

% ====================================================================

% ====================================================================
%
% \subsection{Conclusions}
%
% Here we answer the ten questions posed in
% \autoref{sec:intro:evaluation:caseConclusions}:
%
% \begin{description}
%
% \item[Q1:] {\it Does the science case place any constraints on the
% tradeoff between the sky coverage and coadded depth? For example, should
% the sky coverage be maximized (to $\sim$30,000 deg$^2$, as e.g., in
% Pan-STARRS) or the number of detected galaxies (the current baseline 
% of 18,000 deg$^2$)?}
%
% \item[A1:] ...
%
% \item[Q2:] {\it Does the science case place any constraints on the
% tradeoff between uniformity of sampling and frequency of  sampling? For
% example, a rolling cadence can provide enhanced sample rates over a part
% of the survey or the entire survey for a designated time at the cost of
% reduced sample rate the rest of the time (while maintaining the nominal
% total visit counts).}
%
% \item[A2:] ...
%
% \item[Q3:] {\it Does the science case place any constraints on the
% tradeoff between the single-visit depth and the number of visits
% (especially in the $u$-band where longer exposures would minimize the
% impact of the readout noise)?}
%
% \item[A3:] ...
%
% \item[Q4:] {\it Does the science case place any constraints on the
% Galactic plane coverage (spatial coverage, temporal sampling, visits per
% band)?}
%
% \item[A4:] ...
%
% \item[Q5:] {\it Does the science case place any constraints on the
% fraction of observing time allocated to each band?}
%
% \item[A5:] ...
%
% \item[Q6:] {\it Does the science case place any constraints on the
% cadence for deep drilling fields?}
%
% \item[A6:] ...
%
% \item[Q7:] {\it Assuming two visits per night, would the science case
% benefit if they are obtained in the same band or not?}
%
% \item[A7:] ...
%
% \item[Q8:] {\it Will the case science benefit from a special cadence
% prescription during commissioning or early in the survey, such as:
% acquiring a full 10-year count of visits for a small area (either in all
% the bands or in a  selected set); a greatly enhanced cadence for a small
% area?}
%
% \item[A8:] ...
%
% \item[Q9:] {\it Does the science case place any constraints on the
% sampling of observing conditions (e.g., seeing, dark sky, airmass),
% possibly as a function of band, etc.?}
%
% \item[A9:] ...
%
% \item[Q10:] {\it Does the case have science drivers that would require
% real-time exposure time optimization to obtain nearly constant
% single-visit limiting depth?}
%
% \item[A10:] ...
%
% \end{description}

\navigationbar


% --------------------------------------------------------------------

% ====================================================================
%+
% NAME:
%    short_exposures.tex
%
% CHAPTER:
%    specialsurveys.tex
%
% ELEVATOR PITCH:
%
% AUTHORS:
%    Chris Stubbs (@astrostubbs))
%-
% ====================================================================

\section{Short Exposure Surveying}
\def\secname{shortexp}\label{sec:\secname}

\credit{astrostubbs}

The current LSST requirements stipulate a minimum exposure time of 5
seconds, with an expected default exposure time of 15 seconds. This
document advocates for decreasing the minimum exposure time requirement
from 5 to 0.1 seconds. This would increase the dynamic range for bright
sources (compared to the default 15 sec time) by about 5 magnitudes, to
a total of 13 astronomical magnitudes (where dynamic range is the
difference between the brightest unsaturated source and the faintest
point source detectable at 5 sigma). This is a large factor, and would
enable a wide range of science goals, outlined below. One interesting
aspect of this is that it would allow us to operate the LSST system
during twilight times that would otherwise saturate the array due to
background sky brightness. This would allow a number of the goals
described below to be carried out without impacting the primary survey
by conducting observations during twilight sky conditions that would
saturate the array at longer exposure times.


% ----------------------------------------------------------------------

\subsection{Introduction}
\label{sec:\secname:intro}

Since the twilight sky brightness is an important factor discussed
below, we provide here a very brief outline of the temporal evolution of
the background sky brightness.

\citet{1993AJ....105.1206T}
provide a simple framework that serves our purposes well. They provide
observational data as well as a simple model for the evolution of
twilight sky brightness. Figure~1 from that paper is included below, as \autoref{fig:Tyson}.
They show that a good model for the sky brightness evolution is given by
an exponential with
$\log_{10}(S)=(k/\tau)t+C$,
where S is the sky brightness in electrons per pixel per second, C is
the dark sky background, k = (10.6 minutes)$^{-1}$  is a universal
(band-independent) timescale during which the sky's surface brightness
changes by a factor of ten (at latitude $-$30 degrees), and $\tau$ is a
season-dependent factor that ranges from 1.0 at the equinox to 1.07 in
austral winter and 1.20 in austral summer. So the rule of thumb is that
we should expect it to take 4.25 minutes for the sky background to
change by one magnitude per square arc sec. (In what follows we'll
ignore the increased twilight time in summer and winter.)

For current generation typical astronomical camera systems that take
over a minute to read out, this 4.2 minute time scale means that only a
handful of images can be obtained during twilight time. But for the LSST
camera with a 2 second readout time, we can obtain hundreds of short
exposures during twilight. Even if we are limited to a 15 second cadence
due to thermal stability or data transfer limitations there is a large
amount of time opened up that we can use.

What do we stand to gain in operational time with shorter exposures? If
the standard survey terminates taking 15 second exposures due to some
sky brightness criterion, by shifting to 0.1 sec images at that point we
will have changed the sky flux per pixel by 2.5 $\log_{10}(150)$ = 5.4
magnitudes. This brings us back into a high dynamic range regime, as
described below.

\begin{figure}[htbp]
\begin{center}
\includegraphics[trim = 0 7cm 0 1mm, clip, width=\textwidth]{figs/Stubbs_Fig1.pdf}
\caption{(reproduced from Tyson et al, 1993). This plot shows the
  twilight sky surface brightness as a function of local time for four
  broadband filters (C, B, V and R) and different pointing directions.
  The surface brightness changes by one magnitude in a 4.2 minute interval,
essentially independently of the passband and pointing.}
\label{fig:Tyson}
\end{center}
\end{figure}

\autoref{fig:twilight} illustrates the principles that underpin this proposal. LSST is
a unique combination of hardware and software, that will deliver
reliable catalogs of both the static and the dynamic sky. By pushing
towards shorter integration times we can greatly expand the scientific
reach of the system.

The dynamic range in magnitudes that we can achieve for a given
integration time depends on the sky background, the read noise, and the
full well depth per pixel. We will adopt a typical value of 100Ke for
the full well depth, but the arguments presented below are essentially
independent of this value. The dynamic range in magnitudes is limited on
the bright end by the point source whose PSF peak exceeds full well, and
on the faint end by the 5$\sigma$ point source sensitivity, which
depends on sky brightness per pixel. So we are squeezed between the two
parameters of full well depth and sky background.

\begin{figure}[htbp]
\begin{center}
\includegraphics[width=6in]{figs/Stubbs_Fig2.pdf}
\caption{Twilight dynamic range. As we enter morning twilight time, the increasing sky brightness requires brighter sources for 5 sigma detection, and also limits unsaturated objects to increasingly fainter sources. Eventually the gap between these goes to zero. But operating at shorter exposure times allows us to push useful survey operations into brighter twilight time, and also to increase the dynamic range of the LSST survey products. The black lines correspond to 15 second integrations (nominally in the r band), the red lines to 5 second exposures, and the blue curves to 0.1 second exposures. The upper lines in each case represent the 5 sigma point source detection threshold while the lower line corresponds to the source brightness that produces saturation in the peak pixel of the PSF. Adding shorter exposure times increases our dynamic range in flux, and adds valuable observing time.}
\label{fig:twilight}
\end{center}
\end{figure}

The 5-sigma limiting flux scales as the square root of the sky
brightness, while the saturation flux decreases linearly as sky
brightness increases. So the two curves in \autoref{fig:twilight} have
slopes that differ by a factor of two. Operating during bright-sky time
with short exposures adds about 20 minutes of observing per twilight, or
40 minutes per night. This is a non-trivial resource!

\autoref{fig:twilight} shows one reason why it is not advantageous to go
below 0.1 second exposures- we would lose the overlap between a twilight
survey and the standard LSST object catalog.


% ----------------------------------------------------------------------

\subsection{Science Drivers for Shorter Exposures}
\label{sec:\secname:drivers}

Having set the stage for the opportunity to operate at shorter exposure
times either during dark sky time, or during twilight, or both, we now
describe some of the scientific motivations for doing so.


\subsubsection{Discovery space at short time scales.}

LSST is a time domain discovery machine. It is hard to anticipate the
importance of being able to detect astronomical variability on short
time scales. By extending the time domain sensitivity to phenomena with
a characteristic time of less than 5 seconds, we will have added 1.5
orders of magnitude in time domain sensitivity.

Taking short exposures does not necessarily imply a requirement on fast
image cadence. Periodic variability can be readily detected and
characterized with a succession of short images that do not satisfy the
Nyquist criterion, as long as we know the time associated with each data
point to adequate accuracy. But it does seem appropriate to investigate
the maximum possible rapid-fire imaging rate for LSST, presumably
limited by either data transfer bottlenecks or by thermal issues within
the camera.

\subsubsection{Distances to Nearby SN Ia- an essential ingredient in using supernovae to probe dark energy.}

The determination of the equation of state parameter of the Dark Energy
using type Ia supernovae entails measuring the redshift dependence of
the luminosity distances to objects over a range of redshifts. The low
end of this redshift range is limited by peculiar velocities to
considering supernovae at redshifts z$>$0.01. At this distance (distance
modulus of $\mu$ =33) the peak brightness of a type Ia supernova is r=15
and exceeds the expected LSST point source saturation limit.

Moreover, the rate on the sky of these bright nearby supernovae is so
low that in the standard cadence we don't expect to obtain well-sampled
multiband light curves for them. But we will discover many of them on
the rise. Using twilight time with short exposures to obtain appropriate
temporal and passband coverage will allow us to extend the LSST SN
Hubble diagram across the entire redshift range of 0.01 to 1.

It is vitally important that we obtain these nearby-SN light curves on
the same photometric system, reduced with the same data reduction
pipeline, as the distant sample. This means we really must use the LSST
instrument and software in order to avoid systematic errors arising from
differences in photometric systems or algorithmic issues.

We stress that this twilight SN followup campaign can be accomplished
without impacting the main survey, during the roughly 20 minutes per
night of twilight that would otherwise unusable at the default exposure
time. We would use the brighter twilight time to obtain pointed
observations on nearby supernovae, motivated by the importance of
photometric uniformity described above.


\subsubsection{A Bright Star Survey for Galactic Science.}

We could also use the added twilight time to conduct a bright star
survey, and the precise astrometry and photometry from LSST can then be
used in conjunction with archived data ranging from 11th to 27th AB
magnitudes. This short-exposure domain would extend the LSST dynamic
range in fluxes by two orders of magnitude, towards the bright end.
Moreover, obtaining precise positions, fluxes and variability at these
brighter magnitudes would greatly increase the overlap with the
historical archive of astronomical information, including from digitized
plate data. We would be able to obtain astrometric and color information
to high precision, as well time series for variability studies.

An example of an application to Milky Way structure studies comes from
RR Lyrae variable stars. With a saturation magnitude of around 16th in
the standard LSST survey, RR Lyrae closer than 20 kpc will be saturated
in the standard LSST images. So we will lose nearly all Galactic RR
Lyrae. Extending the survey's bright limit to 11th magnitude will allow
us to collect light curves for RR Lyrae beyond $\sim$ 100 parsecs,
collecting essentially all Southern hemisphere Galactic RR Lyrae.

Another application for stellar population studies is measuring the
fraction of binary stars as a function of stellar type, metallicity, age
and environment. By conducting a variability survey in the 11-18
magnitude range we can capitalize on temperature and metallicity data
already in hand for many of these objects.

Another application of a bright star survey would be to search for
planetary transits in the magnitude range appropriate for radial
velocity followup observations using 30 meter class telescopes. For high
dispersion spectrographs at the 4m aperture class, most targets are
currently around 8th magnitude, so we should expect 30m telescopes to
attain similar radial velocity precisions for sources of magnitude  8 +
5log(30/4) = 12. By going to shorter exposures we obtain almost an
hour's additional observing time per night when these sources don't
saturate, whereas they are far beyond saturation in the default 15
second LSST survey images.

A typical (r$-$K) color between SDSS and 2MASS is r$-$K=3. The 2MASS
catalog is complete down to K$\sim$14 which corresponds to r$\sim$17. So
most 2MASS stars will be saturated in the standard LSST 15 second
observations. A bright star survey will allow a multiband match to the
2MASS data, as well as an astrometric comparison between the two
catalogs.

Finally, the apparent magnitude of solar system objects depends on their
distance from us and from the sun, as well as illumination and
observation geometry. Extending the bright limit will allow us to track
asteroid positions as they approach opposition.


% ----------------------------------------------------------------------

\subsection{Counterarguments}
\label{sec:\secname:counter}


\subsubsection{What About Scintillation Effects?}

Short exposure times suffer from scintillation effects. An estimate for
uncertainty due to scintillation is provided by
\url{http://astro.corlan.net/gcx/scint.txt}. For a 0.1 second
integration we expect a fractional flux uncertainty of  0.15 at 2
airmasses and 0.043 at 1 airmass, for a 10 cm aperture. Scaling this up
to the 8.5m aperture of LSST by a factor D$^{2/3}$ predicts fractional
flux variations of below one percent, even at two airmasses, for a 0.1
second exposure. So scintillation should not impact our ability to make
precision measurements of flux and position.

\subsubsection{What about just doing this with smaller telescopes?}

A possible counter-argument to the proposal of allowing for shorter
exposure times is that much of this can be done with smaller telescopes.
But it's important to bear in mind that LSST is a system, and the data
reduction and dissemination tools are as important as the hardware. We
intend to deliver accessible, high-quality, well-calibrated photometry
on a common photometric system and correspondingly good positions. If we
do so from a co-added point source depth of 27th to the short-exposure
bright limit of 11th magnitude we will span over six decades in flux on
a well-calibrated flux scale. We would also have the ability to study
astrophysical variability on time scales from 0.1 second to 10 years,
which is nine decades in the time domain. This combination of temporal
and flux dynamic range would be a truly remarkable  achievement, and
would yield science benefits far beyond the illustrative examples
provided above. Much of this discovery space is enabled by going to
shorter exposures.

\subsection{Proposed Implementation and Impacts}

The implementation of this would simply entail taking short-exposure
images during twilight time that would otherwise go unused. The data
rate would go up, and the number of shutter cycles per night would also
increase.

%
%\section{References}
%
%Tyson and Gal, An Exposure Guide for Taking Twilight Flats with Large Format CCDs, AJ {\bf 105}, 1026 (1003).

% ====================================================================
%
% \subsection{Conclusions}
%
% Here we answer the ten questions posed in
% \autoref{sec:intro:evaluation:caseConclusions}:
%
% \begin{description}
%
% \item[Q1:] {\it Does the science case place any constraints on the
% tradeoff between the sky coverage and coadded depth? For example, should
% the sky coverage be maximized (to $\sim$30,000 deg$^2$, as e.g., in
% Pan-STARRS) or the number of detected galaxies (the current baseline
% of 18,000 deg$^2$)?}
%
% \item[A1:] ...
%
% \item[Q2:] {\it Does the science case place any constraints on the
% tradeoff between uniformity of sampling and frequency of  sampling? For
% example, a rolling cadence can provide enhanced sample rates over a part
% of the survey or the entire survey for a designated time at the cost of
% reduced sample rate the rest of the time (while maintaining the nominal
% total visit counts).}
%
% \item[A2:] ...
%
% \item[Q3:] {\it Does the science case place any constraints on the
% tradeoff between the single-visit depth and the number of visits
% (especially in the $u$-band where longer exposures would minimize the
% impact of the readout noise)?}
%
% \item[A3:] ...
%
% \item[Q4:] {\it Does the science case place any constraints on the
% Galactic plane coverage (spatial coverage, temporal sampling, visits per
% band)?}
%
% \item[A4:] ...
%
% \item[Q5:] {\it Does the science case place any constraints on the
% fraction of observing time allocated to each band?}
%
% \item[A5:] ...
%
% \item[Q6:] {\it Does the science case place any constraints on the
% cadence for deep drilling fields?}
%
% \item[A6:] ...
%
% \item[Q7:] {\it Assuming two visits per night, would the science case
% benefit if they are obtained in the same band or not?}
%
% \item[A7:] ...
%
% \item[Q8:] {\it Will the case science benefit from a special cadence
% prescription during commissioning or early in the survey, such as:
% acquiring a full 10-year count of visits for a small area (either in all
% the bands or in a  selected set); a greatly enhanced cadence for a small
% area?}
%
% \item[A8:] ...
%
% \item[Q9:] {\it Does the science case place any constraints on the
% sampling of observing conditions (e.g., seeing, dark sky, airmass),
% possibly as a function of band, etc.?}
%
% \item[A9:] ...
%
% \item[Q10:] {\it Does the case have science drivers that would require
% real-time exposure time optimization to obtain nearly constant
% single-visit limiting depth?}
%
% \item[A10:] ...
%
% \end{description}

\navigationbar


% --------------------------------------------------------------------

\navigationbar


% --------------------------------------------------------------------

\chapter[Tensions and Trade-offs]{Tensions and Trade-offs}
\label{chp:tradeoffs}

\noindent {\it
...
}

Discussion and conclusions chapter, at the end, highlighting the
issues that we will need to figure out. Possible topics include the
cost/benefit tradeoffs between competing objectives


% --------------------------------------------------------------------

\bibliographystyle{apj}
\bibliography{references}

% --------------------------------------------------------------------

\end{document}

% ====================================================================
