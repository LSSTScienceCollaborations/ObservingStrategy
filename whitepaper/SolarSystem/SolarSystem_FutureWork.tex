% ====================================================================
%+
% SECTION:
%    SolarSystem_FutureWork.tex
%
% CHAPTER:
%    solarsystem.tex
%
% ELEVATOR PITCH:
%    Ideas for future metric investigation, with quantitaive analysis
%    still pending.
%-
% ====================================================================

\section{Future Work}
\def\secname{\chpname:future}\label{sec:\secname}

In this section we provide a short compendium of science cases that
are either still being developed, or that are deserving of quantitative
MAF analysis at some point in the future.

% ====================================================================

% \section{Orbital Accuracy}
% \def\secname{\chpname:orbits}\label{sec:\secname}

\subsection{Orbital Accuracy}

After initial discovery, investigation of a solar system object quickly
moves to inference of its orbit. Immediate questions include: How secure
is the orbit? Is it going to hit us? Can we find it after X years for
further study? Longer term, we are interested in questions such as: What
is the libration amplitude distribution for TNOs? Can we identify the
source region for NEOs within the main belt?

Quantifying the performance of a given simulated observing pattern for
any of these  science questions will involve somehow capturing the ease
with which  orbital parameters can be inferred.

% ====================================================================

% \section{Detecting Comet Activity}
% \def\secname{\chpname:activity}\label{sec:\secname}

\subsection{Detecting Comet Activity}

Comets are the remnant building blocks of the Solar System
that have been stored at cold temperatures beyond the ice
line, either in the Kuiper belt or the Oort cloud, since their
formation.  Measuring the evolution of cometary activity over
a range of heliocentric distances with LSST will allow to
understand the overall comet activity and to link these
observations with the physical and chemical conditions in the
early solar nebula during planet formation.  Comets are
classified in two main dynamical families, Jupiter Family
comets (JFCs) that have low-inclination orbits with periods
less than 20 years, and Long-period comets (LPCs) that
originate in the Oort Cloud at a distance of more than 10000
AU and have large orbital eccentricities and nearly isotropic
distribution of inclinations.  Currently there are over 400
Jupiter-family comets known, most of which are faint compared
with the LPCs.  LSST will observe about $10^4$ individual
comets repeatedly including measurements of known objects over
its 10-year survey \citep{2010PhDT.......241S}. The
determination of their activity levels at various heliocentric
distances will be used to study the time evolution of each
object individually and to find the connection between comet
families and their formation region in the Solar System.

Several cometary volatiles result in strong emission bands
excited by solar radiation that emit by resonant fluorescence
at optical and near-ultraviolet wavelengths.  The LSST $u$
filter peaks near the CN (0--0) emission band at 3880 \r{A}.
Although CN is not the most abundant daughter species from
cometary volatiles and the OH (0--0) emission band at 3080
\r{A} is generally stronger, CN production rates provide an
excellent proxy of the level of overall gas activity in
comets. Other bands such as $r$, $i$, and $z$ will detect
continuum brightness that is produced by reflected radiation
from dust particles in the coma. Thus, it will be possible to
obtain the evolution of the gas-to-dust production ratio at
high cadence as a function of heliocentric distance in
different comet families. The greatly increased sample size
compared with previous catalogs \citep{1995Icar..118..223A}
will allow for statistical comparison of the comet families
and to link them to other small body populations in the Solar
System.

% --------------------------------------------------------------------
%
% \subsection{Target measurements and discoveries}
% \label{sec:\secname:targets}

LSST will make an exceptionally large number of comet
observations.  About $10^4$ comets will be observed on average
of 50 times by LSST during its main survey, while a few objects
will be observed more than 1000 times
\citep{2010PhDT.......241S}.  Simulations of characteristic
comet orbits have shown that LSST will observe some Jupiter
Family comets (JFCs) hundreds of times over their full orbits
\citep{2010PhDT.......241S}.  Individual LPCs are predicted to
be observed by LSST with dozens of observations as they
approach or recede from the center of the Solar System or
during their perihelion passage.  Thus, these observations
will trace the onset of outgassing from quiescence at large
heliocentric distances and the decline of activity after
perihelion.  LSST will offer the unique opportunity to produce
a large database of CN production rates for the observed
comets that will vastly improve existing ones \citep[see
e.g.][]{1995Icar..118..223A,2012ApJ...758...29A}.

A recently discovered population of main-belt asteroids eject
dust and produce coma and tails giving them the appearance of
comets \citep{2012AJ....143...66J}.  This so-called main-belt
comets or active asteroids have the orbital characteristics of
asteroids with $T_J > 3$ and lose mass during part of their
orbits. The cometary activity observed in these objects may be
driven by primordial water water ice that is trapped near the
surface and sublimates when it is exposed to sunlight.
Main-belt comets are important because they may have been able to
preserve water ice despite the effect of solar radiation and
heating from the decay of short-lived radioactive nuclei.  The
asteroids in the outer regions of the main belt can therefore
have a substantial fraction of water and other volatiles that
may have supplied the volatile content of terrestrial planets.
Most of the main-belt-comets are faint with very weak comae
that are active during part of their orbits. Given the
expected flux sensitivity of LSST, the transient cometary
activity of main-belt asteroids will be observable including
many objects that could be below the detection limits of
current photometric surveys.  The LSST observations will thus
help to understand the overlap between different populations
in the Solar System such as the relationship between comets
and asteroids.

% Now, describe their response to the observing strategy. Qualitatively,
% how will the science project be affected by the observing schedule and
% conditions? In broad terms, how would we expect the observing strategy
% to be optimized for this science?
%
% % --------------------------------------------------------------------
%
% \subsection{Metrics}
% \label{sec:\secname:metrics}
%
% Quantifying the response via MAF metrics: definition of the metrics,
% and any derived overall figure of merit.
%
%
% % --------------------------------------------------------------------
%
% \subsection{OpSim Analysis}
% \label{sec:\secname:analysis}
%
% OpSim analysis: how good would the default observing strategy be, at
% the time of writing for this science project?
%
%
% % --------------------------------------------------------------------
%
% \subsection{Discussion}
% \label{sec:\secname:discussion}
%
% Discussion: what risks have been identified? What suggestions could be
% made to improve this science project's figure of merit, and mitigate
% the identified risks?
%
% Different discussion / risks for each science case within this general metric?
%
% \navigationbar
%
% ====================================================================
%
% \section{Asteroid Light Curves and Rotation Periods}
% \def\secname{\chpname:lightcurves}\label{sec:\secname}

\subsection{Asteroid Light Curves and Rotation Periods}

Two Solar System science projects require a series of photometric
measurements. These are (1) measuring lightcurves and therefore shapes
of minor bodies and (2) measuring the colors and therefore compositions
of minor bodies. This section and the next describe the science and the
metrics for these experiments.

% % --------------------------------------------------------------------
%
% \subsection{Target measurements and discoveries}
% \label{sec:\secname:targets}

In general, minor bodies are aspherical, and therefore observations of
those bodies produce lightcurves with non-zero amplitudes. Constant
monitoring of such a body would reveal the detailed lightcurve, which
can be inverted to derive the effective observed shape at that epoch.
Observations over multiple epochs allow for observations at different
aspects, which can be used to determine the three dimensional shape and
pole orientation of the minor body. All of this information can be used
to understand, broadly, the orbital and physical evolution of minor
bodies in the Solar System.

LSST observations of minor bodies in the Solar System will not, however,
necessarily be dense in time (with the exception of observations made in
Deep Drilling Fields; see below). Therefore, lightcurves of minor bodies
must be combined across arbitrary rotational phase. Without knowing the
phase, the amplitude of the lightcurve (a proxy for asteroid shape) can
simply be determined. More complicated lightcurve inversion analysis xxx
ref xxx can be carried out, given a sufficient number of points.


% % --------------------------------------------------------------------
%
% \subsection{Metrics}
% \label{sec:\secname:metrics}

The metric for lightcurve analysis is therefore related to the number of
observations. The specific requirement is at least $\sim$100
measurements of an asteroid over $\sim$years, calibrated with a
photometric accuracy of $\sim$5\% (SNR=20) or better. In these cases, a
coarse shape model can be derived. The sparse data inversion gives
correct results for both fast (0.2--2~h) and slow ($>$24~h) rotators
\citep{2007IAUS..236..191D}.


% % --------------------------------------------------------------------
%
% \subsection{OpSim Analysis}
% \label{sec:\secname:analysis}

The present Baseline Cadence does pretty well for this project. Our
estimate from existing metrics is that some $10^5$~main belt asteroids
will be observed $>$500~times in the nominal survey.


% % --------------------------------------------------------------------
%
% \subsection{Discussion}
% \label{sec:\secname:discussion}

The success of this experiment is predicated on the idea that MOPS will
work as advertised, that is, that the linking of tracklets and tracks
will be successful, so that each object's photometric time series can be
identified.

% \navigationbar
%
% ====================================================================
%
% \section{Asteroid Colors}
% \def\secname{\chpname:colors}\label{sec:\secname}

\subsection{Asteroid Colors}

The varying compositions of asteroids result in a range of optical
colors. Sloan filters in general are sufficiently diagnostic to
discriminate among different compositional class xxx ref xxx. Therefore,
when a Solar System minor body is observed in griz (Solar System objects
are generally quite faint in u band and many fewer will be detected; Y
band xxx), the color can be used to determine the composition and,
downstream, composition as a function of asteroid size, family
membership, orbital elements, or many other parameters.

One obstacle to determining asteroid colors is that asteroid rotation
periods are on the order of 2--20~hours, so that after an initial
measurement all further measurements (in the same filter, or other
filters) are obtained at an arbitrary rotational phase. However, as
described above, asteroid lightcurves in a single band (presumably r
band, which will likely have the most detections) can be derived. Any
observing of that asteroid in a different filter can be corrected for
the derived lightcurve brightness offset at the time of the non-r band
observation. The intrinsic color of the asteroid can therefore be
measured.

% % --------------------------------------------------------------------
%
% \subsection{Target measurements and discoveries}
% \label{sec:\secname:targets}

% % --------------------------------------------------------------------
%
% \subsection{Metrics}
% \label{sec:\secname:metrics}

The metric of interest is the number of observations in each band. If an
asteroid has more than $\sim$100 observations in r band, then the
lightcurve can be determined, and the color can then be derived. The
fidelity of the color will depend on the number of measurements in each
band with SNR$>$20 (giving colors good to 5\%).

% % --------------------------------------------------------------------
%
% \subsection{OpSim Analysis}
% \label{sec:\secname:analysis}
%
% OpSim analysis: how good would the default observing strategy be, at
% the time of writing for this science project?
%
%
% % --------------------------------------------------------------------
%
% \subsection{Discussion}
% \label{sec:\secname:discussion}
%
% Discussion: what risks have been identified? What suggestions could be
% made to improve this science project's figure of merit, and mitigate
% the identified risks?

% \navigationbar
%
% ====================================================================
%
% other metrics? binary detection?
%
% ====================================================================
%
% \section{Deep Drilling Observations}
% \def\secname{\chpname:dd}\label{sec:\secname}

\subsection{Deep Drilling Observations}

Deep drilling observations provide the opportunity, via digital
shift-and-stack techniques, to discover Solar System Objects fainter
than the individual image limiting magnitude. These fainter objects
will be smaller, more distant, or lower albedo (or some combination of these)
than the general population found with individual images. Discovering smaller
objects is useful for constraining the size distribution to smaller
sizes; this provides constraints for collisional models and insights
into planetesimal formation. More distant objects are interesting in
terms of extending our understanding of each population over a wider
range of space; examples would be discovering very distant
Sedna-like objects or comets at larger distances from the Sun before
the onset of activity. Lower albedo objects may be useful to
understand the distribution of albedos, particularly to look for
trends with size.

Variations on the basic method of shift-and-stack have been used to
detect faint TNOs.
% XXX Allen, Bernstein, Gladman, Fuentes, ? XXX.
Computational limitations on these methods mean that, roughly and in
general for images taken at opposition, images taken over the timespan
of about an hour can be combined and searched for main belt asteroids,
and images taken over the timespan of about 3 days can be combined and
searched for more distant objects like TNOs.

With extragalactic deep drilling fields as in the baseline cadence,
where observations are taken in a series of filters (g, r, and i
would be useful for this purpose) each night, every three or four
days, we could use shift-and-stack to coadd the 50 images obtained in
gri bandpasses in a single night. This would allow detection of
objects about 2 magnitudes fainter than in the regular survey, or
approximately $r=26.5$.
% This is cool and a range of ecliptic
% latitudes is interesting.  But, we would like to do better.
%
% Recap solar system DD white paper.
%
% Describe how we will evaluate DD proposal, with TNO population +
% estimate on number of times objects observed (but not doing actual
% shift-and-stack). Need large-i populations to test if useful, probably.

% ====================================================================

\navigationbar
