% --------------------------------------------------------------------

\chapter[The Milky Way Galaxy]{The Milky Way Galaxy}
\def\chpname{galaxy}\label{chp:\chpname}

Chapter editors:
\credit{willclarkson},
\credit{akvivas}.

Contributing authors:
\credit{bethwillman},
\credit{dnidever},
\credit{ivezic},
\credit{ctslater},
\credit{pmmcgehee},
\credit{cbritt4},
\credit{dgmonet},
\credit{caprastro},
\credit{DanaCD},
\credit{jgizis},
\credit{mliu},
\credit{vpdebattista}
\credit{chomiuk}
\credit{yoachim}
% {\it and others to follow}

\section*{Summary}
\addcontentsline{toc}{section}{~~~~~~~~~Summary}

Galactic science cases fall roughly into two observational categories
based on stellar density and/or Galactic latitude. (i) Strategy
assessment for the {\it high-density or low-latitude} cases is dominated
by large variation in total time allocation for the inner Plane (where
most of the Galaxy's stars are found), since the current strategy
options tend to complete their inner-Plane observations within the first
$\sim 7$~months of the survey (see Chapter \ref{chp:cadexp}). Any
science case requiring more than a years' coverage in these regions will
not be well-served by most of the currently-run strategy simulations.
Quantitatively, figures of merit for these science cases therefore
suggest the baseline cadence to be factors $\sim 6-60$~worse than the
two comparison strategies chosen that allocate more time to the inner
Plane (Tables \ref{tab_SummaryMWDisk} \& \ref{tab_SummaryMWAstrometry}).
While figures of merit for a number of science cases in this category
have been specified and are in the implementation phase (Section
\ref{sec:MW_Disk:MW_Disk_metrics}), at present it is at least as
important that a wider range of strategies be specified and run {\it
with inner-Plane coverage spread over the full ten-year survey time
baseline.} We encourage the reader to supply suggestions for simulations
meeting this need. (ii). For the {\it any-latitude} category (including
the Solar Neighborhood), basic figures of merit including astrometric
characterization have been run for three choices of strategy (Section
\ref{sec:MW_Astrometry:MW_Astrometry_metrics}). While some degree of
calibration is ongoing, at present roughly $12\%$~of fields show
problematic degeneracy against differential chromatic refraction for all
three strategies. Effort is now needed to implement the science figures
of merit in Section \ref{sec:MW_Astrometry:MW_Astrometry_metrics} that
are based on these astrometry indications. Assessment of the science
strategy impact on the Halo (Section \ref{sec:MW_Halo}) largely rests on
the implementation of a robust star/galaxy separation metric into MAF.
This development is ongoing, led by one of the authors of this Chapter
(CTS).



\section{Introduction}

\def\secname{MW_Intro}\label{sec:\secname}

LSST should produce significant contributions to most areas
of Galactic astronomy. LSST Milky Way science cases cover lengthscales
ranging from a few pc (such as sensitive surveys of low-mass objects in
the Solar Neighborhood), up to many tens of kpc (such as surveys for
low-mass satellite galaxies of the Milky Way and their post-disruption
remnant streams, and beyond this, investigations of resolved stellar
populations in the Local Volume). In this chapter, we investigate a limited set of representative Galactic science cases, with the aims of demonstrating the scientific trade-offs of different observing strategies and of motivating readers to contribute to consderations of LSST's observing strategy.  The LSST Science Book
(particularly chapters 6 and 7) and \citet[][in particular Sections
2.1.4 and 4.4]{IvezicEtal2008} present a broader treatment of both science questions and science cases relevant to Galactic science.\footnote{We do however provide
motivating details for certain science cases in this Chapter,
particularly in \autoref{sec:MW_Disk}, as those cases are not
emphasized in the LSST Science Book or the relevant sections of
\citet{IvezicEtal2008}.}.

Concern about observations towards the inner Galactic Plane has for
several years been a common theme in feedback on LSST's observing
strategy, as the Baseline survey currently expends relatively little
observing time per field at low Galactic latitudes (~30 visits per filter, closely spaced in time).  With few visits per filter and a condensed time sampling, populations found in scientifically interesting numbers only at low
Galactic latitudes might be detected with low efficiency
by LSST under this Baseline strategy. (An example is the probing of the
mass function of moderate-separation extrasolar planets via intra-disk
planetary microlensing, as argued forcefully in \citealt{gould13}).

%At the time of
%writing, observations towards the inner Plane still seem to be
%compromised further by the way OpSim arranges shorter programs to
%completion at early times in the simulated survey. This reduces the time
%baseline available for inner-Plane measurements by a significant factor
%compared to observations away from the Plane.

We explore the scientific
impact of shallow Plane observations by comparing Metrics and Figures of
Merit evaluated for the Baseline cadence (\opsimdbref{db:baseCadence}),
for the PanSTARRS-like strategy (\opsimdbref{db:opstwoPS}) which has
essentially uniform depth at all Galactic latitudes observed, and for
the \opsimdbref{db:NormalGalacticPlane} strategy in which the plane is part of
the Wide-Fast-Deep survey. The evaluation of
Figures of Merit generated by these three observing strategies will
quantify how a range of Milky Way science cases could be
substantially improved by selecting a strategy with increase Plane
coverage, without significant cost to the rest of LSST's scientific
investigations.


\subsection{Chapter terminology and structure}

To tame the diversity of science cases, we have picked
representative cases and grouped them within broad scientific areas,
devoting one Section of this chapter to each grouping of cases. A small
number of Figures of Merit (FoMs) have been described for each case.
At present, science cases are grouped in the following way:
\autoref{sec:MW_Disk} assesses the impact of observing strategy on
LSST's ability to map some representative astrophysically important
populations that are found mostly or exclusively in the Plane.
% \autoref{sec:MW_SFH} discusses the use of LSST to probe star formation
% histories through mostly young populations (see also Section 5.6.).
% \autoref{sec:MW_Dust} discusses the impact of observing strategy on
% ISM constraints.
Several observational challenges for LSST find their
sharpest expression in Milky Way science, including (but not limited to)
measurements of stellar parallax, absolute astrometry, and proper
motions (including the tie-in to the reference frame which will be
provided by the \textit{gaia} mission). For this reason, specific issues
relating to precision astrometry are developed in
\autoref{sec:MW_Astrometry}.
\autoref{sec:MW_Halo} assesses the degree to which structures in the Milky Way's halo can be discriminated and mapped, using tracer populations distinguished by variability and/or derived stellar parameters.
Finally, \autoref{sec:MW_future}
presents descriptions of investigations that are needed to properly
determine LSST's utility for Milky Way science, but which
% at this date (late-April 2016)
are as yet relatively incompletely developed.

Summary Tables are provided that present the figure of merit for each
science case within a given section (one row per figure of merit)
evaluated for each tested observing strategy (one column per
strategy). This summary information appears in
\autoref{tab_SummaryMWDisk} (the Disk),
% \autoref{tab_SummaryMWDust} (the ISM),
% \autoref{tab_SummaryMWHalo} (the Halo), and
\autoref{tab_SummaryMWAstrometry} (Astrometry).


% Examples
% identified at this stage include the uses of LSST to set constraints on
% Galactic components (including the structure of the Bulge, and the
% impact of radial migration in the Disk) and the study of resolved
% stellar populations in the Local Group.

%\subsection{Summary tables for Figures of Merit}
%
%The tables below organize the comparison of Figures of Merit for all
%the science cases considered in this chapter:
%begin{itemize}
 % \item Mapping the Milky Way Halo: Table \ref{tab_SummaryMWHalo}
%  \item Mapping the Milky Way Disk: Table \ref{tab_SummaryMWDisk}
%    \item The ISM: Table \ref{tab_SummaryMWDust}
%  \item Astrometry with LSST: Table \ref{tab_SummaryMWAstrometry}
%\end{itemize}

%\subsection{Needed input}
%
%While many of the diagnostic Metrics are relatively well-developed,
%implementation is needed of Figures of Merit (FoMs) that depend on
%these metrics. In some Sections we have sketched out such figures of
%merit, in others the development of a practical FoM is still a topic
%of active development.

% ====================================================================
%+
% SECTION:
%    section-name.tex  % eg lenstimedelays.tex
%
% CHAPTER:
%    chapter.tex  % eg cosmology.tex
%
% ELEVATOR PITCH:
%    Explain in a few sentences what the relevant discovery or
%    measurement is going to be discussed, and what will be important
%    about it. This is for the browsing reader to get a quick feel
%    for what this section is about.
%
% COMMENTS:
%
%
% BUGS:
%
%
% AUTHORS:
%    Phil Marshall (@drphilmarshall)  - put your name and GitHub username here!
%-
% ====================================================================

\section{Mapping the Milky Way Disk}
\def\secname{MW_Disk}\label{sec:\secname} % For example, replace "keyword" with "lenstimedelays"

\noindent{\it Will Clarkson, Jay Strader, Chris Britt}  % (Writing team)

% This individual section will need to describe the particular
% discoveries and measurements that are being targeted in this section's
% science case. It will be helpful to think of a ``science case" as a
% ``science project" that the authors {\it actually plan to do}. Then,
% the sections can follow the tried and tested format of an observing
% proposal: a brief description of the investigation, with references,
% followed by a technical feasibility piece. This latter part will need
% to be quantified using the MAF framework, via a set of metrics that
% need to be computed for any given observing strategy to quantify its
% impact on the described science case. Ideally, these metrics would be
% combined in a well-motivated figure of merit. The section can conclude
% with a discussion of any risks that have been identified, and how
% these could be mitigated.

%A short preamble goes here. What's the context for this science
%project? Where does it fit in the big picture?

Many populations of great importance to Astronomy exist predominantly
in or near the Galactic Plane, and yet are sufficiently
sparsely-distributed (and/or faint enough) that LSST is likely to be
the only facility in the forseeable future that will be able to
identify a statistically meaningful sample. Some (such as the novae
that allow detailed study of the route to Type Ia Supernovae) offer
unique laboratories to study processes of fundamental importance to
astrophysics at all scales. Others (like intra-disk microlensing
events) offer the {\it only} probe of important populations. 

An important collateral benefit of studies in the plane with an
LSST-like facility, is improved mapping of the distribution and
observational effects of the ISM (particularly dust), which is of
importance to all IR/Optical/UV observational studies. \new{Due to its
  importance for studies of {\it all} Milky Way Astronomy, LSST's use
  to constrain the distribution of interstellar dust in three
  dimensions is presented in Section \ref{sec:MW_Dust}.}

% --------------------------------------------------------------------

\subsection{Target measurements and discoveries}
\label{sec:\secname:MW_Disk_targets}

%Describe the discoveries and measurements you want to make.

%Now, describe their response to the observing strategy. Qualitatively,
%how will the science project be affected by the observing schedule and
%conditions? In broad terms, how would we expect the observing strategy
%to be optimized for this science?

We have identified \new{four} science cases within the general area of Milky
Way Disk studies, that will have a diversity of dependencies on
observing strategy (e.g. slow intrinsic variability vs fast intrinsic
variability vs no variability). When the figures of merit have been
computed for these science cases, the results will be summarized in a Table in Section~\ref{sec:\secname:MW_Disk_discussion}.

\begin{itemize}
  \item 1. Quantifying the large quiescent compact binary population via variability;
  \item 2. New insights into the behavior of Novae and the route to Type Ia Superovae;
  \item 3. The next Galactic Supernova;
  \item 4. Measuring population parameters of planets outside the Snow Line with Microlensing;
  %\item 5. A three-dimensional Dust map and improvements in the reddening law
\end{itemize}

\new{Below we provide more detail on these science cases, including
  qualitative discussion of the expected impact of the choice of
  observing strategy, as these science cases are not discussed in
  detail elsewhere (in the LSST Science Book or the Ivezic et al. 2008
  summary paper).}

{\bf 1. Probing quiescent compact binaries via variability:} Of the
millions of stellar-mass black holes formed through the collapse of
massive stars over the lifetime of the Milky Way, only $\sim 20$ have
been dynamically confirmed through spectroscopic measurements
\citep[e.g.,][]{2015arXiv151008869C}.  Many questions central to modern
astrophysics can only be answered by enlarging this sample: which
stars produce neutron stars and which black holes; whether there is a
true gap in mass between neutron stars and black holes; whether
supernova explosions result in large black hole kicks. 

There is expected to be a large population of black hole binaries in quiescence
with low X-ray luminosities from $\sim 10^{30}$--$10^{33}$ erg/s.
Such systems can be identified as optical variables that show unique,
double-humped ellipsoidal variations of typical amplitude $\sim 0.2$
mag due to the tidal deformation of the secondary star, which can be a
giant or main sequence star. In some cases analysis of the light curve
alone can point to a high mass ratio between the components,
suggesting a black hole primary; in other cases the accretion disk
will make a large contribution to the optical light which results in
intrinsic, random, and fast variations in the light curve. The disk
contribution to optical light can change over time, and several years
of data is necessary to properly subtract the accretion disk
contribution in order to properly fit ellipsoidal veriations
\citep{2010ApJ...710.1127C}.
The brighter sources will be amenable to spectroscopy
with the current generation of 4-m to 10-m telescopes to dynamically
confirm new black holes; spectroscopy of all candidates should be
possible with the forthcoming generation of large telescopes. Thus,
LSST would trigger a rich variety of observational investigations of
the accretion/outflow process through studies of this large, dark
population.

While we have focused above on black hole binaries, we note that LSST would be
crucial for investigations of neutron star and white dwarf binaries. For
example, the total number of compact binaries is presently poorly
understood---Population models of neutron star X-ray binaries diverge by orders
of magnitude, largely due to uncertainties in the common envelope phase of
binary evolution
\citep[e.g.,][]{2003ApJ...597.1036P,2006MNRAS.369.1152K,2015A&A...579A..33V}.
This is poorly constrained but has a large impact on, for example,
LIGO event rates. A simple test case of common envelope evolution is available
in the number of dwarf novae (DNe) (accretion disk instability outbursts around
white dwarfs), a population that does not suffer from some of the complicating
factors that neutron star and black hole binaries do (e.g. supernova kicks).
Theoretical estimates routinely yield a significantly higher number of DNe than
are observed in the solar neighborhood. Understanding the true specific
frequency of these systems provides a key check on common envelope evolution.
LSST will detect dwarf novae, which last at least several days with typical
amplitudes of 4--6 mag, out to kpc scales. This will allow a test of not only
the number of cataclysmic variables, but also of the 3D distribution within the
Galaxy and dependence on metallicity gradients \citep{2015MNRAS.448.3455B}.

{\it Response to observing strategy:} Since most black hole candidates
have been identified near the plane in the inner Milky Way (68\%, 92\%
within $5^{\circ}, 10^{\circ}$~of the Plane), this science case {\it
    requires} that LSST observe the plane with sufficient cadence to
  detect the $\sim$hundreds of quiescent black-hole binaries by virtue
  of their variability. The natural choice for a survey for
  low-luminosity black hole binaries would be to extend the
  Wide-Fast-Deep survey throughout the Plane in the direction of the
  inner Milky Way. The orbital period of these systems is short (typically $<1$ day), so that a rolling cadence
  for at least parts of the Plane should be considered. For dwarf novae, the cadence of observations
  is critical in obtaining an accurate measure of the population of
  cataclysmic variables, as a long baseline is necessary to recover
  low duty cycle systems while widely-space observations
 would miss short outbursts.

%Describe the discoveries and measurements you want to make.

%Now, describe their response to the observing strategy. Qualitatively,
%how will the science project be affected by the observing schedule and
%conditions? In broad terms, how would we expect the observing strategy
%to be optimized for this science?

{\bf 2. Novae and the route to Type Ia Supernovae:} Only $\sim 15$
novae (explosions on the surfaces of white dwarfs) are discovered in
the Milky Way each year, while observations of external galaxies show
that the rate should be a factor of $\sim 3$ higher
\citep{2014ASPC..490...77S}.
Evidently, we are missing 50--75\% of novae due to their
location in crowded, extinguished regions, where they are not bright
enough to be discovered at the magnitude limits of existing transient
surveys. Fundamental facts about novae are unknown: how much mass is
ejected in typical explosions; whether white dwarfs undergoing novae
typically gain or lose mass; whether the binary companion is important
in shaping the observed properties of nova explosions. Novae can serve
as scaled-down models of supernova explosions that can be tested in
detail, e.g., in the interaction of the explosion with circumstellar
material \citep[e.g.,][]{2015arXiv151007662C}.  Further, since accreting white
dwarfs are prime candidates as progenitors of Type Ia supernovae, only
detailed study of novae can reveal whether particular systems are
increasing toward the Chandrasekhar mass as necessary in this
scenario.

{\it Response to observing strategy:} Most novae occur in the Galactic
Plane and Bulge, and therefore the inclusion of the Plane in a survey
of sufficient cadence to find these events promptly is of paramount
importance for this science. These events will trigger
multi-wavelength follow-up ranging from the radio to X-ray and
$\gamma$-rays; these data are necessary for accurate measurements of
the ejected mass.

{\bf 3. The First Galactic Supernova:} A supernova in the Milky Way
would be among the most important astronomical events of our lifetime,
with enormous impacts on stellar astrophysics, compact objects,
nucleosynthesis, and neutrino and gravitational wave astronomy. The
estimated rate of supernovae (both core-collapse and Type Ia) in the
Milky Way is about 1 per 20--25 years \citep{2013ApJ...778..164A}; hence there
is a 40--50\% chance that this would occur during the 10-year LSST
survey. If fortunate, such an event will be located relatively close
to the Sun and will be an easily observed (perhaps even naked-eye)
event. However, we must be cognizant of the likelihood that the
supernova could go off in the mid-Plane close to the Galactic Center
or on the other side of the Milky Way---both regions covered by
LSST. While any core-collapse event will produce a substantial
neutrino flux, alerting us to its existence, such observations will
not offer precise spatial localization. The models of Adams et
al.~(2013) indicate that LSST is the \emph{only} planned facility that
can offer an optical transient alert of nearly all Galactic
supernovae. 

{\it Response to observing strategy:} Even if the supernova is not too
faint, LSST will likely be the sole facility with synoptic
observations preceding the explosion, providing essential photometric
data leading up to the event---but only if LSST covers the Plane at a
frequent cadence. Just {\it how} frequent is open to exploration at
present, but the prospect of high-sensitivity observations of the
location of such a supernova {\it before} it takes place are clearly
of enormous scientific value. A secondary issue is the prospect that
an easily-observed Milky Way supernova might be too bright for LSST to
measure precisely with its planned exposure time, with a roughly 82\%
chance of a core-collapse supernova reaching one or two magnitudes
brighter than LSST's nominal saturation limit (with a 1/3 chance that
a ccSN would reach $m_V \sim 5$~(Adams et al. 2013). For a Type Ia in
the Milky Way, 
\citet{2013ApJ...778..164A} estimate $m_{V, max} \lesssim 13.5$~in 92\% of
cases.

{\bf 4. Population parameters of planets beyond the Snow Line with Microlensing:} Gould
(2013) shows that LSST could be an effective intra-disk microlensing
survey (in which disk stars are lensed by other objects in the disk,
such as exoplanets, brown dwarfs, or compact objects). The lower
stellar density compared to past bulge-focused microlensing surveys
would be offset by the larger area covered by LSST. The predicted rate
of high magnification microlensing events that are very sensitive to
planets would be $\sim 25$ per year. This survey would be able to
detect planets at moderate distances from their host stars, a regime
poorly probed by standard Doppler and transit techniques. The LSST
data alone would not be sufficient: the detection of a slow ($\sim$
days) timescale increase in brightness of a disk star would need to
trigger intensive photometric observations from small (1-m to 2-m
class) telescopes that would observe at high cadence for the 1--2
months of the microlensing event. This would represent an excellent
synergy between LSST and the wider observing community, and would
directly take advantage of the capabilities unique to LSST.

{\it Response to observing strategy:} To catch lensing events as they
start to brighten, with sufficient fidelity to trigger the intensive
follow-up required, the models of Gould (2013) suggest each field
should be observed once every few nights. With sparser coverage,
the survey would lose sensitivity to microlensing events in progress.

% --------------------------------------------------------------------

\subsection{Figures of Merit}
\label{sec:\secname:MW_Disk_metrics}

%Quantifying the response via MAF metrics: definition of the metrics,
%and any derived overall figure of merit.

\new{Here we describe the Figures of Merit (FoMs) we intend to
  implement and evaluate for the candidate observing strategies of
  interest. Where these FoM have already been evaluated, we provide
  their evaluation in Table \ref{tab_SummarMWDisk}. Those FoM:}

\begin{itemize}
  \item FoM 1.1 - Fraction of quiescent black hole binaries detectable through ellipsoidal variability;
    \item FoM 1.2 - Uncertainty on the recurrence time distribution of Dwarf Novae;
      \item FoM 2.1 - Uncertainty in the number of Novae detected over the LSST survey interval;
        \item FoM 3.1 - Fraction of Galactic supernovae for which LSST would detect variability {\it before} the main Supernova event.
          \item FoM 4.1 - Fraction of accurately-triggered Microlens candidates
            \item FoM 4.2 - Uncertainty in the mass function of intra-disk microlensed planets;
\end{itemize}

\new{In these FoMs, ``uncertainty'' can be taken to mean both random
  and systematic uncertainty, likely recorded as separate numbers for
  each FoM. We anticipate determining the FoMs that record population
  parameter-uncertainty in a Monte Carlo sense. This is particularly
  relevant for FoMs in which the event rate per pointing may be low
  ($\lesssim 1$~event per pointing per decade) but not so low that
  only 1-few events are expected over the whole sky over the lifetime
  of the survey (as is the case for FoM 3.1, the First Galactic
  Supernova). In very rare-event cases, the FoM can scale with the
  stellar density and the recovery fraction of that particular
  transient, and need only be evaluated once for the entire survey.}

\new{Since (at the time of writing) evaluating a Metric with relaxed
  SQL constraints typically takes half an hour or more, we do not
  expect to re-evaluate the metric for each trial in a Monte Carlo
  evaluation. Instead, the best strategy seems to be to evaluate the
  run of a particular metric against a parameter of interest (apparent
  magnitude, say, which is also expected to lead to a turnover in the
  importance of confusion error), and the investigator's preferred
  Monte Carlo framework for their population of interest can
  interpolate the stored Metric values at the time of trial-population
  generation. At the present date (2016-04-26) we have begun
  investigating the use of these ``Vector Metrics'' for Figures of
  Merit. We describe the anticipated FoMs in these cases below.}

\new{WIC 2016-04-26 7pm EDT: more to follow a little later this evening!}


{\bf FoM 1.1 - Fraction of quiescent
  black hole binaries detectable through ellipsoidal variability, as a function of location on sky and distance.}
Dependencies:
\begin{itemize}
  \item Monte Carlo in period, phase and shape parameters (ASCII input?) for variables as measured in a particular OpSim run. Likely run Monte Carlo for a representative number (ten?) of well-chosen orbital periods within the 0.1-5d range;
  \item (Since these are short-period objects): the ``PeriodicMetric'' of Lund et al. (2015);
  \item Will likely need reasonably high-spatial-resolution HEALPIX slices and a prescription for population density as a function of position on-sky (can be analytic).
\end{itemize}
Possible higher-order FoM: errors on the population size (mass
function??) derived from a survey under a given observing
strategy. Can imagine just adding up the ``recovered'' qLMXB
population and comparing it to that simulated. Some white noise componant of varying strengths could be
added to the light curves to simulate various contributions of the accretion disk to the continuum light. 
Note that the survey will necessarily be highly incomplete (inclination effects, etc.), it
is the likely {\it uncertainty} on the completeness-correction that
would be crucial in this case. 

{\bf FoM 1.2 - Fraction of dwarf novae detected in a given survey
  run as a function of location and distance.}
Dependencies:
\begin{itemize}
  \item Monte Carlo in distribution of maximum brightness and rise/decay timescale;
    \item "Triples" without filter constraints (given a prior detection in each filter)- what fraction are recovered?
    \item Histogram of duty cycle recovery efficiency versus duration of outburst and recurrence time.
    \item Histogram of recovery efficiency of maximum brightness of dwarf nova versus duration and recurrence time (if assume subsequent outbursts have similar profiles).
    \end{itemize}
Higher-level FoM: Uncertainty in LIGO event rates due to
uncertainties in common envelope evolution, which drives uncertainties in both LIGO event rates and DN population.

{\bf FoM 2.1 - Fraction of identified Novae as a function of location on-sky, distance, and time since initial rise.}
Dependencies:
\begin{itemize}
  \item Is the ``Triplets'' metric sufficient (i.e. is this ``just'' a case of supplying the metric the correct $\Delta t$~parameter values)?;
    \item What is the maximum interval since initial rise that would be acceptable? (Is this a function of waveband for followup?)
\end{itemize}
Possible higher-order FoM: error on inferred rate of Type Ia supernovae?

%{\bf FoM 3.1 - Maximum time-interval {\it before} triggering of a SN in the Milky Way that LSST would h%ave taken precursor data.}
%Dependencies:
%\begin{itemize}
%  \item This FoM would probably be very easy to calculate (just estimate the mean time between observat%ions). However the acceptable limits still need thought:
%  \item How many colors are sufficient? Any observations at all before the SN goes off, or would a comp%lete set in all filters be needed to characterize the candidate progenitor?
%    \item What level of variability sensitivity is really needed? Would just an extremely deep image of% the SN field before the event be sufficient?
%\end{itemize}
%{\bf FoM 3.2 - Maximum time-interval {\it after} the SN event for triggering followup.}
%Dependencies:
%\begin{itemize}
%  \item Similar to FoM 3.1. 
%    \item Is it important to know the discovery space for LSST? If a
%      supernova at $m_V~15$~goes off, other facilities are likely to
%      spot it...
%\end{itemize}


\new{\bf FoM 3.1 - Fraction of Galactic supernovae for which LSST would detect variability {\it before} the main Supernova event. Dependencies:}
\begin{itemize}
  \item \new{Assume the probability of a SN going off scales with the number of stars along the line of sight to some depth. The ``starcount'' metric in sims-maf-contrib (by Mike Lund and collaborators) calculates this value out to some fiducial distance. Additionally}; 
\item \new{This FoM requires an assessment of the ability of LSST to detect a particular lightcurve shape. Several options are possible at this time; we have selected ``metrics.TransientMetric'' from the baseline lsst-sims. Therefore also};
\item \new{a lightcurve shape for the pre-SN variability must be assumed.}
\end{itemize}

% WIC - removed this one since it's a bit low-level...
%
%{\bf FoM 3.2 - Maximum time-interval {\it after} the SN event for triggering followup.}
%Dependencies:
%\begin{itemize}
%  \item Similar to FoM 3.1. 
%    \item Is it important to know the discovery space for LSST? If a
%      supernova at $m_V~15$~goes off, other facilities are likely to
%      spot it...
%\end{itemize}



{\bf FoM 4.1 - Fraction of accurately-triggered Microlens candidates as a function of location, distance, lens mass.}
Dependencies:
\begin{itemize}
  \item Analytic (?) model for microlens lightcurve for representative sample of microlens parameters;
    \item transientASCIIMetric may well already do what we need! 
\end{itemize}

{\bf FoM 4.2 - errors in the (mass function, distance distribution) of intra-disk microlensed planets.}
Dependencies:
\begin{itemize}
  \item Event rate with sufficient-cadence observations to trigger followup as a function of (distance, lens mass);
\end{itemize}

%{\bf FoM 5.1 - Errors in derived $E(B-V)$, $n_H$~as a function of
%  location in the Plane.}
%Dependencies:
%\begin{itemize}
%  \item SNR scaling with apparent magnitude
%    \item For the M dwarf based technique, the relation between
%      reddening-invariant index $[Q_{gri}]$~and intrinsic $g-i$~are
%      expressed as polynomials, so expect non-linear relation with
%      photometric error.  
%      \item This uses a 5th-order polynomial to describe the
%        $(Q_{gri}, g-i)$~stellar locus for M dwarfs $(g-i > 1.6)$.
%        \item Care must be taken to correctly propagate errors through
%          the various indices used - not trivial with so many choices
%          of flux ratio used.
%          \item Uncertainties in the parameterizations used for e.g. color-$M_V$~relationships.
%            \item The above are all for every location probed on the map.
%\end{itemize}

% --------------------------------------------------------------------

\subsection{OpSim Analysis}
\label{sec:\secname:MW_Disk_analysis}

\new{(To stimulate input from collaborators, we have implemented a mock-up of a simple figure of merit for the Galactic Supernova case. We choose FoM 3.1 - the fraction of galactic supernovae for which an SN2010mc-like pre-SN outburst could be discovered by LSST. The results of all FoM's for this Section are collected in Table \ref{tab_SummarMWDisk}.)}

\new{{\bf FoM 3.1: Fraction of Galactic supernovae for which LSST would detect variability {\it before} the main Supernova brightening:}}
\begin{itemize}
\item \new{{\it Definition:} 
\begin{equation}
  FoM_{preSN} \equiv \frac{ \sum^{sightlines}_{i} f_{var, i} N_{\ast, i} } {\sum^{sightlines}_{i} N_{\ast, i}}
\end{equation}
where $f_{var, i}$~is the fraction of transient events that LSST would detect for observing strategy including the $i$'th sightline, $N_{\ast,i}$~the number of stars present along the $i$'th sightline, and the FoM is normalized by the total number of stars returned by the density model over all sightlines. (For the two OpSim runs tested here, \opsimdbref{db:baseCadence} and \opsimdbref{db:opstwoPS}, the normalization factors differ by $\sim 2\%$.) Therefore $0.0 \le FoM_{preSN} \le 1.0$.}
\item \new{{\it Assumptions (observational):} Pre-SN variability similar to the
    pre-SN outburst of SN2010mc \citep{2013Natur.494...65O}. The pre-SN variability is
modeled as a sawtooth lightcurve (in apparent magnitude). We assume this
transient event will always reach brightness sufficient for LSST to observe, so
opt for a very bright peak apparent magnitude in all filters. We assume that
the probability of a supernova going off is proportional to the number of stars
along a particular line of sight.}
\item \new{{\it Parameters used:} The pre-SN outburst is simulated using the lsst-sims metrics.TransientMetric module. The lightcurve used has the following parameters: rise slope $-2.4$; time to peak; $20$~days; decline slope: $0.08$; total transient duration: 80 days. All filters are used in the detections, and 20 evenly-spaced phases are simulated for sensitivity to pathological cases (parameter nPhaseCheck=20). Peak apparent magnitudes used: $\{ 11,9,8,7,6,6\}$~in $\{u,g,r,i,z,y\}$. Then, $f_{var, i}$~is taken as the ``Sawtooth Alert'' quantity returned by metrics.TransientMetric. The starcounts metric (in sims-maf-contrib) is used to estimate the number of stars along a given line of sight, using fiducial distance limits ($10$pc $\le d \le 80$kpc).}
\item {\it \bf Results:} $FoM_{preSN}$(\opsimdbref{db:baseCadence}) = 0.129, while $FoM_{preSN}$(\opsimdbref{db:opstwoPS})=0.826.\footnote{\new{2016-04-25 For comparison, when run on 2015-era OpSim runs {\tt enigma\_1189} (Baseline strategy) and {\tt ops2\_1092} (PanSTARRS-like strategy) the results were 0.251 (Baseline) and 0.852 (PanSTARRS-like strategy). So the 2016-era OpSim runs show a sharper disadvantage than before to the Baseline cadence for the Galactic Supernova case.}} See Figure \ref{f_opSim_GalacticSN} for a breakdown of this figure of merit across sightlines.

\end{itemize}
\begin{figure}
\begin{center}
  \includegraphics[width=7cm]{./figs/milkyway/galacticSN_SkyMap_Baseline.png}
  \includegraphics[width=7cm]{./figs/milkyway/galacticSN_SkyMap_PanSTARRS.png}
%  \includegraphics[width=6cm]{./figs/milkyway/galacticSN_Histogram_1092.pdf}
%  \includegraphics[width=6cm]{./figs/milkyway/galacticSN_Histogram_1189.pdf} 
  \caption{Figure of merit $FoM_{preSN}$~describing LSST's sensitivity to any pre-Supernova outburst, broken down by sightline, as sky-maps. $FoM_{preSN}$~is estimated for two OpSim runs (to-date); \opsimdbref{db:baseCadence} (left; Baseline cadence) and \opsimdbref{db:opstwoPS} (right; PanSTARRS-like strategy). The normalizing factors $N_{\ast, total}$ are $3.692\times 10^{10}$~for \opsimdbref{db:opstwoPS} and $3.793 \times 10^{10}$~for \opsimdbref{db:baseCadence}. The imprint of reduced sampling towards the inner plane can be clearly seen for \opsimdbref{db:baseCadence}. Notice the difference in scale between the left and right panels. See Section \ref{sec:MW_Disk:MW_Disk_analysis}}
\end{center}
\label{f_opSim_GalacticSN}
\end{figure}


% The Figures of Merit listed above must now be implemented and applied to the OpSim databases.

%The metrics listed above should be carefully compared between our proposed run and the baseline cadence.


% --------------------------------------------------------------------

\subsection{Discussion: required work}
\label{sec:\secname:MW_Disk_discussion}

The Figures of Merit listed above must now be implemented within the
sims\_maf framework and applied to representative science
cases. \new{See Table \ref{tab_SummaryMWDisk} at the end of this subsection for initial
  efforts along these lines.}

We welcome input and volunteers for this effort. 

Qualitatively, however, we can note immediately that the current
baseline cadence (\opsimdbref{db:baseCadence}) partially excludes the
Galactic Plane from the deep-wide-fast survey and instead adopts a
nominal 30 visits per filter as part of a special proposal - which
also tends to cluster the visits in the inner Plane within the first
few years of the survey. This already seriously compromises the time
baseline (see figure 4.3 of Section
\ref{sec:MW_Astrometry:MW_Astrometry_OpSim} for a demonstration applied to
proper motions).

We have proposed an OpSim run that includes the Galactic Plane in the
deep-wide-fast survey:

\url{https://github.com/LSSTScienceCollaborations/ObservingStrategy/blob/master/opsim/Proposal_GP.md}

\begin{table}
  \begin{tabular}{l|p{6cm}|c|c|c|c|p{5cm}}
    FoM & Brief description & {\rotatebox{90}{\opsimdbref{db:baseCadence}}} & {\rotatebox{90}{\opsimdbref{db:opstwoPS}}} & {\rotatebox{90}{future run 1}} &  {\rotatebox{90}{future run 2}} & Notes \\
    \hline
    1.1 & \footnotesize{LMXB ellipsoidal variations}      & - & - & - & - & - \\
    1.2 & \footnotesize{Uncertainty in dwarf nova duty cycle}   & - & - & - & - &  \footnotesize{LSST as initial trigger} \\
    2.1 & \footnotesize{Fraction of Novae detected}       & - & - & - & - &  - \\
    3.1 & \footnotesize{Galactic Supernova pre-variability} & 0.13 & {\bf 0.83} & - & - & \footnotesize{Fraction of SN2010mc-like outbursts that LSST would detect; $FoM_{preSN} = f_{var} \times N_{\ast}$} \\
    4.1 & \footnotesize{Fraction of triggered microlens candidates} & - & - & - & - & - \\
    4.2 & \footnotesize{Uncertainty in disk-disk microlens distribution parameters due to missed events} & - & - & - & - & \footnotesize{LSST as initial microlens trigger} \\
%    5.1a & \footnotesize{Median (over sight-lines) of the uncertainty in $E(B-V)$} & - & - & - & - & \footnotesize{(Most useful FoM probably a spatial map of the uncertainty.)} \\
%    5.1b & \footnotesize{Variance (over sight-lines) of the uncertainty in $E(B-V)$} & - & - & - & - & - \\
  \end{tabular}
\caption{Summary of figures-of-merit for the Galactic Disk science cases. The best value of each FoM is indicated in bold. Runs \opsimdbref{db:baseCadence} and \opsimdbref{db:opstwoPS} refer to the Baseline and PanSTARRS-like strategies, respectively. See Section \ref{sec:MW_Disk}. }
\label{tab_SummaryMWDisk}
\end{table}


%Discussion: what risks have been identified? What suggestions could be
%made to improve this science project's figure of merit, and mitigate
%the identified risks?



% ====================================================================

\navigationbar


% PJM: Perry's content is to be merged with Pat's, in Section 5.6
% % ====================================================================
%+
% SECTION:
%    section-name.tex  % eg lenstimedelays.tex
%
% CHAPTER:
%    chapter.tex  % eg cosmology.tex
%
% ELEVATOR PITCH:
%    Explain in a few sentences what the relevant discovery or
%    measurement is going to be discussed, and what will be important
%    about it. This is for the browsing reader to get a quick feel
%    for what this section is about.
%
% COMMENTS:
%
%
% BUGS:
%
%
% AUTHORS:
%    Phil Marshall (@drphilmarshall)  - put your name and GitHub username here!
%-
% ====================================================================

\section{Star Formation History of the Milky Way}
\def\secname{MW_SFH}\label{sec:\secname} % For example, replace "keyword" with "lenstimedelays"

\noindent{\it Peregrine M. McGehee} % (Writing team)

% This individual section will need to describe the particular
% discoveries and measurements that are being targeted in this section's
% science case. It will be helpful to think of a ``science case" as a
% ``science project" that the authors {\it actually plan to do}. Then,
% the sections can follow the tried and tested format of an observing
% proposal: a brief description of the investigation, with references,
% followed by a technical feasibility piece. This latter part will need
% to be quantified using the MAF framework, via a set of metrics that
% need to be computed for any given observing strategy to quantify its
% impact on the described science case. Ideally, these metrics would be
% combined in a well-motivated figure of merit. The section can conclude
% with a discussion of any risks that have been identified, and how
% these could be mitigated.

Summary: this important topic does not seem to have been developed in
previous versions of the LSST science book or Ivezic et
al. (2008). LSST gives the opportunity to survey extensive areas
around star formation regions in the Southern hemisphere. Among
others, it would allow to study the Initial Mass Function down to the
sub-stellar limit across different environments. Young stars are
efficiently identified by their variability.

A short preamble goes here. What's the context for this science
project? Where does it fit in the big picture?

% --------------------------------------------------------------------

\subsection{Target measurements and discoveries}
\label{sec:\secname:targets}


Variability is one of the distinguishing features of pre-main sequence stars and can result from a
diverse collection of physical phenomena including rotational modulation of large starspots due to
kiloGauss magnetic fields, hot spots formed by the impact of accretion streams onto the stellar
photosphere, variations in the mass accretion rate, thermal emission from the circumstellar disk,
and changes in the line of sight extinction. These physical processes generate irregular variability
across the entire LSST wavelength range (320–1040 nm) with amplitudes of tenths to several
magnitudes on timescales ranging from minutes to years and will be detectable by LSST.

Due to its sensitivity and anticipated ten-year operations lifetime, LSST will also address the issue
of the eruptive variability found in a rare class of young stellar objects - the FUor and EXor stars.
FUor and EXor variables are named after the prototypes FU Orionis (Hartmann & Kenyon 1996)
and EX Lupi (Herbig et al. 2001) respectively. These stars exhibit outburst behavior characterized
by an up to 6 magnitude increase in optical brightness, with high states persisting from several years
to many decades. Both classes of objects are interpreted as pre-main sequence stars undergoing
significantly increased mass accretion rate possibly due to instabilities in the circumstellar accretion
disk. The mass accretion rates during eruption have been observed to increase by 3 to 4 orders
of magnitude over the $\sim 10^{-9}$
to $10^{-7} M_{\odot}$ per year typical of Classical T Tauri stars. Whether
FUor/EXor eruptions are indeed the signature of an evolutionary phase in all young stars and
whether these outbursts share common mechanisms and differ only in scale is still an open issue.

To date only about 10 FUors, whose eruptions last for decades, having been observed to transition
into outburst (Aspin et al. 2009) with the last major outburst being that of V1057 Cyg (Herbig
1977). Repeat outbursts of several EXors have been studied, including those of EX Lupi (Herbig
et al. 2001) and V1647 Ori (Aspin et al. 2009), the latter erupting in 1966, 2003, and 2008. The
outbursts of EXors only persist for several months to roughly a year in contrast those of FUors,
which may last for decades: for example, the prototype FU Ori has been in a high state for over
70 years. These eruptions can occur very early in the evolution of a protostar as shown by the
detection of EXor outbursts from a deeply embedded Class I protostar in the Serpens star formation
region (Hodapp et al. 1996). The observed rarity of the FUor/EXor phenomenon may be due to
the combination of both the relatively brief (less than 1 Myr) duration of the pre-T Tauri stage
and the high line of sight extinction to these embedded objects hampering observation at optical
and near-IR wavelengths.

V1647 Ori is a well-studied EXor found in the Orion star formation region (m − M = 8) and
thus is a suitable case study for discussion of LSST observations. 
The
inferred extinction is $A_r \sim 11$ magnitudes which coupled with the observed r range of 23 to nearly
17 during outburst (McGehee et al. 2004) suggest that $M_r$ varies from 4 to −2 magnitudes.
The LSST single visit 5$\sigma$ depth for point sources is r $\sim$ 24.7, thus analogs of V1647 Ori will
be detectable in the r band during quiescence to (m − M) + $A_r$ = 20.5 and at maximum light
to (m − M) + $A_r$ = 26.5. The corresponding distance limits are 800 pc to 12 kpc assuming
$A_r$ = 11. For objects at the distance of Orion the extinction limits for LSST r-band detections of
a V1647 Ori analog are $A_r$ = 12.5 and $A_r$ = 17.5. These are conservative limits as V1647 Ori was
several magnitudes brighter at longer wavelengths (iz bands) during both outburst and quiescence
indicating that the LSST observations in izy will be even more sensitive to embedded FUor/EXor
stars.

LSST will increase the sample size for detailed follow-up observations due its ability to survey
star formations at large heliocentric distances and to detect variability in embedded and highly
extincted young objects that would otherwise be missed in shallower surveys. During its operations
LSST will also provide statistics on the durations of high states, at least for the shorter duration
EXor variables.

%Describe the discoveries and measurements you want to make.

%Now, describe their response to the observing strategy. Qualitatively,
%how will the science project be affected by the observing schedule and
%conditions? In broad terms, how would we expect the observing strategy
%to be optimized for this science?


% --------------------------------------------------------------------

\subsection{Metrics}
\label{sec:\secname:metrics}

Quantifying the response via MAF metrics: definition of the metrics,
and any derived overall figure of merit.


In order to assess the ability of LSST to 1) identify and 2) classify
YSO we need to quantify the variability timescales and amplitudes of
both Class I/II (stars with disks) and Class III (WTTS). Inclusion of
eruptive variables (FUor/Exor) is appropriate as well.

In brief, WTTS are quasi-periodic with amplitudes of 0.1 to 0.3 mag
and periods 1 to $\sim$15 days - so comparable to gamma Dor stars (see
Figure 8.17 in the SB). Given the temporal evolution of cool spots, a
period recovery analysis such as shown for RRL stars (se Figure 8.20
in the SB) is likely difficult. The embedded systems and CTTS are
irregular variables but shown have distinctive colors due to
extinction + UB/blue excess arising from to accretion shocks.

% --------------------------------------------------------------------

\subsection{OpSim Analysis}
\label{sec:\secname:analysis}

OpSim analysis: how good would the default observing strategy be, at
the time of writing for this science project?


% --------------------------------------------------------------------

\subsection{Discussion}
\label{sec:\secname:discussion}

Discussion: what risks have been identified? What suggestions could be
made to improve this science project's figure of merit, and mitigate
the identified risks?


% ====================================================================

\navigationbar


% PJM: moved the following to FutureWork, while the metric(s) is/are being implemented
% % ====================================================================
%+
% SECTION:
%    section-name.tex  % eg lenstimedelays.tex
%
% CHAPTER:
%    chapter.tex  % eg cosmology.tex
%
% ELEVATOR PITCH:
%    Explain in a few sentences what the relevant discovery or
%    measurement is going to be discussed, and what will be important
%    about it. This is for the browsing reader to get a quick feel
%    for what this section is about.
%
% COMMENTS:
%
%
% BUGS:
%
%
% AUTHORS:
%    Phil Marshall (@drphilmarshall)  - put your name and GitHub username here!
%-
% ====================================================================

\section{Dust in the Milky Way}
\def\secname{MW_Dust}\label{sec:\secname} % For example, replace "keyword" with "lenstimedelays"

\noindent{\it Peregrine M. McGehee} % (Writing team)

% This individual section will need to describe the particular
% discoveries and measurements that are being targeted in this section's
% science case. It will be helpful to think of a ``science case" as a
% ``science project" that the authors {\it actually plan to do}. Then,
% the sections can follow the tried and tested format of an observing
% proposal: a brief description of the investigation, with references,
% followed by a technical feasibility piece. This latter part will need
% to be quantified using the MAF framework, via a set of metrics that
% need to be computed for any given observing strategy to quantify its
% impact on the described science case. Ideally, these metrics would be
% combined in a well-motivated figure of merit. The section can conclude
% with a discussion of any risks that have been identified, and how
% these could be mitigated.

Interstellar dust is a significant constituent of the Galaxy. Its composition and associated extinction
properties tell us about the material and environments in which stars and their planets are formed.
Dust also presents an obstacle for a wide-range of astronomical observations, causing light from
stars in the plane of the Milky Way to be severely dimmed and causing the apparent colors of
objects observed in any direction to be shifted from their intrinsic values. These color shifts
are dependent upon the dust column density along the line of sight and the radiative transport
properties of the dust grains.

To first order, i.e. neglecting the effects of heterochromatic extinction, the absorption of light
in each band due to dust is dependent upon the column density, related to $E(B-V)$, and the
nature of the dust grains, as parameterized by the ratio of general to selection extinction 
in the Johnson $B$ and $V$ bands, defined as $R_V = A_V /E(B − V)$. 
In the low-density diffuse ISM, $R_V$ has a value $\sim 3.1$, while in
dense molecular clouds $R_V$ can be higher with values $4 < R_V < 6$.

In general, however, the use of broad band photometry requires attention to the intrinsic
SEDs of the background stars in order to correct for heterochromatic variations in the
effective reddening law. As discussed in the LSST Science Book, possession of an accurate
dust map is important to many astrophysical studies. The two most significant all-sky maps generated
in the past two decades are the SFD98 maps based on IRAS observations, and the recent thermal
dust maps derived from Planck submillimeter data. The angular resolutions of both maps are similar - 
between 4 to 6 arcminutes.

Both of the aforementioned maps are strictly two-dimensional and conntain no information about the 
distribution of dust along the line of sight. A third dimension can be obtained by analysis of
accurate stellar photometry which constrain both the reddening $E(B-V)$ and $R_V$ towards 
individual stars. This approach requires determination of the intrinsic stellar colors and the photometric
parallax of each star in the presence of an unknown amount and law of extinction.
Recent work on 3-D maps include the Bayesian analysis method based on Pan-STARRS 1 data 
(Green et al. 2015, ApJ, 810, 25) and an alternative technique using SDSS photometry of 
M dwarfs (McGehee et al. 2016, in preparation). 

% --------------------------------------------------------------------

\subsection{Target measurements and discoveries}
\label{sec:\secname:targets}

The use of stellar samples to create three-dimensional extinction maps has an established history
beginning with the work of Neckel \& Klare (1980); however these, including studies based on SDSS and
PS1 photometry, are typically limited to heliocentric distances of $\sim$4 kpc. In the full co-added survey,
LSST will be able to map dust structures out to distances exceeding 40 kpc, thus revealing a
detailed picture of this component of the Milky Way Galaxy.

The Pan-STARRS1 survey (PS1) has
produced a three-dimensional dust map of the region of the sky covered
in their 3$\pi$ survey (which excludes a large part of the Galactic
Plane toward the south). Such maps are necessary to accurately measure
the intrinsic luminosities and colors of both Galactic and
extragalactic sources. 
Green et al. (2015) estimated $R_V$ along sightlines having higher 
reddening values as well as reddening values.
The PS1 map saturates at
extinctions $E(B-V) > 1.5$ as their tracer stars fall out of the
survey catalogs fainter than $g\sim 22$, meaning that this
high-fidelity map does not extend uniformly to within a few degrees of
the midplane. In addition, it only extends to a distance of about 4.5
kpc. Deep LSST data will allow this map to be extended to much higher
extinctions and larger distances. Owing to the high extinction and the
use of blue filters, this project is less affected by crowding than
other projects requiring photometry in the Plane. 

In comparison, the SDSS survey makes use of M dwarf locus in $(g-r,r-i)$ being
nearly perpendicular to the reddening vector in that color-color space. This 
allows mapping of a reddening-invariant index to the intrinsic stellar $g-i$ color
and subsequent deterimation of the light-of-sight reddening. This approach assumes
a set extinction law, i.e $R_V = 3.1$, in order compute the reddening-invariant 
index from the observed $g-r$ and $r-i$ colors. Given the relative faintness of M dwarfs,
this technique is distance limited to $\sim$1 kpc when based on SDSS data.

The LSST will be in a unique position to measure the changes in the observed reddening vector
due to $R_V$ variations due to its superb photometric accuracy. 
Both of the dust survey techniques mentioned here can be used on LSST data, and perhaps other 
methods will be developed before the start of survey operations. T

% --------------------------------------------------------------------

\subsection{Metrics}

\label{sec:\secname:metrics}

Production of a 3-D map of the dust component of the ISM based on LSST photometry will tell us 
how much dust is present, what type it is, and where it is along the line of sight. 
The latter concern brings in issues of how to determine stellar photometric parallaxes ($\mu = m-M$) under
an unknown reddening.

The dust maps that are created will consist of the median and variance of $E(B-V)$ and $R_V$ expressed as functions of 
$\mu$ under a suitable binning scheme. We can create simple Figure of Merit maps that lose the 
$\mu$ dependency by computing the mean and variance of the measured variances in $E(B-V)$ and $R_V$ 
over the $\mu$ bins.

With the possible exception of sightlines towards star formation regions, the spacing in time of the visits 
doesn't matter for dust studies. In the case of active star formation regions it is possible that changes in the
ISM could be apparent over the lifetime of the survey.
Pushing to fainter magnitudes (which means both better seeing and longer exposures) matters, 
both because we want more stars, and in particular, we want more stars behind the dust.  


{\bf Metric 1: Uncertainty and bias in $E(B-V)$~estimates as a
  function of location on-sky.} Dependencies:

\begin{itemize}
  \item Stellar population throughout the survey (e.g. Knut / Peter developments; TRILEGAL?);
    \item Dust map throughout the survey region;
    \item Scale photometric error predictions for each band from program requirements per exposure;
      \item Produce formal estimate on the error in extinction and reddening as a function of position on-sky within the survey.
\end{itemize}


% --------------------------------------------------------------------

\subsection{OpSim Analysis}
\label{sec:\secname:analysis}

OpSim analysis: how good would the default observing strategy be, at
the time of writing for this science project?


% --------------------------------------------------------------------

\subsection{Discussion}
\label{sec:\secname:discussion}

Discussion: what risks have been identified? What suggestions could be
made to improve this science project's figure of merit, and mitigate
the identified risks?


% ====================================================================

\navigationbar


% ====================================================================
%+
% SECTION:
%    MW_Astrometry.tex
%
% CHAPTER:
%    galaxy.tex
%
% ELEVATOR PITCH:
%
%-
% ====================================================================

\section{Astrometry with LSST: Positions, Proper Motions, and Parallax}
\def\secname{MW_Astrometry}\label{sec:\secname}

\credit{dgmonet}, \credit{DanaCD}, \credit{jgizis}, \credit{mliu},
\credit{caprastro}, \credit{willclarkson}, \credit{yoachim}

A number of Milky Way science cases of interest to the Astronomical
community will depend critically on the astrometric accuracy LSST will
deliver. While ``astrometry'' is not a science case in the framework
of this white paper, LSST's astrometric performance will be sensitive
to the particular choice of observing strategy.
%While astrometry is not a science case, high astrometric accuracy enables
%a large number of science cases.
Hence, the LSST Observing Strategy needs to be examined for systematic
trends that might limit or even preclude precise measures of
stellar positions, proper motions, parallaxes, and perturbations that
arise from unseen companions.

\autoref{sec:\secname:MW_Astrometry_measurements} highlights two
science cases at opposite scales of distance from the Sun that require
accurate and precise astrometry and/or proper motion
measurements. \autoref{sec:\secname:MW_Astrometry_metrics} presents
Metrics for LSST's astrometric performance, and discusses Figures of
Merit for the two highlighted science cases. These metrics are applied to two example OpSim runs in
\autoref{sec:\secname:MW_Astrometry_OpSim}. Finally in Section
\ref{sec:\secname:MW_Astrometry_furtherwork}, the work that is still
needed is discussed, both in terms of the Metrics and the Figures of
Merit that depend on them.

%Each of these cases stresses different aspects of the LSST hardware, software,and observing strategies.

%, here we highlight three representative science cases.
%that illustrate the various impacts of the observing strategy might
%be:
%To highlight the
%various astrometric impacts of the strategy, three science cases have
%been chosen for particular attention:

%\subsection{Introduction: Astrometry as a special case}
%\label{sec:\secname:MW_Astrometry_intro}

\subsection{Target Measurements and Discoveries}
\label{sec:\secname:MW_Astrometry_measurements}

%\begin{itemize}
%\item[1.] Identification of Streams in the Galactic Halo using proper motions.
%\item[2.] A complete sample of stars in the solar neighborhood.
%\end{itemize}
%\item The tie between the Radio and Optical realizations of the International Celestial Reference System.
%\item The specific and ensemble agreement between LSST and Gaia parallaxes.

{\bf 1. Identification of Streams in the Galactic Halo Using Proper Motions}

Much of the Milky Way's stellar halo was built by the accretion of smaller galaxies. Given that these galaxies
were generally of low mass, their tidal debris should still form coherent structures in phase space, especially
in the outer Galaxy where dynamical times are long. The identification of these streams would allow
a reconstruction of the accretion history of the Milky Way. Tides also lead to the dissolution of globular clusters,
leaving notably thin streams that serve as sensitive tracers both of the Galactic potential and of the presence of dark
subhalos.

A relatively small number of streams, originating from both dwarfs and globular clusters, have been identified via photometry
of individual stars in large surveys such as SDSS\@. However, only the highest surface brightness structures can be found
in this manner, and it is often difficult to trace the streams over their full extent. LSST will enable streams to be identified
by stellar proper motions, and combined with targeted follow-up spectroscopy, will yield full 6-D position and velocity measurements suitable for dynamical modeling.
Further, it will allow the discovery of tidal debris that is no longer spatially coherent but which can be unambiguously identified in phase space.

Finally, streams and other kinematically-distinct halo substructure
can be identified and characterized by combining proper motions and
photometry in reduced proper-motion diagrams \citep[e.g.,][]{carlin12},
and by analyzing proper-motions of tracers such as
RR Lyrae and giants over large portions of the sky \citep[e.g.,][]{casettidinescu15}.

{\it Response to observing strategy:} Most stars in streams will be main-sequence stars, and the old main sequence turnoff  is located at $r\sim24$ at a distance of 100 kpc.
The nominal LSST proper motion precision at this magnitude is 1 mas yr$^{-1}$, corresponding to about 475 km s$^{-1}$ at this distance. The proper motion
measurements will be better for brighter stars, but in general ensembles of stars will be necessary for accurate measurements. To make accurate proper motion measurements for faint stars, several key components are required. First, a zero point must be established, possibly via background galaxies located in each field. Next, the observations must cover a sufficient range of epochs to reliably detect linear proper motions.

To identify streams over their full lengths of many degrees of the sky, relative astrometry over small fields will not be sufficient. Therefore the absolute astrometric frame is important. Matching the optical astrometry to the radio International Celestial Reference System (ICRS) relies on measuring accurate positions for objects visible in both wavelength regimes.
These are typically distant QSOs. Unfortunately, many QSOs have detectable optical or radio structures that degrade the positions or suggests a displacement between the location of the sources of the radio and optical radiation. LSST will need to identify a large number of point-like QSOs based on their colors and variability.

Since the number of galaxies is overwhelming toward faint magnitudes,
these must be exploited to produce a reliable absolute
proper-motion zero point. By using Gaia stars at the bright end
as absolute proper-motion calibrators we can quantify the precision
and accuracy of background galaxies as a secondary link to an inertial reference system, and thus improve the calibration at the faint end of the survey.

%The tie between the radio and optical reference frames relies on measuring accurate positions for objects visible in both wavelength regimes.  Whereas there are optical variable stars with radio emission, most have associated optical nebulosity that degrades the accuracy of the optical positions. The typical radio+optical object is a QSO.  Unfortunately, many QSOs have detectable optical or radio structures that degrade the positions or suggests a displacement between the location of the sources of the radio and optical radiation.  The major contribution from LSST will be the identification of a large number of QSOs based on their colors that have minimal (if any) spatially extended structure.  The impact of this search has no obvious impact on the cadence other than temporal coverage to identify variability.

{\bf 2. A Complete Sample of Stars in the Solar Neighborhood}

The direct solar neighborhood offers our only chance to get make a complete sample of stars, brown dwarfs, and stellar remnants that encompass the entire formation and dynamical history of the Milky Way. While Gaia will offer parallax measurements for perhaps billions of stars, its faint magnitude limit of $G\sim 20$ will limit its measurements of the lowest-mass objects
and remnants to nearby objects, much less than the thin disk scale height of $\sim 300$ pc. For example, Gaia can only measure parallaxes for $0.2 M_{\odot}$ M dwarfs to about 100 pc
and $0.1 M_{\odot}$ M dwarfs to only \emph{10 pc}, showing that Gaia is ill-suited for studies of the coolest dwarfs. By contrast, LSST can measure parallaxes for $> 10^5$ M dwarfs and thousands of L/T brown dwarfs (the coolest Y dwarfs are too faint even for LSST; little contribution is likely here beyond the sample provided by WISE). Gaia will likewise be limited to cool white dwarfs within $\sim 100$ pc with which to estimate the age of the disk, and the thick disk and halo will be out of reach. LSST can directly compare white dwarf luminosity functions to determine precise differential ages for the thin disk, thick disk, and halo.

{\it Response to observing strategy:} Successfully completing this project will require parallax measurements much fainter than possible with Gaia as well as a verification that the LSST and Gaia parallax measurements are consistent in the overlapping magnitude range.

The measurement of stellar parallax puts the substantial constraints on the observing cadence. There are two major issues: the need to sample a wide range of parallax factor (related to time of year), and breaking the correlation between differential color refraction and parallax factor.

``Parallax factors" characterize the ellipse of the star's apparent motion as seen over the course of a year. The shape of the ellipse is given by the Earth's orbit and is not a free parameter in the astrometric solution. The amplitude of the right ascension parallax factor is close to unity while the amplitude of the declination parallax factor is dominated by the sine of ecliptic latitude.
The right ascension parallax factor has maximum amplitude when the star is approximately six hours from the Sun, so the optimum time for parallax observing is when the
star is on the meridian near evening or morning twilight. Atmospheric refraction displaces the star's apparent position in the direction of the zenith by an amount dependent on both the wavelength of the light and the distance to the zenith. Whereas the measured position of star is a function of the total refraction, the measurement of parallax
and proper motion depends on the differences in the refraction as a function of the color of each star and the circumstances of the observations.  This
dependence is called differential color refraction. The combination of parallax factor and differential color refraction leads to two rules: (i) Observations need to cover the widest possible range in parallax
factor, and (ii) The correlation between parallax factor and hour angle in the observations needs to be minimized.

%with respect to the meanmotion of the reference frame.

%The search for faint proper motion stars has two key components.  The first is the need to identify stars that move from the ensemble of other image features that can cause confusion.  For example, a compact group of stars that contains one or more stars of variable brightness can confuse the catalog correlation algorithm.  The other is the need to establish the zero point. For the case of relative astrometry, meaning the measurement of relative positions in an image, the question remains on how to remove the mean motion of the reference frame.  For example, astrometry on certain classes of galaxies might produce a zero point of sufficient accuracy.  This leads to a third constraint on the observing cadence.
% \begin{itemize}
%\item [3)] Observations must cover a sufficient range of epochs so that stars with
%linear or periodic motions can be identified at a high level of confidence.
%\end{itemize}


%\subsection{Sensitivity of parallax measurements to observing strategy}
%\label{sec:\secname:MW_Astrometry_cadence}

%\medskip


\subsection{Metrics and Figures of Merit for LSST's delivered astrometric accuracy}
\label{sec:\secname:MW_Astrometry_metrics}

%\medskip

First we discuss metrics for the observing strategy that affect all of
LSST's astrometric measurements, then discuss figures of merit for the
two science cases. (The three general metrics were identified years
ago and are already in the suite of MAF utilities, and they should be
reviewed prior to making final decisions. For this reason, in addition
to the Figures of Merit later in the chapter, we present spatial maps
and histograms for the metrics themselves in Section
\ref{sec:\secname:MW_Astrometry_OpSim}, for representative OpSim
strategies.)

\begin{itemize}
\item[A)] For each LSST field, the parallax factors at each epoch of
observation need to be computed.  The ensemble of these must be checked for
sufficient coverage of the parallactic ellipse.  In particular, the number of
measures with RA parallax factor less than --0.5 and greater than +0.5
needs to be tallied because these carry the most weight in the solution
for the amplitude (parallax).
\item[B)] For each LSST field,
%the hour angle of the observation needs to be
%computed, and
the correlation between hour angle and parallax factor
needs to be examined for significance.  The observing strategy must minimize
the number of fields with this correlation.
\item[C)] The epochs of observation for each field must be checked for a
reasonable coverage over the duration of the survey and to avoid
collections of too many visits during a few short intervals.
\end{itemize}

Within sims\_maf, metrics A (parallax factor distribution) and B
  (hour angle and parallax correlation) are implemented in a slightly
  different manner from the prescription above. We describe the
  implemented metrics here.

{\it Parallax factor coverage:} This is {\tt
    calibrationMetrics.ParallaxCoverageMetric} in sims\_maf. The
  inverse-variance weighted mean parallax offset is subtracted from
  the set of parallax offsets for an object at a given location with
  unit parallax amplitude, and the inverse-variance weighted mean
  $\langle r \rangle$~of the resulting residuals is returned, scaled
  to the range $0 \le \langle r \rangle \le 1$. For each measurement,
  the variance used in the weighting is the estimate of the
  (uncrowded) astrometric uncertainty returned by OpSim for a star of
  specified fiducial magnitude at the center of the HEALPIX of
  interest. What constitutes a ``good'' value for $\langle r \rangle$~depends on the location
  of the star in ecliptic co-ordinates. Near either ecliptic pole a
  star with uniform parallax coverage would have $\langle r \rangle
  \approx 1.0$~while on the ecliptic uniform coverage would produce
  $\langle r \rangle \approx 0.5$. For any location, $\langle r
  \rangle \approx 0$~would mean all the observations were taken with
  identical parallax factor and therefore any attempt to fit the
  parallax amplitude would be completely degenerate with the object's
  position.

{\it Parallax-Hour angle correlation:} This is metric {\tt
    calibrationMetrics.ParallaxDcrDegenMetric}. At the level of tens
  of milliarcsec, Differential Chromatic Refraction (DCR) shifts the
  apparent location of the star in a color-dependent manner. Depending
  on the hour-angle distribution of observations throughout the year,
  motion due to parallax can become degenerate with motion due to the
  pattern of DCR values sampled. This metric returns the Pearson
  correlation coefficient $\rho$~between the best-fit parallax
  amplitude and DCR amplitude, returning values in the range $-1.0 \le
  \rho \le +1.0$. The range of acceptable values for this metric is
  still under investigation; Monte Carlo simulation by one of us (DGM)
  suggests the parallax error becomes independent of
  $\rho$~(i.e. other effects dominate) for values $|\rho| \lesssim
  0.7$.



For the stream project discussed above, a simple to state (but perhaps complex to implement) figure of merit
is the number of streams that can be discovered in LSST via their proper motions. As a first
attempt, it would be reasonable to assume about 100 halo streams from old, metal-poor dwarf galaxies with
stellar masses $10^5-10^7 M_{\odot}$ distributed as $r^{-3.5}$. The stream widths and internal velocity
dispersions can be set from galaxy scaling relations, and their 3-D velocities consistent with a simple Galactic mass
model at their radii. Setting the stream lengths is more complicated, but should cover a large range from a few to many kpc.
Over a given area, the stream ``S/N" can roughly be taken as the number of stream stars (identified via proper motion, color, and magnitude)
divided by the square root of the number of field stars. For globular clusters, a similar number of streams could be included, but these should have much smaller widths (10s of pc)
and typical masses $10^4-10^5 M_{\odot}$. Eventually it would be desirable to use actual simulated stream parameters taken from cosmological models of the Milky Way (e.g.,
from the Aquarius simulation).

Solar neighborhood projects will be sensitive to the general parallax and proper motion metrics discussed above. More specific science figures of merit are {\it required} at this stage.  For example, the precision of the differential age measurement between the thin disk and halo, which would depend on the number of white dwarfs that can be isolated
from each population.

\subsection{OpSim Analysis}
\label{sec:\secname:MW_Astrometry_OpSim}

Here we present initial analysis of LSST's astrometric
performance. Two example strategies are assessed: the current baseline
strategy, \opsimdbref{db:baseCadence}, and the new cadence
\opsimdbref{db:NormalGalacticPlane}, which extends the Wide-Fast-Deep
survey to the Galactic Plane (see Section \autoref{sec:cadexp:alternatives}
for more detail on this run).

%the PanSTARRS-like cadence,
%\opsimdbref{db:opstwoPS}, which greater spatial uniformity and
%superior coverage of the Galactic Plane.

\subsubsection{Metrics: Parallax and proper motion precision}

Here we present the expected astrometric performance of LSST as a function of
location on-sky, for two main cuts on the survey strategies:
\begin{itemize}
  \item By time: objects detected in $g,r,i,z$, after years 1, 2 and 10
    of the survey (Figures \ref{fig_astrom_ByTime_PACoverage} -
    \ref{fig_astrom_ByTime_paError});
\item By filter: objects detected in $g,r,i,z$, or in $u$ only, or $y$ only, over the full 10 years of the survey (Figures~\ref{fig_astrom_ByFilter_PACoverage} - \ref{fig_astrom_ByFilter_paError}).
\end{itemize}

Astrometric performance for parallax is quantified using the following
metrics:
\begin{itemize}
  \item[1.] Parallax factor coverage (following metric A of \autoref{sec:\secname:MW_Astrometry_metrics}); values farther from 0 are better). See Figures \ref{fig_astrom_ByTime_PACoverage} \&  \ref{fig_astrom_ByFilter_PACoverage};
    \item[2.] Parallax-Hour angle correlation (metric B of \autoref{sec:\secname:MW_Astrometry_metrics}; values closer to 0 are better). See Figures \ref{fig_astrom_ByTime_PADegen} \& \ref{fig_astrom_ByFilter_PADegen};
      \item[3.] Proper motion error, for a star at apparent magnitude 21.0 in the filter specified (this addresses the distribution of measurement epochs, as recommended in Metric C in \autoref{sec:\secname:MW_Astrometry_metrics}; smaller values are better). See Figures \ref{fig_astrom_ByTime_pmError} \& \ref{fig_astrom_ByFilter_pmError};
        \item[4.] Parallax error, for a star at apparent magnitude 21.0 in the filter specified (smaller values are better). See Figures \ref{fig_astrom_ByTime_paError} \& \ref{fig_astrom_ByFilter_paError}.
\end{itemize}

{\it Limitations of the results presented in Figures \ref{fig_astrom_ByTime_PACoverage} to \ref{fig_astrom_ByFilter_paError}.:}
\begin{itemize}
  \item[i.] The spatial maps are clipped at $95\%$~in order to keep
    the color-scale at a sensible range; in some cases this has had
    the side effect of removing parts of the spatial coverage in the
    \opsimdbref{db:baseCadence} maps.

  \item[ii.] This analysis neglected spatial confusion in high-density regions. While this
    confusion would be the same whatever observing strategy was
    chosen, the measurement uncertainties for proper motion and parallax uncertainty
    should be regarded as lower limits.

    \item[iii.] The choice of fiducial apparent magnitude $r = u = y =
      21.0$~is arbitrary. It
      would be informative to repeat the analysis for a range of
      target apparent magnitudes that are better-matched to the
      specific science cases.

      \item[iv.] The comparison between single-filter and $griz$
        detections likely overestimates the measurement precision for
        the $u$-only and $y$-only detections, as an object only
        detected in a single filter may well not be detected in all
        images taken in that filter. While the comparison between
        filter subsets for a given strategy may therefore be highly
        approximate, the comparison between strategies for the same
        filter should be more reliable.

% WIC 2016-06-01 - item below removed, now that we are comparing two strategies with
% similar sky coverage.

%  \item[v.] We have not yet subdivided the samples by a meaningful
%    spatial co-ordinate (galactic latitude would be the obvious
%    choice). A large part of the breadth of the various metric values in
%    \opsimdbref{db:baseCadence} as compared to \opsimdbref{db:opstwoPS} may be
%      due to spatial nonuniformity of the sampling; replotting the
%      histograms coded by galactic latitude would be highly informative in this context.

\end{itemize}

{\it Indications at this date:} Despite these limitations, we note the following:

% WIC 2016-06-01 - updated for comparing wfdPlane to Baseline, not PanSTARRS-1 as was the case.

\begin{itemize}
  \item[I1.] Taking snapshots of the survey at various stages of completion (Figures \ref{fig_astrom_ByTime_PACoverage} -  \ref{fig_astrom_ByTime_paError}), strategy \opsimdbref{db:NormalGalacticPlane} is not significantly worse than \opsimdbref{db:baseCadence};
%\item[I2.] As might be expected, the distribution of metric values for the PanSTARRS-like cadence is narrower than for \opsimdbref{db:baseCadence} - thus astrometric survey uniformity is improved;
%\item[I2.] For the extremes of object color (objects detected only in the bluest or only in the reddest filter), the differences between strategies is weaker. The histogram of run \opsimdbref{db:baseCadence} still shows a population with poorer parallax measures (although this might be due to coverage of difficult-to-observe regions that are not covered at all by the PanSTARRS-like strategy).
\item[I2.] To first order, proper motion and parallax error are dominated by the total time coverage, as might be expected.
\end{itemize}

%% In the current incarnation, these will be big figures on the page. Consider
%% finding a way to summarize them!
\begin{figure}[ht]
  \begin{center}
  \includegraphics[width=2.0in]{./figs/milkyway/astromPanels/MW_Astrom_paCovge_Baseline_01y_map.png}
  \includegraphics[width=2.0in]{./figs/milkyway/astromPanels/MW_Astrom_paCovge_Baseline_02y_map.png}
  \includegraphics[width=2.0in]{./figs/milkyway/astromPanels/MW_Astrom_paCovge_Baseline_10y_map.png}
  \end{center}
  \begin{center}
  \includegraphics[width=2.0in]{./figs/milkyway/astromPanels/MW_Astrom_paCovge_wfdPlane_01y_map.png}
  \includegraphics[width=2.0in]{./figs/milkyway/astromPanels/MW_Astrom_paCovge_wfdPlane_02y_map.png}
  \includegraphics[width=2.0in]{./figs/milkyway/astromPanels/MW_Astrom_paCovge_wfdPlane_10y_map.png}
  \end{center}

  \begin{center}
  \includegraphics[width=2.0in]{./figs/milkyway/astromPanels/MW_Astrom_paCovge_Baseline_01y_hst.png}
  \includegraphics[width=2.0in]{./figs/milkyway/astromPanels/MW_Astrom_paCovge_Baseline_02y_hst.png}
  \includegraphics[width=2.0in]{./figs/milkyway/astromPanels/MW_Astrom_paCovge_Baseline_10y_hst.png}
  \end{center}
  \begin{center}
  \includegraphics[width=2.0in]{./figs/milkyway/astromPanels/MW_Astrom_paCovge_wfdPlane_01y_hst.png}
  \includegraphics[width=2.0in]{./figs/milkyway/astromPanels/MW_Astrom_paCovge_wfdPlane_02y_hst.png}
  \includegraphics[width=2.0in]{./figs/milkyway/astromPanels/MW_Astrom_paCovge_wfdPlane_10y_hst.png}
  \end{center}
  \caption{Parallax coverage achieved at different epochs within the survey. {\it Top and Third row:} OpSim run \opsimdbref{db:baseCadence}. {\it Second and bottom row:} OpSim run \opsimdbref{db:NormalGalacticPlane} (wide-fast-deep extended to much of the inner Plane). Reading left-right, columns represent: {\it Left column:} all observations within the first 365 days of operation; {\it Middle column:} first two years; {\it right column:} the full 10-year survey. Spatial maps are clipped at 95\%, with histogram horizontal limits (0.0 - 1.0).}
  \label{fig_astrom_ByTime_PACoverage}
\end{figure}

\begin{figure}[ht]
  \begin{center}
  \includegraphics[width=2.0in]{./figs/milkyway/astromPanels/MW_Astrom_paDcrDegen_Baseline_01y_map.png}
  \includegraphics[width=2.0in]{./figs/milkyway/astromPanels/MW_Astrom_paDcrDegen_Baseline_02y_map.png}
  \includegraphics[width=2.0in]{./figs/milkyway/astromPanels/MW_Astrom_paDcrDegen_Baseline_10y_map.png}
  \end{center}
  \begin{center}
  \includegraphics[width=2.0in]{./figs/milkyway/astromPanels/MW_Astrom_paDcrDegen_wfdPlane_01y_map.png}
  \includegraphics[width=2.0in]{./figs/milkyway/astromPanels/MW_Astrom_paDcrDegen_wfdPlane_02y_map.png}
  \includegraphics[width=2.0in]{./figs/milkyway/astromPanels/MW_Astrom_paDcrDegen_wfdPlane_10y_map.png}
  \end{center}

  \begin{center}
  \includegraphics[width=2.0in]{./figs/milkyway/astromPanels/MW_Astrom_paDcrDegen_Baseline_01y_hst.png}
  \includegraphics[width=2.0in]{./figs/milkyway/astromPanels/MW_Astrom_paDcrDegen_Baseline_02y_hst.png}
  \includegraphics[width=2.0in]{./figs/milkyway/astromPanels/MW_Astrom_paDcrDegen_Baseline_10y_hst.png}
  \end{center}
  \begin{center}
  \includegraphics[width=2.0in]{./figs/milkyway/astromPanels/MW_Astrom_paDcrDegen_wfdPlane_01y_hst.png}
  \includegraphics[width=2.0in]{./figs/milkyway/astromPanels/MW_Astrom_paDcrDegen_wfdPlane_02y_hst.png}
  \includegraphics[width=2.0in]{./figs/milkyway/astromPanels/MW_Astrom_paDcrDegen_wfdPlane_10y_hst.png}
  \end{center}
  \caption{Correlation coefficient $\rho$~between parallax and Differential Chromatic Refraction (DCR) up to different epochs within the survey. {\it Top and Third row:} OpSim run \opsimdbref{db:baseCadence}. {\it Second and bottom row:} OpSim run \opsimdbref{db:NormalGalacticPlane} (wide-fast-deep extended to much of the inner Plane). Reading left-right, columns represent: {\it Left column:} all observations within the first 365 days of operation; {\it Middle column:} first two years; {\it right column:} the full 10-year survey. Spatial maps are clipped at 95\%, with histogram horizontal scale set to the range $-1.0 \le \rho \le +1.0$.}
  \label{fig_astrom_ByTime_PADegen}
\end{figure}



\begin{figure}[ht]
  \begin{center}
  \includegraphics[width=2.0in]{./figs/milkyway/astromPanels/MW_Astrom_pmError_Baseline_01y_map.png}
  \includegraphics[width=2.0in]{./figs/milkyway/astromPanels/MW_Astrom_pmError_Baseline_02y_map.png}
  \includegraphics[width=2.0in]{./figs/milkyway/astromPanels/MW_Astrom_pmError_Baseline_10y_map.png}
  \end{center}
  \begin{center}
  \includegraphics[width=2.0in]{./figs/milkyway/astromPanels/MW_Astrom_pmError_wfdPlane_01y_map.png}
  \includegraphics[width=2.0in]{./figs/milkyway/astromPanels/MW_Astrom_pmError_wfdPlane_02y_map.png}
  \includegraphics[width=2.0in]{./figs/milkyway/astromPanels/MW_Astrom_pmError_wfdPlane_10y_map.png}
  \end{center}

  \begin{center}
  \includegraphics[width=2.0in]{./figs/milkyway/astromPanels/MW_Astrom_pmError_Baseline_01y_hst.png}
  \includegraphics[width=2.0in]{./figs/milkyway/astromPanels/MW_Astrom_pmError_Baseline_02y_hst.png}
  \includegraphics[width=2.0in]{./figs/milkyway/astromPanels/MW_Astrom_pmError_Baseline_10y_hst.png}
  \end{center}
  \begin{center}
  \includegraphics[width=2.0in]{./figs/milkyway/astromPanels/MW_Astrom_pmError_wfdPlane_01y_hst.png}
  \includegraphics[width=2.0in]{./figs/milkyway/astromPanels/MW_Astrom_pmError_wfdPlane_02y_hst.png}
  \includegraphics[width=2.0in]{./figs/milkyway/astromPanels/MW_Astrom_pmError_wfdPlane_10y_hst.png}
  \end{center}
  \caption{Proper motion error for a star at $r=21.0$, for different epochs within the survey. Crowding errors are ignored. {\it Top and Third row:} OpSim run \opsimdbref{db:baseCadence}.  {\it Second and bottom row:} OpSim run \opsimdbref{db:NormalGalacticPlane} (wide-fast-deep extended to much of the inner Plane). Reading left-right, columns represent: {\it Left column:} all observations within the first 365 days of operation; {\it Middle column:} first two years; {\it right column:} the full 10-year survey. Spatial maps are clipped at 95\% and a log-scale is used for the maps and histograms. Reading left-right, the horizontal upper limits on the histograms are (25, 10, 3.0) mas yr$^{-1}$, respectively. Note that the histograms do not include the full range of values reported in the maps.}
  \label{fig_astrom_ByTime_pmError}
\end{figure}


\begin{figure}[ht]
  \begin{center}
  \includegraphics[width=2.0in]{./figs/milkyway/astromPanels/MW_Astrom_paError_Baseline_01y_map.png}
  \includegraphics[width=2.0in]{./figs/milkyway/astromPanels/MW_Astrom_paError_Baseline_02y_map.png}
  \includegraphics[width=2.0in]{./figs/milkyway/astromPanels/MW_Astrom_paError_Baseline_10y_map.png}
  \end{center}
  \begin{center}
  \includegraphics[width=2.0in]{./figs/milkyway/astromPanels/MW_Astrom_paError_wfdPlane_01y_map.png}
  \includegraphics[width=2.0in]{./figs/milkyway/astromPanels/MW_Astrom_paError_wfdPlane_02y_map.png}
  \includegraphics[width=2.0in]{./figs/milkyway/astromPanels/MW_Astrom_paError_wfdPlane_10y_map.png}
  \end{center}

  \begin{center}
  \includegraphics[width=2.0in]{./figs/milkyway/astromPanels/MW_Astrom_paError_Baseline_01y_hst.png}
  \includegraphics[width=2.0in]{./figs/milkyway/astromPanels/MW_Astrom_paError_Baseline_02y_hst.png}
  \includegraphics[width=2.0in]{./figs/milkyway/astromPanels/MW_Astrom_paError_Baseline_10y_hst.png}
  \end{center}
  \begin{center}
  \includegraphics[width=2.0in]{./figs/milkyway/astromPanels/MW_Astrom_paError_wfdPlane_01y_hst.png}
  \includegraphics[width=2.0in]{./figs/milkyway/astromPanels/MW_Astrom_paError_wfdPlane_02y_hst.png}
  \includegraphics[width=2.0in]{./figs/milkyway/astromPanels/MW_Astrom_paError_wfdPlane_10y_hst.png}
  \end{center}
  \caption{Parallax error for a star at $r=21.0$, for different epochs within the survey. Crowding errors are ignored. {\it Top and Third row:} OpSim run \opsimdbref{db:baseCadence}. {\it Second and bottom row:} OpSim run \opsimdbref{db:NormalGalacticPlane} (wide-fast-deep extended to much of the inner Plane). Reading left-right, columns represent: {\it Left column:} all observations within the first 365 days of operation; {\it Middle column:} first two years; {\it right column:} the full 10-year survey. Spatial maps are clipped at 95\%.  Reading left-right, the horizontal upper limits on the histograms are (10, 10, 2.0) mas, respectively.}
  \label{fig_astrom_ByTime_paError}
\end{figure}

%%% Now for the metrics by filter.
\begin{figure}[ht]
  \begin{center}
  \includegraphics[width=2.0in]{./figs/milkyway/astromPanels/MW_Astrom_paCovge_Baseline_u_map.png}
  \includegraphics[width=2.0in]{./figs/milkyway/astromPanels/MW_Astrom_paCovge_Baseline_y_map.png}
  \includegraphics[width=2.0in]{./figs/milkyway/astromPanels/MW_Astrom_paCovge_Baseline_10y_map.png}
  \end{center}
  \begin{center}
  \includegraphics[width=2.0in]{./figs/milkyway/astromPanels/MW_Astrom_paCovge_wfdPlane_u_map.png}
  \includegraphics[width=2.0in]{./figs/milkyway/astromPanels/MW_Astrom_paCovge_wfdPlane_y_map.png}
  \includegraphics[width=2.0in]{./figs/milkyway/astromPanels/MW_Astrom_paCovge_wfdPlane_10y_map.png}
  \end{center}

  \begin{center}
  \includegraphics[width=2.0in]{./figs/milkyway/astromPanels/MW_Astrom_paCovge_Baseline_u_hst.png}
  \includegraphics[width=2.0in]{./figs/milkyway/astromPanels/MW_Astrom_paCovge_Baseline_y_hst.png}
  \includegraphics[width=2.0in]{./figs/milkyway/astromPanels/MW_Astrom_paCovge_Baseline_10y_hst.png}
  \end{center}
  \begin{center}
  \includegraphics[width=2.0in]{./figs/milkyway/astromPanels/MW_Astrom_paCovge_wfdPlane_u_hst.png}
  \includegraphics[width=2.0in]{./figs/milkyway/astromPanels/MW_Astrom_paCovge_wfdPlane_y_hst.png}
  \includegraphics[width=2.0in]{./figs/milkyway/astromPanels/MW_Astrom_paCovge_wfdPlane_10y_hst.png}
  \end{center}
  \caption{Parallax coverage achieved for three extremes of object color, over the full 10-year survey. {\it Top and Third row:} OpSim run \opsimdbref{db:baseCadence}. {\it Second and bottom row:} OpSim run \opsimdbref{db:NormalGalacticPlane} (wide-fast-deep extended to much of the inner Plane). Reading left-right, columns represent: {\it Left column:} Objects detected only in the bluest filter; {\it Middle column:} objects detected only in the reddest filter; {\it Right column:} objects detected in all filters. Spatial maps are clipped at 95\%, with histogram horizontal limits (0.0 - 1.0).}
  \label{fig_astrom_ByFilter_PACoverage}
\end{figure}

\begin{figure}[ht]
  \begin{center}
  \includegraphics[width=2.0in]{./figs/milkyway/astromPanels/MW_Astrom_paDcrDegen_Baseline_u_map.png}
  \includegraphics[width=2.0in]{./figs/milkyway/astromPanels/MW_Astrom_paDcrDegen_Baseline_y_map.png}
  \includegraphics[width=2.0in]{./figs/milkyway/astromPanels/MW_Astrom_paDcrDegen_Baseline_10y_map.png}
  \end{center}
  \begin{center}
  \includegraphics[width=2.0in]{./figs/milkyway/astromPanels/MW_Astrom_paDcrDegen_wfdPlane_u_map.png}
  \includegraphics[width=2.0in]{./figs/milkyway/astromPanels/MW_Astrom_paDcrDegen_wfdPlane_y_map.png}
  \includegraphics[width=2.0in]{./figs/milkyway/astromPanels/MW_Astrom_paDcrDegen_wfdPlane_10y_map.png}
  \end{center}

  \begin{center}
  \includegraphics[width=2.0in]{./figs/milkyway/astromPanels/MW_Astrom_paDcrDegen_Baseline_u_hst.png}
  \includegraphics[width=2.0in]{./figs/milkyway/astromPanels/MW_Astrom_paDcrDegen_Baseline_y_hst.png}
  \includegraphics[width=2.0in]{./figs/milkyway/astromPanels/MW_Astrom_paDcrDegen_Baseline_10y_hst.png}
  \end{center}
  \begin{center}
  \includegraphics[width=2.0in]{./figs/milkyway/astromPanels/MW_Astrom_paDcrDegen_wfdPlane_u_hst.png}
  \includegraphics[width=2.0in]{./figs/milkyway/astromPanels/MW_Astrom_paDcrDegen_wfdPlane_y_hst.png}
  \includegraphics[width=2.0in]{./figs/milkyway/astromPanels/MW_Astrom_paDcrDegen_wfdPlane_10y_hst.png}
  \end{center}
  \caption{Correlation coefficient $\rho$~between parallax and Differential Chromatic Refraction (DCR), selecting filters for three extremes of object color, over the full 10-year survey. {\it Top and Third row:} OpSim run \opsimdbref{db:baseCadence}. {\it Second and bottom row:} OpSim run \opsimdbref{db:NormalGalacticPlane} (wide-fast-deep extended to much of the inner Plane). Reading left-right, columns represent: {\it Left column:} Objects detected only in the bluest filter; {\it Middle column:} objects detected only in the reddest filter; {\it Right column:} objects detected in all filters. Spatial maps are clipped at 95\%, with histogram horizontal scale set to the range $-1.0 \le \rho \le +1.0$.}
  \label{fig_astrom_ByFilter_PADegen}
\end{figure}

\begin{figure}[ht]
  \begin{center}
  \includegraphics[width=2.0in]{./figs/milkyway/astromPanels/MW_Astrom_pmError_Baseline_u_map.png}
  \includegraphics[width=2.0in]{./figs/milkyway/astromPanels/MW_Astrom_pmError_Baseline_y_map.png}
  \includegraphics[width=2.0in]{./figs/milkyway/astromPanels/MW_Astrom_pmError_Baseline_10y_map.png}
  \end{center}
  \begin{center}
  \includegraphics[width=2.0in]{./figs/milkyway/astromPanels/MW_Astrom_pmError_wfdPlane_u_map.png}
  \includegraphics[width=2.0in]{./figs/milkyway/astromPanels/MW_Astrom_pmError_wfdPlane_y_map.png}
  \includegraphics[width=2.0in]{./figs/milkyway/astromPanels/MW_Astrom_pmError_wfdPlane_10y_map.png}
  \end{center}

  \begin{center}
  \includegraphics[width=2.0in]{./figs/milkyway/astromPanels/MW_Astrom_pmError_Baseline_u_hst.png}
  \includegraphics[width=2.0in]{./figs/milkyway/astromPanels/MW_Astrom_pmError_Baseline_y_hst.png}
  \includegraphics[width=2.0in]{./figs/milkyway/astromPanels/MW_Astrom_pmError_Baseline_10y_hst.png}
  \end{center}
  \begin{center}
  \includegraphics[width=2.0in]{./figs/milkyway/astromPanels/MW_Astrom_pmError_wfdPlane_u_hst.png}
  \includegraphics[width=2.0in]{./figs/milkyway/astromPanels/MW_Astrom_pmError_wfdPlane_y_hst.png}
  \includegraphics[width=2.0in]{./figs/milkyway/astromPanels/MW_Astrom_pmError_wfdPlane_10y_hst.png}
  \end{center}
  \caption{Proper motion error for a star at apparent magnitude $m=21.0$, for three extremes of object color and assessed over the full survey. Crowding errors are ignored. {\it Top and Third row:} OpSim run \opsimdbref{db:baseCadence}. {\it Second and bottom row:} OpSim run \opsimdbref{db:NormalGalacticPlane} (wide-fast-deep extended to much of the inner Plane). Reading left-right, columns represent: {\it Left column:} Objects detected only in the bluest filter; the fiducial object has apparent magnitude $u=21.0$; {\it Middle column:} objects detected only in the reddest filter (so $y = 21.0$); {\it Right column:} objects detected in all filters (using $r=21.0$~and a ``flat'' spectrum within sims\_maf). Spatial maps are clipped at 95\% and a log-scale is used for both the spatial maps and histograms. Reading left-right, the horizontal upper limits on the histograms are (4.0, 4.0, 3.0) mas yr$^{-1}$, respectively. Note that the histograms do not include the full range of values reported in the maps.}
  \label{fig_astrom_ByFilter_pmError}
\end{figure}

\begin{figure}[ht]
  \begin{center}
  \includegraphics[width=2.0in]{./figs/milkyway/astromPanels/MW_Astrom_paError_Baseline_u_map.png}
  \includegraphics[width=2.0in]{./figs/milkyway/astromPanels/MW_Astrom_paError_Baseline_y_map.png}
  \includegraphics[width=2.0in]{./figs/milkyway/astromPanels/MW_Astrom_paError_Baseline_10y_map.png}
  \end{center}
  \begin{center}
  \includegraphics[width=2.0in]{./figs/milkyway/astromPanels/MW_Astrom_paError_wfdPlane_u_map.png}
  \includegraphics[width=2.0in]{./figs/milkyway/astromPanels/MW_Astrom_paError_wfdPlane_y_map.png}
  \includegraphics[width=2.0in]{./figs/milkyway/astromPanels/MW_Astrom_paError_wfdPlane_10y_map.png}
  \end{center}

  \begin{center}
  \includegraphics[width=2.0in]{./figs/milkyway/astromPanels/MW_Astrom_paError_Baseline_u_hst.png}
  \includegraphics[width=2.0in]{./figs/milkyway/astromPanels/MW_Astrom_paError_Baseline_y_hst.png}
  \includegraphics[width=2.0in]{./figs/milkyway/astromPanels/MW_Astrom_paError_Baseline_10y_hst.png}
  \end{center}
  \begin{center}
  \includegraphics[width=2.0in]{./figs/milkyway/astromPanels/MW_Astrom_paError_wfdPlane_u_hst.png}
  \includegraphics[width=2.0in]{./figs/milkyway/astromPanels/MW_Astrom_paError_wfdPlane_y_hst.png}
  \includegraphics[width=2.0in]{./figs/milkyway/astromPanels/MW_Astrom_paError_wfdPlane_10y_hst.png}
  \end{center}
  \caption{Parallax error for a star at apparent magnitude $m=21.0$, for three extremes of object color and assessed over the full survey. Crowding errors are ignored. {\it Top and Third row:} OpSim run \opsimdbref{db:baseCadence}. {\it Second and bottom row:} OpSim run \opsimdbref{db:NormalGalacticPlane} (wide-fast-deep extended to much of the inner Plane). Reading left-right, columns represent: {\it Left column:} Objects detected only in the bluest filter; the fiducial object has apparent magnitude $u=21.0$; {\it Middle column:} objects detected only in the reddest filter (so $y = 21.0$); {\it Right column:} objects detected in all filters (using $r=21.0$~and a ``flat'' spectrum within sims\_maf). Spatial maps are clipped at 95\%. Reading left-right, the horizontal upper limits on the histograms are (10, 10, 2.0) mas, respectively. Note that the histograms do not include the full range of values reported in the maps.}
  \label{fig_astrom_ByFilter_paError}
\end{figure}

\subsubsection{Figures of Merit depending on the Metrics}

Building on the first-order metrics above, this subsection communicates scientific figures of merit for the cases identified in \autoref{sec:\secname:MW_Astrometry_measurements} above.

Table \ref{tab_SummaryMWAstrometry} summarizes the Figures of Merit
(FoMs) for Astrometry science cases. At the time of writing, FoMs have
been implemented to summarize the random uncertainty in proper motion
and parallax, for two regions experiencing extreme values of these
quantities: the inner Plane (conservatively defined in this section as
$|b| \lesssim 7^o$~and $|l| \lesssim 80^o$), and the main survey
(excluding the inner plane and the Southern Polar region, taken
here as $\delta_{2000.0} < -60.0^o$). Figure
\ref{fig_astrom_RegionSelKey} illustrates these selection-regions on
the sky. These form FoM 1.1-1.4, and have to-date been run for the
OpSim runs \opsimdbref{db:baseCadence} (Baseline cadence),
\opsimdbref{db:opstwoPS} (similar to PanSTARRS-1), and the
recently-completed \opsimdbref{db:NormalGalacticPlane} (which applies
Wide-Fast-Deep cadence to much of the inner Galactic Plane). From the point of
view of parallax and proper motion, the latter two strategies do not
negatively impact the non-plane regions, but they {\it substantially}
improve the sampling for proper motions and parallax (again,
neglecting the effects of spatial crowding).

FoM 1.5 in Table \ref{tab_SummaryMWAstrometry} reports the total number of fields with Parallax/Hour-angle correlation $|\rho| < 0.7$.

At the time of writing, FoMs 2-5 in Table
\ref{tab_SummaryMWAstrometry} are still at the specification stage,
and are described in Section
\ref{sec:\secname:MW_Astrometry_furtherwork}.

%%%% Figures used as ``key'' for the astrometry FoMs:

\begin{figure}[h]
  \begin{center}
    \includegraphics[width=2.0in]{./figs/milkyway/astromPanels/MW_Astrom_FoM_properMotion_minion_1016_all_skymap.png}
  \includegraphics[width=2.0in]{./figs/milkyway/astromPanels/MW_Astrom_FoM_properMotion_minion_1016_plane_skymap.png}
  \includegraphics[width=2.0in]{./figs/milkyway/astromPanels/MW_Astrom_FoM_properMotion_minion_1016_nonPlane_skymap.png}
    \end{center}
  \caption{Selection regions for the Astrometry Figures of Merit (FoMs) 1.1-1.4. Figures of Merit 1.1 and 1.3 refer to the ``main survey'' region shown in the middle panel (which for the FoM also avoids the region of the South Galactic Pole). The right panel shows the inner Plane region to which FoMs 1.2 \& 1.4 refer. The left-hand panel shows the entire survey region for reference. This example shows run \opsimdbref{db:baseCadence}. See Table \ref{tab_SummaryMWAstrometry} and Section \ref{sec:\secname:MW_Astrometry_metrics}.}
  \label{fig_astrom_RegionSelKey}
\end{figure}

\subsection{Topics that will need to be addressed}
\label{sec:\secname:MW_Astrometry_furtherwork}

Here we present suggestions for further work, first on figures of
merit for the science cases, and then on additional Metrics for LSST's
astrometric performance.

\subsubsection{Further work on science Figures of Merit}
%\medskip

At the time of writing, the Figures of Merit for both the highlighted
Science cases need to be implemented and applied to OpSim output,
preferably in a format that can be summarized in a single Table in
this section. These figures of merit are discussed above in Section
\ref{sec:\secname:MW_Astrometry_metrics} (particularly for the Halo
Streams science project). Figures of merit for the two science cases
might be:
\begin{itemize}
  \item[1.] Number of streams that LSST can discover via their proper motions;
\item[2.] Uncertainty and bias in the thin and thick disk differential age measurement when using white dwarfs from each population as tracers.
\end{itemize}

Given the diversity of science cases that use local Solar Neighborhood
populations as tracers, it may be advantageous to subdivide the Solar
Neighborhood projects into further figures of merit. Two further example
figures of merit might then be:
\begin{itemize}
  \item[3.] Uncertainty and bias in the Brown Dwarf mass function using Solar Neighborhood tracers;
   \item[4.] Uncertainty and bias in the thickness in the main sequence of M-dwarfs within 25pc from the Sun, once variability has been characterized and removed.
\end{itemize}

%%%% SUMMARY TABLE FOR THIS SECTION

\begin{table}
  \begin{tabular}{l|p{4.8cm}|p{1.1cm}|p{1.1cm}|p{1.1cm}|c|p{3.5cm}}
    FoM & Brief description & {\rotatebox{90}{\opsimdbref{db:baseCadence} }} & {\rotatebox{90}{\opsimdbref{db:opstwoPS} }} & {\rotatebox{90}{\opsimdbref{db:NormalGalacticPlane}   }} &  {\rotatebox{90}{future run 2}} & Notes \\
    \hline
    1.1 & \footnotesize{Median parallax error at $r=21$ (main survey)}      & 0.69  & 0.72 & 0.69 & - &
%\footnotesize{Summarize the presentation in Figures \ref{fig_astrom_ByTime_PACoverage}-\ref{fig_astrom_ByFilter_paError} }
\footnotesize{See region definitions in Figure \ref{fig_astrom_RegionSelKey}.}
\\
    1.2. & \footnotesize{Median parallax error at $r=21$ (plane)}   & 2.68 & {\bf 0.91} & {\bf 0.89} & - &
\footnotesize{Smaller values are better.}\\
    1.3. & \footnotesize{Median proper motion error at $r=21$ (main survey)}  & 0.19 & 0.19 & 0.19 & - &
%\footnotesize{Take median of Figure \ref{fig_astrom_ByTime_pmError} over the ``plane'' region.}
\\
    1.4. & \footnotesize{Median proper motion error at $r=21$ (plane)} & 16.7
%$^\dagger$
& {\bf 0.26} & {\bf 0.25} & - &
%\footnotesize{$^\dagger$no, this is not a typo.}
\\
1.5. & \footnotesize{Fields with Parallax-DCR correlation coefficient $\rho \ge 0.7$~/ total fields} & \footnotesize{ \bf{3486} / \bf {31116} } & \footnotesize{3586 / 30107} & \footnotesize{3690 / 31116} & - & \footnotesize{Smaller is better. Value reported after full 10 years of survey for $griz$~detections.}  \\
    \hline
    2.1. & \footnotesize{Number of streams LSST can discover via proper motions} & - & - & - & - &  - \\
    3.1. & \footnotesize{Uncertainty and bias in thin- and thick-disk differential age measurement via white dwarfs} & - & - & - & - &  - \\
    4.1. & \footnotesize{Uncertainty and bias in brown dwarf mass function from the Solar Neighborhood}  & - & - & - & - & \footnotesize{Using astrometry metrics for objects detected only in the reddest filter(s)} \\
    4.2. & \footnotesize{Uncertainty and bias in white dwarf mass function from the Solar Neighborhood}  & - & - & - & - & \footnotesize{Using astrometry metrics for objects detected only in the bluest filter(s)} \\
    5.1. & \footnotesize{Uncertainty and bias in Solar Neighborhood M-dwarf thickness on the MS}  & - & - & - & - &  - \\
\end{tabular}
\caption{Summary of Figures of Merit for the Milky Way Astrometry science cases. The best value of each FoM is indicated in bold. Runs \opsimdbref{db:baseCadence} and \opsimdbref{db:opstwoPS} refer to the Baseline and PanSTARRS-like strategies, respectively. Column \opsimdbref{db:NormalGalacticPlane} refers to a recently-completed OpSim run that includes the Plane in Wide-Fast-Deep observations. See \autoref{sec:MW_Astrometry}.}
\label{tab_SummaryMWAstrometry}
\end{table}


%%%% SUMMARY TABLE FINISHES HERE


\subsubsection{Further work on Astrometry Metrics}

The MAF metrics presented in Sections \ref{sec:\secname:MW_Astrometry_OpSim} and \ref{sec:\secname:MW_Astrometry_metrics} are only part of the
study of LSST's predicted astrometric performance.  Detailed simulations
and studies need to be done in many other areas as part of the
prediction and verification of LSST's astrometric performance.  Among
the most important are the following.
\begin{itemize}
\item How well do galaxies perform as astrometric reference objects? Are certain shapes or colors better than others? What is the
surface density of ``good" astrometric reference galaxies as a function of filter?
\item How well can we identify optically point-like QSOs that will be useful in matching the optical reference frame to the ICRS?
%\item Given the LSST exposure time, site, and physical characteristics, how can we mitigate the limitations on astrometric accuracy imposed by the seeing and local atmospheric turbulence?
\item How does the astrometric performance depend on stellar density? If there are fields in which photometry is only possible via difference imaging, what are the limitations
on astrometry in these fields?
%\item What tools do we need to compare the general and specific agreement between the {\it Gaia} results and the LSST results?
\item Does the ``brighter-wider" effect in the deep-depletion CCDs introduce a magnitude term into the centroid positions?
\end{itemize}

% ====================================================================

\subsection{Conclusions}

Here we answer the ten questions posed in
\autoref{sec:intro:evaluation:caseConclusions}:

\begin{description}

\item[Q1:] {\it Does the science case place any constraints on the
tradeoff between the sky coverage and coadded depth? For example, should
the sky coverage be maximized (to $\sim$30,000 deg$^2$, as e.g., in
Pan-STARRS) or the number of detected galaxies (the current baseline 
of 18,000 deg$^2$)?}

\item[A1:] We do expect tradeoffs between depth and sky
  coverage, but we do not yet have the FoM evaluations to set
  quantitative constraints. For example, we expect some combination of
  depth and survey volume would optimize the completeness to objects
  among the populations in the Solar Neighborhood. More generally,
  perhaps, in fields away from the galactic mid-plane, the
  lengthscales over which the proper motion zeropoints can be
  accurately constrained will depend on the spatial density of
  well-measured background galaxies (finer lengthscale corresponding
  to greater co-added depth). The depth must therefore be sufficient
  to sample enough of these galaxies to constrain variations of
  astrometric zeropoint on lengthscales at least as fine as those
  imposed by the LSST system itself (or the atmosphere, whichever is
  finer). We anticipate that this tradeoff can be informed by
  simulation under a set of assumptions for these variations.

\item[Q2:] {\it Does the science case place any constraints on the
tradeoff between uniformity of sampling and frequency of  sampling? For
example, a rolling cadence can provide enhanced sample rates over a part
of the survey or the entire survey for a designated time at the cost of
reduced sample rate the rest of the time (while maintaining the nominal
total visit counts).}

\item[A2:] Yes, although for astrometry the language of these
  constraints is slightly different. The parallactic ellipse must be
  sufficiently covered, the correlation between hour angle and
  parallax factor must be minimized, and the visits must be
  sufficiently distributed (both within a year and over the ten-year
  time baseline) to produce the best precision in both proper motion and parallax. See Section \ref{sec:MW_Astrometry:MW_Astrometry_metrics}.

\item[Q3:] {\it Does the science case place any constraints on the
tradeoff between the single-visit depth and the number of visits
(especially in the $u$-band where longer exposures would minimize the
impact of the readout noise)?}

\item[A3:] More visits at the standard exposure time are
  generally preferred to a few visits with longer exposures, in order
  to achieve as broad a temporal coverage as possible (see Section
  \ref{sec:MW_Astrometry:MW_Astrometry_metrics}). The $u$-band itself
  is likely to be of limited use for astrometry (except possibly for
  extremely blue objects with little signal in any of the other
  filters) due to differential chromatic refraction (DCR), however of
  course $u$-band will still be useful for photometric constraints.

\item[Q4:] {\it Does the science case place any constraints on the
Galactic plane coverage (spatial coverage, temporal sampling, visits per
band)?}

\item[A4:] Not for the example of detecting Galactic Halo
  streams via proper motions (Section
  \ref{sec:MW_Astrometry:MW_Astrometry_measurements}). For the Solar
  Neighborhood populations, avoiding the inner Galactic mid-plane
  would obviously reduce the completeness of the census of nearby
  objects with parallax determinations due to the reduction in total
  area surveyed. However, this reduction may be incremental rather
  than serious. The impact of an inner-plane zone of avoidance on the
  recovery of the parameters describing these constituent populations
  has not yet been evaluated. Of course, this all changes for
  astrometry of objects of interest that lie in the inner plane (see
  also Section \ref{sec:MW_Disk}), where (for example) the reduced
  proper motion will be a useful diagnostic. At present, however, the
  performance of the LSST software stack towards crowded fields is
  as-yet unknown. As this performance becomes better understood, it
  will be possible to quantitatively compare strategies for astrometry
  towards the inner plane.

\item[Q5:] {\it Does the science case place any constraints on the
fraction of observing time allocated to each band?}

\item[A5:] Yes, but indirectly through the requirement to
  measure all the populations of interest in the Solar
  Neighborhood. Making the assumption that this science case requires
  parallax measurements for extremely blue objects as well as
  extremely red objects, which might each be measurable only in a
  single very red or blue filter, would suggest at a minimum that the
  coverage considerations of Section
  \ref{sec:MW_Astrometry:MW_Astrometry_metrics} be applied to
  observations in $u$~and $Y$ filters separately, as well as at least
  one mid-range filter. However the quantitative impact on population
  recovery from various filter-distributions has yet to be assessed at
  this date. Further work is needed to determine if the increased
  sensitivity of $u$-band astrometry to DCR relative to $g$-band would
  prevent its use for astrometry.

\item[Q6:] {\it Does the science case place any constraints on the
cadence for deep drilling fields?}

\item[A6:] If precision astrometry is desired for the deep drilling fields, then the considerations of Section \ref{sec:MW_Astrometry:MW_Astrometry_metrics} apply to those fields as well.

\item[Q7:] {\it Assuming two visits per night, would the science case
benefit if they are obtained in the same band or not?}

\item[A7:] While detailed investigation is still pending, we
  expect that using different filters within the same night would be
  preferred to allow better constraint of DCR effects. Doing different
  filters on the same night might reduce the number of free parameters
  (like seeing and parallax factor) and give more pairs for direct
  filter-A vs. filter-B astrometry.

\item[Q8:] {\it Will the case science benefit from a special cadence
prescription during commissioning or early in the survey, such as:
acquiring a full 10-year count of visits for a small area (either in all
the bands or in a  selected set); a greatly enhanced cadence for a small
area?}

\item[A8:] It is vital for astrometry that at least a few fields
  be observed with both sufficient parallax factor coverage and
  sufficient number of visits, early in the survey, to demonstrate
  parallax precision specified in the Science Requirements
  Document. In these fields, sufficient exposures must be reserved for
  the entire 10-year survey baseline so that proper motion precision
  is not too badly compromised in these fields. This combination of
  factors may require dedicated commissioning observations of these
  fields in addition to the 10-year survey operations. In addition,
  however, at least a few fields must be observed at a variety of
  values of single-visit achieved depth, and FWHM, in order to
  constrain the degree to which FWHM will actually predict the
  achieved astrometric precision (see also the answer to Q10
  below). This second set of requirements may also be best served by
  dedicated commissioning observations.

\item[Q9:] {\it Does the science case place any constraints on the
sampling of observing conditions (e.g., seeing, dark sky, airmass),
possibly as a function of band, etc.?}

\item[A9:] The observations need to be planned in such a way
  that the correlation between parallax and hour-angle is minimized,
  to avoid deneracies between the motion due to atmospheric refraction and the motion that is sought due to parallax. See
  Section \ref{sec:MW_Astrometry:MW_Astrometry_metrics}.

\item[Q10:] {\it Does the case have science drivers that would require
real-time exposure time optimization to obtain nearly constant
single-visit limiting depth?}

\item[A10:] While optimization on an exposure-to-exposure basis
  is perhaps unlikely, {\it selection} between observations in
  response to conditions (on a timescale of perhaps 10 minutes) will
  be crucial to maximize achieved astrometric precision. The rules by
  which this selection would proceed, still need to be charted. For
  example, while maintaining limiting depth might suggest shorter
  exposure times when the FWHM is narrow, this may not translate to
  improved astrometric error across an LSST chip, because the
  lengthscales of the turbulence driving the FWHM is not the same as
  that of the turbulence driving astrometric error across an LSST chip.

\end{description}

\navigationbar

% --------------------------------------------------------------------


% PJM: moved the following to FutureWork, while the metric(s) is/are being implemented
% % ====================================================================
%+
% SECTION:
%    section-name.tex  % eg lenstimedelays.tex
%
% CHAPTER:
%    chapter.tex  % eg cosmology.tex
%
% ELEVATOR PITCH:
%    Explain in a few sentences what the relevant discovery or
%    measurement is going to be discussed, and what will be important
%    about it. This is for the browsing reader to get a quick feel
%    for what this section is about.
%
% COMMENTS:
%
%
% BUGS:
%
%
% AUTHORS:
%    Phil Marshall (@drphilmarshall)  - put your name and GitHub username here!
%-
% ====================================================================

\section{Mapping the Milky Way Halo}
\def\secname{MW_Halo}\label{sec:\secname} % For example, replace "keyword" with "lenstimedelays"

\noindent{\it Kathy Vivas, Colin Slater, David Nidever}  % (Writing team)

The study of the Halo of the Milky Way is of the most importance not only to understand
the formation and early evolution of our own galaxy, but also to test 
test current models of hierarchical galaxy formation. 
LSST will provide an unprecedented combination of
area, depth, multi-band, multi-epoch information for pursuing detail studies
of the structure of this old Galactic component. We focus here in three
specific projects that can be pursued with LSST. We define metrics that can
be calculated in order to quantify the feasibility of the projects under different
observational strategies. We expect more projects will join later.

RR Lyrae stars have been known for several decades as excellent tracers
of the halo population. They are not only old stars ($>10$ Gyrs) but they are
also excellent standard candles that allow to build 3-dimensional maps. 
The halo of the Milky Way has been now surveyed in a very large extension up to 
$\sim 60-80$ kpc from the Galactic center (refs). Beyond that, the halo is
mostly uncharted territory.
From these RR Lyrae surveys, we have learned that the halo is filled with substructures
which are usually interpreted as debris from destroyed satellite galaxies. The smooth 
component of the RR Lyrae distribution is well described
with a power-law of the mean number density of RR Lyrae stars as a function of
galactocentric distance, which gets steeper after $\sim 30$ kpc (refs). 
Thus, beyond $\sim 60$ kpc, few field RR Lyrae stars are expected. However, we expect that 
any RR Lyrae star beyond this distance may be part of either debris material or distant
satellite galaxies of low luminosity that have been escaped detection until now (refs). 
LCDM models predict debris as far as $0.5$~Mpc from the galactic center
This is the territory that will be explored by LSST.

Similarly, red giant stars can be used to trace the structure of the halo up to large
distances. They have the advantage 
of being bright and numerous stars. 
%%COLIN, PLEASE STEP IN HERE.

Fainter than these two tracers, main sequence stars stand up as a tool for studying
the Halo. They are the most numerous type of stars available and statistical studies 
are possible. Using the technique of photometric metallicities (Ivezic et al 2008), 
SDSS provided unprecedented maps of the metallicity distribution up to  $\sim 10$ 
kpc from the Galactic center, unveiling not only the mean metallicity distribution 
of the halo but also, sub-structures within the halo. This kind of works will be extended
to the outermost parts of the Galaxy with LSST data.


% --------------------------------------------------------------------

\subsection{Target measurements and discoveries}
\label{sec:keyword:MW_Halo_targets}

The three projects just described require the discovery and/or measurement of the following 
type of objects:

\begin{itemize}

\item RR Lyrae stars: These are bright horizontal-branch variable stars with
periods between 0.2 to 1.0 days and large amplitudes, particularly in the bluer 
bandpasses (g amplitudes $0.5 - 1.5$~mag). Optimal use of LSST for discovering
RR Lyraes involves the simultaneous use of multi-band time series (refs).
Chapter \ref{chp:variables} discusses with details the discovery metrics for RR Lyrae stars.
A particularly valuable measurement for studies in the halo is the infrared mean
magnitudes z and y since they provide the most accurate way to obtain 
distances.

\item Main sequence stars: lacking any distinguishable variability, the
challenge in selecting a large and clean sample of main sequence stars comes
from tremendous number of small and nearly-unresolved galaxies present at
faint magnitudes. Precise star/galaxy separation is thus the limiting factor
on the useful depth of the main sequence sample. In addition to identifying
dwarfs, using dwarfs to map the metallicity distribution of the halo requires
precise u-band data, since it exhibits the strongest metallicity dependence of
the LSST filters.

\item Red Giants: due to their intrinsic luminosity red giants will be sample
a far deeper volume than main sequence stars at similar apparent magnitudes,
but they must first be identified and separated from the very numerous main
sequence stars present in the field. A gravity-sensitive photometric index can
be used for separating efficiently giants from dwarfs (refs). The u magnitude
is an essential ingredient in this process and it is necessary to follow-up
the behavior of the u limiting magnitude under different observational
strategies.

\end{itemize}

% --------------------------------------------------------------------

\subsection{Metrics}
\label{sec:keyword:MW_Halo_metrics}

\textbf{Star-Galaxy Separation:} For main sequence stars, the useful depth of
the survey will likely not be the photometric detection limit but will instead
be set by the ability to differentiate stars from unresolved background
galaxies. Towards faint magnitudes the contamination by galaxies worsens
significantly for several reasons: the number of galaxies is rising
substantially, the angular size of galaxies is shrinking, and our ability to
distinguish stars from marginally resolved galaxies diminishes for faint
sources simply due to photon statistics. While the fundamental properties of
the contaminant sources is beyond our control, our ability to reject these
sources depends on survey parameters such as the distribution of seeing across
visits and the depth of these visits.

Our star galaxy separation metric accounts for these factors, using a modeling
framework described in (s/g paper ref). This model uses the distribution of
galaxies in size and number, derived from HST COSMOS observations, along with
a fully Bayesian model decision formalism to compute the expected completeness
and contamination in star-galaxy separation. Computationally for each position
in the survey footprint we interpolate the results from that work on a grid in
seeing, galaxy size, and coadd depth, then integrate over the distribution of
galaxy sizes.

[… will have more to say about the actual metric once it is fully implemented]


% --------------------------------------------------------------------

\subsection{OpSim Analysis}
\label{sec:keyword:MW_Halo_analysis}

\begin{itemize}

\item Comment on the north ecliptic spur in enigma\_1189. Is it close to the
WFD S/G limit?

\item Pan-STARRS-like cadence ops2\_1092, observing up to dec +15. How much
volume do we gain, and how much do we lose in the WFD survey?

\end{itemize}


% --------------------------------------------------------------------

\subsection{Discussion}
\label{sec:keyword:MW_Halo_discussion}

Discussion: what risks have been identified? What suggestions could be
made to improve this science project's figure of merit, and mitigate
the identified risks?


% ====================================================================

\navigationbar


% Under development:
% \input{MilkyWay/MW_Bulge.tex}

% Under development:
% \input{MilkyWay/MW_LocalVolume}

% ====================================================================
%+
% SECTION:
%    section-name.tex  % eg lenstimedelays.tex
%
% CHAPTER:
%    chapter.tex  % eg cosmology.tex
%
% ELEVATOR PITCH:
%    Explain in a few sentences what the relevant discovery or
%    measurement is going to be discussed, and what will be important
%    about it. This is for the browsing reader to get a quick feel
%    for what this section is about.
%
% COMMENTS:
%
%
% BUGS:
%
%
% AUTHORS:
%    Phil Marshall (@drphilmarshall)  - put your name and GitHub username here!
%-
% ====================================================================

\section{Mapping the Milky Way Halo}
\def\secname{MW_Halo}\label{sec:\secname} % For example, replace "keyword" with "lenstimedelays"

\noindent{\it Kathy Vivas, Colin Slater, David Nidever}  % (Writing team)

The study of the Halo of the Milky Way is of the most importance not only to understand
the formation and early evolution of our own galaxy, but also to test 
test current models of hierarchical galaxy formation. 
LSST will provide an unprecedented combination of
area, depth, multi-band, multi-epoch information for pursuing detail studies
of the structure of this old Galactic component. We focus here in three
specific projects that can be pursued with LSST. We define metrics that can
be calculated in order to quantify the feasibility of the projects under different
observational strategies. We expect more projects will join later.

RR Lyrae stars have been known for several decades as excellent tracers
of the halo population. They are not only old stars ($>10$ Gyrs) but they are
also excellent standard candles that allow to build 3-dimensional maps. 
The halo of the Milky Way has been now surveyed in a very large extension up to 
$\sim 60-80$ kpc from the Galactic center (refs). Beyond that, the halo is
mostly uncharted territory.
From these RR Lyrae surveys, we have learned that the halo is filled with substructures
which are usually interpreted as debris from destroyed satellite galaxies. The smooth 
component of the RR Lyrae distribution is well described
with a power-law of the mean number density of RR Lyrae stars as a function of
galactocentric distance, which gets steeper after $\sim 30$ kpc (refs). 
Thus, beyond $\sim 60$ kpc, few field RR Lyrae stars are expected. However, we expect that 
any RR Lyrae star beyond this distance may be part of either debris material or distant
satellite galaxies of low luminosity that have been escaped detection until now (refs). 
LCDM models predict debris as far as $0.5$~Mpc from the galactic center
This is the territory that will be explored by LSST.

Similarly, red giant stars can be used to trace the structure of the halo up to large
distances. They have the advantage 
of being bright and numerous stars. 
%%COLIN, PLEASE STEP IN HERE.

Fainter than these two tracers, main sequence stars stand up as a tool for studying
the Halo. They are the most numerous type of stars available and statistical studies 
are possible. Using the technique of photometric metallicities (Ivezic et al 2008), 
SDSS provided unprecedented maps of the metallicity distribution up to  $\sim 10$ 
kpc from the Galactic center, unveiling not only the mean metallicity distribution 
of the halo but also, sub-structures within the halo. This kind of works will be extended
to the outermost parts of the Galaxy with LSST data.


% --------------------------------------------------------------------

\subsection{Target measurements and discoveries}
\label{sec:keyword:MW_Halo_targets}

The three projects just described require the discovery and/or measurement of the following 
type of objects:

\begin{itemize}

\item RR Lyrae stars: These are bright horizontal-branch variable stars with
periods between 0.2 to 1.0 days and large amplitudes, particularly in the bluer 
bandpasses (g amplitudes $0.5 - 1.5$~mag). Optimal use of LSST for discovering
RR Lyraes involves the simultaneous use of multi-band time series (refs).
Chapter \ref{chp:variables} discusses with details the discovery metrics for RR Lyrae stars.
A particularly valuable measurement for studies in the halo is the infrared mean
magnitudes z and y since they provide the most accurate way to obtain 
distances.

\item Main sequence stars: lacking any distinguishable variability, the
challenge in selecting a large and clean sample of main sequence stars comes
from tremendous number of small and nearly-unresolved galaxies present at
faint magnitudes. Precise star/galaxy separation is thus the limiting factor
on the useful depth of the main sequence sample. In addition to identifying
dwarfs, using dwarfs to map the metallicity distribution of the halo requires
precise u-band data, since it exhibits the strongest metallicity dependence of
the LSST filters.

\item Red Giants: due to their intrinsic luminosity red giants will be sample
a far deeper volume than main sequence stars at similar apparent magnitudes,
but they must first be identified and separated from the very numerous main
sequence stars present in the field. A gravity-sensitive photometric index can
be used for separating efficiently giants from dwarfs (refs). The u magnitude
is an essential ingredient in this process and it is necessary to follow-up
the behavior of the u limiting magnitude under different observational
strategies.

\end{itemize}

% --------------------------------------------------------------------

\subsection{Metrics}
\label{sec:keyword:MW_Halo_metrics}

\textbf{Star-Galaxy Separation:} For main sequence stars, the useful depth of
the survey will likely not be the photometric detection limit but will instead
be set by the ability to differentiate stars from unresolved background
galaxies. Towards faint magnitudes the contamination by galaxies worsens
significantly for several reasons: the number of galaxies is rising
substantially, the angular size of galaxies is shrinking, and our ability to
distinguish stars from marginally resolved galaxies diminishes for faint
sources simply due to photon statistics. While the fundamental properties of
the contaminant sources is beyond our control, our ability to reject these
sources depends on survey parameters such as the distribution of seeing across
visits and the depth of these visits.

Our star galaxy separation metric accounts for these factors, using a modeling
framework described in (s/g paper ref). This model uses the distribution of
galaxies in size and number, derived from HST COSMOS observations, along with
a fully Bayesian model decision formalism to compute the expected completeness
and contamination in star-galaxy separation. Computationally for each position
in the survey footprint we interpolate the results from that work on a grid in
seeing, galaxy size, and coadd depth, then integrate over the distribution of
galaxy sizes.

[… will have more to say about the actual metric once it is fully implemented]


% --------------------------------------------------------------------

\subsection{OpSim Analysis}
\label{sec:keyword:MW_Halo_analysis}

\begin{itemize}

\item Comment on the north ecliptic spur in enigma\_1189. Is it close to the
WFD S/G limit?

\item Pan-STARRS-like cadence ops2\_1092, observing up to dec +15. How much
volume do we gain, and how much do we lose in the WFD survey?

\end{itemize}


% --------------------------------------------------------------------

\subsection{Discussion}
\label{sec:keyword:MW_Halo_discussion}

Discussion: what risks have been identified? What suggestions could be
made to improve this science project's figure of merit, and mitigate
the identified risks?


% ====================================================================

\navigationbar



% ====================================================================
%+
% SECTION:
%    MW_FutureWork.tex
%
% CHAPTER:
%    galaxy.tex
%
% ELEVATOR PITCH:
%    Ideas for future metric investigation, with quantitaive analysis
%    still pending.
%-
% ====================================================================

\section{Future Work}
\def\secname{MW_future}\label{sec:\secname}

In this section we provide a short compendium of science cases that
are either still being developed, or that are deserving of quantitative
MAF analysis at some point in the future.

% ====================================================================

% ====================================================================
%+
% SECTION:
%    section-name.tex  % eg lenstimedelays.tex
%
% CHAPTER:
%    chapter.tex  % eg cosmology.tex
%
% ELEVATOR PITCH:
%    Explain in a few sentences what the relevant discovery or
%    measurement is going to be discussed, and what will be important
%    about it. This is for the browsing reader to get a quick feel
%    for what this section is about.
%
% COMMENTS:
%
%
% BUGS:
%
%
% AUTHORS:
%    Phil Marshall (@drphilmarshall)  - put your name and GitHub username here!
%-
% ====================================================================

\section{Dust in the Milky Way}
\def\secname{MW_Dust}\label{sec:\secname} % For example, replace "keyword" with "lenstimedelays"

\noindent{\it Peregrine M. McGehee} % (Writing team)

% This individual section will need to describe the particular
% discoveries and measurements that are being targeted in this section's
% science case. It will be helpful to think of a ``science case" as a
% ``science project" that the authors {\it actually plan to do}. Then,
% the sections can follow the tried and tested format of an observing
% proposal: a brief description of the investigation, with references,
% followed by a technical feasibility piece. This latter part will need
% to be quantified using the MAF framework, via a set of metrics that
% need to be computed for any given observing strategy to quantify its
% impact on the described science case. Ideally, these metrics would be
% combined in a well-motivated figure of merit. The section can conclude
% with a discussion of any risks that have been identified, and how
% these could be mitigated.

Interstellar dust is a significant constituent of the Galaxy. Its composition and associated extinction
properties tell us about the material and environments in which stars and their planets are formed.
Dust also presents an obstacle for a wide-range of astronomical observations, causing light from
stars in the plane of the Milky Way to be severely dimmed and causing the apparent colors of
objects observed in any direction to be shifted from their intrinsic values. These color shifts
are dependent upon the dust column density along the line of sight and the radiative transport
properties of the dust grains.

To first order, i.e. neglecting the effects of heterochromatic extinction, the absorption of light
in each band due to dust is dependent upon the column density, related to $E(B-V)$, and the
nature of the dust grains, as parameterized by the ratio of general to selection extinction 
in the Johnson $B$ and $V$ bands, defined as $R_V = A_V /E(B − V)$. 
In the low-density diffuse ISM, $R_V$ has a value $\sim 3.1$, while in
dense molecular clouds $R_V$ can be higher with values $4 < R_V < 6$.

In general, however, the use of broad band photometry requires attention to the intrinsic
SEDs of the background stars in order to correct for heterochromatic variations in the
effective reddening law. As discussed in the LSST Science Book, possession of an accurate
dust map is important to many astrophysical studies. The two most significant all-sky maps generated
in the past two decades are the SFD98 maps based on IRAS observations, and the recent thermal
dust maps derived from Planck submillimeter data. The angular resolutions of both maps are similar - 
between 4 to 6 arcminutes.

Both of the aforementioned maps are strictly two-dimensional and conntain no information about the 
distribution of dust along the line of sight. A third dimension can be obtained by analysis of
accurate stellar photometry which constrain both the reddening $E(B-V)$ and $R_V$ towards 
individual stars. This approach requires determination of the intrinsic stellar colors and the photometric
parallax of each star in the presence of an unknown amount and law of extinction.
Recent work on 3-D maps include the Bayesian analysis method based on Pan-STARRS 1 data 
(Green et al. 2015, ApJ, 810, 25) and an alternative technique using SDSS photometry of 
M dwarfs (McGehee et al. 2016, in preparation). 

% --------------------------------------------------------------------

\subsection{Target measurements and discoveries}
\label{sec:\secname:targets}

The use of stellar samples to create three-dimensional extinction maps has an established history
beginning with the work of Neckel \& Klare (1980); however these, including studies based on SDSS and
PS1 photometry, are typically limited to heliocentric distances of $\sim$4 kpc. In the full co-added survey,
LSST will be able to map dust structures out to distances exceeding 40 kpc, thus revealing a
detailed picture of this component of the Milky Way Galaxy.

The Pan-STARRS1 survey (PS1) has
produced a three-dimensional dust map of the region of the sky covered
in their 3$\pi$ survey (which excludes a large part of the Galactic
Plane toward the south). Such maps are necessary to accurately measure
the intrinsic luminosities and colors of both Galactic and
extragalactic sources. 
Green et al. (2015) estimated $R_V$ along sightlines having higher 
reddening values as well as reddening values.
The PS1 map saturates at
extinctions $E(B-V) > 1.5$ as their tracer stars fall out of the
survey catalogs fainter than $g\sim 22$, meaning that this
high-fidelity map does not extend uniformly to within a few degrees of
the midplane. In addition, it only extends to a distance of about 4.5
kpc. Deep LSST data will allow this map to be extended to much higher
extinctions and larger distances. Owing to the high extinction and the
use of blue filters, this project is less affected by crowding than
other projects requiring photometry in the Plane. 

In comparison, the SDSS survey makes use of M dwarf locus in $(g-r,r-i)$ being
nearly perpendicular to the reddening vector in that color-color space. This 
allows mapping of a reddening-invariant index to the intrinsic stellar $g-i$ color
and subsequent deterimation of the light-of-sight reddening. This approach assumes
a set extinction law, i.e $R_V = 3.1$, in order compute the reddening-invariant 
index from the observed $g-r$ and $r-i$ colors. Given the relative faintness of M dwarfs,
this technique is distance limited to $\sim$1 kpc when based on SDSS data.

The LSST will be in a unique position to measure the changes in the observed reddening vector
due to $R_V$ variations due to its superb photometric accuracy. 
Both of the dust survey techniques mentioned here can be used on LSST data, and perhaps other 
methods will be developed before the start of survey operations. T

% --------------------------------------------------------------------

\subsection{Metrics}

\label{sec:\secname:metrics}

Production of a 3-D map of the dust component of the ISM based on LSST photometry will tell us 
how much dust is present, what type it is, and where it is along the line of sight. 
The latter concern brings in issues of how to determine stellar photometric parallaxes ($\mu = m-M$) under
an unknown reddening.

The dust maps that are created will consist of the median and variance of $E(B-V)$ and $R_V$ expressed as functions of 
$\mu$ under a suitable binning scheme. We can create simple Figure of Merit maps that lose the 
$\mu$ dependency by computing the mean and variance of the measured variances in $E(B-V)$ and $R_V$ 
over the $\mu$ bins.

With the possible exception of sightlines towards star formation regions, the spacing in time of the visits 
doesn't matter for dust studies. In the case of active star formation regions it is possible that changes in the
ISM could be apparent over the lifetime of the survey.
Pushing to fainter magnitudes (which means both better seeing and longer exposures) matters, 
both because we want more stars, and in particular, we want more stars behind the dust.  


{\bf Metric 1: Uncertainty and bias in $E(B-V)$~estimates as a
  function of location on-sky.} Dependencies:

\begin{itemize}
  \item Stellar population throughout the survey (e.g. Knut / Peter developments; TRILEGAL?);
    \item Dust map throughout the survey region;
    \item Scale photometric error predictions for each band from program requirements per exposure;
      \item Produce formal estimate on the error in extinction and reddening as a function of position on-sky within the survey.
\end{itemize}


% --------------------------------------------------------------------

\subsection{OpSim Analysis}
\label{sec:\secname:analysis}

OpSim analysis: how good would the default observing strategy be, at
the time of writing for this science project?


% --------------------------------------------------------------------

\subsection{Discussion}
\label{sec:\secname:discussion}

Discussion: what risks have been identified? What suggestions could be
made to improve this science project's figure of merit, and mitigate
the identified risks?


% ====================================================================

\navigationbar


% ====================================================================

% ====================================================================
%+
% SECTION:
%    section-name.tex  % eg lenstimedelays.tex
%
% CHAPTER:
%    chapter.tex  % eg cosmology.tex
%
% ELEVATOR PITCH:
%    Explain in a few sentences what the relevant discovery or
%    measurement is going to be discussed, and what will be important
%    about it. This is for the browsing reader to get a quick feel
%    for what this section is about.
%
% COMMENTS:
%
%
% BUGS:
%
%
% AUTHORS:
%    Phil Marshall (@drphilmarshall)  - put your name and GitHub username here!
%-
% ====================================================================

\section{Mapping the Milky Way Halo}
\def\secname{MW_Halo}\label{sec:\secname} % For example, replace "keyword" with "lenstimedelays"

\noindent{\it Kathy Vivas, Colin Slater, David Nidever}  % (Writing team)

The study of the Halo of the Milky Way is of the most importance not only to understand
the formation and early evolution of our own galaxy, but also to test 
test current models of hierarchical galaxy formation. 
LSST will provide an unprecedented combination of
area, depth, multi-band, multi-epoch information for pursuing detail studies
of the structure of this old Galactic component. We focus here in three
specific projects that can be pursued with LSST. We define metrics that can
be calculated in order to quantify the feasibility of the projects under different
observational strategies. We expect more projects will join later.

RR Lyrae stars have been known for several decades as excellent tracers
of the halo population. They are not only old stars ($>10$ Gyrs) but they are
also excellent standard candles that allow to build 3-dimensional maps. 
The halo of the Milky Way has been now surveyed in a very large extension up to 
$\sim 60-80$ kpc from the Galactic center (refs). Beyond that, the halo is
mostly uncharted territory.
From these RR Lyrae surveys, we have learned that the halo is filled with substructures
which are usually interpreted as debris from destroyed satellite galaxies. The smooth 
component of the RR Lyrae distribution is well described
with a power-law of the mean number density of RR Lyrae stars as a function of
galactocentric distance, which gets steeper after $\sim 30$ kpc (refs). 
Thus, beyond $\sim 60$ kpc, few field RR Lyrae stars are expected. However, we expect that 
any RR Lyrae star beyond this distance may be part of either debris material or distant
satellite galaxies of low luminosity that have been escaped detection until now (refs). 
LCDM models predict debris as far as $0.5$~Mpc from the galactic center
This is the territory that will be explored by LSST.

Similarly, red giant stars can be used to trace the structure of the halo up to large
distances. They have the advantage 
of being bright and numerous stars. 
%%COLIN, PLEASE STEP IN HERE.

Fainter than these two tracers, main sequence stars stand up as a tool for studying
the Halo. They are the most numerous type of stars available and statistical studies 
are possible. Using the technique of photometric metallicities (Ivezic et al 2008), 
SDSS provided unprecedented maps of the metallicity distribution up to  $\sim 10$ 
kpc from the Galactic center, unveiling not only the mean metallicity distribution 
of the halo but also, sub-structures within the halo. This kind of works will be extended
to the outermost parts of the Galaxy with LSST data.


% --------------------------------------------------------------------

\subsection{Target measurements and discoveries}
\label{sec:keyword:MW_Halo_targets}

The three projects just described require the discovery and/or measurement of the following 
type of objects:

\begin{itemize}

\item RR Lyrae stars: These are bright horizontal-branch variable stars with
periods between 0.2 to 1.0 days and large amplitudes, particularly in the bluer 
bandpasses (g amplitudes $0.5 - 1.5$~mag). Optimal use of LSST for discovering
RR Lyraes involves the simultaneous use of multi-band time series (refs).
Chapter \ref{chp:variables} discusses with details the discovery metrics for RR Lyrae stars.
A particularly valuable measurement for studies in the halo is the infrared mean
magnitudes z and y since they provide the most accurate way to obtain 
distances.

\item Main sequence stars: lacking any distinguishable variability, the
challenge in selecting a large and clean sample of main sequence stars comes
from tremendous number of small and nearly-unresolved galaxies present at
faint magnitudes. Precise star/galaxy separation is thus the limiting factor
on the useful depth of the main sequence sample. In addition to identifying
dwarfs, using dwarfs to map the metallicity distribution of the halo requires
precise u-band data, since it exhibits the strongest metallicity dependence of
the LSST filters.

\item Red Giants: due to their intrinsic luminosity red giants will be sample
a far deeper volume than main sequence stars at similar apparent magnitudes,
but they must first be identified and separated from the very numerous main
sequence stars present in the field. A gravity-sensitive photometric index can
be used for separating efficiently giants from dwarfs (refs). The u magnitude
is an essential ingredient in this process and it is necessary to follow-up
the behavior of the u limiting magnitude under different observational
strategies.

\end{itemize}

% --------------------------------------------------------------------

\subsection{Metrics}
\label{sec:keyword:MW_Halo_metrics}

\textbf{Star-Galaxy Separation:} For main sequence stars, the useful depth of
the survey will likely not be the photometric detection limit but will instead
be set by the ability to differentiate stars from unresolved background
galaxies. Towards faint magnitudes the contamination by galaxies worsens
significantly for several reasons: the number of galaxies is rising
substantially, the angular size of galaxies is shrinking, and our ability to
distinguish stars from marginally resolved galaxies diminishes for faint
sources simply due to photon statistics. While the fundamental properties of
the contaminant sources is beyond our control, our ability to reject these
sources depends on survey parameters such as the distribution of seeing across
visits and the depth of these visits.

Our star galaxy separation metric accounts for these factors, using a modeling
framework described in (s/g paper ref). This model uses the distribution of
galaxies in size and number, derived from HST COSMOS observations, along with
a fully Bayesian model decision formalism to compute the expected completeness
and contamination in star-galaxy separation. Computationally for each position
in the survey footprint we interpolate the results from that work on a grid in
seeing, galaxy size, and coadd depth, then integrate over the distribution of
galaxy sizes.

[… will have more to say about the actual metric once it is fully implemented]


% --------------------------------------------------------------------

\subsection{OpSim Analysis}
\label{sec:keyword:MW_Halo_analysis}

\begin{itemize}

\item Comment on the north ecliptic spur in enigma\_1189. Is it close to the
WFD S/G limit?

\item Pan-STARRS-like cadence ops2\_1092, observing up to dec +15. How much
volume do we gain, and how much do we lose in the WFD survey?

\end{itemize}


% --------------------------------------------------------------------

\subsection{Discussion}
\label{sec:keyword:MW_Halo_discussion}

Discussion: what risks have been identified? What suggestions could be
made to improve this science project's figure of merit, and mitigate
the identified risks?


% ====================================================================

\navigationbar


% ====================================================================

\subsection{Other Ideas}

\credit{willclarkson}, \credit{akvivas}, \credit{vpdebattista}

In this final section we provide an extremely brief list of important science
cases that are still in an early stage of development, but that are
deserving of quantitative MAF analysis in the future:

\begin{itemize}
  \item {\it Formation history of the Bulge and present-day balance of
  populations:} Sensitivity to metallicity and age distribution of Bulge
  objects near the Main Sequence Turn-off;
  \item {\it Migration and heating in the Milky Way disk:} Error and
  bias in the determination of components in the (velocity dispersion vs
  metallicity) diagram, for disk populations along various lines of
  sight (e.g. Loebman et al. 2016 ApJ 818 L6);
  \item Fraction of Local-Volume objects discovered as a function of
  survey strategy.
\end{itemize}

% ====================================================================

\navigationbar

