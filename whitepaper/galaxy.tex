% --------------------------------------------------------------------

\chapter[The Milky Way Galaxy]{The Milky Way Galaxy}
\def\chpname{galaxy}\label{chp:\chpname}

Chapter editors:
\credit{willclarkson},
\credit{akvivas}

Contributing authors:
\credit{bethwillman},
\credit{dnidever},
\credit{ivezic},
\credit{ctslater},
\credit{pmmcgehee},
\credit{cbritt4},
\credit{dgmonet},
\credit{caprastro},
\credit{DanaCD},
\credit{jgizis},
\credit{mliu},
\credit{vpdebattista},
\credit{chomiuk},
\credit{yoachim}
% {\it and others to follow}

\section*{Summary}
\addcontentsline{toc}{section}{~~~~~~~~~Summary}

Galactic science cases fall roughly into two observational categories
based on stellar density and/or Galactic latitude. (i) Strategy
assessment for the {\it high-density or low-latitude} cases is dominated
by large variation in total time allocation for the inner Plane (where
most of the Galaxy's stars are found), since the current strategy
options tend to complete their inner-Plane observations within the first
$\sim 7$~months of the survey (see Chapter \ref{chp:cadexp}). Any
science case requiring more than a years' coverage in these regions will
not be well-served by most of the currently-run strategy simulations.
Quantitatively, figures of merit for these science cases therefore
suggest the baseline cadence to be factors $\sim 6-60$~worse than the
two comparison strategies chosen that allocate more time to the inner
Plane (Tables \ref{tab_SummaryMWDisk} \& \ref{tab_SummaryMWAstrometry}).
While figures of merit for a number of science cases in this category
have been specified and are in the implementation phase (Section
\ref{sec:MW_Disk:MW_Disk_metrics}), at present it is at least as
important that a wider range of strategies be specified and run {\it
with inner-Plane coverage spread over the full ten-year survey time
baseline.} We encourage the reader to supply suggestions for simulations
meeting this need. (ii). For the {\it any-latitude} category (including
the Solar Neighborhood), basic figures of merit including astrometric
characterization have been run for three choices of strategy (Section
\ref{sec:MW_Astrometry:MW_Astrometry_metrics}). While some degree of
calibration is ongoing, at present roughly $12\%$~of fields show
problematic degeneracy against differential chromatic refraction for all
three strategies. Effort is now needed to implement the science figures
of merit in Section \ref{sec:MW_Astrometry:MW_Astrometry_metrics} that
are based on these astrometry indications. Assessment of the science
strategy impact on the Halo (Section \ref{sec:MW_Halo}) largely rests on
the implementation of a robust star/galaxy separation metric into MAF.
This development is ongoing, led by one of the authors of this Chapter
(CTS).



\section{Introduction}

\def\secname{MW_Intro}\label{sec:\secname}

LSST should produce significant contributions to most areas
of Galactic astronomy. LSST Milky Way science cases cover lengthscales
ranging from a few pc (such as sensitive surveys of low-mass objects in
the Solar Neighborhood), up to many tens of kpc (such as surveys for
low-mass satellite galaxies of the Milky Way and their post-disruption
remnant streams, and beyond this, investigations of resolved stellar
populations in the Local Volume). In this chapter, we investigate a limited set of representative Galactic science cases, with the aims of demonstrating the scientific trade-offs of different observing strategies and of motivating readers to contribute to consderations of LSST's observing strategy.  The LSST Science Book
(particularly chapters 6 and 7) and \citet[][in particular Sections
2.1.4 and 4.4]{IvezicEtal2008} present a broader treatment of both science questions and science cases relevant to Galactic science.\footnote{We do however provide
motivating details for certain science cases in this Chapter,
particularly in \autoref{sec:MW_Disk}, as those cases are not
emphasized in the LSST Science Book or the relevant sections of
\citet{IvezicEtal2008}.}.

Concern about observations towards the inner Galactic Plane has for
several years been a common theme in feedback on LSST's observing
strategy, as the Baseline survey currently expends relatively little
observing time per field at low Galactic latitudes (~30 visits per filter, closely spaced in time).  With few visits per filter and a condensed time sampling, populations found in scientifically interesting numbers only at low
Galactic latitudes might be detected with low efficiency
by LSST under this Baseline strategy. (An example is the probing of the
mass function of moderate-separation extrasolar planets via intra-disk
planetary microlensing, as argued forcefully in \citealt{gould13}).

%At the time of
%writing, observations towards the inner Plane still seem to be
%compromised further by the way OpSim arranges shorter programs to
%completion at early times in the simulated survey. This reduces the time
%baseline available for inner-Plane measurements by a significant factor
%compared to observations away from the Plane.

We explore the scientific
impact of shallow Plane observations by comparing Metrics and Figures of
Merit evaluated for the Baseline cadence (\opsimdbref{db:baseCadence}),
for the PanSTARRS-like strategy (\opsimdbref{db:opstwoPS}) which has
essentially uniform depth at all Galactic latitudes observed, and for
the \opsimdbref{db:NormalGalacticPlane} strategy in which the plane is part of
the Wide-Fast-Deep survey. The evaluation of
Figures of Merit generated by these three observing strategies will
quantify how a range of Milky Way science cases could be
substantially improved by selecting a strategy with increase Plane
coverage, without significant cost to the rest of LSST's scientific
investigations.


\subsection{Chapter terminology and structure}

To tame the diversity of science cases, we have picked
representative cases and grouped them within broad scientific areas,
devoting one Section of this chapter to each grouping of cases. A small
number of Figures of Merit (FoMs) have been described for each case.
At present, science cases are grouped in the following way:
\autoref{sec:MW_Disk} assesses the impact of observing strategy on
LSST's ability to map some representative astrophysically important
populations that are found mostly or exclusively in the Plane.
% \autoref{sec:MW_SFH} discusses the use of LSST to probe star formation
% histories through mostly young populations (see also Section 5.6.).
% \autoref{sec:MW_Dust} discusses the impact of observing strategy on
% ISM constraints.
Several observational challenges for LSST find their
sharpest expression in Milky Way science, including (but not limited to)
measurements of stellar parallax, absolute astrometry, and proper
motions (including the tie-in to the reference frame which will be
provided by the \textit{gaia} mission). For this reason, specific issues
relating to precision astrometry are developed in
\autoref{sec:MW_Astrometry}.
\autoref{sec:MW_Halo} assesses the degree to which structures in the Milky Way's halo can be discriminated and mapped, using tracer populations distinguished by variability and/or derived stellar parameters.
Finally, \autoref{sec:MW_future}
presents descriptions of investigations that are needed to properly
determine LSST's utility for Milky Way science, but which
% at this date (late-April 2016)
are as yet relatively incompletely developed.

Summary Tables are provided that present the figure of merit for each
science case within a given section (one row per figure of merit)
evaluated for each tested observing strategy (one column per
strategy). This summary information appears in
\autoref{tab_SummaryMWDisk} (the Disk),
% \autoref{tab_SummaryMWDust} (the ISM),
% \autoref{tab_SummaryMWHalo} (the Halo), and
\autoref{tab_SummaryMWAstrometry} (Astrometry).


% Examples
% identified at this stage include the uses of LSST to set constraints on
% Galactic components (including the structure of the Bulge, and the
% impact of radial migration in the Disk) and the study of resolved
% stellar populations in the Local Group.

%\subsection{Summary tables for Figures of Merit}
%
%The tables below organize the comparison of Figures of Merit for all
%the science cases considered in this chapter:
%begin{itemize}
 % \item Mapping the Milky Way Halo: Table \ref{tab_SummaryMWHalo}
%  \item Mapping the Milky Way Disk: Table \ref{tab_SummaryMWDisk}
%    \item The ISM: Table \ref{tab_SummaryMWDust}
%  \item Astrometry with LSST: Table \ref{tab_SummaryMWAstrometry}
%\end{itemize}

%\subsection{Needed input}
%
%While many of the diagnostic Metrics are relatively well-developed,
%implementation is needed of Figures of Merit (FoMs) that depend on
%these metrics. In some Sections we have sketched out such figures of
%merit, in others the development of a practical FoM is still a topic
%of active development.

% ====================================================================
%+
% SECTION:
%    section-name.tex  % eg lenstimedelays.tex
%
% CHAPTER:
%    chapter.tex  % eg cosmology.tex
%
% ELEVATOR PITCH:
%    Explain in a few sentences what the relevant discovery or
%    measurement is going to be discussed, and what will be important
%    about it. This is for the browsing reader to get a quick feel
%    for what this section is about.
%
% COMMENTS:
%
%
% BUGS:
%
%
% AUTHORS:
%    Phil Marshall (@drphilmarshall)  - put your name and GitHub username here!
%-
% ====================================================================

\section{Mapping the Milky Way Disk}
\def\secname{MW_Disk}\label{sec:\secname} % For example, replace "keyword" with "lenstimedelays"

\noindent{\it Will Clarkson, Peregrine McGehee, Jay Strader, Chris Britt}  % (Writing team)

% This individual section will need to describe the particular
% discoveries and measurements that are being targeted in this section's
% science case. It will be helpful to think of a ``science case" as a
% ``science project" that the authors {\it actually plan to do}. Then,
% the sections can follow the tried and tested format of an observing
% proposal: a brief description of the investigation, with references,
% followed by a technical feasibility piece. This latter part will need
% to be quantified using the MAF framework, via a set of metrics that
% need to be computed for any given observing strategy to quantify its
% impact on the described science case. Ideally, these metrics would be
% combined in a well-motivated figure of merit. The section can conclude
% with a discussion of any risks that have been identified, and how
% these could be mitigated.

%A short preamble goes here. What's the context for this science
%project? Where does it fit in the big picture?

Many populations of great importance to Astronomy exist predominantly
in or near the Galactic Plane, and yet are sufficiently
sparsely-distributed (and/or faint enough) that LSST is likely to be
the only facility in the forseeable future that will be able to
identify a statistically meaningful sample. Some (such as the novae
that allow detailed study of the route to Type Ia Supernovae) offer
unique laboratories to study processes of fundamental importance to
astrophysics at all scales. Others (like intra-disk microlensing
events) offer the {\it only} probe of important populations. An
important collateral benefit of studies in the plane with an LSST-like
facility, is improved mapping of the distribution and observational
effects of the ISM (particularly dust), which is of importance to all
IR/Optical/UV observational studies.

% --------------------------------------------------------------------

\subsection{Target measurements and discoveries}
\label{sec:keyword:MW_Disk_targets}

%Describe the discoveries and measurements you want to make.

%Now, describe their response to the observing strategy. Qualitatively,
%how will the science project be affected by the observing schedule and
%conditions? In broad terms, how would we expect the observing strategy
%to be optimized for this science?

\begin{itemize}
  \item 1. Quantifying the large quiescent compact binary population via variability;
  \item 2. New insights into the behavior of Novae and the route to Type Ia Superovae;;
  \item 3. The next Galactic Supernova;;
  \item 4. Probing planets outside the Snow Line with Microlensing;
  \item 5. A three-dimensional Dust map and improvements in the reddening law
\end{itemize}

Motivation and qualitative description of response to observing strategy:

{\bf 1. Probing quiescent compact binaries via variability:} Of the
millions of stellar-mass black holes formed through the collapse of
massive stars over the lifetime of the Milky Way, only $\sim 20$ have
been dynamically confirmed through spectroscopic measurements (e.g.,
Corral-Santana et al.~2015). Many questions central to modern
astrophysics can only be answered by enlarging this sample: which
stars produce neutron stars and which black holes; whether there is a
true gap in mass between neutron stars and black holes; whether
supernova explosions result in large black hole kicks. 

There is expected to be a large population of black hole binaries in quiescence
with low X-ray luminosities from $\sim 10^{30}$--$10^{33}$ erg/s.
Such systems can be identified as optical variables that show unique,
double-humped ellipsoidal variations of typical amplitude $\sim 0.2$
mag due to the tidal deformation of the secondary star, which can be a
giant or main sequence star. In some cases analysis of the light curve
alone can point to a high mass ratio between the components,
suggesting a black hole primary; in other cases the accretion disk
will make a large contribution to the optical light which results in
intrinsic, random, and fast variations in the light curve. The disk
contribution to optical light can change over time, and several years
of data is necessary to properly subtract the accretion disk
contribution in order to properly fit ellipsoidal veriations (Cantrell
et al. 2010). The brighter sources will be amenable to spectroscopy
with the current generation of 4-m to 10-m telescopes to dynamically
confirm new black holes; spectroscopy of all candidates should be
possible with the forthcoming generation of large telescopes. Thus,
LSST would trigger a rich variety of observational investigations of
the accretion/outflow process through studies of this large, dark
population.

While we have focused above on black hole binaries, we note that LSST
would be crucial for investigations of neutron star and white dwarf
binaries. For example, the total number of dwarf novae (accretion disk
instability outbursts around white dwarfs)
is presently
poorly understood---theoretical estimates routinely yield a
significantly higher number than are observed in the solar
neighborhood. Understanding the true specific frequency of these
systems provides a key check on common envelope evolution, which is
poorly constrained but has a large impact on, for example, LIGO event
rates. LSST will detect dwarf novae, which last at least several days
with typical amplitudes of 4--6 mag,
out to kpc scales. This will allow a test of not only the number of
cataclysmic variables, but also of the 3D distribution within the
Galaxy and dependence on metallicity gradients (Britt et al. 2015).

{\it Response to observing strategy:} Since most black hole candidates
have been identified near the plane in the inner Milky Way (68\%, 92\%
within $5^{\circ}, 10^{\circ}$~of the Plane), this science case {\it
    requires} that LSST observe the plane with sufficient cadence to
  detect the $\sim$hundreds of quiescent black-hole binaries by virtue
  of their variability. The natural choice for a survey for
  low-luminosity black hole binaries would be to extend the
  Wide-Fast-Deep survey throughout the Plane in the direction of the
  inner Milky Way. For dwarf novae, the cadence of observations
  is critical in obtaining an accurate measure of the population of
  cataclysmic variables, as a long baseline is necessary to recover
  low duty cycle systems while widely-space observations
 would miss short outbursts.

%Describe the discoveries and measurements you want to make.

%Now, describe their response to the observing strategy. Qualitatively,
%how will the science project be affected by the observing schedule and
%conditions? In broad terms, how would we expect the observing strategy
%to be optimized for this science?

{\bf 2. Novae and the route to Type Ia Supernovae:} Only $\sim 15$
novae (explosions on the surfaces of white dwarfs) are discovered in
the Milky Way each year, while observations of external galaxies show
that the rate should be a factor of $\sim 3$ higher (Shafter et
al.~2014). Evidently, we are missing 50--75\% of novae due to their
location in crowded, extinguished regions, where they are not bright
enough to be discovered at the magnitude limits of existing transient
surveys. Fundamental facts about novae are unknown: how much mass is
ejected in typical explosions; whether white dwarfs undergoing novae
typically gain or lose mass; whether the binary companion is important
in shaping the observed properties of nova explosions. Novae can serve
as scaled-down models of supernova explosions that can be tested in
detail, e.g., in the interaction of the explosion with circumstellar
material (e.g., Chomiuk et al.~2015). Further, since accreting white
dwarfs are prime candidates as progenitors of Type Ia supernovae, only
detailed study of novae can reveal whether particular systems are
increasing toward the Chandrasekhar mass as necessary in this
scenario.

{\it Response to observing strategy:} Most novae occur in the Galactic
Plane and Bulge, and therefore the inclusion of the Plane in a survey
of sufficient cadence to find these events promptly is of paramount
importance for this science. These events will trigger
multi-wavelength follow-up ranging from the radio to X-ray and
$\gamma$-rays; these data are necessary for accurate measurements of
the ejected mass.

{\bf 3. The First Galactic Supernova:} A supernova in the Milky Way
would be among the most important astronomical events of our lifetime,
with enormous impacts on stellar astrophysics, compact objects,
nucleosynthesis, and neutrino and gravitational wave astronomy. The
estimated rate of supernovae (both core-collapse and Type Ia) in the
Milky Way is about 1 per 20--25 years (Adams et al.~2013); hence there
is a 40--50\% chance that this would occur during the 10-year LSST
survey. If fortunate, such an event will be located relatively close
to the Sun and will be an easily observed (perhaps even naked-eye)
event. However, we must be cognizant of the likelihood that the
supernova could go off in the mid-Plane close to the Galactic Center
or on the other side of the Milky Way---both regions covered by
LSST. While any core-collapse event will produce a substantial
neutrino flux, alerting us to its existence, such observations will
not offer precise spatial localization. The models of Adams et
al.~(2013) indicate that LSST is the \emph{only} planned facility that
can offer an optical transient alert of nearly all Galactic
supernovae. 

{\it Response to observing strategy:} Even if the supernova is not too
faint, LSST will likely be the sole facility with synoptic
observations preceding the explosion, providing essential photometric
data leading up to the event---but only if LSST covers the Plane at a
frequent cadence. Just {\it how} frequent is open to exploration at
present, but the prospect of high-sensitivity observations of the
location of such a supernova {\it before} it takes place are clearly
of enormous scientific value. A secondary issue is the prospect that
an easily-observed Milky Way supernova might be too bright for LSST to
measure precisely with its planned exposure time, with a roughly 82\%
chance of a core-collapse supernova reaching one or two magnitudes
brighter than LSST's nominal saturation limit (with a 1/3 chance that
a ccSN would reach $m_V \sim 5$~(Adams et al. 2013). For a Type Ia in
the Milky Way, Adams et al. (2013) estimate $m_{V, max} \lesssim
13.5$~in 92\% of cases.

{\bf 4. Probing planets beyond the Snow Line with Microlensing:} Gould
(2013) shows that LSST could be an effective intra-disk microlensing
survey (in which disk stars are lensed by other objects in the disk,
such as exoplanets, brown dwarfs, or compact objects). The lower
stellar density compared to past bulge-focused microlensing surveys
would be offset by the larger area covered by LSST. The predicted rate
of high magnification microlensing events that are very sensitive to
planets would be $\sim 25$ per year. This survey would be able to
detect planets at moderate distances from their host stars, a regime
poorly probed by standard Doppler and transit techniques. The LSST
data alone would not be sufficient: the detection of a slow ($\sim$
days) timescale increase in brightness of a disk star would need to
trigger intensive photometric observations from small (1-m to 2-m
class) telescopes that would observe at high cadence for the 1--2
months of the microlensing event. This would represent an excellent
synergy between LSST and the wider observing community, and would
directly take advantage of the capabilities unique to LSST.

{\it Response to observing strategy:} To catch lensing events as they
start to brighten, with sufficient fidelity to trigger the intensive
follow-up required, the models of Gould (2013) suggest each field
should be observed once every few nights. With sparser coverage,
the survey would lose sensitivity to microlensing events in progress.

{\bf 5. Dust in the Milky Way disk:} The Pan-STARRS1 survey (PS1) has
produced a three-dimensional dust map of the region of the sky covered
in their 3$\pi$ survey (which excludes a large part of the Galactic
Plane toward the south). Such maps are necessary to accurately measure
the intrinsic luminosities and colors of both Galactic and
extragalactic sources. The PS1 map (Schlafly et al.~2014) saturates at
extinctions $E(B-V) > 1.5$ as their tracer stars fall out of the
survey catalogs fainter than $g\sim 22$, meaning that this
high-fidelity map does not extend uniformly to within a few degrees of
the midplane. In addition, it only extends to a distance of about 4.5
kpc. Deep LSST data will allow this map to be extended to much higher
extinctions and larger distances. Owing to the high extinction and the
use of blue filters, this project is less affected by crowding than
other projects requiring photometry in the Plane. 

{\it Response to observing strategy:} When focusing on dust in the ISM (as
opposed to time-domain studies, e.g., dust around star-forming
regions or young stars), the main drivers of feasibility are
coverage of the few degrees around the Plane with sufficient photometric depth
and accuracy. This project is less affected by crowding than other
projects requiring photometry in the Plane owing to its use of blue
filters and the high extinction.  Nonetheless, quantiative estimates of
the expected photometric accuracy in coadded $u$ and $g$ images at low
Galactic latitude are desirable.


% --------------------------------------------------------------------

\subsection{Metrics}
\label{sec:keyword:MW_Disk_metrics}

%Quantifying the response via MAF metrics: definition of the metrics,
%and any derived overall figure of merit.

Unpolished notes about likely (reasonably straightforward) metrics
follow, along with (in some cases) possible directions for future
higher-level metrics.  \new{WIC - Note to co-authors: these metrics
  were chosen to build off existing metrics in development. To take
  one example, Peter Y has developed some nice ipython Notebooks that
  illustrate representative Monte Carlo tests for periodicity
  detection including spatial variation - thus I think the metrics
  below should be straightforward to implement by modification of existing tools.}


{\bf Metric 1.1 - Fraction of quiescent
  black hole binaries detectable through ellipsoidal variability, as a function of location on sky and distance.}
Dependencies:
\begin{itemize}
  \item Monte Carlo in period, phase and shape parameters (ASCII input?) for variables as measured in a particular OpSim run. Likely run Monte Carlo for a representative number (ten?) of well-chosen orbital periods within the 0.1-5d range;
  \item (Since these are short-period objects): the ``PeriodicMetric'' of Lund et al. (2015);
  \item Will likely need reasonably high-spatial-resolution HEALPIX slices and a prescription for population density as a function of position on-sky (can be analytic).
\end{itemize}
Possible higher-order metric: errors on the population size (mass
function??) derived from a survey under a given observing
strategy. Can imagine just adding up the ``recovered'' qLMXB
population and comparing it to that simulated. Note that the survey
will necessarily be highly incomplete (inclination effects, etc.), it
is the likely {\it uncertainty} on the completeness-correction that
would be crucial in this case.

{\bf Metric 1.2 - Fraction of dwarf novae detected in a given survey
  run as a function of location and distance.}
Dependencies:
\begin{itemize}
  \item Monte Carlo in distribution of maximum brightness and rise/decay timescale;
    \item Need to know how well the lightcurve shape needs to be characterized. Does the ``Triples'' approach work here, or would something more sophisticated be needed? (E.g. the ``transientASCIIMetric from the ``Explosive Transients'' section.) Can the observations be in different filters to accomplish this?
\end{itemize}
Higher-level metrics: Uncertainty in LIGO event rates due to
uncertainties in DNe characterization.

{\bf Metric 2.1 - Fraction of identified Novae as a function of location on-sky, distance, and time since initial rise.}
Dependencies:
\begin{itemize}
  \item Is the ``Triplets'' metric sufficient (i.e. is this ``just'' a case of supplying the metric the correct $\Delta t$~parameter values)?;
    \item What is the maximum interval since initial rise that would be acceptable? (Is this a function of waveband for followup?)
\end{itemize}
Possible higher-order metric: error on inferred rate of Type Ia supernovae?

{\bf Metric 3.1 - Maximum time-interval {\it before} triggering of a SN in the Milky Way that LSST would have taken precursor data.}
Dependencies:
\begin{itemize}
  \item This metric would probably be very easy to calculate (just estimate the mean time between observations). However the acceptable limits still need thought:
  \item How many colors are sufficient? Any observations at all before the SN goes off, or would a complete set in all filters be needed to characterize the candidate progenitor?
    \item What level of variability sensitivity is really needed? Would just an extremely deep image of the SN field before the event be sufficient?
\end{itemize}
{\bf Metric 3.2 - Maximum time-interval {\it after} the SN event for triggering followup.}
Dependencies:
\begin{itemize}
  \item Similar to metric 3.1. 
    \item Is it important to know the discovery space for LSST? If a
      supernova at $m_V~15$~goes off, other facilities are likely to
      spot it...
\end{itemize}

{\bf Metric 4.1 - Fraction of accurately-triggered Microlens candidates as a function of location, distance, lens mass.}
Dependencies:
\begin{itemize}
  \item Analytic (?) model for microlens lightcurve for representative sample of microlens parameters;
    \item transientASCIIMetric may well already do what we need! 
\end{itemize}

{\bf Metric 4.2 - errors in the (mass function, distance distribution) of intra-disk microlensed planets.}
Dependencies:
\begin{itemize}
  \item Event rate with sufficient-cadence observations to trigger followup as a function of (distance, lens mass);
\end{itemize}

{\bf Metric 5.1 - Errors in derived $E(B-V)$, $n_H$~as a function of
  location in the Plane.}
Dependencies:
\begin{itemize}
  \item SNR scaling with apparent magnitude
    \item For the M dwarf based technique, the relation between
      reddening-invariant index $[Q_{gri}]$~and intrinsic $g-i$~are
      expressed as polynomials, so expect non-linear relation with
      photometric error.  
      \item This uses a 5th-order polynomial to describe the
        $(Q_{gri}, g-i)$~stellar locus for M dwarfs $(g-i > 1.6)$.
        \item Care must be taken to correctly propagate errors through
          the various indices used - not trivial with so many choices
          of flux ratio used.
          \item Uncertainties in the parameterizations used for e.g. color-$M_V$~relationships.
            \item The above are all for every location probed on the map.
\end{itemize}



% --------------------------------------------------------------------

\subsection{OpSim Analysis}
\label{sec:keyword:MW_Disk_analysis}

The current baseline cadence (${\tt enigma\_1189}$) partially excludes the Galactic Plane from the deep-wide-fast survey and instead adopts a nominal 30 visits 
per filter as part of a special proposal. We have proposed an OpSim run that includes the Galactic Plane in the deep-wide-fast survey (see \http{https://github.com/LSSTScienceCollaborations/ObservingStrategy/blob/master/opsim/Proposal_GP.md}). The metrics listed above should be carefully compared between our proposed 
run and the baseline cadence.


% --------------------------------------------------------------------

\subsection{Discussion}
\label{sec:keyword:MW_Disk_discussion}

Discussion: what risks have been identified? What suggestions could be
made to improve this science project's figure of merit, and mitigate
the identified risks?


% ====================================================================

\navigationbar


% PJM: Perry's content is to be merged with Pat's, in Section 5.6
% % ====================================================================
%+
% SECTION:
%    MW_SFH.tex
%
% CHAPTER:
%    galaxy.tex
%
% ELEVATOR PITCH:
%
%-
% ====================================================================

\section{Star Formation History of the Milky Way}
\def\secname{MW_SFH}\label{sec:\secname}

\credit{pmmcgehee}

\label{sec:\secname:targets}

% This individual section will need to describe the particular
% discoveries and measurements that are being targeted in this section's
% science case. It will be helpful to think of a ``science case" as a
% ``science project" that the authors {\it actually plan to do}. Then,
% the sections can follow the tried and tested format of an observing
% proposal: a brief description of the investigation, with references,
% followed by a technical feasibility piece. This latter part will need
% to be quantified using the MAF framework, via a set of metrics that
% need to be computed for any given observing strategy to quantify its
% impact on the described science case. Ideally, these metrics would be
% combined in a well-motivated figure of merit. The section can conclude
% with a discussion of any risks that have been identified, and how
% these could be mitigated.

LSST gives the opportunity to survey extensive areas
around star formation regions in the Southern hemisphere. Among
others, it would allow to study the Initial Mass Function down to the
sub-stellar limit across different environments. Young stars are
efficiently identified by their variability.

Section 8.10.2 in the LSST Science Book (p298--299) provides a
  thorough scientific motivation for the characterization of young
  stars through variability, including discussion of the observational
  signatures of the diverse physical phenomena driving observed time
  variability. The general observable is strong, irregular flaring
  across the entire $ugrizy$~bandpass of LSST. Flaring can last from
  minutes to years, and at amplitudes from a few tenths to several
  magnitudes.

% WIC - the following para has been removed since it's already present
% verbatim at the end of Section 8.10.2 in the LSST Science Book.

%LSST will increase the sample size for detailed follow-up observations
%due its ability to survey star formations at large heliocentric
%distances and to detect variability in embedded and highly extincted
%young objects that would otherwise be missed in shallower
%surveys. During its operations LSST will also provide statistics on
%the durations of high states, for at least one important tracer
%population (the shorter-duration EXor variables).

% --------------------------------------------------------------------

\subsection{Target measurements and discoveries}
\label{sec:\secname:targets}

The nature of the variability in young stars changes with evolutionary
status. For the youngest stars still undergoing significant mass
accretion, FU Orionis and related outbursts can occur due to
circumstellar disk instabilities. As the natal environment dissipates
and the accretion rates drops, the stars take on a Classical T Tauri
appearance where the variability is primarily due to changes in the
accretion flow and rotational modulation of hot spots resulting from
accretion shocks on the protostellar photosphere. Also present are the
signature of cool spots arising from strong magnetic fields. This cool
spot rotational modulation is responsible for the variability in the
disk-less, and older, weak-line T Tauri stars.

Of particular interest are the FUor and EXor variables, which are
  named after the prototype objects FU Orionis \citep{hartmann96}
  and EX Lupi \citep{herbig01} respectively, and for which
  only a relatively small number of examples are known. In these
  pre-main sequence objects, eruptive outbursts of up to 6 magnitudes
  have been observed, with high state durations from years to
  decades. In addition to triggering follow-up observations, LSST
  should be able to set the first population constraints on the
  duration of high states, particularly for the short end of the
  timescale distribution for these eruptive variables.


% \new{(WIC: Material already in LSST Science Book removed.)}


%{\bf CTTS and WTTS material goes here.}

%Here are the references cited above:\\
%Hartmann \& Kenyon 1996, ARA\&A, 34, 207 \\
%Herbig et al. 2001, PASP, 113, 1547 \\
%Herbig 1977, ApJ, 217, 693 \\
%Aspin et al. 2009, ApJ, 692L, 67 \\
%Hodapp et al. 1996, ApJ, 468, 861 \\
%McGehee et al. 2004, ApJ, 616, 1058 \\


%Describe the discoveries and measurements you want to make.

%Now, describe their response to the observing strategy. Qualitatively,
%how will the science project be affected by the observing schedule and
%conditions? In broad terms, how would we expect the observing strategy
%to be optimized for this science?


% --------------------------------------------------------------------

\subsection{Metrics}
\label{sec:\secname:metrics}

In order to assess the ability of LSST to 1) identify and 2) classify
Young Stellar Objects we need to quantify the variability timescales
and amplitudes of both Class I/II (stars with disks, including
Classical T Tauris) and Class III (Weak-line T Tauris).  Inclusion of
eruptive variables (FUor/EXor) is appropriate as well.

In brief, Weak-line T-Tauris are quasi-periodic with amplitudes of 0.1
to 0.3 mag and periods 1 to $\sim$15 days, so their variability is
comparable to that of $\gamma$ Dor stars. Given the temporal evolution
of cool spots, a period recovery analysis such as shown for RR Lyrae
stars is likely difficult.  The embedded systems and Classical T
Tauris are irregular variables but have been shown to have distinctive
colors due to extinction and the ultraviolet and blue excess arising
from accretion shocks.

\autoref{table:pseudoForExor} shows a possible Figure of Merit for the
recovery by LSST of the distribution of EXor high-state duration in
outburst.


\begin{table}
\small
\begin{tabular}{c p{12cm}}
& {\it Figure of Merit for recovery of EXor high--state duration distribution}\\
\hline
1.  & Produce ASCII lightcurve for eruptive outburst \\
2.  & Initialise large array to store the maps of fraction detected as a function of duration and amplitude. \\
2.  & for {\it duration T} in range \{min, max\}:  \\
3.  & ~~~~ for {\it amplitude A} in range \{min, max\}: \\
4.  & ~~~~~~~~~~ run {\tt mafContrib/transientAsciiMetric} \\
5.  & ~~~~~~~~~~ store the spatial map of the fraction detected for this (A, T) pair \\
6.  & Initialise master arrays to hold the run of duration distribution measurements.\\
7. & Produce distribution of high--state durations and amplitudes from which the simulations will be drawn. \\
8.  & for {\it iDraw} in range \{1, nDraws\}:\\
9.  & ~~~~ construct model population with input duration distribution \\
10.  & ~~~~ Apply the stored metrics from 2-5 to measure fraction recovered \\
11.  & ~~~~ Characterize the duration distribution for this draw \\
12. & ~~~~ Fill the {\it iDraw}'th entry in the master arrays. \\
13. & {\bf FoM 1:} Compute the median and variance of the upper/lower quintiles. \\
14. & {\bf FoM 2:} Evaluate the bias between recovered and input high-state duration. \\
\hline
\end{tabular}
\caption{\new{Steps for Figure of Merit recovering the distribution
  for the duration of EXor high states. See Section \ref{sec:MW_SFH:targets} }}
\label{table:pseudoForExor}
\end{table}


% --------------------------------------------------------------------

%\subsection{OpSim Analysis}
%\label{sec:\secname:analysis}

%OpSim analysis: how good would the default observing strategy be, at
%the time of writing for this science project?

% --------------------------------------------------------------------

\subsection{Discussion}
\label{sec:\secname:discussion}

Galactic star formation regions are largely found at low Galactic
latitudes or within the Gould Belt structure. As such study of young
stars with LSST is closely tied to other science goals concerning the
Milky Way Disk and is subject to the concerns of both crowded field
photometry and the observing cadence along the Milky Way.

The embedded and Classical T Tauri stars also undergo significant and
rapid color changes due to accretion processes. The ability of LSST to
track these variations in color could be limited by the interval
between filter changes.

% ====================================================================
%
% \subsection{Conclusions}
%
% Here we answer the ten questions posed in
% \autoref{sec:intro:evaluation:caseConclusions}:
%
% \begin{description}
%
% \item[Q1:] {\it Does the science case place any constraints on the
% tradeoff between the sky coverage and coadded depth? For example, should
% the sky coverage be maximized (to $\sim$30,000 deg$^2$, as e.g., in
% Pan-STARRS) or the number of detected galaxies (the current baseline but
% with 18,000 deg$^2$)?}
%
% \item[A1:] ...
%
% \item[Q2:] {\it Does the science case place any constraints on the
% tradeoff between uniformity of sampling and frequency of  sampling? For
% example, a rolling cadence can provide enhanced sample rates over a part
% of the survey or the entire survey for a designated time at the cost of
% reduced sample rate the rest of the time (while maintaining the nominal
% total visit counts).}
%
% \item[A2:] ...
%
% \item[Q3:] {\it Does the science case place any constraints on the
% tradeoff between the single-visit depth and the number of visits
% (especially in the $u$-band where longer exposures would minimize the
% impact of the readout noise)?}
%
% \item[A3:] ...
%
% \item[Q4:] {\it Does the science case place any constraints on the
% Galactic plane coverage (spatial coverage, temporal sampling, visits per
% band)?}
%
% \item[A4:] ...
%
% \item[Q5:] {\it Does the science case place any constraints on the
% fraction of observing time allocated to each band?}
%
% \item[A5:] ...
%
% \item[Q6:] {\it Does the science case place any constraints on the
% cadence for deep drilling fields?}
%
% \item[A6:] ...
%
% \item[Q7:] {\it Assuming two visits per night, would the science case
% benefit if they are obtained in the same band or not?}
%
% \item[A7:] ...
%
% \item[Q8:] {\it Will the case science benefit from a special cadence
% prescription during commissioning or early in the survey, such as:
% acquiring a full 10-year count of visits for a small area (either in all
% the bands or in a  selected set); a greatly enhanced cadence for a small
% area?}
%
% \item[A8:] ...
%
% \item[Q9:] {\it Does the science case place any constraints on the
% sampling of observing conditions (e.g., seeing, dark sky, airmass),
% possibly as a function of band, etc.?}
%
% \item[A9:] ...
%
% \item[Q10:] {\it Does the case have science drivers that would require
% real-time exposure time optimization to obtain nearly constant
% single-visit limiting depth?}
%
% \item[A10:] ...
%
% \end{description}

% ====================================================================

\navigationbar


% PJM: moved the following to FutureWork, while the metric(s) is/are being implemented
% % ====================================================================
%+
% SECTION:
%    section-name.tex  % eg lenstimedelays.tex
%
% CHAPTER:
%    chapter.tex  % eg cosmology.tex
%
% ELEVATOR PITCH:
%    Explain in a few sentences what the relevant discovery or
%    measurement is going to be discussed, and what will be important
%    about it. This is for the browsing reader to get a quick feel
%    for what this section is about.
%
% COMMENTS:
%
%
% BUGS:
%
%
% AUTHORS:
%    Phil Marshall (@drphilmarshall)  - put your name and GitHub username here!
%-
% ====================================================================

\section{Dust in the Plane of the Milky Way}
\def\secname{MW_Dust}\label{sec:\secname} % For example, replace "keyword" with "lenstimedelays"

\noindent{\it Author Name(s)} % (Writing team)

% This individual section will need to describe the particular
% discoveries and measurements that are being targeted in this section's
% science case. It will be helpful to think of a ``science case" as a
% ``science project" that the authors {\it actually plan to do}. Then,
% the sections can follow the tried and tested format of an observing
% proposal: a brief description of the investigation, with references,
% followed by a technical feasibility piece. This latter part will need
% to be quantified using the MAF framework, via a set of metrics that
% need to be computed for any given observing strategy to quantify its
% impact on the described science case. Ideally, these metrics would be
% combined in a well-motivated figure of merit. The section can conclude
% with a discussion of any risks that have been identified, and how
% these could be mitigated.

A short preamble goes here. What's the context for this science
project? Where does it fit in the big picture?

% --------------------------------------------------------------------

\subsection{Target measurements and discoveries}
\label{sec:keyword:targets}

Describe the discoveries and measurements you want to make.

Now, describe their response to the observing strategy. Qualitatively,
how will the science project be affected by the observing schedule and
conditions? In broad terms, how would we expect the observing strategy
to be optimized for this science?


% --------------------------------------------------------------------

\subsection{Metrics}
\label{sec:keyword:metrics}

Quantifying the response via MAF metrics: definition of the metrics,
and any derived overall figure of merit.


% --------------------------------------------------------------------

\subsection{OpSim Analysis}
\label{sec:keyword:analysis}

OpSim analysis: how good would the default observing strategy be, at
the time of writing for this science project?


% --------------------------------------------------------------------

\subsection{Discussion}
\label{sec:keyword:discussion}

Discussion: what risks have been identified? What suggestions could be
made to improve this science project's figure of merit, and mitigate
the identified risks?


% ====================================================================

\navigationbar


% ====================================================================
%+
% SECTION:
%    section-name.tex  % eg lenstimedelays.tex
%
% CHAPTER:
%    chapter.tex  % eg cosmology.tex
%
% ELEVATOR PITCH:
%    Explain in a few sentences what the relevant discovery or
%    measurement is going to be discussed, and what will be important
%    about it. This is for the browsing reader to get a quick feel
%    for what this section is about.
%
% COMMENTS:
%
%
% BUGS:
%
%
% AUTHORS:
%    Phil Marshall (@drphilmarshall)  - put your name and GitHub username here!
%-
% ====================================================================

\section{Mapping the Milky Way with Positions, Proper Motions, and Parallaxes}
\def\secname{MW_Astrometry}\label{sec:\secname} % For example, replace "keyword" with "lenstimedelays"

\noindent{\it Jay Strader, David Monet, ...}  % (Writing team)

% This individual section will need to describe the particular
% discoveries and measurements that are being targeted in this section's
% science case. It will be helpful to think of a ``science case" as a
% ``science project" that the authors {\it actually plan to do}. Then,
% the sections can follow the tried and tested format of an observing
% proposal: a brief description of the investigation, with references,
% followed by a technical feasibility piece. This latter part will need
% to be quantified using the MAF framework, via a set of metrics that
% need to be computed for any given observing strategy to quantify its
% impact on the described science case. Ideally, these metrics would be
% combined in a well-motivated figure of merit. The section can conclude
% with a discussion of any risks that have been identified, and how
% these could be mitigated.

A short preamble goes here. What's the context for this science
project? Where does it fit in the big picture?

% --------------------------------------------------------------------

\subsection{Target measurements and discoveries}
\label{sec:keyword:MW_Astrometry_targets}

Describe the discoveries and measurements you want to make.

Now, describe their response to the observing strategy. Qualitatively,
how will the science project be affected by the observing schedule and
conditions? In broad terms, how would we expect the observing strategy
to be optimized for this science?


% --------------------------------------------------------------------

\subsection{Metrics}
\label{sec:keyword:MW_Astrometry_metrics}

Quantifying the response via MAF metrics: definition of the metrics,
and any derived overall figure of merit.


% --------------------------------------------------------------------

\subsection{OpSim Analysis}
\label{sec:keyword:MW_Astrometry_analysis}

OpSim analysis: how good would the default observing strategy be, at
the time of writing for this science project?


% --------------------------------------------------------------------

\subsection{Discussion}
\label{sec:keyword:MW_Astrometry_discussion}

Discussion: what risks have been identified? What suggestions could be
made to improve this science project's figure of merit, and mitigate
the identified risks?


% ====================================================================

\navigationbar


% PJM: moved the following to FutureWork, while the metric(s) is/are being implemented
% % ====================================================================
%+
% SECTION:
%    section-name.tex  % eg lenstimedelays.tex
%
% CHAPTER:
%    chapter.tex  % eg cosmology.tex
%
% ELEVATOR PITCH:
%    Explain in a few sentences what the relevant discovery or
%    measurement is going to be discussed, and what will be important
%    about it. This is for the browsing reader to get a quick feel
%    for what this section is about.
%
% COMMENTS:
%
%
% BUGS:
%
%
% AUTHORS:
%    Phil Marshall (@drphilmarshall)  - put your name and GitHub username here!
%-
% ====================================================================

\section{Mapping the Milky Way Halo}
\def\secname{MW_Halo}\label{sec:\secname} % For example, replace "keyword" with "lenstimedelays"

\noindent{\it Kathy Vivas, David Nidever, Colin Slater}  % (Writing team)

% This individual section will need to describe the particular
% discoveries and measurements that are being targeted in this section's
% science case. It will be helpful to think of a ``science case" as a
% ``science project" that the authors {\it actually plan to do}. Then,
% the sections can follow the tried and tested format of an observing
% proposal: a brief description of the investigation, with references,
% followed by a technical feasibility piece. This latter part will need
% to be quantified using the MAF framework, via a set of metrics that
% need to be computed for any given observing strategy to quantify its
% impact on the described science case. Ideally, these metrics would be
% combined in a well-motivated figure of merit. The section can conclude
% with a discussion of any risks that have been identified, and how
% these could be mitigated.

A short preamble goes here. What's the context for this science
project? Where does it fit in the big picture?

% --------------------------------------------------------------------

\subsection{Target measurements and discoveries}
\label{sec:keyword:MW_Halo_targets}

Describe the discoveries and measurements you want to make.

Now, describe their response to the observing strategy. Qualitatively,
how will the science project be affected by the observing schedule and
conditions? In broad terms, how would we expect the observing strategy
to be optimized for this science?


% --------------------------------------------------------------------

\subsection{Metrics}
\label{sec:keyword:MW_Halo_metrics}

Quantifying the response via MAF metrics: definition of the metrics,
and any derived overall figure of merit.


% --------------------------------------------------------------------

\subsection{OpSim Analysis}
\label{sec:keyword:MW_Halo_analysis}

OpSim analysis: how good would the default observing strategy be, at
the time of writing for this science project?


% --------------------------------------------------------------------

\subsection{Discussion}
\label{sec:keyword:MW_Halo_discussion}

Discussion: what risks have been identified? What suggestions could be
made to improve this science project's figure of merit, and mitigate
the identified risks?


% ====================================================================

\navigationbar


% Under development:
% % ====================================================================
%+
% SECTION:
%    section-name.tex  % eg lenstimedelays.tex
%
% CHAPTER:
%    chapter.tex  % eg cosmology.tex
%
% ELEVATOR PITCH:
%    Explain in a few sentences what the relevant discovery or
%    measurement is going to be discussed, and what will be important
%    about it. This is for the browsing reader to get a quick feel
%    for what this section is about.
%
% COMMENTS:
%
%
% BUGS:
%
%
% AUTHORS:
%    Phil Marshall (@drphilmarshall)  - put your name and GitHub username here!
%-
% ====================================================================

\section{Mapping the Milky Way Bulge}
\def\secname{MW_Bulge}\label{sec:\secname} % For example, replace "keyword" with "lenstimedelays"

\noindent{\it Will Clarkson, ...}  % (Writing team)

% This individual section will need to describe the particular
% discoveries and measurements that are being targeted in this section's
% science case. It will be helpful to think of a ``science case" as a
% ``science project" that the authors {\it actually plan to do}. Then,
% the sections can follow the tried and tested format of an observing
% proposal: a brief description of the investigation, with references,
% followed by a technical feasibility piece. This latter part will need
% to be quantified using the MAF framework, via a set of metrics that
% need to be computed for any given observing strategy to quantify its
% impact on the described science case. Ideally, these metrics would be
% combined in a well-motivated figure of merit. The section can conclude
% with a discussion of any risks that have been identified, and how
% these could be mitigated.

A short preamble goes here. What's the context for this science
project? Where does it fit in the big picture?

% --------------------------------------------------------------------

\subsection{Target measurements and discoveries}
\label{sec:keyword:MW_Bulge_targets}

Describe the discoveries and measurements you want to make.

Now, describe their response to the observing strategy. Qualitatively,
how will the science project be affected by the observing schedule and
conditions? In broad terms, how would we expect the observing strategy
to be optimized for this science?


% --------------------------------------------------------------------

\subsection{Metrics}
\label{sec:keyword:MW_Bulge_metrics}

Quantifying the response via MAF metrics: definition of the metrics,
and any derived overall figure of merit.


% --------------------------------------------------------------------

\subsection{OpSim Analysis}
\label{sec:keyword:MW_Bulge_analysis}

OpSim analysis: how good would the default observing strategy be, at
the time of writing for this science project?


% --------------------------------------------------------------------

\subsection{Discussion}
\label{sec:keyword:MW_Bulge_discussion}

Discussion: what risks have been identified? What suggestions could be
made to improve this science project's figure of merit, and mitigate
the identified risks?


% ====================================================================

\navigationbar


% Under development:
% % ====================================================================
%+
% SECTION:
%    section-name.tex  % eg lenstimedelays.tex
%
% CHAPTER:
%    chapter.tex  % eg cosmology.tex
%
% ELEVATOR PITCH:
%    Explain in a few sentences what the relevant discovery or
%    measurement is going to be discussed, and what will be important
%    about it. This is for the browsing reader to get a quick feel
%    for what this section is about.
%
% COMMENTS:
%
%
% BUGS:
%
%
% AUTHORS:
%    Phil Marshall (@drphilmarshall)  - put your name and GitHub username here!
%-
% ====================================================================

\section{Mapping the Local Volume with Resolved Stars}
\def\secname{MW_LocalVolume}\label{sec:\secname} % For example, replace "keyword" with "lenstimedelays"

\noindent{\it ???, ...}  % (Writing team)

% This individual section will need to describe the particular
% discoveries and measurements that are being targeted in this section's
% science case. It will be helpful to think of a ``science case" as a
% ``science project" that the authors {\it actually plan to do}. Then,
% the sections can follow the tried and tested format of an observing
% proposal: a brief description of the investigation, with references,
% followed by a technical feasibility piece. This latter part will need
% to be quantified using the MAF framework, via a set of metrics that
% need to be computed for any given observing strategy to quantify its
% impact on the described science case. Ideally, these metrics would be
% combined in a well-motivated figure of merit. The section can conclude
% with a discussion of any risks that have been identified, and how
% these could be mitigated.

A short preamble goes here. What's the context for this science
project? Where does it fit in the big picture?

% --------------------------------------------------------------------

\subsection{Target measurements and discoveries}
\label{sec:keyword:MW_LocalVolume_targets}

Describe the discoveries and measurements you want to make.

Now, describe their response to the observing strategy. Qualitatively,
how will the science project be affected by the observing schedule and
conditions? In broad terms, how would we expect the observing strategy
to be optimized for this science?


% --------------------------------------------------------------------

\subsection{Metrics}
\label{sec:keyword:MW_LocalVolume_metrics}

Quantifying the response via MAF metrics: definition of the metrics,
and any derived overall figure of merit.


% --------------------------------------------------------------------

\subsection{OpSim Analysis}
\label{sec:keyword:MW_LocalVolume_analysis}

OpSim analysis: how good would the default observing strategy be, at
the time of writing for this science project?


% --------------------------------------------------------------------

\subsection{Discussion}
\label{sec:keyword:MW_LocalVolume_discussion}

Discussion: what risks have been identified? What suggestions could be
made to improve this science project's figure of merit, and mitigate
the identified risks?


% ====================================================================

\navigationbar


% ====================================================================
%+
% SECTION:
%    section-name.tex  % eg lenstimedelays.tex
%
% CHAPTER:
%    chapter.tex  % eg cosmology.tex
%
% ELEVATOR PITCH:
%    Explain in a few sentences what the relevant discovery or
%    measurement is going to be discussed, and what will be important
%    about it. This is for the browsing reader to get a quick feel
%    for what this section is about.
%
% COMMENTS:
%
%
% BUGS:
%
%
% AUTHORS:
%    Phil Marshall (@drphilmarshall)  - put your name and GitHub username here!
%-
% ====================================================================

\section{Mapping the Milky Way Halo}
\def\secname{MW_Halo}\label{sec:\secname} % For example, replace "keyword" with "lenstimedelays"

\noindent{\it Kathy Vivas, David Nidever, Colin Slater}  % (Writing team)

% This individual section will need to describe the particular
% discoveries and measurements that are being targeted in this section's
% science case. It will be helpful to think of a ``science case" as a
% ``science project" that the authors {\it actually plan to do}. Then,
% the sections can follow the tried and tested format of an observing
% proposal: a brief description of the investigation, with references,
% followed by a technical feasibility piece. This latter part will need
% to be quantified using the MAF framework, via a set of metrics that
% need to be computed for any given observing strategy to quantify its
% impact on the described science case. Ideally, these metrics would be
% combined in a well-motivated figure of merit. The section can conclude
% with a discussion of any risks that have been identified, and how
% these could be mitigated.

A short preamble goes here. What's the context for this science
project? Where does it fit in the big picture?

% --------------------------------------------------------------------

\subsection{Target measurements and discoveries}
\label{sec:keyword:MW_Halo_targets}

Describe the discoveries and measurements you want to make.

Now, describe their response to the observing strategy. Qualitatively,
how will the science project be affected by the observing schedule and
conditions? In broad terms, how would we expect the observing strategy
to be optimized for this science?


% --------------------------------------------------------------------

\subsection{Metrics}
\label{sec:keyword:MW_Halo_metrics}

Quantifying the response via MAF metrics: definition of the metrics,
and any derived overall figure of merit.


% --------------------------------------------------------------------

\subsection{OpSim Analysis}
\label{sec:keyword:MW_Halo_analysis}

OpSim analysis: how good would the default observing strategy be, at
the time of writing for this science project?


% --------------------------------------------------------------------

\subsection{Discussion}
\label{sec:keyword:MW_Halo_discussion}

Discussion: what risks have been identified? What suggestions could be
made to improve this science project's figure of merit, and mitigate
the identified risks?


% ====================================================================

\navigationbar



% ====================================================================
%+
% SECTION:
%    MW_FutureWork.tex
%
% CHAPTER:
%    galaxy.tex
%
% ELEVATOR PITCH:
%    Ideas for future metric investigation, with quantitaive analysis
%    still pending.
%-
% ====================================================================

\section{Future Work}
\def\secname{MW_future}\label{sec:\secname}

% ====================================================================

% % ====================================================================
%+
% SECTION:
%    section-name.tex  % eg lenstimedelays.tex
%
% CHAPTER:
%    chapter.tex  % eg cosmology.tex
%
% ELEVATOR PITCH:
%    Explain in a few sentences what the relevant discovery or
%    measurement is going to be discussed, and what will be important
%    about it. This is for the browsing reader to get a quick feel
%    for what this section is about.
%
% COMMENTS:
%
%
% BUGS:
%
%
% AUTHORS:
%    Phil Marshall (@drphilmarshall)  - put your name and GitHub username here!
%-
% ====================================================================

\section{Dust in the Plane of the Milky Way}
\def\secname{MW_Dust}\label{sec:\secname} % For example, replace "keyword" with "lenstimedelays"

\noindent{\it Author Name(s)} % (Writing team)

% This individual section will need to describe the particular
% discoveries and measurements that are being targeted in this section's
% science case. It will be helpful to think of a ``science case" as a
% ``science project" that the authors {\it actually plan to do}. Then,
% the sections can follow the tried and tested format of an observing
% proposal: a brief description of the investigation, with references,
% followed by a technical feasibility piece. This latter part will need
% to be quantified using the MAF framework, via a set of metrics that
% need to be computed for any given observing strategy to quantify its
% impact on the described science case. Ideally, these metrics would be
% combined in a well-motivated figure of merit. The section can conclude
% with a discussion of any risks that have been identified, and how
% these could be mitigated.

A short preamble goes here. What's the context for this science
project? Where does it fit in the big picture?

% --------------------------------------------------------------------

\subsection{Target measurements and discoveries}
\label{sec:keyword:targets}

Describe the discoveries and measurements you want to make.

Now, describe their response to the observing strategy. Qualitatively,
how will the science project be affected by the observing schedule and
conditions? In broad terms, how would we expect the observing strategy
to be optimized for this science?


% --------------------------------------------------------------------

\subsection{Metrics}
\label{sec:keyword:metrics}

Quantifying the response via MAF metrics: definition of the metrics,
and any derived overall figure of merit.


% --------------------------------------------------------------------

\subsection{OpSim Analysis}
\label{sec:keyword:analysis}

OpSim analysis: how good would the default observing strategy be, at
the time of writing for this science project?


% --------------------------------------------------------------------

\subsection{Discussion}
\label{sec:keyword:discussion}

Discussion: what risks have been identified? What suggestions could be
made to improve this science project's figure of merit, and mitigate
the identified risks?


% ====================================================================

\navigationbar


% ====================================================================

% WIC - promoted this back to MW Halo section

% % ====================================================================
%+
% SECTION:
%    section-name.tex  % eg lenstimedelays.tex
%
% CHAPTER:
%    chapter.tex  % eg cosmology.tex
%
% ELEVATOR PITCH:
%    Explain in a few sentences what the relevant discovery or
%    measurement is going to be discussed, and what will be important
%    about it. This is for the browsing reader to get a quick feel
%    for what this section is about.
%
% COMMENTS:
%
%
% BUGS:
%
%
% AUTHORS:
%    Phil Marshall (@drphilmarshall)  - put your name and GitHub username here!
%-
% ====================================================================

\section{Mapping the Milky Way Halo}
\def\secname{MW_Halo}\label{sec:\secname} % For example, replace "keyword" with "lenstimedelays"

\noindent{\it Kathy Vivas, David Nidever, Colin Slater}  % (Writing team)

% This individual section will need to describe the particular
% discoveries and measurements that are being targeted in this section's
% science case. It will be helpful to think of a ``science case" as a
% ``science project" that the authors {\it actually plan to do}. Then,
% the sections can follow the tried and tested format of an observing
% proposal: a brief description of the investigation, with references,
% followed by a technical feasibility piece. This latter part will need
% to be quantified using the MAF framework, via a set of metrics that
% need to be computed for any given observing strategy to quantify its
% impact on the described science case. Ideally, these metrics would be
% combined in a well-motivated figure of merit. The section can conclude
% with a discussion of any risks that have been identified, and how
% these could be mitigated.

A short preamble goes here. What's the context for this science
project? Where does it fit in the big picture?

% --------------------------------------------------------------------

\subsection{Target measurements and discoveries}
\label{sec:keyword:MW_Halo_targets}

Describe the discoveries and measurements you want to make.

Now, describe their response to the observing strategy. Qualitatively,
how will the science project be affected by the observing schedule and
conditions? In broad terms, how would we expect the observing strategy
to be optimized for this science?


% --------------------------------------------------------------------

\subsection{Metrics}
\label{sec:keyword:MW_Halo_metrics}

Quantifying the response via MAF metrics: definition of the metrics,
and any derived overall figure of merit.


% --------------------------------------------------------------------

\subsection{OpSim Analysis}
\label{sec:keyword:MW_Halo_analysis}

OpSim analysis: how good would the default observing strategy be, at
the time of writing for this science project?


% --------------------------------------------------------------------

\subsection{Discussion}
\label{sec:keyword:MW_Halo_discussion}

Discussion: what risks have been identified? What suggestions could be
made to improve this science project's figure of merit, and mitigate
the identified risks?


% ====================================================================

\navigationbar


% ====================================================================

% \subsection{Other Ideas}

\credit{willclarkson}, \credit{akvivas}, \credit{vpdebattista}

In this final section we provide an extremely brief list of important science
cases that are still in an early stage of development, but that are
deserving of quantitative MAF analysis in the future.

\subsection{Further considerations for Milky Way static science}

  One important area of Milky Way science on which further
  community input is still sorely needed, is {\it static science} (a
  category that includes population disentanglement through deep,
  multicolor photometry), particularly in regions outside the main
  ``Wide-Fast-Deep'' (WFD) survey (Sections \ref{sec:MW_Astrometry} \&
  \ref{sec:MW_Halo} include discussion of static science in WFD regions). Since
  static science depends on depth (for, e.g., precise colors near the
  main sequence turn-off of some population) and uniformity over the
  survey (to aid characterization of strong selection functions),
  static science observing requirements may be in tension with (or at
  least not explicitly addressed by) requirements communicated
  elsewhere in this chapter.

  For example, probing deep within spatially crowded
  populations may lead to a sharper requirement on the selection of
  observations in good seeing conditions towards crowded regions, than
  has been apparent to-date. This needs quantification. 

  To pick another example, while we have indicated that co-added
  depth is a lower priority than temporal coverage for
  variability-driven studies in the Galactic Disk (conclusion A.1 in
  Section \ref{sec:MW_Disk}), co-added depth will likely be crucial for
  population disentanglement through photometry. While a
  judiciously-chosen observing strategy should be able to support both
  static and variable science, at this date quantitative trade-offs have not yet been specified.

  In many cases the implementation of figures of merit for static
  science in the Milky Way is complicated by the requirement to
  interface custom population simulations with the observational
  characterizations produced by the \MAF framework. For many
  investigators, the preferred method may be to use \MAF to produce
  parameterizations of the observational quantities of interest - for
  example, the run of photometric uncertainty against apparent
  magnitude, for each location on the sky, and including spatial
  confusion (all of which \MAF can currently produce) - and then use
  these characteristics as input to their own population simulations,
  on which the investigator may have invested substantial time and
  effort.

  To provoke progress, we specify in Table
  \ref{table:strawmanMWstaticScience} a possible Figure of Merit for
  static science in terms of capabilities mostly already provided by
  the \MAF framework, which does not require custom simulation. This
  Figure of Merit - which asks what fraction of fields in a spatial
  region of interest, are sufficiently well-observed to permit
  population disentanglement to some desired level of precision -
  could form the basis of several science FoM's (for example, the
  fraction of fields in which photometric age determinations of
  bulge/bar populations might be attempted). We encourage community
  development and implementation of this and other FoMs for Milky Way
  static science.

\begin{table}[h]
  \small
  \begin{tabular}{c p{12cm}}
    & {\it FoM innerMW-Static: fraction of fields in inner Galactic plane adequately covered for population discrimination} \\
    \hline
    1. & Produce absolute magnitudes $M_{u,g,r,i,z,y}$~of the population of interest, using \MAF's spectral libraries; \\
    2. & For each HEALPIX (i.e. pointing):\\
       & 2.1. Place the fiducial star at appropriate line-of-sight distance, produce apparent magnitudes $m_{u,g,r,i,z,y}$;\\
       & 2.2. Modify $m_{u,g,r,i,z,y}$~for extinction using \MAF's extinction model;\\
    & 2.3. Compute the photometric and astrometric uncertainties due to sampling and random error (from \MAF's ``m52snr'' method);\\
    & 2.4. Convert exposure-by-exposure estimates of photometric and astrometric uncertainty due to spatial confusion, into co-added uncertainties;\\
    & 2.5. Combine the random and confusion uncertainties into final measurement uncertainties on the photometry and astrometry;\\
    & 2.6. Use uncertainty propagation to estimate color uncertainties $u-g, g-r, r-i, i-z$;\\
    3. & Count the fraction of sight-lines for which the color below the threshold needed for "sufficient" accuracy in parameter determination \citep{ivezic08}. {\bf This is the figure of merit.} \\
\hline
    \end{tabular}
 \caption{Description of Figure of merit ``innerMW-Static''}
  \label{table:strawmanMWstaticScience}
\end{table}

Finally, we provide an extremely brief list of important science
cases that are still in an early stage of development, which fall into the ``Static science'' category of Milky Way science:
\begin{itemize}
  \item {\it Formation history of the Bulge and present-day balance of
  populations:} Sensitivity to metallicity and age distribution of Bulge
  objects near the Main Sequence Turn-off;
  \item {\it Migration and heating in the Milky Way disk:} Error and
  bias in the determination of components in the (velocity dispersion vs
  metallicity) diagram, for disk populations along various lines of
  sight \citep[e.g.][]{2016ApJ...818L...6L}.

% WIC 2017-05-24: commented this out in favor of a dedicated local volume subsection.
%
%\item Fraction of Local-Volume objects discovered as a function of
%  survey strategy.
\end{itemize}

\subsection{Short exposures}

  Populations near, or brighter than, LSST's nominal saturation
  limit ($r \sim 16$~with 15s exposures) are likely to be crucial to a
  number of investigations for Milky Way investigations, whether as
  science tracers in their own right, or as contaminants that might
  interfere with measurements of fainter program objects (due, for
  example, to charge bleeds of bright, foreground disk objects).

  Quantitative exploration of these issues now requires involvement from
  the community (e.g. to determine LSST's discovery space for bright
  tracers in context with other facilities and surveys like ZTF, Gaia
  and VVV), and the project (e.g. to determine the parameters of short
  exposures that might be supported by the facility). To provoke
  development, here we list a few example questions regarding short
  exposures that still need resolution:

\begin{itemize}
  \item{What level of bright-object charge-bleeding can be tolerated? For example, is some minimum distribution of position-angles required in order to spread the bleeds azimuthally over the set of observations of a particular line of sight (so that different pixels fall under bleeds in different exposures)?\footnote{Note that with $\sim 30$~exposures per filter per field over ten years towards some regions, charge-bleed directions might not be highly randomized.} What is this minimum?}
    \item{What is the science impact of restricting targets to $r \gtrsim 16$?}
      \item{Will the proposed twilight survey of short exposures be adequate for science cases requiring short exposures? Are there enough observations from other surveys (e.g. DES or other DECam surveys) to cover the bright end of the entire LSST footprint, and are {\it new} observations of very bright objects required scientifically in any case?}
        \item{What is the bright limit required for adequate astrometric cross-calibration against Gaia?}
          \item{To what extent would a given bright cutoff hamper the combination of LSST photometry or astrometry with that from other surveys (like VVV)?}
          \item{How short an exposure time can the facility support? Does the OpSim framework already include all the operational limitations on scheduling short exposures?}
\end{itemize}


\subsection{The Local Volume}

  Finally, we remind the reader that substantial opportunity
  remains to develop science figures of merit for {\it Local Volume}
  science cases. Figures of Merit for Local Volume science likely
  share much common ground with those for the Halo (discussed in
  Section \ref{sec:MW_Halo}), and substantial prior expertise exists
  \citep[e.g.][]{2014ApJ...795L..13H}. One straightforward Figure of
  Merit (FoM) might be the fraction of dwarf galaxies in the Local
  Volume that are correctly identified as a function of survey
  strategy.

% ====================================================================

\navigationbar

