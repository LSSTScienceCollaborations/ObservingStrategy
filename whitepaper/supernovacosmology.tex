% ====================================================================
%+
% SECTION:
%    section-name.tex  % eg lenstimedelays.tex
%
% CHAPTER:
%    chapter.tex  % eg cosmology.tex
%
% ELEVATOR PITCH:
%    Explain in a few sentences what the relevant discovery or
%    measurement is going to be discussed, and what will be important
%    about it. This is for the browsing reader to get a quick feel
%    for what this section is about.
%
% COMMENTS:
%
%
% BUGS:
%
%
% AUTHORS:
%    Phil Marshall (@drphilmarshall)  - put your name and GitHub username here!
%-
% ====================================================================

\section{Supernova Cosmology and Physics}
\def\secname{supernovae}\label{sec:\secname}
% \label{sec:cosmology, supernovae, classification, lenstimedelays, deepdrillingfields }

\noindent{\it Jeonghee Rho, Michelle Lochner, Rahul Biswas} % (Writing team)

% This individual section will need to describe the particular
% discoveries and measurements that are being targeted in this section's
% science case. It will be helpful to think of a ``science case" as a
% ``science project" that the authors {\it actually plan to do}. Then,
% the sections can follow the tried and tested format of an observing
% proposal: a brief description of the investigation, with references,
% followed by a technical feasibility piece. This latter part will need
% to be quantified using the MAF framework, via a set of metrics that
% need to be computed for any given observing strategy to quantify its
% impact on the described science case. Ideally, these metrics would be
% combined in a well-motivated figure of merit. The section can conclude
% with a discussion of any risks that have been identified, and how
% these could be mitigated.

This section is concerned with the detection and characterization over time of
supernovae using the Large Synoptic Sky Telescope (LSST). Supernovae of various
types are visible over a time scale of about a few weeks (eg. Type Ia) to a
 close to a year (Type IIP). During the full ten year survey of LSST, the
telescope will scan the entire Southern Sky repeatedly with a universal Wide Fast Deep
 Candence, and certain specific locations of the sky (the Deep Drilling fields)
with special enhanced cadence. This spatio-temporal window should contain
millions (RB: remember to check) of supernovae, that should be detected with the
 limiting magnitude of LSST. However, the actual sequence of observations in LSST
defined by series of field pointings as a function of time in filter bands
 (along with weather conditions) will determine the extent to which each of such
supernovae can be characterized well. Characterization of these supernovae is at
 the core of a number of science programs that use supernovae as bright, abundant
 objects with empirically determined intrinsic brightness.

The most important of these is the use of SNIa (and potentially core-collapse
supernovae like Type IIP) as standardizable candles to measure the
distance-redshift relation at cosmological distances to confront models leading to late time
 acceleration. For LSST, this goal entails (a) photometric typing of supernovae,
 (b) estimating photometric redshifts of supernovae (or identifying host galaxies,
 and obtaining their redshifts from photometry or follow-up spectroscopy)
(c) estimation of intrinsic brightnesses of the supernovae. The efficacy of
Photometric typing, redshifts and estimation of intrinsic brightnesses are all
dependent on the amount of information available in the observed light curves of supernovae.
 It should be noted that goal of using supernovae to constrain the
cosmology of a statistically homogeneous and isotropic universe is independent
of the spatial extent of the locations of the supernovae, and therefore a target
 precision in terms of constraining cosmology can be met by using a high cadence sampling
 in a relatively small subset of the LSST sky. However, there are other science cases
using such supernovae as tracers of position, redshift and distance simultaneously to probe
 quantities like the distribution of peculiar velocities, or to test whether the universe is
statistically isotropic at redshifts $z \lesssim 1.5$. Such projects which use supernovae as a 
tracer of large scale structure with a good measure of ``distance'' require large spatial extents, 
that could only be provided by the LSST WFD.

{\bfseries Photometric supernova classification}\\
In the past, only spectroscopically typed supernovae have been used for cosmology. Photometric 
typing from the light curve alone has only been used to select candidates for spectroscopic 
follow-up (see for example \citet{Sako2008}). However, LSST will simply produce far too many 
candidates for any chance of following up even a significant fraction of them. In order to avoid 
throwing away the majority of the supernova dataset, we need to use techniques capable of 
determining cosmological parameters from a potentially contaminated photometric supernova dataset.

There have been several techniques proposed in recent literature to solve this problem. One 
approach proposes applying stringent cuts to the photometric dataset to obtain a nearly pure sample 
of type Ia supernovae \citep{Bernstein2012,Campbell2013} and to run the standard supernova analysis 
with this sample. Another approach, BEAMS \citep{Kunz2007,Newling2011,Hlozek2012,Knights2013}, 
makes use of the full dataset, coping with contamination by using a mixture model for the 
likelihood, thus allowing for multiple populations. Whatever the technique ultimately used to for 
cosmological analysis, it will rely on accurate initial classifications of supernova type and 
unbiased estimates for the probability of each type.

The current state-of-the-art photometric classification techniques rely on fitting empirically 
determined templates of supernovae to light curves \citep{Jha2007,Guy2007,Sako2011}. However in 
recent years, new approaches have been published in response to the 2010 `Supernova 
Photometric Classification Challenge' \citep{Kessler2010a}. Many of these use novel light curve 
parameterisation and employ machine learning algorithms to perform the classification (see 
\citet{Kessler2010b} and references therein).

While many of these methods have been tested on standard sets of simulated data and (in some cases) 
on SDSS data, it is still not clear which technique (if any) is superior in all situations. For 
example, some techniques rely heavily on reliable redshift information being available, while others 
are less reliant on it. Some techniques may be more robust to non-representative datasets than 
others and it is not clear how the techniques will respond to changes in cadence, filter sets, SNR 
etc. With this in mind, we propose the use of a multifaceted classification system which employs 
several different methods of extracting features from the light curves (e.g. fitting parametric 
functions or templates) and several different classification algorithms. This system is highly 
modular, allowing the easy addition of new approaches for direct comparison with existing 
techniques. This also allows direct analysis of different observing strategies, without having to 
make an initial choice of classification technique. 

% --------------------------------------------------------------------

\subsection{Target measurements and discoveries}
\label{sec:keyword:targets}

% Describe the discoveries and measurements you want to make.

% Now, describe their response to the observing strategy. Qualitatively,
% how will the science project be affected by the observing schedule and
% conditions? 

% In broad terms, how would we expect the observing strategy
% to be optimized for this science?

In general terms, supernova science will be optimized by obtaining larger number of well sampled 
supernova light curves.



% --------------------------------------------------------------------

\subsection{Metrics}
\label{sec:keyword:metrics}

Quantifying the response via MAF metrics: definition of the metrics,
and any derived overall figure of merit.

\emph{To be added: discussion of the ROC curve as a useful metric for photometric supernova 
classification}

% --------------------------------------------------------------------

\subsection{OpSim Analysis}
\label{sec:keyword:analysis}

OpSim analysis: how good would the default observing strategy be, at
the time of writing for this science project?


% --------------------------------------------------------------------

\subsection{Discussion}
\label{sec:keyword:discussion}

Discussion: what risks have been identified? What suggestions could be
made to improve this science project's figure of merit, and mitigate
the identified risks?


% ====================================================================

\navigationbar
