\chapter[Synergy with WFIRST]{Synergy with WFIRST}
\def\chpname{wfirst}\label{chp:\chpname}

Chapter editor:
\credit{jasondrhodes}.

Contributing authors:
\credit{rubind},
\credit{davidpbennett},
\credit{mtpenny},
{\it Rachel Street}.

\section*{Summary}
\addcontentsline{toc}{section}{~~~~~~~~~Summary}

WFIRST will launch in $\sim$ 2025 for a 5 year mission to explore dark energy, find and characterize exoplanets, and take wide, deep infrared surveys of the galactic and extragalactic sky.  WFIRST was recognized by the Astro2010 Decadal Survey as an excellent NIR complement to LSST's optical capabilities. More recent work has recognized the strong synergy these two projects will have in helping to address cosmological questions in the 2020s \citep{2015arXiv150107897J}.
  Together, the two observatories can accomplish significantly more (and better) science than either can alone in the areas of weak lensing, large-scale structure studies, strong lensing, supernova studies, exoplanet investigations,  and photometric redshift determination. Accomplishing this will require coordinated observations.  We have identified three areas of proposed coordination: 1. Early coverage of the $>2000$ square degree WFIRST High Latitude Survey to the full optical depth for enhanced photometric redshifts for both LSST and WFIRST; 2. Coordinated LSST optical observations in the WFIRST supernova discovery fields; 3. Precursor, simultaneous, and follow-up observations of the WFIRST microlensing fields near the galactic bulge.

  We note here that the focus of the WFIRST  input to this paper is on suggesting changes or enhancements to the LSST observing strategy (cadence or survey overlap) that will provide mutual benefit for WFIRST and LSST.  The observing strategy suggestions here are not exhaustive of the possibilities for synergy between WFIRST and LSST; rather the suggestions here are based on the planned primary WFIRST surveys that are driving the WFIRST mission requirements.  As the plans and possibilities for WFIRST Guest Observer surveys evolve, and the ideas for additional science investigations that will make use of planned WFIRST survey data mature, the community should continue to look for areas of survey synergy.  Both WFIRST and LSST should consider survey modifications that would increase the global combined science output of the two surveys.


% --------------------------------------------------------------------

\section{Introduction}
\label{sec:wfirst:intro}

% Introduce, with a very broad brush, this chapter's science projects,
% and why it makes sense for them to be considered together.

The Wide Field Infrared Survey Telescope (WFIRST) is a NASA mission that
entered Phase A in February 2016.  WFIRST was the highest recommendation
for large space missions in the 2010 New Worlds New Horizons Decadal
Survey.  That recommendation envisioned a wide-field observatory with
near infrared (NIR) capabilities to complement LSST's optical
capabilities; the Decadal Survey recognized the obvious synergy between
WFIRST and LSST.  WFIRST's design has evolved since 2010 and the design
being pursued for a mid-2020s launch uses an existing $2.4$m telescope
donated to NASA, giving WFIRST capabilities not envisioned by the
Decadal Survey.  WFIRST has 3 primary science objectives:

\begin{itemize}
\item Determine the nature of the dark energy that is driving the
current accelerating expansion of the universe using a combination of
weak lensing, galaxy clustering (including Baryon Acoustic Oscillations
and Redshift Space Distortions), and supernovae type Ia (SN).
\item Study exoplanets through a statistical microlensing survey and via
direct imaging and spectroscopy with a coronagraph.
\item Perform NIR surveys of the galactic and extragalactic sky via a
Guest Observer program.
\end{itemize}

WFIRST will be at L2 to enable the thermal stability needed for the
precise astrometric, photometric, and morphological measurements
required for these science goals. The baseline WFIRST mission
architecture is described in detail in the final report of the WFIRST
Science Definition Team \citep{2015arXiv150303757S}
%(arxiv/1503.03757).
The WFIRST Wide Field
Instrument(WFI) has a NIR focal plane with a $\sim0.28$ square degree
field of view made up 18 4k$\times$4k Teledyne H4RG NIR detectors and will
have imaging capabilities from $0.5-2$ microns and grism spectroscopy
capabilities from $1-2$ microns with $R\sim461\lambda$.  The WFI
also contains an Integral Field Channel (IFC) spectrometer with $R\sim100$
resoluton over the range $0.6-2$ microns for SN follow up. The exoplanet
coronagraph will have imaging ($0.43-0.97$ microns) and spectroscopic
($0.6-0.97 $ microns) capabilities with a contrast ratio of 1 part in a
billion.

WFIRST's  5 year primary mission is envisioned to have $\sim2$years dedicated to a
$\sim2200$ square degree High Latitude Survey (HLS) for weak lensing and
galaxy clustering,  $\sim1$ year of microlensing observations divided into 6
seasons, $\sim0.5$ years of SN search and follow-up, $\sim0.8$ years dedicated to
the coronagraph and the remaining time dedicated to competitively selected Guest
Observer observations. WFIRST has no expendables that would prevent an
extended mission of 10 years or longer, and an extended mission will likely be
given over entirely to Guest Observer observations.

The synergy with LSST is very promising indeed. In this chapter we aim
to  lay out three  specific projects in the three main WFIRST science
areas, and test the simulated LSST Observing
Strategies for their performance in each case. Then, we use these
results to design a suite of modified LSST Observing Strategies, which
we propose as new \OpSim simulation runs.



% --------------------------------------------------------------------

% ====================================================================
%+
% SECTION:
%    WFIRST_weaklensing.tex
%
% CHAPTER:
%    wfirst.tex
%
% ELEVATOR PITCH:
%    Explain in a few sentences what the relevant discovery or
%    measurement is going to be discussed, and what will be important
%    about it. This is for the browsing reader to get a quick feel
%    for what this section is about.
%
% COMMENTS:
%
%
% BUGS:
%
%
% AUTHORS:
%    Phil Marshall (@drphilmarshall)  - replace with your name and GitHub username!
%-  Jason Rhodes @jasondrhodes
% ====================================================================

\section{Cosmological Weak Lensing with WFIRST and LSST}
\def\secname{\chpname:weaklensing}\label{sec:\secname}

%note to Phil-  I am not sure if there will be another section that serves as an intro to WFIRST, but I am doing that here
%also, I think we should make this about more than just WL.  Can we change the name and focus of the section to the High Latitude Survey.  The driver %is still probably weak lensing!



\credit{jasondrhodes},
{\it and others to follow}
\textbf{Intro to WFIRST} 
The Wide Field Infrared Survey Telescope (WFIRST) is a NASA mission that entered Phase A in February 2016.  WFIRST was the highest recommendation for large space missions in the 2010 New Worlds New Horizons Decadal Survey.  That recommendation envisioned a wide-field observatory with near infrared (NIR) capabilities to complement LSST's optical capabilities; the Decadal Survey recognized the obvious synergy between WFIRST and LSST.  WFIRST's design has evolved since 2010 and the design being pursued for a mid-2020s launch uses an existing $2.4$m telescope donated to NASA, giving WFIRST capabilities not envisioned by the Decadal Survey.  WFIRST has 3 primary science objectives:
\begin{itemize}
\item Determine the nature of the dark energy that is driving the current accelerating expansion of the universe using a combination of weak lensing, galaxy clustering (including Baryon Acoustic Oscillations and Redshift Space Distortions), and supernovae type Ia (SN).
\item Study exoplanets through a statistical microlensing survey and via direct imaging and spectroscopy with a coronagraph. 
\item Perform NIR surveys of the galactic and extragalactic sky via a Guest Observer program. 
\end{itemize}
WFIRST will be at L2 to enable the thermal stability required for the precise astrometric, photometric, and morphological measurements required for these science goals. The baseline WFIRST mission architecture is described in detail in the final report of the WFIRST Science Definition Team (arxiv/1503.03757). The Wide Field Instrument(WFI) has a NIR focal plane with a $\sim0.28$ square degree field of view made up 18 4k$\times$4k Teledyne H4RG NIR detectors will have imaging capabilities from $0.7-2$ microns and grism spectroscopy capabilities from $1.35-1.89$ microns with $R\sim461\lambda$.  The WFI also contains an Integral Field Unit (IFU) spectrometer with $R\sim100$ resoluton over the range $0.6-2$ microns for SN follow up. The exoplanet coronagraph will have imaging ($0.43-0.97$ microns) and spectroscopic ($0.6-0.97 $ microns) capabilities with a contrast ratio of 1 part in a billion.

WFIRST's  6 year primary mission will have 2 years dedicated to a $\sim2200$ square degree High Latitude Survey (HLS) for weak lensing and galaxy clustering,  1 year of microlensing observations divided into 6 seasons, $0.6$ years of SN search and follow-up, one year dedicated to the coronagraph and 1.4 years dedicated to competitively selected Guest Observer observations. WFIRST has no expendables that would prevent an extended mission of 10 years or longer, and an extended mission would be given over entirely to Guest Observer observations. 

 \textbf{WFIRST's High Latitude Survey (HLS)}
 WFIRST's HLS will cover 2200 square degrees in 4 NIR photometric filters (3 of which will be sufficiently sampled for weak lensing shape measurements) and NIR grism spectroscopy.  The benefits of overlapping spectroscopic and photometric surveys for dark energy constraints and systematics mitigation are strong.  The primary scientific driver of the photometric portion of the WFIRST HLS is weakg gravitational lensing, but there is a wide range of ancillary science that will be possible with the publicly available WFIRST HLS data (see for instance, the SDT report mentioned above).  However, the requirements on the HLS are largely set by constraints from weak lensing measurements.  Each galaxy in the WFIRST weak lesing survey needs to have an accurate photometric redshift.  This requires optical photometry that reaches the depth of the NIR photometry WFIRST will acquire ($J~27AB$).  \emph{Thus, the WFIRST weak lensing survey will require the full  10-year LSST depth in 4 optical bands for optimal photometric redsfhift determination}. 
 
There is strong benefit not jsut to WFIRT, but to LSST, in coordinating observations of the WFIRST HLS survey field. The combination of full-depth LSST data and WFIRST HLS NIR data will provide the gold standard in photo-zs.  Furthermore, WFIRST grism observations over the same area will provide many millions of high quality slitless spectra and WFIRST’s IFU can be run in parallel with WFI observations to provide many more very accurate spectroscopic redshifts in the survey area.  Thus, the WFIRST photometric data will help to provide better LSST photo-zs and  WFIRST will also provide many of the spectra needed for a training set to calibrate the photo-zs for both missions.  A further benefit to LSST might be the reduced need for LSST observations at the reddest end of the LSST wavelength range (the z and y filters), where both the atmosphere and the physics of CCDs make ground-based observations less efficient than what WFIRST can achieve. Finally, the joint processing of LSST and WFIRST data will provide better object deblending parameters than LSST can achieve alone; WFIRST will be able to provide a morphological prior for the deblending of LSST images.

% --------------------------------------------------------------------

\subsection{Target measurements and discoveries}
\label{sec:\secname:targets}

We propose an acceleration of the LSST survey over about $10\%$ of the LSST survey area (the $\sim2200$ WFIRST HLS) such that the full LSST ten years survey depth is reached on a timescale that maximizes the joint usefulness of LSST and WFIRST data on that area.  Assuming the two year WFIRST HLS is taken in the first four years of a WFIRST mission that launches in 2024, this would require reaching full LSST depth over that area in $\sim2028$ rather than $\sim2032$. Since the HLS area is roughly $1/8$ as large as the LSST ``Main Survey"'' region, this could be achieved by devoting 1.25 years of LSST observations to the HLS area, assuming that it covers a wide enough range of Right Ascension.  More practically, it could be achieved by devoting 25\% of LSST observing time to this area during each of the first 5 years of the LSST survey, which doubles the time it would naturally be observed during those years at a modest reduction in coverage of the rest of the Main Survey area during that time period.   Given existing plans to speed up the LSST cadence over small sub-areas of the LSST survey, this may only require coordination of the locations of the accelerated LSST area and the WFIRST HLS. As LSST and WFIRST progress, there is a mutual benefit in continuing discussions about the optimal joint observation schedule.

It is possible that the WFIRST data might allow for shallower LSST data in the reddest LSST filter in the overlap region, and this must be quantified. 


% --------------------------------------------------------------------

\subsection{Metrics}
\label{sec:\secname:metrics}
A simple, first order metric would be the amount of LSST/WFIRST overlapping survey area that reaches the full LSST depth when the WFIRST HLS is completed.  Such a metric is straightforward, but not quantitative until the 2020s, when the WFIRST launch date and survey plan is more definite.  A slightly more complicated metric could include the pace at which the overlapping LSST/WFIRST survey areas are both taken to full depth, since this would make each data set maximally useful to the US community (or anyone with immediate access to both WFIRST and LSST data).  WFIRST data is unlikely to have any proprietary period.  Current plans call for the WFIRST HLS to be conducted in multple passes, but the exact survey pattern is still undecided, so this metric is also not quantifiable yet.   

There may be some reduced need for the the LSST reddest bands in the WFIRST HLS overlap area, which should also be folded into the metric. 

% --------------------------------------------------------------------

\subsection{OpSim Analysis}
\label{sec:\secname:analysis}

The default survey strategy would only achieve the full LSST photometric depth over the WFIRST HLS after 10 years of survey ($\sim2032).  


% --------------------------------------------------------------------

\subsection{Discussion}
\label{sec:\secname:discussion}

Increasing the cadence of the LSST survey over $\~sim10\%$ of the LSST survey has science benefits that go far beyond the LSST/WFIRST synergy described here.  There are benefits to certain aspects of time-domain science.  Every effort should be made to coordinate all discussions of increased survey cadence (resulting in full LSST depth well before 10 years) over sub-areas of the LSST survey footprint.  Specific attention should be paid to whether the accelerated portions of the LSSt survey can completely overlap the WFIRST HLS, and whether the position of the WFIRST HLS can be determined, in part, by other science drivers within LSST.  This will require close LSST and WFIRST coordination at the Project levels.


% ====================================================================

\navigationbar


% PJM: commented out pending check-in from Dave Rubin
% ====================================================================
%+
% SECTION:
%    WFIRST_supernovae.tex
%
% CHAPTER:
%    wfirst.tex
%
% ELEVATOR PITCH:
%-
% ====================================================================

\section{Supernova Cosmology with WFIRST and LSST}
\def\secname{\chpname:supernovae}\label{sec:\secname}

\credit{rubind}

% This individual section will need to describe the particular
% discoveries and measurements that are being targeted in this section's
% science case. It will be helpful to think of a ``science case" as a
% ``science project" that the authors {\it actually plan to do}. Then,
% the sections can follow the tried and tested format of an observing
% proposal: a brief description of the investigation, with references,
% followed by a technical feasibility piece. This latter part will need
% to be quantified using the MAF framework, via a set of metrics that
% need to be computed for any given observing strategy to quantify its
% impact on the described science case. Ideally, these metrics would be
% combined in a well-motivated figure of merit. The section can conclude
% with a discussion of any risks that have been identified, and how
% these could be mitigated.
%
% A short preamble goes here. What's the context for this science
% project? Where does it fit in the big picture?

The WFIRST SN survey seeks to measure thousands of SNe Ia with excellent systematics control over a two-year period. The Science Definition Team (SDT) outlined a three-tiered cadenced imaging survey: wide to $z=0.4$ (27.44 square degrees), medium to $z=0.8$ (8.96 square degrees), deep to $z=1.7$ (5.04 square degrees). SNe discovered in the imaging would be followed with IFU spectrophotometry, helping to monitor changes in SN physical parameters and the extinction distribution with redshift. However, due to the slew time and high read noise in short exposures, the wide survey would be very inefficient, spending a bit more than half of its time on slews, while the medium survey would spend a significant fraction of its time slewing. However, the LSST DDFs offer a path to high signal-to-noise, well calibrated, multi-band optical imaging over an even larger area than WFIRST can survey. If the wide and medium tiers are replaced with LSST DDF discoveries, then WFIRST can offer spectrophotometry (with good host-galaxy subtraction) for $\sim$ 2,000 LSST SNe, with screening spectra for $\sim$ 1-2,000 more. As the WFI and IFU operate in parallel, this survey could provide sparsely sampled NIR imaging for $\sim$ 5,000 SNe up to $z = 1$ at the same time as the spectroscopy. The joint survey would thus provide systematics control (almost certainly better than either survey alone), as well as a cross-check of LSST photometric typing and host-galaxy-only redshift assignment.


% --------------------------------------------------------------------

\subsection{Target measurements and discoveries}
\label{sec:\secname:targets}

% Describe the discoveries and measurements you want to make.
%
% Now, describe their response to the observing strategy. Qualitatively,
% how will the science project be affected by the observing schedule and
% conditions? In broad terms, how would we expect the observing strategy
% to be optimized for this science?


The targets of the measurements are related to those enumerated in Section~\ref{sec:supernovae:targets}. The SNe must be detected $\sim$ 10 observer-frame days before maximum light, so that there is time for a shallow screening spectrum before deeper spectrophotmetry around maximum. There should be enough visits per filter so that some photometric screening can be done before WFIRST triggers any spectroscopy. There should be an identification of the host galaxy (if seen), so that joint WFIRST/LSST photometric redshifts can be used to provide a distance-limited sample (minimizing selection effects). Finally, the light curve should continue after the SN has been sent to WFIRST, so that important light-curve parameters (date of maximum, rise time and decline time, etc.) can be measured.

%All these goals can be likely be met with $\sim 3$ day rest-frame cadence ($\sim 5$ observer-frame days). LSST would measure NUV to rest-frame $V$-band (with WFIRST providing redder wavelength coverage), or observer-frame $grizY$. For a plausible SN Ia (based on the rising light curve), a series of typing/sub-typing spectra would be triggered, with increasing depth, as the confidence grew that the transient was a SN Ia. DR: in my simulations, I've assumed a depth for each filter of 26th magnitude (probably not realistic for $Y$-band, but very feasible for the other filters); is this too shallow? LSST would contribute 4 transients per day to the pool of objects observed by WFIRST. In practice, the LSST DDFs will contain more SNe Ia than this, so a random sample (perhaps sculpted in redshift) should be sent for observations.


% --------------------------------------------------------------------

\subsection{Metrics}
\label{sec:\secname:metrics}

% Quantifying the response via MAF metrics: definition of the metrics,
% and any derived overall figure of merit.

The primary metrics are based on constraining cosmological parameters; the DETF FoM is standard. For the joint observations proposed here, this FoM increases about 20\%, from $\sim 300$ for WFIRST alone (with a Stage IV CMB constraint) to $\sim 370$. However, the number of SNe at $z \sim 0.5$ increases by $\sim$ 50\% over a WFIRST-only survey, improving some $w(z)$-derived FoM values by 40\%.

The cosmological metric will essentially depend on the number of SNe meeting the above targets. It will degrade if core-collapse SNe are mistakenly sent to WFIRST for followup, if SNe Ia are sent to WFIRST but the LSST light curve is lost due to weather gaps, or if the cadence and depth simply do not allow the measurement of light curve parameters. These metrics will be strongly related to those in Section~\ref{sec:supernovae:metrics}, but with more emphasis on the rising portion of the light curve.

% --------------------------------------------------------------------

%\subsection{OpSim Analysis}
%\label{sec:\secname:analysis}

% OpSim analysis: how good would the default observing strategy be, at
% the time of writing for this science project?


% --------------------------------------------------------------------

%\subsection{Discussion}
%\label{sec:\secname:discussion}

% Discussion: what risks have been identified? What suggestions could be
% made to improve this science project's figure of merit, and mitigate
% the identified risks?

% ====================================================================
%
\subsection{Conclusions}

Here we answer the ten questions posed in
\autoref{sec:intro:evaluation:caseConclusions}. As WFIRST will only cadence a few 10's of square degrees, we assume that the overlap region will be contained in the DDFs, not in the main survey. We thus place no constraints on the main survey.

\begin{description}

\item[Q1:] {\it Does the science case place any constraints on the
tradeoff between the sky coverage and coadded depth?}

\item[A1:] In terms of direct overlap with WFIRST, there is no constraint. There are secondary considerations (such as constraining  Milky Way extinction and probing isotropy) that may prefer a larger number of square degrees, but this has not been investigated.

\item[Q2:] {\it Does the science case place any constraints on the
tradeoff between uniformity of sampling and frequency of sampling?}

\item[A2:] No constraints.

\item[Q3:] {\it Does the science case place any constraints on the
tradeoff between the single-visit depth and the number of visits
(especially in the $u$-band where longer exposures would minimize the
impact of the readout noise)?}

\item[A3:] In the DDFs, the SN Ia science would not be harmed by having longer exposures and less-frequent visits, at least if there was one observation every few days in each filter. No constraint on the main survey.

\item[Q4:] {\it Does the science case place any constraints on the
Galactic plane coverage (spatial coverage, temporal sampling, visits per
band)?}

\item[A4:] No constraints.

\item[Q5:] {\it Does the science case place any constraints on the
fraction of observing time allocated to each band?}

\item[A5:] This has not been quantitatively investigated. Based on experience with other programs, the reddest filters (z and Y) should receive the most time.

\item[Q6:] {\it Does the science case place any constraints on the
cadence for deep drilling fields?}

\item[A6:] To obtain good light curves, we would want at least one observation every few days per filter. This is also a requirement for photometric typing to obtain reasonable purity if WFIRST spectroscopy is to be triggered. Note: of all the questions, this one is the most important to the program.

\item[Q7:] {\it Assuming two visits per night, would the science case
benefit if they are obtained in the same band or not?}

\item[A7:] No benefit.

\item[Q8:] {\it Will the case science benefit from a special cadence
prescription during commissioning or early in the survey, such as:
acquiring a full 10-year count of visits for a small area (either in all
the bands or in a  selected set); a greatly enhanced cadence for a small
area?}

\item[A8:] For the DDFs, such observations might help provide deep SN-free (template) images or host-galaxy photometric redshifts. This has not been quantitatively investigated. No constraints on the main survey.

\item[Q9:] {\it Does the science case place any constraints on the
sampling of observing conditions (e.g., seeing, dark sky, airmass),
possibly as a function of band, etc.?}

\item[A9:] As long as the DDF observations reach the targeted depths, there is no benefit to the SN Ia science from having specific observing conditions.

\item[Q10:] {\it Does the case have science drivers that would require
real-time exposure time optimization to obtain nearly constant
single-visit limiting depth?}

\item[A10:] We do not currently know of any benefit to the SN Ia science from doing this.

\end{description}

% ====================================================================

\navigationbar


% ====================================================================
%+
% SECTION:
%    WFIRST/microlensing.tex
%
% CHAPTER:
%    wfirst.tex
%
% ELEVATOR PITCH:
%    Explain in a few sentences what the relevant discovery or
%    measurement is going to be discussed, and what will be important
%    about it. This is for the browsing reader to get a quick feel
%    for what this section is about.
%
% COMMENTS:
%
%
% BUGS:
%
%
% AUTHORS:
%    Phil Marshall (@drphilmarshall)  - replace with your name and GitHub username!
%-
% ====================================================================

\section{Exoplanetary Microlensing with WFIRST and LSST}
\def\secname{\chpname:microlensing}\label{sec:\secname}

\credit{davidbennett},
\credit{rachelstreet},
{\it and others to follow}

This individual section will need to describe the particular
discoveries and measurements that are being targeted in this section's
science case. It will be helpful to think of a ``science case" as a
``science project" that the authors {\it actually plan to do}. Then,
the sections can follow the tried and tested format of an observing
proposal: a brief description of the investigation, with references,
followed by a technical feasibility piece. This latter part will need
to be quantified using the MAF framework, via a set of metrics that
need to be computed for any given observing strategy to quantify its
impact on the described science case. Ideally, these metrics would be
combined in a well-motivated figure of merit. The section can conclude
with a discussion of any risks that have been identified, and how
these could be mitigated.

A short preamble goes here. What's the context for this science
project? Where does it fit in the big picture?


% --------------------------------------------------------------------

\subsection{Target measurements and discoveries}
\label{sec:\secname:targets}

Describe the discoveries and measurements you want to make.

Now, describe their response to the observing strategy. Qualitatively,
how will the science project be affected by the observing schedule and
conditions? In broad terms, how would we expect the observing strategy
to be optimized for this science?


% --------------------------------------------------------------------

\subsection{Metrics}
\label{sec:\secname:metrics}

Quantifying the response via MAF metrics: definition of the metrics,
and any derived overall figure of merit.


% --------------------------------------------------------------------

\subsection{OpSim Analysis}
\label{sec:\secname:analysis}

OpSim analysis: how good would the default observing strategy be, at
the time of writing for this science project?


% --------------------------------------------------------------------

\subsection{Discussion}
\label{sec:\secname:discussion}

Discussion: what risks have been identified? What suggestions could be
made to improve this science project's figure of merit, and mitigate
the identified risks?


% ====================================================================

\navigationbar


% --------------------------------------------------------------------

% ====================================================================
%+
% SECTION:
%    WFIRST_proposals.tex
%
% CHAPTER:
%    wfirst.tex
%
% ELEVATOR PITCH:
%    Maximizing the overlap between LSST and WFIRST is likely to be a fruitful
%    approach to modifying the LSST observing strategy. Let's pull together the
%    findings from the three WFIRST science cases and propose some OpSim
%    experiments.
%
%-
% ====================================================================
% 
% \section{Maximizing the Synergy between WFIRST and LSST}
% \def\secname{\chpname:proposals}\label{sec:\secname}
%
% \credit{jasondrhodes}
%
% In the previous sections, we introduced figures of merit for each WFIRST
% science project, and tested the existing LSST observing strategies for
% their performance. In the process we learned some of the shortcomings of the
% baseline LSST strategy, and suggested some alternative cadence
% options. In this section, we will pull those suggestions together to propose a
% suite of new \OpSim experiments.
%
% % Make table here.
%
% % ====================================================================
%
% \navigationbar


% --------------------------------------------------------------------

\navigationbar
