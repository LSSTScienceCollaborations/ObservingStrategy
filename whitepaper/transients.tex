% --------------------------------------------------------------------

\chapter[Explosive Transients]{Explosive Transients}
\def\chpname{transients}\label{chp:\chpname}

% \noindent {\it
% Mike Lund, Ashish Mahabal, Stephen Ridgway, Lucianne Walkowicz, Rahul Biswas, Michelle Lochner, Jeonghee Rho, Eric Bellm...
% }

Chapter editors:
\credit{ebellm}?,
\credit{fedhere}?,


% --------------------------------------------------------------------

\section{Introduction}

Explosive transients such as novae, supernovae (SNe), and gamma-ray bursts (GRBs) probe the final stages of stellar evolution.
Cadence choices are vital to determining LSST's ability to discover and characterize these events.
Different types and time scales of phenomena benefit from different sampling strategies---sometimes significantly different.  Competing objectives described in this chapter are at the heart of LSST observing strategy and cadence design.

When evaluating a particular observation or series of observations in light of how they perform for a specific science case, it may be helpful to think of metrics as lying along a continuum between discovery and characterization. Discovery requires a minimum amount of information to recognize an event or object as a candidate of interest, which necessarily involves some level of bare-bones characterization (upon which said recognition is based); rich characterization, on the other hand, implies that an event may not only be recognized as a candidate of interest, but basic properties of the event or object may be determined from the observation (e.g. including but not limited to classification of the event). The interpretation of a given metric along this continuum has implications for the subsequent action and analysis required, particularly as regards possible follow-up observations with other facilities.

In this chapter we focus on LSST's potential to advance the science of transients as astrophysical objects; the use of SNe for cosmology is discussed in \autoref{chp:cosmo}.

% --------------------------------------------------------------------

\section{Transient Events}
\def\secname{\chpname:transients}\label{sec:\secname}

\credit{StephenRidgeway},
\credit{AshishMahabal},
\credit{ohadshemmer}

Transient events may benefit from substantial temporal sampling
(matched to the time constant of the event) with color information
(perhaps contemporaneous) to support characterization and
classification, obtained over the limited duration of interest.
Transient events slower than $\sim$ weeks may be adequately sampled by
a uniform LSST cadence.  Faster events may require special scheduling
strategies.  For some event types, LSST can only be expected to
provide a discovery service, and followup will necessarily be
performed elsewhere.

% --------------------------------------------------------------------

\subsection{Targets and Measurements}
\label{sec:\secname:targets}

The class of transients includes a heterogeneous assortment of objects and phenomena.

\begin{center}
\begin{tabular}{| p{3cm} | p{8cm} | l | l |}
\hline Transient Type & Examples of target science & Amplitude & Time Scale\\
\hline
Flare stars & Flare frequency, energy, stellar age & large & min\\
Cataclysmic variables  & Interacting binaries, stellar evolution, compact objects, explosive events & small & min\\
Supernovae & SN physics, mass loss, distance scale, cosmology& large & days\\
Active galactic nuclei & Galaxy evolution, reverberation mapping, black hole physics& large & weeks\\
Stellar microlensing & Exoplanet statistics& large & hours\\
Gamma ray bursts & Optical discovery and characterization& large & min\\
LIGO detections & Source position and characterization& unknown & min\\
Serendipity & Discovery and characterization& unknown & unknown\\
Tidal Disruption Events & Discovery and characterization & large & days\\
 \hline \end{tabular}
 \end{center}

Among the targets in this list, only AGN are likely to be sampled with sufficient resolution by a uniform LSST cadence - in fact for AGN, a challenge may be to spread visits sufficiently in time to avoid excessive seasonal gaps.

For very short lived phenomena (stellar flares, CV outbursts, GRBs, LIGO events) it appears that the function of LSST will be to provide discoveries and/or simple characterization.  Followup to discovery/identification, if required, will surely take place elsewhere.

For events requiring intensive monitoring (stellar microlensing, exoplanet transits), the followup will certainly take place elsewhere.

Supernovae fall in an intermediate time range.  LSST will provide multiple visits in multiple filters during the typical SN duration.  This sampling may be insufficient for many (including key) science objectives.  However, a moderate, and feasible, change to LSST observing strategy, may enhance the sampling for part of the sky part of the time, greatly enhancing the usefulness of SN observations.

For Tidal Disruption Events, where the fading time-scale is much more gradual (over weeks to months) than the rise time-scale it will be worth checking - through a metric - how many will be missed (as alerts). Ref. Science Book: 10.6.1. Also ref. recent papers.

Serendipitous discoveries are of course harder to plan for.  An ideal transient discovery survey would include heavy coverage of all time scales. LSST will cover longer time periods well, but will have to make some choices of emphasis in coverage of shorter time-scales.


% --------------------------------------------------------------------

\subsection{Metrics}
\label{sec:\secname:metrics}

\begin{center}
\begin{tabular}{| p{5cm} |p{10cm} |}
\hline Metric & Description\\
\hline
SNe & Number of events adequately sampled\\
Serendipity & Histogram of median visit series length vs maximum visit spacing within the series\\
  \hline \end{tabular}
 \end{center}

The metrics for SNe will be highly specialized and based on the best available understanding of SN light curve analysis and the expected event population.

The suggested metric set for serendipity is based on the simple-minded idea that a novel transient will be characterized by a band-limited, finite waveform, and that a useful observation series will consist of a series of samples extending over the duration of the event, with at least critical sampling of the fastest variations.  Since for some event durations the number of useful time series will be small, it may be useful to look not at the median length, but the median length of a subset size preselected as possibly useful (e.g. the$10^3$ longest series).

Lund et al. (2015; \url{http://arxiv.org/pdf/1508.03175.pdf}) discuss three metrics that have been incorporated into the MAF. Two of these metrics deal explicitly with time variable behavior: a) observational triplets, and b) detection of periodic variability.

\subsubsection{Observational triplets (TripletMetric)}

This metric provides a means of evaluating whether a transient event on some timescale of interest has been detected, by testing for a sequence of three observations. The object must be detected in quiescence, followed by two subsequent detections above some threshold; this sequence of observations allows the magnitude of the change to be measured, as well as its timescale.

This metric may be used for a variety of astrophysical phenomena, in particular transient events on variable objects (e.g. novae, stellar flares), in that it is general with respect to the amplitude of the brightness variation as well as the timescale of said change. The requirement of a detection prior to outburst does constrain it to objects that have already been detected in quiescence (in other words, not necessarily ``true'' transients), although there may be some cases where this is not the case (e.g. a supernova occurring on a previously detected galaxy). In practice, the time lapse between the first and second and second and third observations must be comparable (between 10$^2$ and 10$^5$ seconds) for discovery. This metric may be calculated for a given OpSim run and then further reduced to a histoogram in logarithmic time bins; the minimum number of bins to construct an interesting sample of objects is source-dependent.

\subsubsection{Transient Metric (transientAsciiMetric)}

Calculate what fraction of transients would be detected using an ascii input file for the lightcurve.

\subsection{Proposed Metrics}

The following is a raw list of metric ideas; these need specificity and further description.

The triplet metric may also be altered to include filter constraints, such that the triplets are drawn from a single filter or subset of filters.

Color evolution constraint: triplets of observations in a specific color (really requirement of two triplets in multiple filters)

  2D Histogram of delta t?s between observations constituting a triplet

Histogram of median visit series length vs maximum visit spacing within the series

Number of events adequately sampled

% --------------------------------------------------------------------

\subsection{OpSim Analysis}
\label{sec:\secname:analysis}

Analysis shows that current simulations provide  poor coverage in any one filter for transient events longer than a deep drilling session ($\sim$30 minutes) and shorter than $\sim$ weeks.

Simulated performance for SN observations must be analyzed for both main survey and mini-survey (deep drilling) productivity.  It is considered that current simulated schedules give inadequate performance for SN science.



% --------------------------------------------------------------------

\subsection{Discussion}
\label{sec:\secname:discussion}

Community studies are providing improving SNe metrics, and continuing communication between the SN and LSST communities is essential to tuning the observing strategy to deliver the SN time series that are needed and possible.

Improving LSST science return for SNe will also improve sampling of all transients with similar or somewhat shorter characteristic times.  Non-uniform survey strategies (rolling cadence) can significantly improve the LSST performance for faster transients.  Interpretation of multiple filters for novel events may be powerful, or problematic, since color may be uncertain.

Some insight into fast transients may be available from image pairs  or triples (as opposed to more complete series).  These include the pair of images in a visit - which could be useful in studying the rise time of an extremely fast event.  This includes the characteristic grouping of visits (typically 0.5 to 1.0 hour separation) planned for purposes of identifying asteroids.  It also includes fortuitous multiple sampling due to field overlap, providing additional sampling, which may be random or systematic, depending on the scheduling, on a time scale of minutes to hours.  The sampling benefits of this fortuitous overlap have not yet been investigated.


\navigationbar

% --------------------------------------------------------------------

% ====================================================================
%+
% SECTION:
%    section-name.tex  % eg lenstimedelays.tex
%
% CHAPTER:
%    chapter.tex  % eg cosmology.tex
%
% ELEVATOR PITCH:
%    Explain in a few sentences what the relevant discovery or
%    measurement is going to be discussed, and what will be important
%    about it. This is for the browsing reader to get a quick feel
%    for what this section is about.
%
% COMMENTS:
%
%
% BUGS:
%
%
% AUTHORS:
%    Phil Marshall (@drphilmarshall)  - put your name and GitHub username here!
%-
% ====================================================================

\section{Gravitational Wave Sources}
\def\secname{gw}\label{sec:\secname}

\noindent{\it Raffaella Margutti, Z. Doctor, W. Fong, Z. Haiman, V. Kalogera, V. Trimble, B.~A. Zauderer } % (Writing team)

% This individual section will need to describe the particular
% discoveries and measurements that are being targeted in this section's
% science case. It will be helpful to think of a ``science case" as a
% ``science project" that the authors {\it actually plan to do}. Then,
% the sections can follow the tried and tested format of an observing
% proposal: a brief description of the investigation, with references,
% followed by a technical feasibility piece. This latter part will need
% to be quantified using the MAF framework, via a set of metrics that
% need to be computed for any given observing strategy to quantify its
% impact on the described science case. Ideally, these metrics would be
% combined in a well-motivated figure of merit. The section can conclude
% with a discussion of any risks that have been identified, and how
% these could be mitigated.

The first detection of Gravitational Waves (GW) by the advanced LIGO/Virgo collaboration \citep{Abbott16, Abbott09, Acernese08} has recently opened a new window of exploration into our Universe. The amount of information that can be revealed by the properties of the GW emission is immense and holds promises for revolutionary insights, including accurate masses and spins of neutron stars and black holes, tests of General Relativity and an accurate census of the neutron star (NS) and black hole (BH) populations that might challenge our current understanding of massive stellar evolution. However, GW events are poorly localized (10-100 deg$^2$ at the time of LSST operations). The identification of EM counterparts would provide precise localization and distance measurements, in addition to the necessary astrophysical context (e.g. host galaxy properties, connection to specific stellar populations) to fully exploit the revolutionary power of this new GW era.


% --------------------------------------------------------------------

\subsection{Target measurements and discoveries}
\label{sec:\secname:targets}

The first GW event was found to be associated with the merger of two black holes \citep{Abbott16,Abbott16b}. Although no EM counterpart was expected to accompany a black-hole black-hole (BBH) merger, it seems now possible that even BBH mergers  might produce short GRB-like EM emission \citep{Connaughton16, Loeb16,Zhang16,Perna16,Stone16}. Indeed, in analogy with supermassive BH mergers, shocks might develop in the just-formed circumbinary accretion disk (if a disk forms), which can produce a bright afterglow following the BBH merger (e.g. \citealt{Lippai08,Corrales10,Schnittman13}). Albeit speculative in nature, it is advisable to keep an open mind about the possibility of EM counterparts to BBH mergers. 

The most promising and better understood EM counterparts to GW events are ``kilonovae" \citep{Li98, Metzger10, Metzger12,Kasen13,Barnes13}. Kilonovae are short-lived (typical time scale of one week), apparently faint ($z\sim21$ mag at peak at 120 Mpc), red ($i-z\approx1$ mag), isotropic transients (Fig. \ref{Fig:kilonova}) due to the radioactive decay of r-process elements synthesized in the merger ejecta of a NS-NS or NS-BH system. These merging systems are the favored progenitors of short GRBs. Indeed, the signature of a kilonova emission has been recently found following the short GRB\,130603B \citep{Berger13,Tanvir13}. The key piece of information that enabled the discovery of kilonova-like emission associated with  this short GRB was its sub-arcsecond localization enabled by the detection of the optical afterglow, which allowed for an effective kilonova search with the Hubble Space Telescope (Fig. \ref{Fig:kilonova}). In contrast, the typical localization region of GW events in the LSST era is expected to be of the order of a few tens of square degrees \citep{aaa+13}. It is thus clear that the major challenges faced by the optical follow-up of GW events is represented by the combination of poor localizations with faint and fast evolving red electromagnetic counterparts.

The detection of an optical counterpart in conjunction with a GW event will significantly leverage the GW signal.
LSST, with its the wide FOV, wavelength coverage and exquisite sensitivity is uniquely poised to identify and characterize counterparts to GW events. 

\begin{figure}
\vskip -0.0 true cm
\centering
\includegraphics[scale=0.85]{figs/transients/kilonovaBerger.png}
\caption{Kilonova signature in the short GRB\,130603B as revealed by the Hubble Space Telescope (HST). The Magellan and Gemini telescopes sampled the optical afterglow of the GRB (dotted lines). The kilonova light starts to dominate the emission in the H band around a few days after the merger. Thick and dashed lines: theoretical kilonova models from \cite{Barnes13} showing that kilonovae are fast-evolving, faint and red transients. The light-curve of the SN\,2006aj associated with the long GRB\,060218  is also shown for comparison. From \cite{Berger13}.}
\label{Fig:kilonova}
\end{figure}


% --------------------------------------------------------------------



% --------------------------------------------------------------------

\subsection{OpSim Analysis and Discussion}
\label{sec:\secname:analysis}

Effective follow up of GW triggers relies on the capability to sample a relatively large portion of the sky, repeatedly, over a time scale $<1$ week, with different filters \citep{Cowperthwaite15}. In the optical band, the kilonova signature is expected to be more prominent in the $i$, $z$  and $y$ filters, which we identify as the most promising filters for the kilonova search. We emphasize however that another set of contemporaneous observations in  a ``bluer" filter is necessary to acquire color information and distinguish kilonovae from other fast-evolving transients.

We use the median inter-night gap  for visits in the same filter derived from the candidate Baseline Cadence \texttt{minion\_1016} to show that, in the absence of a Target of Opportunity (ToO) capability, it is \emph{not} possible for LSST  to play a major role in the identification of EM counterparts of GW triggers.  

To identify kilonova candidates we need at least 2 observations acquired within $\sim 1$ week  of the GW event \citep{Cowperthwaite15}.
Using the inter-night gap distribution for visits in the $y$ filter (which is the most promising filter for a kilonova search), the area of the sky covered with cadence  $\Delta t<7$ days at any given time, is $A_{sky}\sim 3000$ deg$^2$ (including deep drilling fields).  This is the area that can be searched for fast evolving transients.  Two important considerations follow:

\begin{itemize}
\item[(1)] $A_{sky}$ only covers $P\sim7$\% of the sky. The  probability that the \emph{entire} GW localization region is contained, by chance,  within $A_{sky}$ is thus very small.
\item[(2)] Even if LSST is able to cover a meaningful portion of the GW region, we would still not have color information, and we would thus be unable to filter out contaminating transients.
\end{itemize}

\textbf{We conclude that relying on the serendipitous alignment of the LSST fields with the GW localization map is not an effective strategy to follow up GW triggers and identify their EM counterparts. We thus strongly recommend a ToO capability as part of the baseline LSST operations strategy.}

Ideally, the ToO capability will allow for imaging of the GW localization map at least twice over $\Delta t\lesssim$1 week with a ``red" filter ($i$, $z$  or $y$),  and  will include the possibility to designate a desired set of filters to obtain color information. By the time of LSST operation the typical size of the GW localization region is expected to be 10-100 deg$^2$, which would require a small number of LSST re-pointings. We thus do \emph{not} anticipate a significantly disruptive impact on other LSST campaigns (especially if only the GW triggers with the best localizations in the southern sky are selected for LSST ToOs).

\textbf{At the price of re-shuffling a reasonably small number of fields, \emph{if} equipped with ToO capabilities, LSST can be the premier player in the era of EM follow up to GW sources.}






% ====================================================================

\navigationbar


% --------------------------------------------------------------------

% ====================================================================
%+
% SECTION:
%    grb.tex
%
% CHAPTER:
%    transients.tex
%
% ELEVATOR PITCH:
%-
% ====================================================================

\section{Gamma-Ray Burst Afterglows}
\def\secname{grbs}\label{sec:\secname}

\credit{ebellm}

Gamma-ray bursts (GRBs) are relativistic explosions typically classified
by the temporal duration of their initial gamma-ray emission: Long GRBs,
that mark the endpoint of the lives of some massive stars, and short
GRBs, believed to originate from the merger of binary neutron stars. GRB
emission is known to be beamed: the initial prompt gamma-ray emission is
seen only for observers looking at the jet axis. The longer-wavelength
X-ray, optical, and radio afterglow may be seen both by on- and off-axis
observers.  The latter case is known as an orphan afterglow, due to the
absence of gamma-ray emission. On- and off-axis afterglows are predicted
to have different temporal signatures in the optical: On-axis events
decay as a power-law until a jet break, while off-axis events should be
fainter and show an initial rise (Figure \ref{fig:afterglow_lcs}).
Despite systematic searches, no convincing orphan afterglow candidates
have yet been discovered, limiting our knowledge of the beaming fraction
of GRBs and hence their true rates. Well-sampled orphan afterglow
lightcurves would also permit study of the GRB jet structure.

\begin{figure}[hbt]
\centerline{
\includegraphics[width=0.6\textwidth]{figs/transients/predicted_afterglow_lcs_mag.pdf}
}
\caption{ Predicted light curves of GRB afterglows by off-axis angle
with respect to the jet axis $\theta_{\rm obs}$ \citep[Figure
8.8,][]{2009arXiv0912.0201L}. The forward shock model is derived from
\citet{2002ApJ...576..120T} and assumes a jet half opening angle
$\theta_j = 4^\circ$, the isotropic equivalent energy of $E_{\rm iso} =
5\times10^{53} \rm erg$, ambient medium density $n = 1$ g cm$^{-3}$, and
the slope of the electron energy distribution $\rm p = 2.1$. The
apparent AB $r$-band magnitudes assume a source redshift $z = 1$. }
\label{fig:afterglow_lcs}
\end{figure}

Because of their rarity, in all but one case \citep{2015ApJ...803L..24C}
to date GRBs have been discovered using their prompt emission by hard
X-ray or gamma-ray all-sky monitors. This selection imposes biases on
the population of relativistic explosions we observe. Baryon-loading in
the GRB jet---a ``dirty fireball'' \citep{2003ApJ...591.1097R}---can
lead to on-axis events without gamma-ray emission.  Only one plausible
candidate has been identified to date \citep{2013ApJ...769..130C}.
Discovery of new dirty fireballs---if distinguished from off-axis
events--would clarify the rates of these events and enhance our
understanding of the diversity of stellar death.

LSST is the survey most capable of resolving these decades-old
questions.  Due to its large aperture and etendue, LSST can detect
faint, fast-fading, and rare cosmological events, potentially enabling
population studies of the high-redshift universe.
\citet{2015A&A...578A..71G} estimated LSST could detect 50 orphan
afterglows each year, more than any other planned survey.

%deep survey helps due to time dilation

%beaming fraction and true rates; jet structure; dirty fireballs?
%GRB-SN connection; probe high-z star formation?

%other fast transients: Fast transients and SN shock breakout?  flash spectroscopy

The challenge of detecting and recognizing GRB afterglows in the LSST data in
real time makes this science case a useful proxy for other fast transient
science cases that benefit from $N > 2$ visits per night.  In particular, this
includes discovering supernovae soon after explosion for flash spectroscopy or
shock breakout searches.

% need appropriate cadences to support value of realtime alert stream

% --------------------------------------------------------------------

\subsection{Target measurements and discoveries}
\label{sec:\secname:targets}

GRB afterglow discovery is among the science cases that places the
greatest stress on the LSST cadence.  Because afterglows fade
rapidly---dropping several magnitudes in the first few hours---high
cadence observations are required to detect the fast fading. If an
afterglow candidate can be recognized in real time, it will be possible
to trigger TOO spectroscopy (to measure a redshift and confirm the event
is cosmological), X-ray and radio observations (to detect a high-energy
counterpart and the presence of a jet), and additional photometry (to
characterize the lightcurve evolution).  If there is no source at the
location of the transient in the coadded reference image, two
consecutive observations in the same filter separated by an hour or two
are the minimum required to potentially trigger followup of a
fast-fading event. However, a third observation within a night or
two---ideally in the same filter---would improve the purity of the
sample and reduce the reliance on triggered followup. Observations in
other bands at high cadence are less useful because they require
assumptions about the event's SED and its evolution to determine if a
source is truly fading.

Distinguishing orphan afterglows from on-axis events (whether conventional
GRBs or dirty fireballs) will also require more than two detections.
Orphan events may prove harder to recognize in real time, because they are
intrinsically fainter than on-axis events and show an initial rise rather
than a rapid decay (Figure \ref{fig:afterglow_lcs}).  Additionally, because
of relativistic time dilation, high redshift events are easier to detect,
but these events will be fainter and more difficult to follow up.
Accordingly, population studies of orphan afterglow candidates may by
necessity be conducted with LSST photometry alone.  Such studies may only
be productive if LSST has sufficiently frequent revisits to a field in a
single filter.

% --------------------------------------------------------------------

\subsection{Metrics}
\label{sec:\secname:metrics}

The core figure of merit for GRB afterglows is simply the raw number of
on- and off-axis events detectable in two, three, or more observations,
preferably in a single filter.

The appropriate way to derive these detections is to conduct a Monte
Carlo simulation of a cosmological population of GRBs and fold it
through the LSST observing cadence \citep[cf.][]{2011PASP..123.1034J}.
We are developing this infrastructure for the MAF framework.

In the meantime, simplified metrics can give us a general idea of how well
a given cadence can characterize fast-evolving transients such as GRBs.  We
have created a new metric, \texttt{GRBTransientMetric}, that replaces the
linearly rising and decaying lightcurve in \texttt{TransientMetric} with
the $F \sim t^{-\alpha}$ decay characteristic of on-axis afterglows.  (For
the time being, we neglect the jet break that steepens the rate of decay;
this implies that our detectability estimates are optimistic.)

We simulate random on-axis afterglows using the parameters of
\citet{2011PASP..123.1034J}: the R-band apparent magnitude at 1 minute
after explosion is randomly drawn from a Gaussian with $\mu=15.35$,
$\sigma=1.59$ and decays with $\alpha=1.0$.  For these estimates we
simply assume zero color difference between in all LSST bands.
There are roughly 300 on-axis GRBs per year with these parameters;
we calculate the average fraction of these events which have at least one,
two, or three detections in any single filter.

% Can use https://github.com/lsst/sims_maf/blob/master/python/lsst/sims/maf/metrics/tgaps.py or https://github.com/lsst/sims_maf/blob/master/python/lsst/sims/maf/metrics/cadenceMetrics.py (Inter/Intra-night) to get histograms.  Would be nice to extend to single-band, N-offset

% --------------------------------------------------------------------

\subsection{OpSim Analysis}
\label{sec:\secname:analysis}

We ran \texttt{GRBTransientMetric} on several OpSim v3.3.5 runs with a range of
characteristics:  \opsimdbref{db:baseCadence}, the baseline cadence;
\opsimdbref{db:NEOswithVisitTriplets}, with three visits per WFD field;
\opsimdbref{db:NoVisitPairs}, with no visit pairs; and
\opsimdbref{db:opstwoPS}, a PanSTARRS-like cadence.

\autoref{tab:SummaryGRBs} lists the fraction of on-axis afterglows
detected in at least one, two, and three visits in a single filter.

Because of its wider areal coverage, the PanSTARRS-like cadence of
\opsimdbref{db:opstwoPS} maximizes the fraction of events detected in
one and two epochs.  Not surprisingly, the triplet-visit WFD cadence of
\opsimdbref{db:NEOswithVisitTriplets} maximizes the three-epoch detection
rate.


\begin{table}
  \begin{tabular}{l|p{6cm}|c|c|c|c|p{5cm}}
    FoM & Brief description & {\rotatebox{90}{\opsimdbref{db:baseCadence}}}
	  & {\rotatebox{90}{\opsimdbref{db:NEOswithVisitTriplets}}} &
	  {\rotatebox{90}{\opsimdbref{db:NoVisitPairs}}} &
	  {\rotatebox{90}{\opsimdbref{db:opstwoPS}}} & Notes \\
    \hline
    \thesection-1 & \footnotesize{\texttt{GRBTransientMetric},
    \texttt{nPerFilter}\,$=1$}      & 0.17 & 0.16 & 0.20 & \textbf{0.21} &
    \footnotesize{Fraction of GRB-like transients detected in at least one
    epoch.} \\
    \thesection-2     & \footnotesize{\texttt{GRBTransientMetric},
    \texttt{nPerFilter}\,$=2$}      & 0.12 & 0.10 & 0.09 & \textbf{0.14} &
    \footnotesize{Fraction of GRB-like transients detected in at least two
    epochs in any single filter.} \\
    \thesection-3     & \footnotesize{\texttt{GRBTransientMetric},
    \texttt{nPerFilter}\,$=3$}      & 0.05 & \textbf{0.08} & 0.04 & 0.04 &
    \footnotesize{Fraction of GRB-like transients detected in at least
	    three epochs in any single filter.}
\end{tabular}
\caption{Mean figures-of-merit (FoMs) for on-axis Gamma-Ray Bursts for one,
two, and three detections in a filter.
The best value of each FoM is indicated in bold.
The wider areal coverage of \opsimdbref{db:opstwoPS} improves its detection
rate of GRBs in one and two epochs, while the triplet visits
in \opsimdbref{db:NEOswithVisitTriplets} naturally improve the
three-detection efficiency.
}
\label{tab:SummaryGRBs}
\end{table}

% --------------------------------------------------------------------

\subsection{Discussion}
\label{sec:\secname:discussion}

An LSST cadence purely designed for discovering GRB afterglows would
include three or more visits to each field every night, with the visits
separated by an hour or two. Moreover, it would be conducted in a single
filter in order to best identify the lightcurve shape of off-axis
events.

While the current surveys simulated are far from this ideal
(usually just two closely spaced visits, with subsequent revisits days
later), nonetheless an appreciable number of GRBs are detectable.
\opsimdbref{db:NEOswithVisitTriplets} would detect about 25 events each
year in three epochs, already potentially the largest sample of untriggered
afterglows.

However, some care is required in interpreting these values:
while the GRB afterglow fades rapidly over the first day of the explosion
(Figure \ref{fig:afterglow_lcs}), at later times a 30 minute visit
separation is not enough to reveal significant evolution in the lightcurve.
We intend to enhance our metric to require that detections are counted only
if significant evolution is statistically distinguishable with 1\%
photometry.

In future work we intend to simulate cosmological populations of on- and
off-axis in order to better determine how many events could be discovered
in time to trigger real-time followup.

\begin{figure}[hbt]
\centerline{
\includegraphics[width=0.6\textwidth]{figs/transients/afterglow_cdf.png}
}
\caption{ Cumulative fraction of GRB on-axis afterglows fainter than
magnitude 24.7 at a given time after the burst. We use an $\alpha=1$
decay with no jet breaks and the brightness parameters of
\citet{2011PASP..123.1034J}. }
\label{fig:afterglow_visibility}
\end{figure}

Thanks to LSST's depth, GRBs can be visible for weeks (Figure
\ref{fig:afterglow_visibility}).  Accordingly,
modest enhancements to the intra- and inter-night revisit rate with
single-filter rolling cadences should substantially improve LSST's
discovery and characterization of relativistic explosions.


% ====================================================================

\navigationbar


% --------------------------------------------------------------------
