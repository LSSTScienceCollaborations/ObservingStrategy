% --------------------------------------------------------------------

\chapter[Eruptive and Explosive Transients]{Eruptive and Explosive Transients}
\def\chpname{transients}\label{chp:\chpname}

% \noindent {\it
% Mike Lund, Ashish Mahabal, Stephen Ridgway, Lucianne Walkowicz, Rahul Biswas, Michelle Lochner, Jeonghee Rho, Eric Bellm, Federica B. Bianco...
% }

Chapter editors:
\credit{ebellm},
\credit{fedhere},

Contributing Authors:
\credit{AshishMahabal},
\credit{StephenRidgway},
\credit{ohadshemmer},
\credit{paulaszkody},
\credit{nathansmith},
\credit{tommatheson}

% --------------------------------------------------------------------

\section{Introduction}


Explosive and eruptive transients include a diverse ensemble of
object, diverse both physically and phenomenolocigally. Transients
such as novae, supernovae (SNe), and gamma-ray bursts (GRBs) probe the
final stages of stellar evolution, Tidal Disruption Events (TDEs), GRBs, Cataclysmic Variables (CVs) give us the
opportunity to study compact and binary objects, massive star
eruptions allow us to understand mass loss mechanisms, and chemical
enrichment. Transients impact the evolutionary history of the
Universe. The brightest transients (GRBs, TDEs, SNe) are light beams
that can be seen over cosmic distances, and some transients (Type Ia
SNe in the first place) are cosmological tracers.  Cadence choices
will determine LSST's ability to discover, classify, and characterize
these events. However, different types and time scales of phenomena
will benefit from different sampling strategies---sometimes
significantly different, and at times orthogonal.  Competing
objectives described in this chapter are at the heart of LSST
observing strategy and cadence design.

When evaluating a particular observation or series of observations in
light of how they perform for a specific science case, it may be
helpful to think of metrics as lying along a continuum between
discovery and characterization. Discovery requires a minimum amount of
information to recognize an event or object as a candidate of
interest, which necessarily involves some level of bare-bones
characterization; rich characterization, on the other hand, implies
that an event may not only be recognized as a candidate of interest,
but basic properties of the event or object may be determined from the
LSST observations (including but not limited to the classification of
the event). The level of characterization accessible through LSST data
will of course evolve with time for all the transients that have
characteristic time scales longer than a few days.

Here we foucs on cadences feasible within the main survey.
Many science objectives related to transients may be better, and possibly only,
achieved within the DDF program.
Transient events may benefit from substantial temporal sampling
(matched to the time constant of the event) with color information
(perhaps contemporaneous) to support characterization and
classification, obtained over the limited duration of interest.
Transient events slower than $\sim$ weeks may be adequately sampled by
a uniform LSST cadence.  Faster events may require special scheduling
strategies.  For some event types, LSST can only be expected to
provide a discovery service, and followup will necessarily be
performed elsewhere.  For some events, for example detecting LIGO
counterparts, a serendipitous discovery is extremely unlikely, but
enabling a TOO program would provide the opportunity for LSST to
contribution significantly to this science.

%The interpretation of a given metric along this continuum has
%implications for the subsequent action and analysis required,
%particularly as regards possible follow-up observations with other
%facilities.

We consider a non-exhaustive set of ``astronomical transients'' in the
paragraphs that follow.

\subsection{Targets and Measurements}
\label{sec:\chpname:targets}

%The class of transients includes a heterogeneous assortment of objects
%and phenomena.
The table below is a \emph{non-exhaustive} list of
phenomena to which we are referring as \emph{Eruptive and Explosive
  transients} in this document.

\begin{center}
  \begin{tabular}{| p{2.5cm} | p{6cm} | l | l | l |}
    \hline

    Transient Type & Science drivers & Amplitude & Time Scale & Event Rate\\
\hline

Flare stars & Flare frequency, energy, stellar age & large & min & ?\\

X-ray Novae & Interacting binaries, stellar evolution, SN progenitors, nuclear physics & large & weeks & ?\\

Cataclysmic variables & Interacting binaries, stellar evolution, compact objects & large & min - days & ?\\

LBV variability & Late stages stellar evolution, Mass loss, SN progenitors & large & weeks-years & ?\\

Massive star eruptions & Late stages stellar evolution, Mass loss, SN progenitors & extreme & weeks-years & rare\\

SN & stellar evolution, feedback, chemical enrichment, cosmology & extreme & days - months & very common\\

GRBs & SN connection, stellar evolution, cosmology & extreme & min - days & ?\\

TDEs & Massive BH demographic & large & months & ?\\

LIGO detections & EM characterization & unknown & unknown & ?\\

Unknown & Discovery & unknown & unknown & rare\\

 \hline \end{tabular}
\end{center}

WE SOULD ADD A FIGURE OF THE PHASE SPACE + EVENT RATE


{\bf What else?}


%templates and stacks

%confusion 

%


% --------------------------------------------------------------------


\subsection{Transient time scales}

For very short lived phenomena (stellar flares, CV outbursts, GRBs,
LIGO events) the main function of LSST will be to provide discoveries
and/or simple characterization.  Followup to discovery/identification,
if required, will surely take place elsewhere.

LBV's and massive star eruptions (such as the 
Gap transients??

SN fall in an intermediate time range.  LSST will provide
multiple visits in multiple filters during the typical SN duration
(months).  This sampling may be insufficient for many science
objectives.  However, moderate, and feasible, changes to LSST
observing strategy, may enhance the sampling for part of the sky part
of the time, greatly improving the usefulness of SN observations. 
Metrics that assess the discovery rate of SN are included in the
cosmology chapter. Here we are interested in assessing the ability of
LSST to discriminate SN from other transients, SN subtypes from one
another, and to identify particularly interesting SNe: for example
those that show signature of shock break-out, or companion interaction
in the early light curve, and would be candidates for \emph{flash
  spectroscopy} follow-up (ref here). In addition metrics that
quantify LSST's ability to constraint SN physics in a statistically
large sample of SN are needed (offered hopefully).


**** STILL WANT THIS? CAN WE DO IT?
TDEs, whose fading time-scale is much more gradual
(over weeks to months) than the rise time-scale and we want to asses -
through a metric - how many are detected while still on the
rise. Ref. Science Book: 10.6.1. Also ref. recent papers.

A phenomenon that falls under this document's definition of explosive
transients, but is not discussed here, is AGN. AGN are likely to be
sampled with sufficient resolution by a uniform LSST cadence and they
are discussed in Chapter~\ref{chp:agn}.

In addition we hope that LSST will provide a wealth of serendipitous
discoveries of yet-to-be-observed transients.  An ideal transient
discovery survey would include heavy coverage of all time scales. LSST
will cover longer time periods well, but will have to make some
choices of emphasis in coverage of shorter time-scales. (metric
here??)


In the paragraphs that follow we will use case studies to asses:

\begin{itemize}
\item
  The discerning power of a given LSST observing cadence.
  While early classification of transients is extremely important for
  transient science with LSST, it is also an incerdibly difficult task,
  on which many astronomers and statisticians are working. It is unfeasible,
  and beyond the scope of this work to attempt a naive classification metric.
  However a crucial input to follo-up strategy design is the early
  identification of young, fast evolving transients. We identify a region of
  the phase-space of rising speed and color that is characteristif of a
  variety of transients in their early phases and assess the ability of LSST's
  cadences to place transients within this phase-space. (Eric, Stefano, Iair, Fed...)
  %For this topic we will address the ability to distinguish LBVs, SNe, and TDEs.
\item
  The ability to identify an object of interest for follow-up and
  trigger prompt follow-up observations. For this topic GRBs are used as
  a case study. ***ERIC
\item
  The insight that a cadence gives into a single transient class. The
  case study will be
  **massive star eruptions if Nathan gets it done
  **CVs if Laura gets it done
\item 
  The statistical constraints to a transient class that can be obtained
  over the course of the LSST survey, from the LSST survey data alone
  (assuming a successful classification). SN Ia early interaction
  signatures or IIb shock break-out (Fed).
\item
  The ability of LSST, given a cadence, to provide insightful coverage
  for long duration transients. This metric favors cadences with small
  seasonal gaps, which allow us to gather complete color evolution of
  slow evolving transients, such as LBV and TDEs. (Ryan)

\end{itemize}

In this chapter we focus on LSST's potential to advance the science of
eruptive and explosive transients; the use of SNe for cosmology is
discussed in \autoref{chp:cosmo}.


% --------------------------------------------------------------------

\subsection{Metrics}
\label{sec:\chpname:metrics}

\begin{center}
\begin{tabular}{| p{5cm} |p{10cm} |}
\hline Metric & Description\\ \hline SNe & Number of events adequately
sampled\\ Serendipity & Histogram of median visit series length vs
maximum visit spacing within the series\\ \hline \end{tabular}
 \end{center}


%%WHO WROTE THIS? I AM NOT SURE WHAT THIS METRIC DESCRIPTION MEANS
The metrics for SNe will be highly specialized and based on the best
available understanding of SN light curve analysis and the expected
event population.

%%WHAT DOES THE PARAGRAPH BELOW MEAN?
The suggested metric set for serendipity is based on the simple-minded
idea that a novel transient will be characterized by a band-limited,
finite waveform, and that a useful observation series will consist of
a series of samples extending over the duration of the event, with at
least critical sampling of the fastest variations.  Since for some
event durations the number of useful time series will be small, it may
be useful to look not at the median length, but the median length of a
subset size preselected as possibly useful (e.g. the$10^3$ longest
series).

\citet{2015arXiv150803175L} discuss
three metrics that have been incorporated into the MAF. Two of these
metrics deal explicitly with time variable behavior: a) observational
triplets, and b) detection of periodic variability. While in this
chapter w do not deal with periodic phenomena, the first of these
metrics is useful to generically assess the performance of LSST for
transients of all types. More specialized transients metrics can be
obtained by simply using a more realistic light curve shape. A metric
that uses input ASCII files is provided, to enable the assessment of
discovery and characterization of an arbitrarily shaped transient.

\subsubsection{Observational triplets (TripletMetric)}

This metric provides a means of evaluating whether a transient event
on some time scale of interest has been detected, by testing for a
sequence of three observations. The object must be detected in
quiescence, followed by two subsequent detections above some
threshold; this sequence of observations allows the magnitude of the
change to be measured, as well as its time scale.

This metric may be used for a variety of astrophysical phenomena, in
particular transient events on variable objects (e.g. novae, stellar
flares), in that it is general with respect to the amplitude of the
brightness variation as well as the time scale of said change. The
requirement of a detection prior to outburst does constrain it to
objects that have already been detected in quiescence (in other words,
not necessarily ``true'' transients), although there may be some cases
where this is not the case (e.g. a supernova occurring on a previously
detected galaxy). In practice, the time lapse between the first and
second and second and third observations must be comparable (between
10$^2$ and 10$^5$ seconds) for discovery. This metric may be
calculated for a given OpSim run and then further reduced to a
histogram in logarithmic time bins; the minimum number of bins to
construct an interesting sample of objects is source-dependent.

\subsubsection{Transient Metric (transientAsciiMetric)}

Calculate what fraction of transients would be detected using an ASCII
input file for the light curve.

\subsection{Proposed Metrics}

The following is a raw list of metric ideas; these need specificity
and further description.

The triplet metric may also be altered to include filter constraints,
such that the triplets are drawn from a single filter or subset of
filters.

Color evolution constraint: triplets of observations in a specific
color (really requirement of two triplets in multiple filters)

  2D Histogram of delta t?s between observations constituting a triplet

Histogram of median visit series length vs maximum visit spacing
within the series

Number of events adequately sampled

% --------------------------------------------------------------------

\subsection{OpSim Analysis}
\label{sec:\chpname:analysis}

Analysis shows that current simulations provide  poor coverage in any one filter for transient events longer than a deep drilling session ($\sim$30 minutes) and shorter than $\sim$ weeks.

Simulated performance for SN observations must be analyzed for both
main survey and mini-survey (deep drilling) productivity.  It is
considered that current simulated schedules give inadequate
performance for SN science.



% --------------------------------------------------------------------

\subsection{Discussion}
\label{sec:\chpname:discussion}

Community studies are providing improving SNe metrics, and continuing
communication between the SN and LSST communities is essential to
tuning the observing strategy to deliver the SN time series that are
needed and possible.

Improving LSST science return for SNe will also improve sampling of
all transients with similar or somewhat shorter characteristic times.
Non-uniform survey strategies (rolling cadence) can significantly
improve the LSST performance for faster transients.  Interpretation of
multiple filters for novel events may be powerful, or problematic,
since color may be uncertain.

Some insight into fast transients may be available from image pairs or
triples (as opposed to more complete series).  These include the pair
of images in a visit - which could be useful in studying the rise time
of an extremely fast event.  This includes the characteristic grouping
of visits (typically 0.5 to 1.0 hour separation) planned for purposes
of identifying asteroids.  It also includes fortuitous multiple
sampling due to field overlap, providing additional sampling, which
may be random or systematic, depending on the scheduling, on a time
scale of minutes to hours.  The sampling benefits of this fortuitous
overlap have not yet been investigated.


\navigationbar

% --------------------------------------------------------------------

% ====================================================================
%+
% SECTION:
%    cv.tex
%
% CHAPTER:
%    transients.tex
%
% ELEVATOR PITCH:
%    Explain in a few sentences what the relevant discovery or
%    measurement is going to be discussed, and what will be important
%    about it. This is for the browsing reader to get a quick feel
%    for what this section is about.
%
% AUTHORS:
%    Federica Bianco (@fedhere)
%
% ====================================================================

% \section{Cataclismic Variables}
\subsection{Cataclismic Variables}
\def\secname{\chpname:CVtransients}\label{sec:\secname}

\credit{paulaszkody},
\credit{fedhere}

Cataclysmic Variables (CVs) encompass a broad group of objects
including novae, dwarf novae, novalikes, and AM CVn systems, all with different
amplitudes and rate of variability. The one thing they all have in
common is active mass transfer from a late type companion to a
white dwarf. These create variability on a wide range of timescales:

\begin{itemize}
	\item \textit{minutes} flickering in
dwarf novae and novalikes, pulsations in accreting white dwarfs in
the instability strip, orbital periods of AM CVn systems
\item \textit{hours} orbital periods of novae, dwarf novae and novalikes
\item \textit{days} normal outburst lengths of dwarf novae
\item \textit{weeks} outburst length of superoutbursts in short orbital period
dwarf novae, outburst recurrence time of normal outbursts in short
orbital period dwarf novae
\item \textit{months} outburst recurrence time of
longer period dwarf novae, various state changes in novalikes, declines
in novae
\item \textit{years} for the outburst recurrence timescales of the
shortest period dwarf novae and the recurrence times in recurrent novae
\end{itemize}
The
amplitudes range from tenths of mags for flickering and pulsations to 4 mags
for normal dwarf novae and changes in novalike states up to 9-15 mags for the
largest amplitude dwarf novae and classical novae.

These large differences make correct classification with LSST difficult
but necessary in order to reach goals of assessing the correct number
of types of objects for population studies of the end points of
binary evolution. Multiple filters (especially the blue $u$ and $g$)
along with amplitude and recurrence of variation provide the best
discrimination, as all CVs are bluer during outburst and high states of
accretion. Long term, evenly sampled observations can provide indications
of the low amplitude random variability and catch some of the more frequent
outbursts, but higher sampling is needed to determine whether an object
has a normal or superoutburst, to catch a rise to outburst or to a
different accretion state or to follow a nova. Novae typically
have rise times of a few days, while the decline time and shape provide
information as to the mass, distance and composition. The time to decline
by 2-3 magnitudes is correlated with composition,
%
% FED: what is a range of time scales for this decline? days? months?
%
WD mass and location in
the galaxy, thus enabling a study of Galactic chemical evolution.  As with SN,
the diagnostic power for all these systems rests on color and sampling.

Metrics to be developed would assess the abilities of  a given observing
strategy to distinguish between new novae and dwarf novae outbursts and
identify high and low states.  This discriminiation is provided by
measurement of the shapes and recurrence times of large variations as well
as blue colors to distinguish low amplitude variability that would indicate
new pulsators or novalikes. Population studies rely on the numbers of long
orbital period (low amplitude, wide outbursts) vs. short orbital period
(patterns of short outbursts followed by larger, longer superoutbursts)
dwarf novae at different places in the galaxy, as well as the numbers of
recurrent (1-10 yrs) vs. normal novae (10,000 yrs, about 35/galaxy/yr).
Objects particulary worthy of later followup are containing highly magnetic
white dwarfs. These objects can be identified in a large sample when the
magnitude for a majority of the years is a faint (low) state and a small
percentage of time is a bright (high) state, combined with a red color (due
to cyclotron emission from the magnetic accretion column).

% ====================================================================
%
\subsection{Conclusions}
%
% Here we answer the ten questions posed in
% \autoref{sec:intro:evaluation:caseConclusions}:
%
 \begin{description}

 \item[Q1:] {\it Does the science case place any constraints on the
 tradeoff between the sky coverage and coadded depth? For example, should
 the sky coverage be maximized (to $\sim$30,000 deg$^2$, as e.g., in
 Pan-STARRS) or the number of detected galaxies (the current baseline
 of 18,000 deg$^2$)?}

 \item[A1:] Extending the sky area is not a priority for this science.

 \item[Q2:] {\it Does the science case place any constraints on the
 tradeoff between uniformity of sampling and frequency of  sampling? For
 example, a rolling cadence can provide enhanced sample rates over a part
 of the survey or the entire survey for a designated time at the cost of
 reduced sample rate the rest of the time (while maintaining the nominal
 total visit counts).}

 \item[A2:] Intervals of higher cadence are extremely valuable and a rolling cadence is a
satisfactory approach, subject to cadence details.

 \item[Q3:] {\it Does the science case place any constraints on the
 tradeoff between the single-visit depth and the number of visits
 (especially in the $u$-band where longer exposures would minimize the
 impact of the readout noise)?}

 \item[A3:] Increasing the number of $u$-band visits will improve the characterization of CV phenomena.

 \item[Q4:] {\it Does the science case place any constraints on the
 Galactic plane coverage (spatial coverage, temporal sampling, visits per
 band)?}

 \item[A4:] Most CVs will be detected in the galactic plane, and a long, rich series of visits is needed.

 \item[Q5:] {\it Does the science case place any constraints on the
 fraction of observing time allocated to each band?}

 \item[A5:] $u$-band is diagnostic, especially $u$-$g$.

 \item[Q6:] {\it Does the science case place any constraints on the
 cadence for deep drilling fields?}

 \item[A6:] For deep drilling in the galactic plane or for local group galaxies, the best cadences would obtain several epochs  per night in
 each filter, rather than concentrating all acquisition with a filter in a single rapid burst.

 \item[Q7:] {\it Assuming two visits per night, would the science case
 benefit if they are obtained in the same band or not?}

 \item[A7:] Same filter and different filter each offer valuable information, and a mix of these two options would be preferred
pending test of both cadences.

 \item[Q8:] {\it Will the case science benefit from a special cadence
 prescription during commissioning or early in the survey, such as:
 acquiring a full 10-year count of visits for a small area (either in all
 the bands or in a  selected set); a greatly enhanced cadence for a small
 area?}

 \item[A8:] It would be very helpful to CV studies - and many other areas of transient science - to understand variability across all timescales.
 Especially valuable would be a cadence that would cover one (cluster, rich star field) with all the timescales that will not be
strongly represented  in the main survey, starting at 15 seconds.

 \item[Q9:] {\it Does the science case place any constraints on the
 sampling of observing conditions (e.g., seeing, dark sky, airmass),
 possibly as a function of band, etc.?}

 \item[A9:] No.

 \item[Q10:] {\it Does the case have science drivers that would require
 real-time exposure time optimization to obtain nearly constant
 single-visit limiting depth?}

 \item[A10:] No.

 \end{description}

 \navigationbar


% --------------------------------------------------------------------

% ====================================================================
%+
% SECTION:
%    sn.tex
%
% CHAPTER:
%    transients.tex
%
% ELEVATOR PITCH:
%
%-
% ====================================================================

\section{Supernovae as Transients}
\def\secname{\chpname:SNtransients}\label{sec:\secname}

\credit{fedhere}

Supernovae (SNe) represent the final dramatic stages of the life of many
stars. The term SN we covers a diverse set of phenomena: explosion of
low mass stars in binary systems, thermonuclear SN or SN Ia (also
discussed in \autoref{sec:supernovae}), and explosion of high mass stars,
Core collapse (CC) SNe, and even terminal explosions of more exotic
systems, yet to be understood, like Super Luminous SNe
(SLSNe). Phenomenologically, the observables of the explosion are
also diverse. The transient duration ranges between weeks,
months, even years. The electromagnetic energy radiated ranges between
$\sim0.1$ (faintest CC SNe), to $\sim1$ (SN Ia) and $\sim100$ (SLSNe)
$\times 10^{49}$ erg, corresponding to absolute magnitudes at peak
ranging between $\sim-19$ and $\sim-22$.

LSST's contribution to SNe studies can be substantial. Synoptic
surveys such as SDSS, SNLS, PTF, PanSTARRS have revolutionized our
understanding of SN time and again, exposing their diversity,
and revealing different progenitor channels. LSST's first crucial
input will be discovery: the normal type Ia SN rate out to redshift
$z=1$ is estimated to be $\sim200 ~(\mathrm{sq. deg.})^{-1}$ per
year\footnote{\url{http://www.lsst.org/sites/default/files/docs/Wood-Vasey_086.11.pdf}},
and SN~Ia represent only about 1/4 of all SN
events~\citep{Li11b}: tens of millions of stars will explode
within the LSST footprint every year. The main factors affecting 
LSST SN science concern:
\begin{enumerate}
\item
LSST's SN discovery power,
\item
LSST's discrimination power,
\item
the quality of the statistical sample over time.
\end{enumerate}
Items 1 and 2 are \emph{time sensitive}, while the latter
is not, although it is interesting to understand the pace at which a
science question can be advanced in the lifetime of LSST.

{\bf \emph{Discovery:}} the SN~Ia discovery is rate is a standard LSST time-domain metric: a
fraction of $\sim40\%$ SN~Ia $z\lesssim0.5$ are expected to be discovered pre-peak
luminosity within the standard LSST survey
(e.g. \opsimdbref{db:baseCadence}~Figure~\ref{fig:enigmaEarlySNe}). The
topic of SNe discovery is discussed in further detail in
\ref{sec:supernovae}.

The next step is then {\bf\emph{discrimination}}, and the question we need
to answer, for SNe as well as for most other transients, is: will LSST
photometry allow to distinguish SN from other transients, and to
distinguish the different types of SN? And further: will this be
achievable in time to appropriately direct follow-up efforts? This is
particularly difficult considering that photometric classification
schemes have only achieved modest performances in distinguishing, for
example, SN~Ic from SN~Ia. As mentioned earlier in this chapter, the issue
of prompt classification is at the heart of the success of LSST
transient science, but it is too complex to tackle in this chapter,
while LSST's ability to assess whether a new transient is young is discussed
in \autoref{sec:\chpname:transientsAge}.

When a large statistical sample of SNe is generated, LSST's photometry
may allow setting constraints on the diversity of the sample, even as a
standalone survey, without the aid of follow-up efforts.  {\bf Thus
  LSST \emph{alone} can shed light on the diversity within the
  population of SN}, which in turn may constrain the genesis of the
explosion.\footnote{Reliable typing of a SN and redshift determination
  would still require auxiliary data.} For SN~Ia, where the exploding
star is a carbon-oxygen White Dwarf (WD), outstanding questions
that can be answered by an LSST photometric sample include, for
example, what is the percentage of SN Ia that arise from a
\emph{double-degenerate} (DD) progenitor system -- a carbon-oxygen WD-WD binary
--, from a \emph {single-degenerate} (SD) system -- a WD-Main Sequence
 or WD-Red Giant (RG) binary--, or a \emph{merger} -- a WD-WD
 binary with a He and a carbon-oxygen WD.
 Answering this question would reduce the
scatter in the Hubble diagram if SNe from different progenitors are
shown to require different standardization~\citep{Scolnic2014}. On the
CC~SN side: the diversity of SN sub-classes, and the relationship
between them (is there a phenomenological continuum or are they actually
distinct classes, e.g. between IIp and IIL, or Ib and IIb?) is yet to
be understood. Exceptionally well-studied objects may answer these
questions: individual SN Ia with tight constraints on the progenitor
system show, for example, that both single and double degenerate
progenitors exist (e.g. SN 2011fe, \citealt{Li11}, ~\citealt{Olling15}
and PTF 11kx, \citealt{Dilday12}). However, a statistical sample is
needed to set constraints on populations~\citep{Hayden2010, Bianco11}.

Thus the technical question to be answered is: how much detail can be
sacrificed in favor of sample size without compromising diagnostic
power? And the diagnostic power relies on color and sampling: thus
what is the trade-off between cadence in the same filter, and
observations in different filters. Specifically, transients can be
distinguished early from two photometric characteristics: rise time
and color. There is a tension between these observables, as discussed
in Section~\ref{sec:\chpname:transientsAge}. Obtaining colors relies
of course on obtaining photometry in different bands as close as
possible to \emph{simultaneously}.  However, assessing the rise slope
is best done with a single filter, so prompt characterization also
needs multiple epochs within a night, although separated by at least a
few hours, in the same filter, as observing with different
filters it is impossible (or very hard) to separate shape from
color. Colors are an important diagnostic for the
statistical sample: as long as the epoch of peak is reliably assessed
coadded light curves can be studied, which is the goal of the analysis
that follows.

\subsection{Distinguishing progenitor scenarios}

In this chapter we envision and design a SN related metric that works
on a large sample (months-to-years of LSST data) and assesses the
ability to characterize the contribution of SNe with specific features
to the global population: as a test case we will use the presence of
an early blue excess for SN type Ia, signature of interaction with a
companion, and thus of a SD progenitor. Equivalently, the presence of
an early blue excess in CC~SNe could be assessed, the signature of
shock breakout which directly measures the radius of the progenitor
star. We perform the simulation on SN~Ia since statistical studies of
samples that set constraints on progenitor fractions (fraction of DD
vs SD progenitors) exist and can be used as a
benchmark~\citep{Hayden2010, Bianco11}.  What we evaluate as a
\emph{figure of merit} (FoM) for this science deviates from the
guidelines for figures of merit, since LSST will surely be able to
answer this question \emph{at some point} and we measure \emph{how
  fast} LSST can answer this question. The FoM is time
within the survey required to achieve a sufficiently large sample of
SNe to enable us to distinguish
populations with different contribution from DD and SD progenitors.
We rely on simulations of the observables of the population for
different sample sizes, and on the \texttt{transientAsciiMetric} to
determine the detectability of interacting vs non-interacting
SNe. We are developing a metric (\texttt{colorGapMetric}) to assess
the gap between of detections in 2 filters. In the meantime, we rely on
the estimated of the gap between observations in a single filter, and
in any filters (see~\autoref{sec:\chpname:analysis}).

We simulate interacting SNe from the Nugent templates \citep{Nugent02}
injecting the angle-dependent effects of interaction with a companion
as simulated by \citep{Kasen10}, for a $2~M_\odot$ and a $6~M_\odot$
MS companion stars, and a $1~M_\odot$ RG companion, following the
procedure designed in ~\citep{Bianco11}. We create synthetic progenitor
populations with a fraction of single degenerate progenitor systems
$0.05 \leq f_\mathrm{SD} \leq 0.6 $ in 0.05 intervals, and random lines of
sight with respect to the binary's geometry. One such lightcurve, with
maximal interaction effects, is shown in \autoref{fig:kasenlc}, also
indicating how it may be observed by LSST. For each population, we
simulate the observation of colors by selecting random epochs with a
granularity of 1 day within the first 10 days after explosion, and
subtracting the magnitude in different filters at the same epoch
$\pm~1$~day for each SN, and we include the effects of observational
noise by generating datapoints from a draw within a Gaussian
distribution centered at the color measured in the previous step and
with standard deviation $\sigma_\mathrm{pop} = 0.1$, 0.3, and 0.5.
The SNR requirement is
translated into a requirement on each
observation of $\mathrm{SNR} >
\frac{1.0}{\sqrt{2.0}~\sigma_\mathrm{pop}}$.
We generate populations of $N_\mathrm{pop}=100$,~1000,~10000 $z~=~0.5$ SNe,
observed in $g'-r'$, as a representative case. Because the effect is
heavily chromatic, and it dissipates becoming essentially negligible
by $r$ band, $u'-i'$ gives the most leverage. However $g'$ and $r'$
are the best observed LSST bands in most cadences. An extension of
this work should then consider $g'-r'$, $u'-r'$, $g'-i'$, and $u'-i'$.

\begin{figure}[hbt]
\centerline{
\includegraphics[width=0.6\textwidth]{figs/transients/LSST_Kasen_lcv0.pdf}
}
\caption{ A normal SN Ia lightcure at z=0.5 showing interaction with a
  RG companion as seen from the most favorable viewing angle: the
  effect of interaction as simulated by \citet{Kasen10} is added on
  top of a lightcurve simulated from the \citealt{Nugent02}
  templates. The data points represent one possible set of LSST
  observations of this transient, obtained by running the
  \texttt{transientAsciiMetric}.  This particular event is detected in
  $g'$, $r'$, and $i'$ within the first 10 days.}
\label{fig:kasenlc}
\end{figure}

We perform Kolmogorov-Smirnoff ($KS$) an Anderson-Darling ($AD$) tests
to evaluate our diagnostic power as a function of sample quality,
$SNR$, and sample size, $N_\mathrm{pop}$.  In
\autoref{tab:SNprogenitors} we report the ability to distinguish a
population with a $f_\mathrm{SD} > x$ from $f_\mathrm{SD}=0.05$; \emph{the
number reported is the SN~Ia fraction from SD progenitors that can be
distinguished at a $p\mathrm{-value}~\leq ~0.05$}.

\begin{table}
\begin{center}
  %\begin{tabular}{ c | c| c| c | c | c| c| c | }
  \begin{tabular}{ c | c| c| c |  }
$g-r$&\bf{$N_\mathrm{pop}$=100}&\bf{$N_\mathrm{pop}$=1,000}&\bf{$N_\mathrm{pop}$=10,000}\\%& $g-i$&\bf{$N$=100}&\bf{$N$=1,000}&\bf{$N$=10,000} \\
  \hline
  {\bf $\mathrm{SNR}~\geq~1.4$}&  -  & 0.2 & 0.1 \\%& &  -  & -  &  0.2 \\
  {\bf $\mathrm{SNR}~\geq~2.0$}&  - & 0.2 & 0.1 \\%& &  -  & 0.2 & 0.1 \\
  {\bf $\mathrm{SNR}~\geq~7.0$}& 0.2 & 0.1 & 0.1 \\%& & 0.4 & 0.1 & 0.1 \\
%&&&&&&&\\ $u-i$&\bf{$N$=100}&\bf{$N$=1,000}&\bf{$N$=10,000} & $u-r$&\bf{$N$=100}&\bf{$N$=1,000}&\bf{$N$=10,000}\\
% \bf{$\sigma_\mathrm{pop} = 2.0$}&  -  & -  &  0.2 & &  -  & -  &  0.2 \\
% \bf{$\sigma_\mathrm{pop} = 1.0$}&  -  & 0.2 & 0.1 & &  -  & 0.2 & 0.1 \\
% \bf{$\sigma_\mathrm{pop} = 0.5$}& 0.4 & 0.1 & 0.1 & & 0.4 & 0.1 & 0.1 \\

 \hline
  \end{tabular}
  \caption{Minimum fraction of single-degenerate (SD) SN~Ia in a sample of size $N_\mathrm{pop}$ of $z~=~0.5$ SNe that can be distinguished from a population with a fraction of 0.95 double-degenerate (DD) and 0.05 SD SNe~Ia, for a given quality cut on each observed datapoint ($\sigma_\mathrm{pop}$).}
\label{tab:SNprogenitors}
\end{center}
\end{table}

Now we can evaluate how long it will take for a given LSST
cadence to obtain a sufficient number of observations in the 2 desired
bands, separated by less than 1 day, that pass the SNR requirements.
This should be done in a full Monte Carlo simulation, injecting
light curves with the proper light curve shape at the proper rate.  Note
that, because the early light curves of interacting SD SN~Ia are
brighter, they should be more easily detected. However, at this stage we
can take some shortcuts. \emph{First shortcut}: we evaluate the
relative observability of SNe with excess, and SNe without excess at
$z~=~0.5$ and adjust the number of detections according to
the injected ratio.  The relative detectability can be assessed with
the \texttt{transientAsciiMetric}, which allows us to see how OpSims
recovers observations of transients with realistic shapes. We conclude
that for RG-WD progenitors the detectability is enhanced by $\sim50\%$ in
 $g'$ compared to SD progenitors, and slightly less in $r'$. 
Then we extract from the \texttt{transientAsciiMetric}, the number of
\emph{color observations}, i.e. observations in 2 bands within 1 day
of each other, each fulfilling our SNR requirement for the color for
3-, 6-, and 12 months of survey in year 1. 

With the goal of distinguishing a SD contribution of 10\% to the SN Ia
population from a 5\% contribution to a three-sigma level ($p$-value
$<0.05$) we need more than 1000 detections within 1 day in 2 filters
at a $\mathrm{SNR}~\geq~7$: \autoref{tab:SNprogenitors}. But the pairs
of observations we recovered in the previous steps are within the
first 10 days but with any gap in time. \emph{Second shortcut}: To
include the constraint that the detections should be within 24 hours
we use to the \texttt{InterNightGapsMetric}, which is plotted in
~\autoref{fig:enigmaGapAll}.  For the \opsimdbref{db:baseCadence} we
estimate $~\sim10\%$ of the observation are revisited within a
night. With the assumption that this is likely to happen in two
different filters, which is \emph{non-conservative}, but neglecting
intra-night observations that may happen in the two different filters,
which is a \emph{conservative} assumption our numbers drop by a factor
10. The lightcurves are injected with an event rate designed to be
consistent with the discovery rate measured in \ref{sec:supernovae}.

With all these assumptions standing, we find that that only 3 months
of survey are sufficient to provide a sufficiently large and
sufficiently high SNR sample for our purpose, and improve on the
findings on this topic that were achieved with
SDSS~II~\citep{Hayden2010}, and 3 years of SNLS data~\citep{Bianco11}
with \opsimdbref{db:baseCadence} or the
\opsimdbref{db:NEOswithVisitTriplets}. The
\opsimdbref{db:NEOswithVisitTriplets} requires three visits, thus
increasing the timeline for inter-night observations. Although it does not
require the observations to be in any specific filters, and with the
addition of the third visit within the same night, it increases the
typical intra-night gap, it outperforms \opsimdbref{db:baseCadence} slightly.
It is possible that a detailed investigation
of the true \emph{inter-night gap between different filters}, or the
addition of a requirement in the cadence that one of the night filters
be different than the others (possibly requiring an increased gap
between two of the three images to minimize filter changes) would
provide valuable data for this kind of studies even faster.

\emph{This exercise demonstrates the power of LSST in collecting large high SNR samples of transients, but we must remind the reader that these conclusions, and generally large sample analysis, rely on having properly identified both the transient class (normal SN~Ia) and the date of maximum! This, once more, highlights the importance of prompt identification and classification: for SN~Ia this likely will limit this work to objects that could be identified spectroscopically, enhancing the importance of follow-up.}


\autoref{fig:sndetect} shows the detection rate for SN~Ia at $z=0.5$
in absence of shock interaction as a function of SNR (obtained by summing in quadrature the errors on $g'$
and $r'$) for 3, 6 months, and a year of
\opsimdbref{db:baseCadence} and \opsimdbref{db:NEOswithVisitTriplets}.
% (ideally I will plot it for the other survey as well tonight)



\begin{table}
  \begin{tabular}{l|p{8cm}|c|c|p{3cm}}
    FoM & Brief description & {\rotatebox{90}{\opsimdbref{db:baseCadence}}}
	  & {\rotatebox{90}{\opsimdbref{db:NEOswithVisitTriplets}}} & Notes \\
    \hline
    \thesection-1 & \footnotesize{\texttt{SNIaprojenitorMetric},
    \nolinebreak{\texttt{1,000 detections}}}      & 3 & $<3$ &
    \footnotesize{Time in month to collect 1,000 relevant observations to distinguish a 5\% from a 10\% SD contribution} \\
    \thesection-2     & \footnotesize{\texttt{SNIaprojenitorMetric},
    \texttt{10,000 detections}}      & $>12$ & $>12$ &
    \footnotesize{Time in month to collect 10,000 relevant observations in months  to distinguish a 5\% from a 10\% SD contribution.}\\
\end{tabular}
\caption{FoMs for statistical SN Ia progenitor studies to assess the contribution of SD progenitors to the SN Ia population.
}
\label{tab:SummarySNprojs}
\end{table}

\begin{figure}[hbt]
  \centerline{
    \includegraphics[width=0.6\textwidth]{figs/transients/LSST_Iadetected_wcolor.png}
  }
  \caption{
    Normal SN Ia light cure at z=0.5 detected by the \opsimdbref{db:baseCadence} (solid lines) and \opsimdbref{db:NEOswithVisitTriplets} cadence (dashed line) in 3 months, 6 months, and 1 year, that provide color information useful to constrain the progenitor distribution. Line are third-degree polynomial fits.}
  \label{fig:sndetect}
\end{figure}

% ====================================================================
%
 \subsection{Conclusions}

 Here we answer the ten questions posed in
 \autoref{sec:intro:evaluation:caseConclusions}:

 \begin{description}

 \item[Q1:] {\it Does the science case place any constraints on the
 tradeoff between the sky coverage and coadded depth? For example, should
 the sky coverage be maximized (to $\sim$30,000 deg$^2$, as e.g., in
 Pan-STARRS) or the number of detected galaxies (the current baseline but
 with 18,000 deg$^2$)?}

 \item[A1:] Yes, although this question can be answered with a full simulation that includes a coordinate dependent event rate while the current simulation assumes uniform probability of a transient in the field-of-view and cannot evaluate this trade-off

 \item[Q2:] {\it Does the science case place any constraints on the
 tradeoff between uniformity of sampling and frequency of sampling? For
 example, a rolling cadence can provide enhanced sample rates over a part
 of the survey or the entire survey for a designated time at the cost of
 reduced sample rate the rest of the time (while maintaining the nominal
 total visit counts).}

 \item[A2:] Yes: this science case is sensitive to both. A more sophisticated simulation which also measures the ability to correctly identify the epoch of maximum would be more powerful to answer the question.
   

 \item[Q3:] {\it Does the science case place any constraints on the
 tradeoff between the single-visit depth and the number of visits
 (especially in the $u$-band where longer exposures would minimize the
 impact of the readout noise)?}

 \item[A3:] Yes, because the diagnostic power depends on both the SNR of each observation and the gap between observations. 

 \item[Q4:] {\it Does the science case place any constraints on the
 Galactic plane coverage (spatial coverage, temporal sampling, visits per
 band)?}

 \item[A4:] No.

 \item[Q5:] {\it Does the science case place any constraints on the
 fraction of observing time allocated to each band?}

 \item[A5:] Yes, since it relies on obtaining observations in at least 2 filters.

 \item[Q6:] {\it Does the science case place any constraints on the
 cadence for deep drilling fields?}

 \item[A6:] Yes, although the results have not been analyzed separately for WFD and DD fields.

 \item[Q7:] {\it Assuming two visits per night, would the science case
 benefit if they are obtained in the same band or not?}

 \item[A7:] No. Although  we would benefit greatly from 2 visits in the same filters, and one visit in a different filter to constrain simultaneously shape and color."

 \item[Q8:] {\it Will the case science benefit from a special cadence
 prescription during commissioning or early in the survey, such as:
 acquiring a full 10-year count of visits for a small area (either in all
 the bands or in a  selected set); a greatly enhanced cadence for a small
 area?}

 \item[A8:] A full event rate needs to be inculded in the simulation to answer this question.

 \item[Q9:] {\it Does the science case place any constraints on the
 sampling of observing conditions (e.g., seeing, dark sky, airmass),
 possibly as a function of band, etc.?}

 \item[A9:] Yes since detection depends on SNR.

 \item[Q10:] {\it Does the case have science drivers that would require
 real-time exposure time optimization to obtain nearly constant
 single-visit limiting depth?}

 \item[A10:] No.

 \end{description}


% --------------------------------------------------------------------

% ====================================================================
%+
% SECTION:
%    grb.tex
%
% CHAPTER:
%    transients.tex
%
% ELEVATOR PITCH:
%-
% ====================================================================

\section{Gamma-Ray Burst Afterglows}
\def\secname{grbs}\label{sec:\secname}

\credit{ebellm}

Gamma-ray bursts (GRBs) are relativistic explosions typically classified
by the temporal duration of their initial gamma-ray emission: Long GRBs,
that mark the endpoint of the lives of some massive stars, and short
GRBs, believed to originate from the merger of binary neutron stars. GRB
emission is known to be beamed: the initial prompt gamma-ray emission is
seen only for observers looking at the jet axis. The longer-wavelength
X-ray, optical, and radio afterglow may be seen both by on- and off-axis
observers.  The latter case is known as an orphan afterglow, due to the
absence of gamma-ray emission. On- and off-axis afterglows are predicted
to have different temporal signatures in the optical: On-axis events
decay as a power-law until a jet break, while off-axis events should be
fainter and show an initial rise (Figure \ref{fig:afterglow_lcs}).
Despite systematic searches, no convincing orphan afterglow candidates
have yet been discovered, limiting our knowledge of the beaming fraction
of GRBs and hence their true rates. Well-sampled orphan afterglow
lightcurves would also permit study of the GRB jet structure.

\begin{figure}[hbt]
\centerline{
\includegraphics[width=0.6\textwidth]{figs/transients/predicted_afterglow_lcs_mag.pdf}
}
\caption{ Predicted light curves of GRB afterglows by off-axis angle
with respect to the jet axis $\theta_{\rm obs}$ \citep[Figure
8.8,][]{2009arXiv0912.0201L}. The forward shock model is derived from
\citet{2002ApJ...576..120T} and assumes a jet half opening angle
$\theta_j = 4^\circ$, the isotropic equivalent energy of $E_{\rm iso} =
5\times10^{53} \rm erg$, ambient medium density $n = 1$ g cm$^{-3}$, and
the slope of the electron energy distribution $\rm p = 2.1$. The
apparent AB $r$-band magnitudes assume a source redshift $z = 1$. }
\label{fig:afterglow_lcs}
\end{figure}

Because of their rarity, in all but one case \citep{2015ApJ...803L..24C}
to date GRBs have been discovered using their prompt emission by hard
X-ray or gamma-ray all-sky monitors. This selection imposes biases on
the population of relativistic explosions we observe. Baryon-loading in
the GRB jet---a ``dirty fireball'' \citep{2003ApJ...591.1097R}---can
lead to on-axis events without gamma-ray emission.  Only one plausible
candidate has been identified to date \citep{2013ApJ...769..130C}.
Discovery of new dirty fireballs---if distinguished from off-axis
events--would clarify the rates of these events and enhance our
understanding of the diversity of stellar death.

LSST is the survey most capable of resolving these decades-old
questions.  Due to its large aperture and etendue, LSST can detect
faint, fast-fading, and rare cosmological events, potentially enabling
population studies of the high-redshift universe.
\citet{2015A&A...578A..71G} estimated LSST could detect 50 orphan
afterglows each year, more than any other planned survey.

%deep survey helps due to time dilation

%beaming fraction and true rates; jet structure; dirty fireballs?
%GRB-SN connection; probe high-z star formation?

%other fast transients: Fast transients and SN shock breakout?  flash spectroscopy

The challenge of detecting and recognizing GRB afterglows in the LSST data in
real time makes this science case a useful proxy for other fast transient
science cases that benefit from $N > 2$ visits per night.  In particular, this
includes discovering supernovae soon after explosion for flash spectroscopy or
shock breakout searches.

% need appropriate cadences to support value of realtime alert stream

% --------------------------------------------------------------------

\subsection{Target measurements and discoveries}
\label{sec:\secname:targets}

GRB afterglow discovery is among the science cases that places the
greatest stress on the LSST cadence.  Because afterglows fade
rapidly---dropping several magnitudes in the first few hours---high
cadence observations are required to detect the fast fading. If an
afterglow candidate can be recognized in real time, it will be possible
to trigger TOO spectroscopy (to measure a redshift and confirm the event
is cosmological), X-ray and radio observations (to detect a high-energy
counterpart and the presence of a jet), and additional photometry (to
characterize the lightcurve evolution).  If there is no source at the
location of the transient in the coadded reference image, two
consecutive observations in the same filter separated by an hour or two
are the minimum required to potentially trigger followup of a
fast-fading event. However, a third observation within a night or
two---ideally in the same filter---would improve the purity of the
sample and reduce the reliance on triggered followup. Observations in
other bands at high cadence are less useful because they require
assumptions about the event's SED and its evolution to determine if a
source is truly fading.

Distinguishing orphan afterglows from on-axis events (whether conventional
GRBs or dirty fireballs) will also require more than two detections.
Orphan events may prove harder to recognize in real time, because they are
intrinsically fainter than on-axis events and show an initial rise rather
than a rapid decay (Figure \ref{fig:afterglow_lcs}).  Additionally, because
of relativistic time dilation, high redshift events are easier to detect,
but these events will be fainter and more difficult to follow up.
Accordingly, population studies of orphan afterglow candidates may by
necessity be conducted with LSST photometry alone.  Such studies may only
be productive if LSST has sufficiently frequent revisits to a field in a
single filter.

% --------------------------------------------------------------------

\subsection{Metrics}
\label{sec:\secname:metrics}

The core figure of merit for GRB afterglows is simply the raw number of
on- and off-axis events detectable in two, three, or more observations,
preferably in a single filter.

The appropriate way to derive these detections is to conduct a Monte
Carlo simulation of a cosmological population of GRBs and fold it
through the LSST observing cadence \citep[cf.][]{2011PASP..123.1034J}.
We are developing this infrastructure for the MAF framework.

In the meantime, simplified metrics can give us a general idea of how well
a given cadence can characterize fast-evolving transients such as GRBs.  We
have created a new metric, \texttt{GRBTransientMetric}, that replaces the
linearly rising and decaying lightcurve in \texttt{TransientMetric} with
the $F \sim t^{-\alpha}$ decay characteristic of on-axis afterglows.  (For
the time being, we neglect the jet break that steepens the rate of decay;
this implies that our detectability estimates are optimistic.)

We simulate random on-axis afterglows using the parameters of
\citet{2011PASP..123.1034J}: the R-band apparent magnitude at 1 minute
after explosion is randomly drawn from a Gaussian with $\mu=15.35$,
$\sigma=1.59$ and decays with $\alpha=1.0$.  For these estimates we
simply assume zero color difference between in all LSST bands.
There are roughly 300 on-axis GRBs per year with these parameters;
we calculate the average fraction of these events which have at least one,
two, or three detections in any single filter.

% Can use https://github.com/lsst/sims_maf/blob/master/python/lsst/sims/maf/metrics/tgaps.py or https://github.com/lsst/sims_maf/blob/master/python/lsst/sims/maf/metrics/cadenceMetrics.py (Inter/Intra-night) to get histograms.  Would be nice to extend to single-band, N-offset

% --------------------------------------------------------------------

\subsection{OpSim Analysis}
\label{sec:\secname:analysis}

We ran \texttt{GRBTransientMetric} on several OpSim v3.3.5 runs with a range of
characteristics:  \opsimdbref{db:baseCadence}, the baseline cadence;
\opsimdbref{db:NEOswithVisitTriplets}, with three visits per WFD field;
\opsimdbref{db:NoVisitPairs}, with no visit pairs; and
\opsimdbref{db:opstwoPS}, a PanSTARRS-like cadence.

\autoref{tab:SummaryGRBs} lists the fraction of on-axis afterglows
detected in at least one, two, and three visits in a single filter.

Because of its wider areal coverage, the PanSTARRS-like cadence of
\opsimdbref{db:opstwoPS} maximizes the fraction of events detected in
one and two epochs.  Not surprisingly, the triplet-visit WFD cadence of
\opsimdbref{db:NEOswithVisitTriplets} maximizes the three-epoch detection
rate.


\begin{table}
  \begin{tabular}{l|p{6cm}|c|c|c|c|p{5cm}}
    FoM & Brief description & {\rotatebox{90}{\opsimdbref{db:baseCadence}}}
	  & {\rotatebox{90}{\opsimdbref{db:NEOswithVisitTriplets}}} &
	  {\rotatebox{90}{\opsimdbref{db:NoVisitPairs}}} &
	  {\rotatebox{90}{\opsimdbref{db:opstwoPS}}} & Notes \\
    \hline
    \thesection-1 & \footnotesize{\texttt{GRBTransientMetric},
    \texttt{nPerFilter}\,$=1$}      & 0.17 & 0.16 & 0.20 & \textbf{0.21} &
    \footnotesize{Fraction of GRB-like transients detected in at least one
    epoch.} \\
    \thesection-2     & \footnotesize{\texttt{GRBTransientMetric},
    \texttt{nPerFilter}\,$=2$}      & 0.12 & 0.10 & 0.09 & \textbf{0.14} &
    \footnotesize{Fraction of GRB-like transients detected in at least two
    epochs in any single filter.} \\
    \thesection-3     & \footnotesize{\texttt{GRBTransientMetric},
    \texttt{nPerFilter}\,$=3$}      & 0.05 & \textbf{0.08} & 0.04 & 0.04 &
    \footnotesize{Fraction of GRB-like transients detected in at least
	    three epochs in any single filter.}
\end{tabular}
\caption{Mean figures-of-merit (FoMs) for on-axis Gamma-Ray Bursts for one,
two, and three detections in a filter.
The best value of each FoM is indicated in bold.
The wider areal coverage of \opsimdbref{db:opstwoPS} improves its detection
rate of GRBs in one and two epochs, while the triplet visits
in \opsimdbref{db:NEOswithVisitTriplets} naturally improve the
three-detection efficiency.
}
\label{tab:SummaryGRBs}
\end{table}

% --------------------------------------------------------------------

\subsection{Discussion}
\label{sec:\secname:discussion}

An LSST cadence purely designed for discovering GRB afterglows would
include three or more visits to each field every night, with the visits
separated by an hour or two. Moreover, it would be conducted in a single
filter in order to best identify the lightcurve shape of off-axis
events.

While the current surveys simulated are far from this ideal
(usually just two closely spaced visits, with subsequent revisits days
later), nonetheless an appreciable number of GRBs are detectable.
\opsimdbref{db:NEOswithVisitTriplets} would detect about 25 events each
year in three epochs, already potentially the largest sample of untriggered
afterglows.

However, some care is required in interpreting these values:
while the GRB afterglow fades rapidly over the first day of the explosion
(Figure \ref{fig:afterglow_lcs}), at later times a 30 minute visit
separation is not enough to reveal significant evolution in the lightcurve.
We intend to enhance our metric to require that detections are counted only
if significant evolution is statistically distinguishable with 1\%
photometry.

In future work we intend to simulate cosmological populations of on- and
off-axis in order to better determine how many events could be discovered
in time to trigger real-time followup.

\begin{figure}[hbt]
\centerline{
\includegraphics[width=0.6\textwidth]{figs/transients/afterglow_cdf.png}
}
\caption{ Cumulative fraction of GRB on-axis afterglows fainter than
magnitude 24.7 at a given time after the burst. We use an $\alpha=1$
decay with no jet breaks and the brightness parameters of
\citet{2011PASP..123.1034J}. }
\label{fig:afterglow_visibility}
\end{figure}

Thanks to LSST's depth, GRBs can be visible for weeks (Figure
\ref{fig:afterglow_visibility}).  Accordingly,
modest enhancements to the intra- and inter-night revisit rate with
single-filter rolling cadences should substantially improve LSST's
discovery and characterization of relativistic explosions.


% ====================================================================

\navigationbar


% --------------------------------------------------------------------

% ====================================================================
%+
% SECTION:
%    section-name.tex  % eg lenstimedelays.tex
%
% CHAPTER:
%    chapter.tex  % eg cosmology.tex
%
% ELEVATOR PITCH:
%    Explain in a few sentences what the relevant discovery or
%    measurement is going to be discussed, and what will be important
%    about it. This is for the browsing reader to get a quick feel
%    for what this section is about.
%
% COMMENTS:
%
%
% BUGS:
%
%
% AUTHORS:
%    Phil Marshall (@drphilmarshall)  - put your name and GitHub username here!
%-
% ====================================================================

\section{Gravitational Wave Sources}
\def\secname{gw}\label{sec:\secname}

\noindent{\it Raffaella Margutti, Z. Doctor, W. Fong, Z. Haiman, V. Kalogera, V. Trimble, B.~A. Zauderer } % (Writing team)

% This individual section will need to describe the particular
% discoveries and measurements that are being targeted in this section's
% science case. It will be helpful to think of a ``science case" as a
% ``science project" that the authors {\it actually plan to do}. Then,
% the sections can follow the tried and tested format of an observing
% proposal: a brief description of the investigation, with references,
% followed by a technical feasibility piece. This latter part will need
% to be quantified using the MAF framework, via a set of metrics that
% need to be computed for any given observing strategy to quantify its
% impact on the described science case. Ideally, these metrics would be
% combined in a well-motivated figure of merit. The section can conclude
% with a discussion of any risks that have been identified, and how
% these could be mitigated.

The first detection of Gravitational Waves (GW) by the advanced LIGO/Virgo collaboration \citep{Abbott16, Abbott09, Acernese08} has recently opened a new window of exploration into our Universe. The amount of information that can be revealed by the properties of the GW emission is immense and holds promises for revolutionary insights, including accurate masses and spins of neutron stars and black holes, tests of General Relativity and an accurate census of the neutron star (NS) and black hole (BH) populations that might challenge our current understanding of massive stellar evolution. However, GW events are poorly localized (10-100 deg$^2$ at the time of LSST operations). The identification of EM counterparts would provide precise localization and distance measurements, in addition to the necessary astrophysical context (e.g. host galaxy properties, connection to specific stellar populations) to fully exploit the revolutionary power of this new GW era.


% --------------------------------------------------------------------

\subsection{Target measurements and discoveries}
\label{sec:\secname:targets}

The first GW event was found to be associated with the merger of two black holes \citep{Abbott16,Abbott16b}. Although no EM counterpart was expected to accompany a black-hole black-hole (BBH) merger, it seems now possible that even BBH mergers  might produce short GRB-like EM emission \citep{Connaughton16, Loeb16,Zhang16,Perna16,Stone16}. Indeed, in analogy with supermassive BH mergers, shocks might develop in the just-formed circumbinary accretion disk (if a disk forms), which can produce a bright afterglow following the BBH merger (e.g. \citealt{Lippai08,Corrales10,Schnittman13}). Albeit speculative in nature, it is advisable to keep an open mind about the possibility of EM counterparts to BBH mergers. 

The most promising and better understood EM counterparts to GW events are ``kilonovae" \citep{Li98, Metzger10, Metzger12,Kasen13,Barnes13}. Kilonovae are short-lived (typical time scale of one week), apparently faint ($z\sim21$ mag at peak at 120 Mpc), red ($i-z\approx1$ mag), isotropic transients (Fig. \ref{Fig:kilonova}) due to the radioactive decay of r-process elements synthesized in the merger ejecta of a NS-NS or NS-BH system. These merging systems are the favored progenitors of short GRBs. Indeed, the signature of a kilonova emission has been recently found following the short GRB\,130603B \citep{Berger13,Tanvir13}. The key piece of information that enabled the discovery of kilonova-like emission associated with  this short GRB was its sub-arcsecond localization enabled by the detection of the optical afterglow, which allowed for an effective kilonova search with the Hubble Space Telescope (Fig. \ref{Fig:kilonova}). In contrast, the typical localization region of GW events in the LSST era is expected to be of the order of a few tens of square degrees \citep{aaa+13}. It is thus clear that the major challenges faced by the optical follow-up of GW events is represented by the combination of poor localizations with faint and fast evolving red electromagnetic counterparts.

The detection of an optical counterpart in conjunction with a GW event will significantly leverage the GW signal.
LSST, with its the wide FOV, wavelength coverage and exquisite sensitivity is uniquely poised to identify and characterize counterparts to GW events. 

\begin{figure}
\vskip -0.0 true cm
\centering
\includegraphics[scale=0.85]{figs/transients/kilonovaBerger.png}
\caption{Kilonova signature in the short GRB\,130603B as revealed by the Hubble Space Telescope (HST). The Magellan and Gemini telescopes sampled the optical afterglow of the GRB (dotted lines). The kilonova light starts to dominate the emission in the H band around a few days after the merger. Thick and dashed lines: theoretical kilonova models from \cite{Barnes13} showing that kilonovae are fast-evolving, faint and red transients. The light-curve of the SN\,2006aj associated with the long GRB\,060218  is also shown for comparison. From \cite{Berger13}.}
\label{Fig:kilonova}
\end{figure}


% --------------------------------------------------------------------



% --------------------------------------------------------------------

\subsection{OpSim Analysis and Discussion}
\label{sec:\secname:analysis}

Effective follow up of GW triggers relies on the capability to sample a relatively large portion of the sky, repeatedly, over a time scale $<1$ week, with different filters \citep{Cowperthwaite15}. In the optical band, the kilonova signature is expected to be more prominent in the $i$, $z$  and $y$ filters, which we identify as the most promising filters for the kilonova search. We emphasize however that another set of contemporaneous observations in  a ``bluer" filter is necessary to acquire color information and distinguish kilonovae from other fast-evolving transients.

We use the median inter-night gap  for visits in the same filter derived from the candidate Baseline Cadence \texttt{minion\_1016} to show that, in the absence of a Target of Opportunity (ToO) capability, it is \emph{not} possible for LSST  to play a major role in the identification of EM counterparts of GW triggers.  

To identify kilonova candidates we need at least 2 observations acquired within $\sim 1$ week  of the GW event \citep{Cowperthwaite15}.
Using the inter-night gap distribution for visits in the $y$ filter (which is the most promising filter for a kilonova search), the area of the sky covered with cadence  $\Delta t<7$ days at any given time, is $A_{sky}\sim 3000$ deg$^2$ (including deep drilling fields).  This is the area that can be searched for fast evolving transients.  Two important considerations follow:

\begin{itemize}
\item[(1)] $A_{sky}$ only covers $P\sim7$\% of the sky. The  probability that the \emph{entire} GW localization region is contained, by chance,  within $A_{sky}$ is thus very small.
\item[(2)] Even if LSST is able to cover a meaningful portion of the GW region, we would still not have color information, and we would thus be unable to filter out contaminating transients.
\end{itemize}

\textbf{We conclude that relying on the serendipitous alignment of the LSST fields with the GW localization map is not an effective strategy to follow up GW triggers and identify their EM counterparts. We thus strongly recommend a ToO capability as part of the baseline LSST operations strategy.}

Ideally, the ToO capability will allow for imaging of the GW localization map at least twice over $\Delta t\lesssim$1 week with a ``red" filter ($i$, $z$  or $y$),  and  will include the possibility to designate a desired set of filters to obtain color information. By the time of LSST operation the typical size of the GW localization region is expected to be 10-100 deg$^2$, which would require a small number of LSST re-pointings. We thus do \emph{not} anticipate a significantly disruptive impact on other LSST campaigns (especially if only the GW triggers with the best localizations in the southern sky are selected for LSST ToOs).

\textbf{At the price of re-shuffling a reasonably small number of fields, \emph{if} equipped with ToO capabilities, LSST can be the premier player in the era of EM follow up to GW sources.}






% ====================================================================

\navigationbar

