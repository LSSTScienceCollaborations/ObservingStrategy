% ====================================================================

\chapter{The Magellanic Clouds}
\def\chpname{mc}\label{chp:\chpname}

Chapter editors:
\credit{knutago},
\credit{dnidever}.

Contributing authors:
\credit{someoneelse},
{\it and others to follow}

% --------------------------------------------------------------------

\section{Introduction}
\label{sec:\chpname:intro}

% Introduce, with a very broad brush, this chapter's science projects,
% and why it makes sense for them to be considered together.

% This individual section will need to describe the particular
% discoveries and measurements that are being targeted in this section's
% science case. It will be helpful to think of a ``science case" as a
% ``science project" that the authors {\it actually plan to do}. Then,
% the sections can follow the tried and tested format of an observing
% proposal: a brief description of the investigation, with references,
% followed by a technical feasibility piece. This latter part will need
% to be quantified using the MAF framework, via a set of metrics that
% need to be computed for any given observing strategy to quantify its
% impact on the described science case. Ideally, these metrics would be
% combined in a well-motivated figure of merit. The section can conclude
% with a discussion of any risks that have been identified, and how
% these could be mitigated.

%A short preamble goes here. What's the context for this science
%project? Where does it fit in the big picture?

The Magellanic Clouds have always had outsized importance for
astrophysics.  They are critical steps in the cosmological distance
ladder, they are a binary galaxy system with a unique interaction
history, and they are laboratories for studying all manner of
astrophysical phenomena.  They are often used as jumping-off points
for investigations of much larger scope and scale; examples are the
searches for extragalactic supernova prompted by the explosion of
SN1987A and the dark matter searches through the technique of
gravitational microlensing.  More than 17,000 papers in the NASA ADS
include the words ``Magellanic Clouds'' in their abstracts or as part
of their keywords, highlighting their importance for a wide variety of
astronomical studies.

Our science goals are as follows:
\begin{enumerate}
\item What are the stellar and dark matter mass profiles of the
Magellanic Clouds?  Map extended disk, halo, debris, and streams.  Use
streams as probes of total mass profile.  RR Lyrae give potential for
three-dimensional stellar profile.
\item What is the satellite population of the Magellanic Clouds?
Discovery of dwarfs by DES and other surveys illuminating for
understanding distribution of dark matter subhalos and how galaxies
form in them (REFS)
\item What are the internal dynamics of the Magellanic Clouds?  Proper
motions from HST and from the ground (REFS) have measured the bulk
motions of the Clouds and have, in combination with spectroscopy,
begun to unravel the three dimensional internal dynamics of the
Clouds...
\item How do exoplanet statistics in the Magellanic Clouds compare to
those in the Milky Way?  Lund calculation shows can measure transits
of Jupiter-like planets, Clouds are lower metallicity environment
\item Identify and characterize the variable star and transient
population of the Clouds.  Population studies, linking to star
formation and chemical enrichment histories, etc, from Szkody et al.
DD white paper.
\item Light echoes from supernovae and explosive events.  Echoes can
give view of such events unavailable by any other means, ref. papers
by Rest et al.
\end{enumerate}

These goals can be categorized under two main overarching science themes:
\begin{enumerate}
\item {\bf Galaxy formation evolution}: The study of the formation and
evolution of the Large and Small Magellanic Clouds (LMC and SMC,
respectively), especially their interaction with each other and the
Milky Way. The Magellanic Clouds (MCs) are a unique local laboratory
for studying the formation and evolution of dwarf galaxies in
exquisite detail.  LSST's large FOV will be able to map out the
three-dimensional structure, metallicity and kinematics in great
detail.
\item {\bf Stellar astrophysics \& Exoplanets}:  The MCs have been
used for decades to study stellar astrophysics, microlensing and other
processes.  The fact that the objects are effectively all at a single
known distance makes it much easier to study them than in, for
example, the Milky Way.  LSST will extend these studies to fainter
magnitudes, higher cadence, and larger area.
\end{enumerate}

Many different types of objects and measurements with their own
cadence ``requirements'' will fall into these two broad categories
(with some overlap).  These will be outlined in the science sections
of this chapter.

A very important aspect of the ``galaxy evolution'' science theme is
not just the cadence but also the sky coverage.  A common
misunderstanding is that the MCs only cover a few degrees on the sky.
That is, however, just the central regions of the MCs akin to the
thinking of the Milky Way as the just the bulge.  The full galaxies
are actually much larger with LMC stars detected at $\sim$21$^{\circ}$
($\sim$18 kpc) and SMC stars at $\sim$10$^{\circ}$ ($\sim$11 kpc) from
their respective centers. The extended stellar debris from their
interaction likely extends to even larger distances.  {\it  Therefore,
to get a complete picture of the complex strucure of the MCs would
require a ``mini-survey'' that covers $\sim$2000 deg$^2$.}  \new{In
this chapter we do  not assume the existence of this special survey,
but instead simply evaluate the existing proposed observing strategies
using our science metrics. We will, however, return to these figures
of merit in \autoref{chp:specialsurveys}, where we suggest some
suitable prooperties of this mini-survey}.  In any case, for our
second science project this is not as much of an issue, since the
large majority of the relevant objects will be located in the
high-density, central regions of the MCs \new{which are covered in the
existing observing strategies}.

\new{Finally, we list some possible observables that might be interesting.}

\begin{enumerate}

\item Deep Color Magnitude Diagrams
%  -Deep CMDs, just a matter of number of visits
%  -do the full SMASH (and relevant DES area) with full spatial coverage, at least to SMASH depths, smaller
%  number of epochs, ~5 sigma at gri~25
% Knut thoughts: I think we want to make sure that we get 1 mag below old turnoff out to 100 kpc in ugriz with 10sigma precision, i.e. ugriz~25


\item Proper Motions
%-Proper Motions, cadence not as much of an issue, just more epochs
%  bulk proper motion
%  LMC spiral motion, streaming motions
%  internal velocity dispersion

%\item Parallaxes
%-Parallaxes, also mostly a function of nubmer of epochs
%  bulk distances
%  internal distance spread

\item Variable stars

%-Variables, RR Lyrae, Cepheids might be too bright, dwarf cepheids/scuti good, many more of them.
%   especially good for getting the 3D structure (out to large distances) of the MCs
%   -eclipsing binaries (get very accurate distances, see OGLE paper), pulsating WDs, CVs, T Tauri stars

\item Transients
%-Transients, dwarf novae

\item Transiting Exoplanets
% -Transiting planets

\item Astrometric binaries
%-Astrometric binaries

\item Gyrochronology
%-Gychronology, need to get periods of the dwarfs, gives age information

\item Astroseismology
%-Astroseismology, dwarfs/giants, giants vary by a couple percent and on "longer" timescales, but
%    probably too bright for LSST, OGLE probably has best data for those. however LSST might be able to do
%    asteroseismology of giants to larger distances, measure masses/ages of halo giants!
%    dwarfs are harder because they vary less and need more higher frequency observations

\end{enumerate}


\navigationbar
