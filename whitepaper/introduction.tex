
\chapter[Introduction]{Introduction}
\def\chpname{intro}\label{chp:\chpname}

\noindent {\it
Zeljko Ivezic, Beth Willman, ...
}

Preamble.

\listofopsimdbs


% --------------------------------------------------------------------

\section{The Baseline Observing Strategy}
\def\secname{intro:baseline}\label{sec:\secname}

Synoptic Surveying with LSST - the basic observing strategy determined
by key projects described in the LSST Science Requirements Document,
and constrained by the LSST's design \citep{IvezicEtal2008}.

Optimizing the Observing Strategy - what perturbations can we
introduce, to maximize the system's science capabilities?


% - - - - - - - - - - - - - - - - - - - - - - - - - - - - - - - - - -

\opsimdb[db:enigma]{enigma1189}

Zeljko's text on the \opsimdbref{db:enigma} goes here.

\navigationbar

% --------------------------------------------------------------------

\section{Some Simulated Alternative Observing Strategies}
\def\secname{intro:alternatives}\label{sec:\secname}

\noindent {\it
Zeljko Ivezic
}

Please see
\url{http://www.astro.washington.edu/users/ivezic/lsst/cadexp2.pdf}
for a summary of the LSST 2015 \OpSim runs. We'll import much of the
material from this document soon.


\navigationbar

% --------------------------------------------------------------------

\section{Evaluation and Optimization}
\def\secname{intro:evaluation}\label{sec:\secname}

The first step towards a science-based optimization of the LSST
observing strategy is a {\it science-based evaluation of the baseline
LSST observing strategy}. Adopting XXXX as this fiducial baseline, we
now need to quantify the value of this observing schedule to each
science team. This is what the LSST DM Sims team's ``Metric Analysis
Framework''  was designed to enable. Once the fiducial strategy has
been evaluated, then any other strategy can be evaluated in the same
terms, using the same code, and we will be able to start optimizing
the strategy through iterations between \OpSim and MAF.

With this program in mind, it makes sense to define {\it one ``Figure
of Merit'' (FoM) per science project}, that captures the value of  the
observing strategy under consideration to that science team. This FoM
will probably be a function of several ``metrics'' that quantify
lower-level features of the observing sequence.  For Figures of Merit
to be directly comparable between disparate science projects,  they
need to be {\it dimensional, and have the same units}. One natural
choice could be the {\it information gained} by the science team, in
bits. This is a well-defined statistical quantity, albeit not yet one
in common use. A given observing schedule's value would then depend on
both this information gain, but also {\it how much that information is
worth to the whole community}. It is at this point that the debate
could become heated: probably the best we can do in Cadence Diplomacy
is to quantify all the information gains implied by each proposed
change to the baseline  observing strategy, combine them to see
whether it makes everyone happy, and iterate. In this way we might
hope to minimize the debates about the less quantifiable worth of each
piece of information.

We are some way from being able to define information-based Figures of
Merit for most science cases -- but the metrics that they will depend
on will be easier to derive. Writing this white paper is an
opportunity to think through the Figure of Merit for each science
project that we as a community want to carry out, and how that measure
of success is likely (or even known) to depend on metrics that
summarize the observing sequence presented to us. Thinking about the
problem in terms of science projects, each with a  Figure of Merit,
encourages us to design {\it modular document
sections,} with one science project and one Figure of Merit per section.

This will have the happy side-effect of allowing the chapters to be
straightforwardly re-arranged as we go, to make the white paper easier
to read. It will also naturally lead to the definition of a suite of
MAF  super-metrics, can be evaluated on any future \OpSim output
database.  A table in each section showing the values of the metrics
and the FoM, for different schedules, for that science project, will
be very helpful. The metric names in these tables should match the
metric class names in the
\href{https://github.com/LSST-nonproject/sims_maf_contrib/wiki}{\simsMafContrib}
module. In principle these tables could be auto-generated by the MAF
framework, and extended as \OpSim is repeatedly reconfigured and run.

For an example of how all this could look, please see the
\hyperref[sec:lenstimedelays]{lens
time delays section}. The MAF subsections are still under development
there, but keep checking back to see it come together during the
August 2015 workshop week.


\navigationbar

% --------------------------------------------------------------------
