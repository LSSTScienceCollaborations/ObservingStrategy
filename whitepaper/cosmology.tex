% --------------------------------------------------------------------

\chapter[Cosmology]{Accurate Cosmological Measurements on the Largest Scales}
\def\chpname{cosmo}\label{chp:\chpname}

Chapter editors:
\credit{egawiser},
\credit{MichelleLochner}.

Contributing authors:
\credit{HumnaAwan},
\credit{egawiser},
\credit{pkurczynski},
\credit{rhiannonlynne},
\credit{jmeyers314},
\credit{tonytyson},
\credit{MelissaGraham},
\credit{SamSchmidt},
\credit{connolly},
\credit{ivezic},
\credit{jhrlsst},
\credit{rbiswas4},
\credit{sethdigel},
\credit{astrofoley},
\credit{lgalbany},
\credit{pgris},
\credit{ReneeHlozek},
\credit{saurabhwjha},
\credit{RickKessler},
\credit{AlexGKim},
\credit{oferlahav},
\credit{aimalz},
\credit{jasonmcewen},
\credit{janewman-pitt-edu},
\credit{hiranyapeiris},
\credit{kponder},
\credit{rlschuhmann},
\credit{astrostubbs},
\credit{wmwv},
\credit{drphilmarshall},
\credit{tanguita}

% \section*{Summary}
% \addcontentsline{toc}{section}{~~~~~~~~~Summary}
%
% Executive summary goes here, highlighting the primary conclusions from
% the chapter's science cases. This should be abstract length, no more:
% say, 200 words.

% --------------------------------------------------------------------

\section{Introduction}
\label{sec:\chpname:intro}

Cosmology is one of the key science themes for which LSST was
designed. Our goal is to measure cosmological parameters, such as the
equation of state of dark energy or departures from General
Relativity, with sufficient accuracy to distinguish one model from
another and hence drive our theoretical understanding of how the
universe works as a whole. To do this will necessarily involve a
variety of different measurements that can act as cross-checks and break
any parameter degeneracies.

The  Dark Energy Science Collaboration (DESC) has identified five
different cosmological probes enabled by the LSST: weak lensing (WL),
large-scale structure (LSS), Type Ia supernovae (SN), strong lensing
(SL), and clusters of galaxies (CL). In all these cases, the primary concern
is the residual systematic error: the shapes and photometric redshifts of
galaxies and the properties of supernova and lensed quasar light
curves will all need to be measured with extraordinary accuracy in
order to properly harness the statistical power available through LSST. This
accuracy will come from the abundance and heterogeneity of the
individual measurements and the degree to which they can be
modeled and understood. This latter point implies a need for uniformity
in the survey, which enables powerful simplifying assumptions to be made
when calibrating on the largest, cosmologically most important scales.
The need for heterogeneity in the measurements also requires uniformity in
the sense that the nuisance parameters that describe the systematic effects
need to be sampled over as wide a range as possible (e.g., the need to
sample a wide range of roll angles to minimize shape error;
observing conditions to understand photometric errors due to the
changing atmosphere).

In this chapter we look at some of the key measurements planned by DESC
and how they depend on the Observing Strategy.


%---------------------------------------------------------------------

% ====================================================================
%+
% SECTION NAME:
%    lss.tex
%
% CHAPTER:
%    cosmology.tex
%
% ELEVATOR PITCH:
%    Large Scale Structure, Weak Lensing, and Clusters all require
%    survey uniformity in the static 10-year survey.  A key contributor
%    to this is the pattern of dithers adopted.
%
% AUTHORS:
%   Eric Gawiser (@egawiser)
%-
% ====================================================================
\newcommand{\sigmaOS}[0]{$\sigma_{\mathrm{C_{\ell, {OS}}}}$}
\newcommand{\CellOS}[0]{$C_{\ell, \rm{OS}}$}
\newcommand{\statFloor}[0]{$\Delta C_\ell$}
\newcommand{\delobs}[0]{\delta_{\mathrm{obs},i}}
\newcommand{\dellss}[0]{\delta_{\mathrm{LSS},i}}
\newcommand{\delos}[0]{\delta_{\mathrm{OS},i}}
\newcommand{\ev}[1]{\left < {#1} \right >}

\section{Large-Scale Structure: Testing Dither Patterns and Timescales to Improve Survey Uniformity}
\def\secname{lss}\label{sec:\secname}

\credit{HumnaAwan},
\credit{egawiser},
\credit{pkurczynski},
\credit{rhiannonlynne}

Three of the key cosmology probes available with LSST represent ``static science'', i.e., insensitive to time-domain concerns.  These are Weak Lensing, Large-Scale Structure, and Galaxy Clusters.  Nonetheless, due to the need to track and correct for the survey window function for these probes, cosmology with LSST will benefit from achieving survey depth as uniform as possible over the WFD area.  OpSim tiles the sky in hexagons inscribed within the nearly-circular LSST FOV. It has been shown in \citet{CarrollEtal2014} that the default LSST survey strategy implemented in OpSim leads to a strongly non-uniform honeycomb pattern due to overlapping regions on the edges of the hexagons receiving nearly double the observing time.  A pattern of large dithers, i.e. telescope-pointing offsets on the scale of the LSST FOV, greatly reduces these overlaps, leading to an increase in the median survey depth in each filter by 0.08 magnitudes.

In this section, we report the results from an investigation into different types of dithers, varying both in terms of the timescales on which dithers are implemented as well as the geometry of the dither positions. The results discussed largely follow the analysis in \citet{AwanEtal2016}, except that here we use the $i$-band-relative mock catalogs and magnitude cuts (as opposed to the $r$-band), and analyze the impacts of different observing strategies using the new baseline cadence \opsimdbref{db:baseCadence} and two other cadences. We also discuss the quantification of the effectiveness of a given observing strategy as a Figure of Merit.

% ====================================================================
% Subsection: Dither Patterns and Timescales
% ====================================================================
\subsection{Dither Patterns and Timescales}
\label{sec:\secname:strategies}
As in \citet{AwanEtal2016}, we consider three timescales to capture the range of time intervals on which dithers can be implemented: by visit, by night, and by season. Both visit and night timescales are well-defined in OpSim \citep[see]{IvezicEtal2008}. For the seasons, we use the \href{https://github.com/lsst/sims_maf/blob/master/python/lsst/sims/maf/stackers/generalStackers.py}{SeasonStacker}, which assigns a season to every observation by tracking when each field's RA is overhead in the middle of the day. The season assignment leads to 11 seasons, and for our purposes, we treat the 0th and 10th seasons the same by assigning them the same dither position.

Another variation in the implementation timescale is added by dithering each field independently as opposed to dithering all fields collectively. For instance, FieldPerNight timescale assigns a new dither position to a field when it is observed on a new night, while PerNight implementation assigns a new dither to all fields every night regardless of whether a particular field is observed or not.

In \citet{AwanEtal2016}, we study five different geometries for the dither positions, where one geometry is specifically for PerSeason assignment and the rest are implemented on three timescales, namely FieldPerVisit, FieldPerNight, and PerNight. Here, we focus only on three combinations of the different geometries and timescales as an illustration of the impacts of these combinations: RepulsiveRandomDitherFieldPerVisit, FermatSpiralDitherPerNight,  and PentagonDitherPerSeason. These geometries are shown in \autoref{fig: dithGeometries}, adapted from Figure 1 in \citet{AwanEtal2016}.

\begin{figure*}[!htb]
      \centering\includegraphics[width=\linewidth, trim={25 570 25 50},clip=true]{figs/awan_dithGeometries.pdf}
\caption{Dither geometries implemented for various timescales. PentagonDither is implemented only on PerSeason timescale, while the rest are implemented on FieldPerVisit, FieldPerNight, and PerNight timescales. Here the green curve is the LSST FOV of radius 0.305 radians; the blue hexagon represents the hexagonal tiling of the sky originally adopted for the undithered observations; and the red points are the dithers. The axes are in radians.}
\label{fig: dithGeometries}
\end{figure*}

Here we note that all the dithers are restricted to lie within the hexagons inscribed in the 3.5$^\circ$ LSST FOV, and that we continue with the naming scheme [Geometry]Dither[Field]Per[Timescale], where the absence of the tag `Field' implies that all fields are assigned the same dither.

% ====================================================================
% Subsection: Metrics
% ====================================================================
\subsection{Metrics}
\label{sec:\secname:metrics}
Our first metric is the \href{https://github.com/lsst/sims_maf/blob/master/python/lsst/sims/maf/metrics/simpleMetrics.py}{CoaddM5Metric}, which we use to calculate the coadded 5$\sigma$ depth resulting from various observing strategies. This allows us to compare the artifacts in the coadded depth induced by the observing strategy. Then we account for the dust extinction, using \href{https://github.com/lsst/sims_maf/blob/master/python/lsst/sims/maf/metrics/exgalM5.py}{ExGalM5Metric}, before estimating the number of galaxies in each pixel at a particular depth. For this purpose, we use a mock LSST catalog \citep{MunozEtal2015} to estimate the power law coefficients for each redshift bin, converting the depth into an estimated number of galaxies for each pixel \citep[Eq. 2]{AwanEtal2016}:
\begin{equation}
	N_{\mathrm{gal}}= 0.5\int_{-\infty}^{m_\mathrm{{max}}} {\mathrm{erfc}[a(m-5\sigma_{\mathrm{stack}})] 10^{c_1m + c_2}dm}
	\label{eq: Ngal}
\end{equation}

Here the \texttt{erfc} function accounts for incompleteness while the constants c$_1$ and c$_2$ are determined from the mock catalog. $m_{\mathrm{max}}$ specifies the magnitude cut, and we modify both $m_\mathrm{{max}}$ and c$_2$ to account for colors (assumed to be $u-g= g-r= r-i= 0.4$). This calculation is carried out using the \href{https://github.com/humnaawan/sims_maf_contrib/blob/master/mafContrib/galaxyCountsMetric_extended.py}{GalaxyCountMetric}.

We then account for the effects of photometric calibrations in our estimated number of galaxies. As discussed in \citet{AwanEtal2016}, we model the calibration uncertainties using a simple ansatz relating the systematic errors in each pixel to the seeing in that pixel (relative to the average seeing across the survey region; $\Delta s_{i}$)  and number of observations $N_{\mathrm{obs},i}$:
\begin{equation}
	\Delta_{i}= \frac{k \Delta s_{i}}{\sqrt{N_{\mathrm{obs}, i}}}
\end{equation}
where $k$ is a constant such that $\sigma_{\Delta_i}$ matches the LSST photometric calibration goal of 1$\%$  \citep{2009arXiv0912.0201L}. Since the calibration uncertainties are pixel-dependent, we use the \href{https://github.com/humnaawan/sims_maf_contrib/blob/master/mafContrib/galaxyCounts_withPixelCalibration.py}{pseudo-GalaxyCountsMetric}, which handles pixel-by-pixel calculation to modify the upper limit in the integral in \autoref{eq: Ngal} to be $m_{\mathrm{max}} + \Delta_{i}$, thereby accounting for the fluctuations in the galaxy counts due to the calibration errors. We then calculate the fluctuations in the galaxy counts in each pixel, $\delta_i= \Delta N_i/\overline{N}$.

% ====================================================================
% Subsection: Figure of Metric
% ====================================================================
\subsection{Figure of Merit}
\label{sec:\secname:FoM}
As derived and discussed in \citet{AwanEtal2016}, the spurious power from the artificial fluctuations in the galaxy counts induced by the observing strategy (OS) represents a bias in our measurement of the LSS. Hence, the uncertainty in this bias becomes the limiting factor in our ability to correct for the OS-induced structure. More quantitatively, for an optimized LSS study, the OS-induced uncertainties, \sigmaOS, must be subdominant to the statistical uncertainty \statFloor\ inherent to the measured power spectrum due to ``cosmic variance'' \citep{Dodelson}:
\begin{equation}
	\Delta C_\ell= C_{\ell,\mathrm{LSS}} \sqrt{\frac{2}{f_{\mathrm{sky}} (2\ell + 1)}}
	\label{eq: statFloor}
\end{equation}
where $f_{\rm{sky}}$ is the fraction of the sky observed, accounting for the reduction in the observed information due to incomplete sky coverage.

Since we do not include any input LSS in our pipeline, the power spectrum we measure for any given band is due to the OS-induced power, \CellOS, for that band. Modeling the overall OS-induced bias as an average across $ugri$ bands, we calculate the OS-induced uncertainties \sigmaOS\ as the standard deviation across \CellOS\ for $ugri$ bands to account for the effects of detecting the galaxy catalog through various bands. We then compare these uncertainties with the statistical floor for various redshift bins, where the statistical floor is based on the galaxy power spectra calculated using the code from \citet{Zhan2006}, which we pixelize to match the HEALPix resolution to account for the finite angular resolution of our simulations.

To quantify the effectiveness of each observing strategy in minimizing \sigmaOS, we construct a Figure of Merit (FoM) as the ratio of the ideal-case uncertainty in the measured power spectrum and the uncertainty where shot noise and OS-induced structure also play a role:
\begin{equation}
	\mathrm{FoM} = \sqrt{\frac{\sum\limits_\ell{\left({\sqrt{\frac{2}{f_{\mathrm{sky, max}} (2\ell + 1)}}C_{\ell, \mathrm{LSS}}} \right)}^2}{\sum\limits_\ell \left[{ \left( { \sqrt{\frac{2}{f_{\mathrm{sky}} (2\ell + 1)}}\left\{{C_{\ell, \mathrm{LSS}} + \frac{1}{\bar{\eta}}} \right\}  } \right ) ^2 + \sigma_{C_{\ell,\mathrm{OS}}}^2  }\right]}}
\label{eq: FoM}
\end{equation}
Here, $\bar{\eta}$ is the surface number density in steradians$^{-1}$, and the term containing it accounts for the contribution from the shot noise to the measured signal \citep{HutererEtal2001,Jing2005}. This FoM measures the percentage of ideal-case information that can be measured in the presence of systematics. We note that the shot noise is negligible even for the shallowest (10-year) surveys we consider.

We define the ideal-case as being based on the largest coverage of the sky with LSST, i.e., $f_{\rm{sky, max}}$ is the largest WFD coverage with the baseline cadence. For \opsimdbref{db:baseCadence}, the observing strategy with RepulsiveRandomDitherFieldPerVisit dithers leads to the largest $f_{\rm{sky}}$ ($\sim 39.5\%$). Note that this fraction is calculated after masking the shallow borders of the main survey; for details, see \citet{AwanEtal2016}. %Therefore, we fix the $f_{\rm{sky, best}}$ to be $39.5\%$ and compare the the different observing strategies and cadences relative to it. 

% ====================================================================
% Subsection: A Comment on Terminology
% ====================================================================
\subsection{A Comment on Terminology}
For clarity, we make a note on the terminology we have introduced. Strictly speaking, the bias caused by the observing strategy is a window function bias, as the survey window function ($W_i$) accounts for the effective survey geometry which scales the fluctuations in the galaxy counts in each pixel: $1+\delobs= W_i(1+\dellss)$. Comparing this with Equation 4 in \citet{AwanEtal2016}, $1+\delobs= (1+\delos)(1+\dellss)$, we see that the OS-induced bias is directly related to the window function: $1 + \delos= W_i$

Then, for the total power, we have
\begin{equation}
\ev{\delobs^2}=\ev{\dellss^2}\ev{(1 + \delos)^2}+ \ev{\delos^2}=\ev{\dellss^2}\ev{W_i^2}+ \ev{(W_i-1)^2}
\end{equation}
where the first equality is based on Equation 6 in \citet{AwanEtal2016} and the second one holds given the relation between $\delos$ and $W_i$. Since the OS-induced bias $\delos^2=  (W_i-1)^2$, the uncertainties in the OS-induced bias are the window function uncertainties.

Generally the window function is assumed to be known perfectly and its uncertainties are not explicitly identified as such. To avoid confusion and focus on the window function uncertainties arising from the observing strategy, we continue using the term OS-induced bias and its uncertainties in favor of window function and its uncertainties.

% ====================================================================
% Subsection: OpSim Analysis and Results
% ====================================================================
\subsection{OpSim Analysis and Results}
\label{sec:\secname: analysis}
For the purposes of our analysis, we use HEALPix resolution of $N_\mathrm{_{side}}= 256$, effectively tiling each $3.5^\circ$ FOV with about 190 HEALPix pixels. Using the metrics discussed in \autoref{sec:\secname:metrics}, we analyze \sigmaOS\ from various observing strategies. First we present the results for the baseline cadence, \opsimdbref{db:baseCadence}.

\begin{figure*}[!htb]
      \centering\hspace*{-3em}\includegraphics[width=0.6\linewidth]{figs/awan_coaddHistogram.pdf}
      \vspace*{-1em}
\caption{Histogram for the $i$-band coadded 5$\sigma$ depth after the full, 10-year survey.}
\label{fig: coaddHistogram}
\end{figure*}

\autoref{fig: coaddHistogram} shows the histogram for the $i$-band coadded 5$\sigma$ depth from \opsimdbref{db:baseCadence} for the four observing strategies. We observe a bimodal distribution for the undithered survey -- the deeper depth mode corresponds to the overlapping regions between the hexagons, while the rest of the survey contributes to the shallower mode. In contrast, all dithered surveys lead to unimodal distributions as the overlapping regions between the fields change frequently, leading to more uniformity. We also note that frequent dithering leads to deeper regions as we observe more peaked histograms for FieldPerVisit and PerNight strategies.

\autoref{fig: coaddSkymaps} shows the plots for the $i$-band coadded 5$\sigma$ depth for the observing strategies. As in \citet{AwanEtal2016}, we find that the undithered survey leads to a strong honeycomb pattern which is much weaker in all of the dithered surveys. We again observe that the dithered surveys are deeper than the undithered survey in terms of the median depth across the survey region.

\begin{figure*}[!htb]
      \centering\includegraphics[width=\linewidth, trim={50 470 55 70},clip=true]{figs/awan_minion1016_coaddSkymaps.pdf}
\caption{Plots for the $i$-band coadded 5$\sigma$ depth based on \opsimdbref{db:baseCadence} for various observing strategies. The top left plot shows the Mollweide projection for NoDither while the bottom left shows the corresponding Cartesian projection, restricted to $180^\circ>$RA$>-180^\circ$ (left-right), $-70^\circ<$Dec$<10^\circ$ (bottom-top). Only the latter is shown for the rest of the strategies. }
\label{fig: coaddSkymaps}
\end{figure*}

In order to quantify the angular characteristics observed in the skymaps, we calculate the angular power spectra corresponding to the skymaps for the $i$-band coadded 5$\sigma$ depth. \autoref{fig: coaddPowerSpectrum} shows these spectra for the four observing strategies. We observe a sharp reduction in the artificial power in the dithered surveys when compared to the undithered one: the strong honeycomb pattern in the undithered survey leads to a large peak around $\ell\sim150$, while the peak is about 10 times weaker in the dithered surveys. We do, however, observe variations amongst the various dither strategies: while RepulsiveRandom dithers lead to small power for all timescales, PerSeason dithers lead to large power on larger angular scales, and both PerSeason and FermatSpiral lead to large power around $\ell\sim150$ (which still is $<10\times$ the corresponding peak from the undithered survey).

\begin{figure*}[!htb]
      \centering\includegraphics[width=\linewidth]{figs/awan_coaddpowerspectrum.pdf}
      \vspace*{-2em}
\caption{Angular power spectra for the  $i$-band coadded 5$\sigma$ depth from \opsimdbref{db:baseCadence} for various observing strategies. We note that dithering reduces the spurious power by over 10$\times$.}
\label{fig: coaddPowerSpectrum}
\end{figure*}

We then proceed to calculate the OS-induced bias and its uncertainty from the different observing strategies. First, we examine simulated results after only one year of survey. \autoref{fig: minion1016: 1yr} shows the comparison between \sigmaOS\ and \statFloor\ for $0.66<z<1.0$ after the 1-year survey for two magnitude cuts: $i<24.0$ and $i<25.3$. We observe that the undithered survey leads to \sigmaOS\ 1-5$\times$ the statistical floor around $\ell\sim150$; PerSeason timescale does only slightly better. However, we see an improvement with frequent dithers: both FieldPerVisit and PerNight implementations lead to uncertainties 0.5-1$\times$ the statistical floor, although FermatSpiral dithers on PerNight timescale lead to a peak around $\ell\sim150$ more pronounced than the one from RepulsiveRandom dithers on FieldPerVisit timescale.

\begin{figure*}[!htb]
      \centering\hspace*{1em}\includegraphics[width=\linewidth]{figs/awan_1yr_minion1016_2magCuts.pdf}
       \vspace*{-2em}
\caption{\sigmaOS\ comparison with the minimum statistical uncertainty \statFloor\ for $0.66<z<1.0$ for different magnitude cuts after only one year of survey based on \opsimdbref{db:baseCadence}.}
\label{fig: minion1016: 1yr}
\end{figure*}

The trends are captured in the Figure of Merit, which we calculate using \autoref{eq: FoM} over the range $100<\ell<300$. We observe a smaller FoM for the shallower survey -- realistic given that although there is less structure and therefore weaker OS-induced artifacts, the shot noise becomes significant and makes the FoM smaller. For the deeper survey, we find that FermatSpiralDitherPerNight outperforms all others with the highest FoM, while RepulsiveRandomDitherFieldPerVisit is more effective than PerSeason dithers. The undithered survey, as expected, performs the worst.

In \autoref{fig: minion1016: 10yr}, we show simulated results after the full, 10-year survey for $0.66<z<1.0$ for three different magnitude cuts: $i<24.0$, $i<25.3$ and $i<27.5$. We observe stark differences between the undithered and dithered surveys: the former leads to large uncertainties in the OS-induced bias while the latter is effective in bringing \sigmaOS\ well below the statistical floor. The effectiveness of all three dithered surveys in minimizing the uncertainties implies more flexibility in choosing the dither strategy for years 2-10.

%For 10yr, 3 magnitude cuts with minion1016, NoDither was always bad but PerSeason dithers were giving us comparable FoM to PerNight and FieldPerVisit dithers. That is not the case anymore; PerSeason dithers are now doing slightly better than before though still not as good as PerNight and FieldPerVisit dithers.

Analyzing the FoM more closely, we observe that the gold sample leads to smaller FoM than both the shallower and deeper catalogs. The larger FoM for shallower catalog is realistic, given less structure with shallow depth leads to weaker artifacts and the shot noise is negligible over the full ten-year survey, but the out-of-trend behavior of gold sample hints at a peculiarity of the variance across the $ugri$ bands at that depth for the baseline cadence. We investigate this behavior briefly and find that the $u$-band-induced artifacts add the most to the uncertainties in the bias induced by the observing strategy, as the gold sample $u$-band cadence in the \opsimdbref{db:baseCadence} is different from $gri$ cadences. This issue still needs to be further investigated, with potentially incorporating the importance of each band to calculate an overall OS-induced bias. We note, however, that this peculiarity is particularly enhanced for the undithered survey.

\begin{figure*}[!htb]
      \centering\hspace*{1em}\includegraphics[width=\linewidth]{figs/awan_10yr_minion1016_3magCuts.pdf}
       \vspace*{-2em}
\caption{\sigmaOS\ comparison with the minimum statistical uncertainty \statFloor\ for $0.66<z<1.0$ for different magnitude cuts after the full, 10-year survey based on \opsimdbref{db:baseCadence}.}
\label{fig: minion1016: 10yr}
\end{figure*}

The trends observed here remain consistent for all five redshift bins. We note that our choice of dithers is particularly important for the one-year survey as only one of the three dither strategies leads to a large FoM. Therefore, in the absence of effective dithers, systematics correction methods will become necessary after the one-year survey. However, these methods may not lead to significant improvements for a dithered 10-year survey as dithers of most kinds are effective in reducing the uncertainties well below the minimum statistical limit.

To further probe the effects of dithers, we run the 1-year and 10-year analyses for two cadences besides the baseline cadence: \opsimdbref{db:NoVisitPairs} which does not require visit pairs, and \opsimdbref{db:opstwoPS} which implements a Pan-STARRS-like observing strategy offering a larger area coverage. In \autoref{fig: cadences: 1yr}, we compare the results from these two cadences with those from \opsimdbref{db:baseCadence} for $0.66<z<1.0$ for the  $i<25.3$ galaxy sample after only one year of survey. We see that the undithered survey leads to large uncertainties in the OS-induced bias with all three cadences, with the peak uncertainty 5-15$\times$ the statistical floor. As expected, the undithered survey with the wider coverage \opsimdbref{db:opstwoPS} cadence leads to stronger artifacts and a much smaller FoM (by $\sim33\%$ in comparison with \opsimdbref{db:baseCadence}), while not requiring visit-pairs is slightly more effective than the baseline (FoM increases by about 6$\%$). We see very similar trends for the three cadences for PerSeason dithers although the peak \sigmaOS\ ranges between 3-9$\times$ the statistical floor; FoM based on \opsimdbref{db:opstwoPS} is worse than that from \opsimdbref{db:baseCadence} by about 25$\%$ and  \opsimdbref{db:NoVisitPairs} improves on the baseline FoM by $\sim5\%$.

% For different cadences, 1yr results, we now have FoM really high for the wider minion1020 for both PerNight and FieldPerVisit dithers while NoDither and PerSeason dithers are still performing poorly. RepRandom dithers with the wider survey gets us F0M=0.99 while FermatSpiral dithers perform better than RepRandom for the other two cadences.

As before, \sigmaOS\ improves with more frequent dithering. It is only about 1-3$\times$ the statistical floor for FermatSpiral dithers on PerNight timescale. In contrast to NoDither and PerSeason dithers, both \opsimdbref{db:opstwoPS} and \opsimdbref{db:NoVisitPairs} perform better than baseline\opsimdbref{db:baseCadence} with PerNight dithers: FoM from the wider coverage cadence is about $4.5\%$ better than for the baseline cadence, while we see a $4\%$ better FoM with \opsimdbref{db:NoVisitPairs}. 

For RepulsiveRandom dithers on FieldPerVisit timescale, we find that the uncertainties in the OS-induced bias are on the same scale as the statistical floor. The wider coverage cadence outperforms the baseline cadence significantly as  the wider survey FoM is about $18\%$ better than the baseline FoM while the improvement is about 3$\%$ when not requiring visit-pairs. We emphasize that the differences between results with different cadences is highly dependent on the observing strategy: the wider coverage with no or infrequent dithers performs quite poorly while it significantly improves the FoM when large, frequent dithers are implemented. On the other hand, not requiring visit-pairs leads to comparatively larger improvement for infrequent dithers than frequent ones (compared to the baseline).

\begin{figure*}[!htb]
      \centering\includegraphics[width=\linewidth]{figs/awan_1yr_goldSample_3cadences.pdf}
       \vspace*{-2em}
\caption{\sigmaOS\ comparison with the minimum statistical uncertainty \statFloor\ for $0.66<z<1.0$ for three different cadences for $i<25.3$ after only one year of survey.}
\label{fig: cadences: 1yr}
\end{figure*}

Finally, we show the simulated results for different cadences after the 10-year survey in \autoref{fig: cadences: 10yr}. As in \autoref{fig: minion1016: 10yr}, we see that all the dithered surveys effectively minimize the uncertainties, regardless of the cadence. We do observe, however, that the wider coverage \opsimdbref{db:opstwoPS} still underperforms significantly for the undithered survey (FoM about 30$\%$ less than baseline FoM)  while all the dithered surveys see a stark improvement (FoM $>$ 1 for all; $\sim 20\%$ improvement on the baseline FoM). The improvement from \opsimdbref{db:NoVisitPairs} is comparable among the four observing strategies. Based on these results, we note than wider coverage offers significant improvements with large dithers on any implementation timescale.

\begin{figure*}[!htb]
      \centering\includegraphics[width=\linewidth]{figs/awan_10yr_goldSample_3cadences.pdf}
       \vspace*{-2em}
\caption{\sigmaOS\ comparison with the minimum statistical uncertainty \statFloor\ for $0.66<z<1.0$ for three different cadences for $i<25.3$ after the full, 10-year survey.}
\label{fig: cadences: 10yr}
\end{figure*}

% ====================================================================
% Science Case Conclusions
% ====================================================================
\subsection{Conclusions}

Here we answer the ten questions posed in
\autoref{sec:intro:evaluation:caseConclusions}:

\begin{description}

\item[Q1:] {\it Does the science case place any constraints on the
tradeoff between the sky coverage and coadded depth? For example, should
the sky coverage be maximized (to $\sim$30,000 deg$^2$, as e.g., in
Pan-STARRS) or the number of detected galaxies (the current baseline but
with 18,000 deg$^2$)?}

\item[A1:] As we see in \autoref{sec:\secname: analysis}, a deeper
catalog is more effective, though it makes the choice of the dither
strategy more important, especially in the first year of survey. We also see 
that the wider-coverage cadence \opsimdbref{db:opstwoPS} performs 
significantly better for LSS systematics with large, frequent dithers  
while it performs much poorly with no or infrequent dithers; this
trend is consistent for both one-year and the full, ten-year surveys. We
note here that one year of the wider coverage (for gold sample) with frequent large
dithers leads to better systematics (as quantized here) than ten years of 
the standard WFD  footprint, strongly supporting the effectiveness of wider area
coverage in the first year of survey for LSS systematics. We are definitely area-
limited more than depth-limited for LSS studies.

\item[Q2:] {\it Does the science case place any constraints on the
tradeoff between uniformity of sampling and frequency of  sampling? For
example, a rolling cadence can provide enhanced sample rates over a part
of the survey or the entire survey for a designated time at the cost of
reduced sample rate the rest of the time (while maintaining the nominal
total visit counts).}

\item[A2:] Depth uniformity is critical for LSS systematics. As we
demonstrated in \autoref{sec:\secname: analysis}, LSS studies will benefit
strongly from large dithers and wide area coverage. We do not have constraints
on the cadence.

\item[Q3:] {\it Does the science case place any constraints on the
tradeoff between the single-visit depth and the number of visits
(especially in the $u$-band where longer exposures would minimize the
impact of the readout noise)?}

\item[A3:] From our investigation into the large uncertainties in the
OS-induced bias observed in the gold sample in the baseline cadence, in
comparison with the shallower and deeper catalogs, we find that
$u$-band-induced artifacts add the most to the uncertainties in the
bias. Hence there could be a significant penalty from reducing the
number of $u$-band visit. At minimum, doing so would make the choice of
dither pattern more important. This issue still needs to be further
investigated.

\item[Q4:] {\it Does the science case place any constraints on the
Galactic plane coverage (spatial coverage, temporal sampling, visits per
band)?}

\item[A4:] LSS systematics do not place any constraints on the Galactic
plane coverage.

\item[Q5:] {\it Does the science case place any constraints on the
fraction of observing time allocated to each band?}

\item[A5:] Increasing the number of visits leads to greater survey
uniformity. At present, this is worst (among $ugri$) in the $u$-band, so
increasing the fraction of $u$-band observing time would likely help.

\item[Q6:] {\it Does the science case place any constraints on the
cadence for deep drilling fields?}

\item[A6:] LSS systematics do not constrain the cadence for deep
drilling fields as long as the main survey dithers are not affected.

\item[Q7:] {\it Assuming two visits per night, would the science case
benefit if they are obtained in the same band or not?}

\item[A7:] We do not see significant difference between obtaining two
visits per night in the same band or not, although we do see a mild
benefit in not obtaining the visits in the same band as it allows
greater variation in atmospheric conditions in each band.

\item[Q8:] {\it Will the case science benefit from a special cadence
prescription during commissioning or early in the survey, such as:
acquiring a full 10-year count of visits for a small area (either in all
the bands or in a  selected set); a greatly enhanced cadence for a small
area?}

\item[A8:] We will request full, 10-year depth during commissioning to
validate our choice of dither pattern.

\item[Q9:] {\it Does the science case place any constraints on the
sampling of observing conditions (e.g., seeing, dark sky, airmass),
possibly as a function of band, etc.?}

\item[A9:] Seeing will play a role in the photometric calibration
errors. However, these errors appear to be subdominant to the artifacts
induced by the observing strategy.

\item[Q10:] {\it Does the case have science drivers that would require
real-time exposure time optimization to obtain nearly constant
single-visit limiting depth?}

\item[A10:] We do not require any real-time exposure time optimization.

\end{description}

% ====================================================================
% Discussion
% ====================================================================
\subsection{Discussion}
\label{sec:\secname:discussion}

In this section, we presented results for the impacts of LSST observing
strategy on LSS studies. Using the OpSim cadence baseline
\opsimdbref{db:baseCadence}, we demonstrate that dithers are necessary
for both 1-year and 10-year surveys. We find that of the three dither strategies
discussed here, FermatSpiral dithers on PerNight  timescale are the most
effective for the gold sample after one-year of survey while  dithers of all kinds
are effective after the ten-year survey. These results imply the need for a very careful
choice of the observing strategy in the first year while there is quite a range of choice 
for years 2-10.

We also analyze two other cadences and find  that frequent dithering with
maximum  sky coverage could allow a significant  fraction LSST-enabled LSS
science after  one-year. Assuming that the quality of photometric redshifts is
fixed (when it actually improves with depth) and that our ansatz for window
function uncertainties is representative,  our results can go as far as implying
that the wider coverage for a few years is far more important than more years of the
baseline  WFD coverage.

Future work will entail improving our analysis to better constrain the
artifacts induced by the observing strategy by, e.g., including
uncertainties in the dust extinction, using improved models for the
photometric calibration uncertainties, more realistic galaxy colors,
incorporating improved mock catalogs to better estimate the galaxy
counts as well as its uncertainties, and a better estimate of the
uncertainties in the OS-induced by a more thorough accounting of the
effects of each band. Also, the development and the analysis of the
effectiveness of various systematics correction methods needs to be
carried out, especially for the 1-year survey, as only a few observing 
strategies reduce the artifacts. Finally, the effectiveness
of various dithers still needs to be assessed for other science probes.


\navigationbar


% --------------------------------------------------------------------

% ====================================================================
%+
% SECTION NAME:
%    wl.tex
%
% CHAPTER:
%    cosmology.tex
%
% ELEVATOR PITCH:
%-
% ====================================================================

\section{Weak Lensing}
\def\secname{wl}\label{sec:\secname}

\credit{tonytyson},
\credit{jmeyers314},
\credit{StephenRidgway}.

Much of LSST cosmology may be limited by systematic errors rather than
photon signal-to-noise. This is especially true of weak gravitational
lensing,  which relies on very accurate (\ie low bias), but very low
signal-to-noise, measurements of the shapes of galaxies, and high
signal-to-noise measurements of PSF calibration stars. As outlined in the SRD,
uniformity of seeing in the bands used for WL and special observing strategies
are required in order to reduce additive and multiplicative shear systematics.

Achieving the ultimate sensitivity of the LSST to weak lensing science places
stringent requirements on our ability to accurately measure galaxy shapes and redshifts,
which in turn demands precise and accurate knowledge of the point spread function,
astrometry, and photometry. These measurements are influenced by the interaction of
light with the Earth's atmosphere, the telescope optics, and the CCD sensors. Sysematics
in the shear are introduced in each case.   Methods have been developed for suppressing
these systematics in current lensing surveys. These and new methods will be applied to
the LSST survey.

Over the sample of 3-4 billion galaxies, the shear systematics must be below
one part in 10,000 for additive shear, and one part in 1000 for multiplicative shear.
Each visit to a sky patch encounters these systematics. Some observing strategies can
effectively randomize these over all visits to a field.  Below we discuss the observing
strategies for suppressing shear systematics and metrics for their success.

\subsection{Target Selection}

Image quality must be uniformly good in the bands used for weak lens shear.  These will be
the $r$ and $i$ bands.   Depending on the current weather and seeing, the scheduler
will have a list of priorities for next-field, based on prior history of coverage.
Nearby fields in need of coverage in these bands will be given high priority if the
seeing is better than some specified value, likely 0.7 arcsec FWHM.


\subsection{Target Measurements}

It is expected that even after maximal optimization of camera optics
and electronics, that systematic image shape errors will be associated
with the orientation of the camera focal plane.  Using data from vendor CCDs, simulations
of LSST observing have shown that a combination of x-y dithering on the sky and
pipeline processing with pixel re-map (to cancel much of the CCD frame fixed
distortions) can get well within a factor of ten of the goal for shear
systematics residuals.  Simulations which add camera angle dithering show
that the goal can be achieved in fields with relatively uniform seeing history.

Thus shear systematics will be partially reduced by randomization of the
orientation of the camera with respect to the sky.  This is
represented by the parameter RotSkyPos: we can construct diagnostic
metrics that quantify the uniformity of its distribution at each sky
position.   Given the spin 2 symmetry of shear, the optimal strategy for shear systematics
will be to aim for uniformity of RotSkyPos mod $\pi$, since angles separated by $\pi$ radians
are degenerate.

Similarly, the telescope optics may harbor systematic aberrations, and
these also could be mitigated by recording images with varying
parallactic angle.  More important is the effect of atmospheric differential chromatic
refraction: re-visits to a given field should be distributed over hour angles, consistent
with airmass and seeing limits.

Uniformity of depth is important, but less so than uniformity in camera
rotator shear suppression.  Simulations have shown that for the Gold sample of galaxies,
uniformity at the 0.2 mag level in limiting magnitude produces little shear bias. The
largest effect comes from bias in weak lens magnification tomography.


\subsection{Metrics}

For characterizing the isotropy of rotational sampling, both for rotSkyPos and the parallactic
angle, we investigate two metrics: the AngularSpreadMetric and the KuiperMetric.  The
AngularSpreadMetric characterizes the balance of a set of angular values, in the sense that opposing
angles, those separated by $\pi$ radians, have zero contribution to the AngularSpread.  The Kuiper
statistic, which is related to the well known Kolmogorov Smirnov statistic, characterizes the
departure of a distribution from uniform, but with the added quality of being invariant under cyclic
transformations of the input set of angles.

The AngularSpread metric is computed as follows:  Given a set of angles $\{\theta\}_{i=1, ..., N}$,
map these angles onto a unit circle: $(x_i, y_i) = (\cos \theta_i, \sin \theta_i)$, and find the 2D
centroid: $(\bar{x}, \bar{y}) = \frac{1}{N} (\sum_i x_i, \sum_i y_i)$.  The AngularSpread is the
distance of the 2D centroid from the unit circle:
$\mathrm{AngularSpread} = 1 - \sqrt{\bar{x}^2 + \bar{y}^2}$.  An AngularSpread of 1 therefore
corresponds to a perfectly balanced distribution, in which the averages of both $\cos \theta$ and
$\sin \theta$ are zero, while an AngularSpread of 0 indicates a maximally anisotropic distribution
in which every angle is identical: $\theta_i = \mathrm{const}$.  As mentioned above, weak lensing
shear systematics cancel to first order when those systematics are separated not by an angle of
$\pi$ radians, but by an angle of $\pi/2$ radians.  To incorporate this spin-2 nature of shear
systematics is simple, we just multiply each angle $\theta_i$ by two before applying the
AngularSpread metric, so that, for example, pairs of angles initially separated by $\pi/2$ radians
become separated by $\pi$ radians and correctly cancel.

While the AngularSpread metric does a good job at characterizing the balance of a distribution
defined on a circle, it isn't directly studying the {\emph uniformity} of said distribution.  For
instance, the AngularSpread of the angles $\{0, 0, 0, 0, \pi, \pi, \pi, \pi\}$ is zero, but the
distribution is far from uniform.  The Kolmogorov Smirnov (KS) test is well known for investigating
whether a set of data are consistent with a given distribution.  The KS statistic, off which the
test is based, is defined as the maximum absolute difference in the empirical cumulative
distribution function (CDF) of the data and the CDF of the distribution being tested.  The Kuiper
statistic is a slight modification of the KS statistic, defined as the sum of the maximum difference
and absolute minimum (maximally negative) difference between the empirical and test CDFs.  This
modification is convenient for characterizing distributions defined on a circle, since it makes the
statistic invariant under rotations of the data.  To incorporate the spin-2 nature of shear
systematics for the Kuiper statistic, we map the values $\theta_i \rightarrow \theta_i \mod \pi$ and
compare to the uniform distribution between 0 and $\pi$.

%
%
% Our figure of merit for weak lensing is $\sigma_{\rm sys}$, the
% residual shear systematic error. This is related to the uniformity of
% the sky survey in ways not yet derived, but for now we can prepare by
% calculating some low-level diagnostic metrics.
%
% A metric is available for RotSkyPos.  The metric computes, for any
% selected filter and simulation, a histogram of the distribution of rms
% values of RotSkyPos computed per field. It also computes basic
% statistics of these distributions.


\subsection{Ancilliary data}

We can use largescale patterns of distortions of the PSF over the 20,000 stars per exposure for
PSF regularization in the per-CCD PSF fitting. In the per CCD fits, there is a benefit to
setting aside some stars for validation tests of PSF extrapolation.
In addition to using all the stars in a given visit, there is useful information in the
wavefront sensors and the guide CCDs that may be used to regularize the PSF
reconstruction in a visit. We might read out guider CCDs in different ways to better
monitor the atmosphere.


\subsection{OpSim Analysis}

% \begin{figure}
% \centering\includegraphics[width=\linewidth]{figs/enigma1189RmsAnglerotSkyPosugrizybandallpropsOPSIComboHistogram.png}
% \caption{The relative angle of the detector plane with respect to the sky, RotSkyPos, as a histogram showing the number of fields vs. rms of the parameter.}
% \label{RotSkyPos}
% \end{figure}

% The distribution of rms values by filter is shown in
% \autoref{RotSkyPos} for the current candidate baseline simulation,
% enigma\_1189.  As shown, the rms values cluster around the value 1
% radian,  with typical values 1 +- 0.3 radian.  This compares to a
% completely uniform distribution over the half circle with an rms of
% 1.14.  As mentioned above, uniformity in cosine squared is the goal.
% Simulated observing of 100 visits to a field show this will produce
% a factor of 10 decrease in CCD-based shear systematics such as edge
% effects and the brighter-fatter x-y anisotropy.




\newcommand\plottwo[2]{{%
\typeout{Plottwo included the files #1 #2}
\centering
\leavevmode
\includegraphics*[width=0.45\columnwidth]{#1}%
\hfil
\includegraphics*[width=0.45\columnwidth]{#2}%
}}%


%  rotSkyPos metrics

\begin{figure}[tbh!]
\plottwo{figs/WL/minion_1016_AngularSpread_rotSkyPos_propID_54_and_i_HEAL_SkyMap.pdf}
        {figs/WL/minion_1016_AngularSpread_rotSkyPos_u_g_r_i_z_y_propID_54_HEAL_ComboHistogram.pdf}
\caption{\textbf{Left:} Sky map showing the distribution of the AngularSpread metric applied to the
    angle rotSkyPos mod $\pi$, where rotSkyPos is the angle between the $+y$ camera direction and
    North, and the modulus with period $\pi$ is taken to account for the degeneracy of angles
    separated by $\pi$ radians for spin-2 shear systematics.  An AngularSpread of 0 indicates a
    maximally anisotropic distribution (all visits have the rotSkyPos angle mod $\pi$), while an
    AngularSpread of 1 indicates that visits are maximally balanced (the fraction of angles
    rotSkyPos at $\theta$ and at $\theta + \pi/2$ are the same.)  For complete definition of the
    AngularSpread metric, please see the text.  To leading order, shear systematics permanently
    imprinted on the camera or lenses cancel when AngularSpread = 1.  \textbf{Right:} Distribution
    of the AngularSpread metric applied to (rotSkyPos mod $\pi$) for all LSST filters.}
\label{fig:WL_AngularSpread_rotSkyPos}
\end{figure}

\begin{figure}[tbh!]
\plottwo{figs/WL/minion_1016_Kuiper_rotSkyPos_propID_54_and_i_HEAL_SkyMap.pdf}
        {figs/WL/minion_1016_Kuiper_rotSkyPos_u_g_r_i_z_y_propID_54_HEAL_ComboHistogram.pdf}
\caption{\textbf{Left:} Sky map showing the distribution of the Kuiper metric (see text for
    definition) applied to the angle rotSkyPos mod $\pi$.  A Kuiper value of 0 indicates an
    isotropic distribution of angles (mod $\pi$), while a Kuiper value of 1 indicates a maximally
    anisotropic distribution. \textbf{Right:} Distribution of the Kuiper metric applied to
    (rotSkyPos mod $\pi$) for all LSST filters.}
\label{fig:WL_Kuiper_rotSkyPos}
\end{figure}

%  ParallacticAngle metrics

\begin{figure}[tbh!]
\plottwo{figs/WL/minion_1016_AngularSpread_ParallacticAngle_propID_54_and_i_HEAL_SkyMap.pdf}
        {figs/WL/minion_1016_AngularSpread_ParallacticAngle_u_g_r_i_z_y_propID_54_HEAL_ComboHistogram.pdf}
\caption{Same as Fig. \ref{fig:WL_AngularSpread_rotSkyPos}, but for the parallactic angle (the angle
    between North and zenith) instead of rotSkyPos.  The isotropy of the parallactic angle affects
    the impact of shear systematics due to differential chromatic refraction.}
\label{fig:WL_AngularSpread_ParallacticAngle}
\end{figure}

\begin{figure}[tbh!]
\plottwo{figs/WL/minion_1016_Kuiper_ParallacticAngle_propID_54_and_i_HEAL_SkyMap.pdf}
        {figs/WL/minion_1016_Kuiper_ParallacticAngle_u_g_r_i_z_y_propID_54_HEAL_ComboHistogram.pdf}
\caption{Same as Fig. \ref{fig:WL_Kuiper_rotSkyPos}, but for the parallactic angle instead of
    rotSkyPos.}
\label{fig:WL_Kuiper_ParallacticAngle}
\end{figure}


\subsection{Discussion}

The RotSkyPos metric analysis shows that the majority of fields have a
good randomization of detector angles projected on the sky.

There are some limitations to this observation.

%First, we do not have at present a quantitative requirement for
%randomization of this parameter.  In future development of weak
%lensing analysis, a criterion should be developed.

A significant fraction of fields  have median values that are
lower or higher than expected for a random distribution, with some far
from uniformly distributed.  Regardless of the $per field$ criterion,
it is desirable to avoid the incidence of individual discrepant
fields.

The recommended criterion for randomization of RotSkyPos is not the
behavior of the majority of the fields, but of the minority with the
least random behavior.  The number of non-random fields should be
minimized.  A recommended metric is the count of fields with median
RMS less then 0.8 or greater than 1.5 radians (these values to be
reviewed again as additional experience is gained with additional
OpSim schedule simulations and weak lensing analysis.)

It is certain that actively controlling the statistics of RotSkyPos
will require additional slewing of the camera rotator.  At present,
the operations plan is to only slew when necessary to prepare for a
filter change - that could be estimated at the equivalent of $\simeq
3$ complete rotations per day.  \autoref{RotSkyPos} shows that to
render the distribution completely uniform would require moving all
observing angles an average of $\simeq 30$ degrees, or 300 complete
rotations per night.  The timing of this has not been considered.
Whether or not this uniformity could be achieved with less slew time
if implemented in scheduling remains to be demonstrated.

A similar metric for RotTelPos should be developed.


% --------------------------------------------------------------------

% ====================================================================
%+
% SECTION NAME:
%    photoz.tex
%
% CHAPTER:
%    cosmology.tex
%
% ELEVATOR PITCH:
%    Photometric redshifts are an intermediate data product that comprises
%    a key input for many investigations of galaxies and cosmology.  They
%    represent "static science", but we need them to have high quality after
%    the first year and at each "data release" thereafter.
%
% COMMENTS:
%    Updated Wed May 18 by MLG.
%    Minor updates to figures and captions, Sep 14 by MLG.
%
% BUGS:
%
%
% AUTHORS:
%   Melissa Graham, Sam Schmidt, Andy Connolly, Zeljko Ivezic
%-
% ====================================================================
\clearpage
\section{Photometric Redshifts}
\def\secname{photoz}\label{sec:\secname}

\credit{MelissaGraham},
\credit{SamSchmidt},
\credit{connolly},
\credit{ivezic}

\subsection{Introduction}

Photometric redshifts are an essential part of
every cosmology probe within LSST.  The principal concern for LSST
photo-$z$ performance is to meet the stringent requirements on redshift
uncertainty, bias, and catastrophic outlier rate as laid out in the
Science Requirements document. Photo-$z$'s are dependent on precise
measurements of galaxy colors, thus cadence and depth variations must be
examined as a function of all six LSST filter bandpasses.  Overall image
depth and signal-to-noise is our primary concern. For studies of Large
Scale Structure, Weak Lensing, Clusters, and Supernova host galaxies,
survey uniformity is desired for the full depth survey, while the
temporal details of how we reach full depth are not as important as
uniformity both as a function of sky position and observing conditions.
However, as we desire science-grade photometric redshifts after one year
of operations, two years, and so forth, the cadence must meet some basic
requirements for the six-band system at least on the timescales of the
yearly data releases.


\textbf{Specifications.} The Science Requirements Document (SRD) defines
the minimum statistical specifications for photometric redshifts for an
$i<25$, magnitude-limited sample of $4\times10^9$ galaxies from
$0.3<z<3.0$ as: (1) the root-mean-square ($\sigma$) error in photo-$z$,
divided by $1+z$, must be $\sigma < 0.02$; (2) the fraction of
``catastrophic" outliers (defined as those with errors exceeding the
larger of 0.06 or 3$\sigma$) must be $<10\%$; and (3) the average bias
must be $\overline{z_{\rm true} - z_{\rm phot}} < 0.003$. With this in
mind, we are developing software to show that our photo-$z$ algorithms
can meet specifications for LSST baseline parameters and to simulate the
impact of deviations from the 10-year baseline plan on photo-$z$
statistics.

\textbf{Planned Experiments.} This software is designed to allow the
user to modify LSST baseline parameters, simulate a set of test galaxy
observations (i.e., magnitudes with errors appropriate for the given
LSST parameters) from a training catalog with ``true" magnitudes and
redshifts and a realistic intrinsic dispersion in color, magnitude, and
redshift, run a photometric redshift algorithm on the test galaxies
(i.e., matching in color-space to the training catalog), and output
statistics for analysis. Modifiable LSST input parameters will include:
the limiting magnitude applied to the galaxy catalogs (e.g., $i<25$);
the number of visits per filter; the number of years of LSST
observations that have passed (this can be a fraction of a year);
systematic offsets to the magnitudes in each filter (default $=0$); and
coefficients for the magnitude uncertainties in each filter (default
$=1$).  Output for user analysis will include catalogs of $z_{\rm true}$
$vs.$ $z_{\rm phot}$ and the aforementioned statistics on the photo-$z$
in any desired redshift range. For example, we will be able to vary the
total number of $u$-band visits and examine how this affects the
fraction of outliers at 1, 5, and 10-years of the survey. In this
software, parameters of the photo-$z$ algorithm itself will also be
modifiable, allowing us to test options in the algorithms against
various LSST observing strategies.

\textbf{Currently implemented photo-$z$ algorithm.} We draw $N_{\rm
test}$ ``test" galaxies from the training catalog, determine their
magnitude uncertainties as appropriate for the LSST parameters, randomly
scatter their magnitudes to induce an observational error, and calculate
the associated colors and color errors. We calculate the Mahalanobis
distance in color space between each test galaxy and all training
catalog galaxies, and identify a color-matched subset using a threshold
defined by the $\chi^2$ percentage point function at 95\%. We draw a
random color-matched training galaxy and use its redshift as the
photo-$z$ for that test galaxy. We then calculate our statistical
metrics on the photometric redshifts for the test sample, using each
test galaxy's original catalog redshift as the ``true'' redshift. This
process is open to substituting alternate photometric redshift
algorithms, a variety of galaxy catalogs, and/or adding priors based on
e.g., apparent magnitude.


\subsection{Metrics}

The primary metrics we will use to evaluate LSST
observing strategies with respect to the SRD photo-z specifications are the
standard deviation, bias, and fraction of outliers. For all test
galaxies we calculate $\Delta z = (z_{\rm true} - z_{\rm phot}) /
(1+z_{\rm true})$, and identify galaxies in the interquartile range of
$\Delta z$. For these interquartile galaxies we determine the standard
deviation, $\sigma$, of the $\Delta z$ distribution, and the bias as the
median value of $\Delta z$. The interquartile range is used to exclude
outliers from influencing these statistics; in other words, they
represent the standard deviation and bias for the subset of ``good"
photo-$z$'s. The ``catastrophic" outliers are identified as those with
$\Delta z$ exceeding the larger of 0.06 or 3$\sigma$.


\subsection{Initial Results}

To demonstrate this software with a
preliminary analysis, we apply the currently implemented photo-$z$
algorithm to a catalog of galaxies that was originally created as an
accurate cosmological simulation for Euclid. We first cull the catalog
to galaxies with $i$-band magnitude $<25.3$, and then randomly draw
training and test galaxy samples (with at least a 4$\times$ more
galaxies in the training sample than the test sample). For this
demonstration we show how the photo-$z$ metrics evolve with respect to
two of the basic LSST parameters: the year of the survey, and the number
of $u$-band visits. When we simulate results in a given year of LSST, we
assume uniform progression in all filters (i.e., the total number of
visits per filter, \texttt{[56, 80, 184, 184, 160, 160]} in
\texttt{[u,g,r,i,z,y]}, is distributed evenly over all years). When we
simulate the LSST 10-year results for a given number of $u$-band visits,
the visits removed/added to $u$-band are added/subtracted evenly to/from
the other five filters. The results of these tests are presented in
Figure~\ref{fig:redshifts} and~\ref{fig:metrics}. For example, in this
demonstration we can see that the $u$-band is necessary to limit scatter
at $z<0.5$ and $z>2.0$.

\begin{figure}[h]
\begin{center}
\includegraphics[width=5cm]{figs/photoz/nyears_cat05.png}
\includegraphics[width=5cm]{figs/photoz/nyears_cat20.png}
\includegraphics[width=5cm]{figs/photoz/nyears_cat100.png}
\includegraphics[width=5cm]{figs/photoz/uvisits_cat1.png}
\includegraphics[width=5cm]{figs/photoz/uvisits_cat4.png}
\includegraphics[width=5cm]{figs/photoz/uvisits_cat6.png}
\caption{Photometric vs. spectroscopic (i.e., catalog truth) redshifts
for our preliminary simulations. Across the top row we show results from
0.5, 2.0 and 10.0 years of the LSST survey using catalogs with 10000 test
and 40000 training galaxies. The photo-$z$'s clearly improve with time
as the survey progresses. Across the bottom row we show results for $1$,
$56$ (baseline), and $96$ $u$-band visits, also using catalogs with 10000 test
and 40000 training galaxies. Between the left-most and middle plot of
the bottom row, representing 1 and 56 (baseline) $u$-band visits
respectively, we see that $u$-band data is necessary to limit scatter in
the photo-$z$'s, especially at $z<0.5$ and $z>2.0$.
\label{fig:redshifts}}
\end{center}
\end{figure}

\begin{figure}[h]
\begin{center}
\includegraphics[width=5cm]{figs/photoz/nyears_IQR.png}
\includegraphics[width=5cm]{figs/photoz/nyears_fout.png}
\includegraphics[width=5cm]{figs/photoz/nyears_bias.png}
\includegraphics[width=5cm]{figs/photoz/uvisits_IQR.png}
\includegraphics[width=5cm]{figs/photoz/uvisits_fout.png}
\includegraphics[width=5cm]{figs/photoz/uvisits_bias.png}
\caption{Three photo-$z$ metrics as a function of LSST parameters. From
left to right, the y-axis is the standard deviation, the fraction of
outliers, and the bias. The top row shows these statistics as a function
of the number of years of LSST survey, and the bottom row shows them as
a function of the number of $u$-band visits. Colors show these relations
for four bins in redshift: 0.3--3.0 (black), 0.0--0.4 (blue), 0.8--1.0
(green), and 2.0--3.0 (red). Dashed lines mark the SRD specification for
each metric. The fraction of catastrophic outliers appears erroneously
low early in the LSST survey (top middle plot) when the dispersion in
$z_{\rm spec}-z_{\rm phot}$ is too large to adequately identify outliers
(e.g., see top left plot of Figure \ref{fig:redshifts}).
\label{fig:metrics}}
\end{center}
\end{figure}


\subsection{Discussion}

\textbf{Additional considerations for observing strategy.} As mentioned
above, overall image depth and signal-to-noise is our primary concern,
so we are not testing changes in e.g., the inter-night gap time or the
exposure time of individual visits.  Our software is instead focused on
modifying other LSST parameters such as systematic offsets to the
magnitudes in each filter and/or coefficients for the magnitude
uncertainties in each filter in order to simulate improvements or
degradations the system throughput, sky background brightness, and other
such factors. We also aim to test airmass distributions (i.e., changes
to the effective filter functions), different progression rates for
filters (e.g., a scenario in which we complete all $u$-band by year 2),
scenarios in which some areas of sky have better/worse coverage at any
given time, and so forth. In all respects we are open to suggestions
from the community.

\textbf{Considerations for building the real training catalog.} All
photometric redshift algorithms require training set data consisting of
objects with secure spectroscopic redshifts.  For LSST, many of these
will be contained in a small number of training/calibration fields (e.g.
COSMOS, VVDS).  Imaging these fields to full depth in all six bands
early in the survey (but under the range of observing conditions
expected for the ten year survey) will be key to characterizing
performance.  Inclusion of these patches of full-depth imaging must be
included in any cadence design. Future simulations of photo-$z$ results
can include varying the quality of the training catalog obtained by
LSST.

\textbf{Integration with MAF.} One way to extend our program to be able
to evaluate observing strategies simulated with \OpSim could be to use
the MAF to enable us to simulate representative samples of galaxies
across the mock LSST sky, and compute the metrics we have defined.
It may be possible to avoid such a large computation by first defining
some intermediate diagnostic metrics, such as the $u$-band coverage, and
working out how our higher level metrics depend on them, using some
approximate interpolation formulae.

\textbf{Connecting to the Dark Energy Figure of Merit.} The metrics we
have defined here should be able to be related to the DETF Figure of
Merit, but because photo-zs affect all of the LSS, WL and CL
cosmological probes, this step may need to wait until a joint
Figure of Merit MAF metric is developed.

\navigationbar

% ====================================================================


% --------------------------------------------------------------------

% ====================================================================
%+
% SECTION:
%    section-name.tex  % eg lenstimedelays.tex
%
% CHAPTER:
%    chapter.tex  % eg cosmology.tex
%
% ELEVATOR PITCH:
%    Explain in a few sentences what the relevant discovery or
%    measurement is going to be discussed, and what will be important
%    about it. This is for the browsing reader to get a quick feel
%    for what this section is about.
%
% COMMENTS:
%
%
% BUGS:
%
%
% AUTHORS:
%    Phil Marshall (@drphilmarshall)  - put your name and GitHub username here!
%-
% ====================================================================

\section{Supernova Cosmology and Physics}
\def\secname{supernovae}\label{sec:\secname}
% \label{sec:cosmology, supernovae, classification, lenstimedelays, deepdrillingfields }

\noindent{\it Jeonghee Rho, Michelle Lochner, Rahul Biswas} % (Writing team)

% This individual section will need to describe the particular
% discoveries and measurements that are being targeted in this section's
% science case. It will be helpful to think of a ``science case" as a
% ``science project" that the authors {\it actually plan to do}. Then,
% the sections can follow the tried and tested format of an observing
% proposal: a brief description of the investigation, with references,
% followed by a technical feasibility piece. This latter part will need
% to be quantified using the MAF framework, via a set of metrics that
% need to be computed for any given observing strategy to quantify its
% impact on the described science case. Ideally, these metrics would be
% combined in a well-motivated figure of merit. The section can conclude
% with a discussion of any risks that have been identified, and how
% these could be mitigated.

This section is concerned with the detection, characterization of supernovae 
over time using the Large Synoptic Sky Telescope (LSST) and the use of these
supernovae for a number of science applications. The most important science 
application is the use of supernovae Type Ia (SNIa) and potentially some core-colapse SN (like Type IIP) to trace the recent expansion history of the universe,
and confront models of the physics driving the late time accelerated expansion
of the universe. 

This objective of supernova cosmology follows (at least for SNIa) several
highly successful surveys; improvement in that knowledge could come from
substantially larger numbers of well-characterized supernovae and potentially
useful redshift distributions of such detected supernovae. In this sense, this
goal is not directly tied to the large survey area that is an unprecedented
characteristic of the LSST. However, we shall argue that in practice, even this
goal would be largely helped by the spatial scale offered by the Wide Fast Deep
(WFD) component of the LSST. 

On the other hand, the WFD component of the LSST survey is potentially the 
first single survey to scan supernovae across the very large area of the
entire Southern sky. Therefore, supernovae detected and well characterized
(a) probing the isotropy of the universe, or (b) using peculiar velocities of 
supernovae to probe the growth of structure and finally (c) this may be the
best avenue for a highly complete sample of supernovae that will enable further
sharpening of our understanding of the properties of the supernova population 
of different types. 
This last point is extremely important for supernova cosmology goals: The success of supernova cosmology has always been based on the emperical model of intrinsic peak brightnesses being related to the certain observable characteristics of
supernovae. While the spatial location of the supernovae is not important, the 
WFD has the potential to dramatically increase the size of the sample 
available to train such an emperical model, as well understand the probability of deviations and scatter from this model. Aside from issues like calibration 
which need to be addressed differently, a larger sample size of such well measured supernovae is probably the only way to address deviations from the emperical
model usually discussed as `systematics'. This can be thought of in two 
components: the low redshift sample of supernovae which is more likely to be complete, and the higher redshift sample that might be able to constrain evolution. 
% --------------------------------------------------------------------

\subsection{Target measurements and discoveries}
\label{sec:keyword:targets}

% Describe the discoveries and measurements you want to make.

Supernovae of different types are visible over a time scale of about a few 
weeks (eg. Type Ia) to close to a year (Type IIP). During the full ten year
 survey of LSST, the telescope will scan the entire Southern Sky repeatedly
 with a universal Wide Fast Deep (WFD) Candence, and certain specific locations
of the sky called the Deep Drilling Fields (DDF) with special enhanced cadence. 

This spatio-temporal window should contain millions (RB: remember to check) of supernovae, that will have apparent magnitudes brighter than the single exposure limiting magnitude of LSST, for at least some time.  However, the actual
 sequence of observations in LSST defined by series of field pointings as a
 function of time in filter bands (along with weather conditions) will
 determine the extent to which each of such supernovae can be detected and
 characterized well.  Characterization of these supernovae is at the core of a
 number of science programs that use supernovae as bright, abundant objects with empirically determined intrinsic brightness. For LSST, this goal entails (a) detection of supernovae (b) photometric typing of supernovae, (c) estimating photometric redshifts of supernovae (or identifying host galaxies,
 and obtaining their redshifts from photometry or follow-up spectroscopy)
(c) estimation of intrinsic brightnesses of the supernovae, and finally use these data in addressing our science goals of cosmological inference, etc.
The efficacy of photometric typing, redshifts and estimation of intrinsic brightnesses are all
dependent on the amount of information available in the observed light curves of supernovae. While these steps are not necessarily independent, it is useful to think of the requirements on some of these steps separately; it is not unlikely  that combining some of these steps would still be affected by similar requirements. 

{\emph{Our first objective is to detect such supernovae}}. By `detection of supernovae', we mean a process
that detects transients from the subsets of LSST detection, and classify them as supernovae (as opposed to an AGN, or an asteroid). In brief, this process 
consists the identification of a set of image subtractions between high 
resolution `template` image of a sky section, and a set of single exposures at
different times (usually of lower resolution) of the same region, after 
attempting to correctly account for the different resolutions of images, and alignments. These sets of image subtractions associated
 with a single object will be used to detect the object as a transient and then
classify the transient as a supernova . Clearly, this step of detecting a supernova depends on the number of such images recorded per object, the number of filters and the signal to noise ratio of these images. One might expect that the efficiency of this step may be summarized as a threshold on the joint properties 
of an astrophysical candidate (apparent brightness, light curve characteristics, background) as well as observing conditions (Seeing etc.).  

{\emph{Our second objective is to photometrically classify different kinds of supernovae}} 
{\bfseries Photometric supernova classification}\\
In the past, only spectroscopically typed supernovae have been used for cosmology. Photometric 
typing from the light curve alone has only been used to select candidates for spectroscopic 
follow-up (see for example \citet{Sako2008}). However, LSST will simply produce far too many 
candidates for any chance of following up even a significant fraction of them. In order to avoid 
throwing away the majority of the supernova dataset, we need to use techniques capable of 
determining cosmological parameters from a potentially contaminated photometric supernova dataset.

There have been several techniques proposed in recent literature to solve this problem. One 
approach proposes applying stringent cuts to the photometric dataset to obtain a nearly pure sample 
of type Ia supernovae \citep{Bernstein2012,Campbell2013} and to run the standard supernova analysis 
with this sample. Another approach, BEAMS \citep{Kunz2007,Newling2011,Hlozek2012,Knights2013}, 
makes use of the full dataset, coping with contamination by using a mixture model for the 
likelihood, thus allowing for multiple populations. Whatever the technique ultimately used to for 
cosmological analysis, it will rely on accurate initial classifications of supernova type and 
unbiased estimates for the probability of each type.

The current state-of-the-art photometric classification techniques rely on fitting empirically 
determined templates of supernovae to light curves \citep{Jha2007,Guy2007,Sako2011}. However in 
recent years, new approaches have been published in response to the 2010 `Supernova 
Photometric Classification Challenge' \citep{Kessler2010a}. Many of these use novel light curve 
parameterisation and employ machine learning algorithms to perform the classification (see \citet{Kessler2010b} and references therein).

While many of these methods have been tested on standard sets of simulated data and (in some cases) 
on SDSS data, it is still not clear which technique (if any) is superior in all situations. For 
example, some techniques rely heavily on reliable redshift information being available, while others 
are less reliant on it. Some techniques may be more robust to non-representative datasets than 
others and it is not clear how the techniques will respond to changes in cadence, filter sets, SNR 
etc. With this in mind, we propose the use of a multifaceted classification system which employs 
several different methods of extracting features from the light curves (e.g. fitting parametric 
functions or templates) and several different classification algorithms. This system is highly 
modular, allowing the easy addition of new approaches for direct comparison with existing  techniques. This also allows direct analysis of different observing strategies, without having to 
make an initial choice of classification technique. 


{\emph{Our third obvective is to characterize supernovae in terms of emperical
    light curve models}}

The ultimate goal of using supernovae for a cosmology analaysis (either SNIa or SNIIP) requires an estimate of the intrinsic brightness of the supernova. The
first (and sometimes only step depending on the light curve model) to this, is
to fit the calibrated fluxes to a light curve model with a set of parameters.
According to the ansatz used in supernova cosmology, the intrinsic brightness of
 supernovae is largely determined by the parameters of the light curve model; 
 hence the uncertainties on the inferred parameters largely determine the
 uncertainties on the inferred peak intrinsic brightness or distance moduli of the supernovae.

% Now, describe their response to the observing strategy. Qualitatively,
% how will the science project be affected by the observing schedule and
% conditions? 

% In broad terms, how would we expect the observing strategy
% to be optimized for this science?





% --------------------------------------------------------------------

\subsection{Metrics}
\label{sec:keyword:metrics}

Quantifying the response via MAF metrics: definition of the metrics,
and any derived overall figure of merit.

\emph{To be added: discussion of the ROC curve as a useful metric for photometric supernova 
classification}




% --------------------------------------------------------------------

\subsection{OpSim Analysis}
\label{sec:keyword:analysis}

OpSim analysis: how good would the default observing strategy be, at
the time of writing for this science project?

As noted above the science goal of trying to characterize supernovae is largely
dependent on how well the light curves of individual supernovae are sampled in
time and filters. To study this, we reindex the opsim output on spatial
locations rather than use the temporal index. There are different methods (which will be merged), and here we will first illustrate this in terms the cadence in an example LSST field.

\begin{figure}
\includegraphics[width=\textwidth]{figs/supernova/fig_firstSeason_0}
\includegraphics[width=\textwidth]{figs/supernova/fig_firstSeason_1}
\includegraphics[width=\textwidth]{figs/supernova/fig_firstSeason_2}
\includegraphics[width=\textwidth]{figs/supernova/fig_firstSeason_3}
\includegraphics[width=\textwidth]{figs/supernova/fig_firstSeason_4}
\label{fig:opsimSummary}
\caption{Cadence in different filters for a few LSST deep drilling fields in
    the the ouptut of OpSim version Enigma 1189. This ignores issues of chip 
gaps or overlaps between LSST. These issues have been addressed in \citep{CarrollEtal2014} and Awan et.al. 2015, in preparation. We will add these to this analysis.}
\end{figure}



% --------------------------------------------------------------------

\subsection{Discussion}
\label{sec:keyword:discussion}

Discussion: what risks have been identified? What suggestions could be
made to improve this science project's figure of merit, and mitigate
the identified risks?


\begin{itemize}
\item Intinsic Dispersion, environmental effects, newer analysis methods
\item Follow-up procedures: What is feasible? Where will our training samples for classification and light curve models come from (other experiments, our own 
sub-samples with spectroscopic follow-up), spectroscopic follow-up of host galaxies. Can hosts be identified?
\item `Systematics': In what ways will the real data not match the assumptions made in analysis. Having a large sample of SN, to understand the astrophysics would be useful for this. 
\end{itemize}


% ====================================================================

\navigationbar


% --------------------------------------------------------------------

% ====================================================================
%+
% SECTION NAME:
%    \secname.tex
%
% CHAPTER:
%    cosmology.tex
%
% ELEVATOR PITCH:
%    Lensed quasars and supernovae provide distance measurements for
%    cosmology. They are a few days to a few weeks in length. To
%    measure them well we need long campaigns (>~3 years) with high
%    night-to-night cadence (better than the standard 5 days if
%    possible, especially as combining all filters might be difficult.)
%
% COMMENTS:
%
%
% BUGS:
%
%
% AUTHORS:
%   Phil Marshall (@drphilmarshall)
%-
% ====================================================================
\clearpage
\section{ Strong Gravitational Lens Time Delays }
\def\secname{lenstimedelays}\label{sec:\secname}

\noindent{\it Phil Marshall} % (Writing team)

% This individual section will need to describe the particular
% discoveries and measurements that are being targeted in this section's
% science case. It will be helpful to think of a ``science case" as a
% ``science project" that the authors {\it actually plan to do}. Then,
% the sections can follow the tried and tested format of an observing
% proposal: a brief description of the investigation, with references,
% followed by a technical feasibility piece. This latter part will need
% to be quantified using the MAF framework, via a set of metrics that
% need to be computed for any given observing strategy to quantify its
% impact on the described science case. Ideally, these metrics would be
% combined in a well-motivated figure of merit. The section can conclude
% with a discussion of any risks that have been identified, and how
% these could be mitigated.

% A short preamble goes here. What's the context for this science
% project? Where does it fit in the big picture?

The multiple images of strongly lensed quasars and supernovae have
delayed arrival times: variability in the first image will be observed
in the second image some time later, as the photons take different
paths around the deflector galaxy, and through different depths of
gravitational potential. If the lens mass distribution can be modeled
independently, using a combination of high resolution imaging of the
distorted quasar/SN host galaxy and stellar dynamics in the lens
galaxy, the measured time delays can be used to infer the``time delay
distance'' in the system. This distance enables a direct measurement
of the Hubble constant, independent of the distance ladder.

% --------------------------------------------------------------------

\subsection{Target measurements and discoveries}
\label{sec:\secname:targets}

% Describe the discoveries and measurements you want to make.

For this cosmological probe to be competitive with LSST's others, the
time delays of several hundred systems (which will be distributed
uniformly over the extragalactic sky) will need to be measured with
bias below the sub-percent level, while the precision required is a
few percent per lens.  In galaxy-scale lenses, the kind that are most
accurately modeled, these time delays are typically between several
days and several weeks long, and so are measurable in monitoring
campaigns having night-to-night cadence of between one and a few days,
and seasons lasting several months or more.

% Now, describe their response to the observing strategy.
% Qualitatively, how will the science project be affected by the
% observing schedule and conditions? In broad terms, how would we
% expect the observing strategy to be optimized for this science?

To obtain accurate as well as precise lensed quasar time delays, several monitoring seasons are required. Lensed supernova time delays have not yet been measured, but their transient nature means that their time delay measurements may be more sensitive to cadence than season or campaign length.

% --------------------------------------------------------------------

\subsection{Metrics}
\label{sec:\secname:metrics}

% Quantifying the response via MAF metrics: definition of the metrics,
% and any derived overall figure of merit.

Anticipating that the time delay accuracy would depend on night-to-night cadence, season length, and campaign length, we carried out a large scale simulation and measurement program that coarsely sampled these schedule properties. In \citet{LiaoEtal2015}, we simulated 5 different light curve datasets, each containing 1000 lenses, and presented them to the strong lensing community in a ``Time Delay Challenge.'' These 5 challenge ``rungs'' differed by their schedule properties, in the ways shown in \autoref{tab:tdcrungs}. Focusing on the best challenge submissions made by the community, we derived a simple power law model for the variation of each of the time delay accuracy, time delay precision, and useable sample fraction, with the schedule properties cadence, season length and campaign length. These models are shown in \autoref{fig:tdcresults}, reproduced from \citet{LiaoEtal2015}, and are given by the following equations:
\begin{align}
|A|_{\rm model} &\approx 0.06\% \left(\frac{\rm cad} {\rm 3 days}  \right)^{0.0}
                          \left(\frac{\rm sea}  {\rm 4 months}\right)^{-1.0}
                          \left(\frac{\rm camp}{\rm 5 years} \right)^{-1.1} \notag \\
  P_{\rm model} &\approx 4.0\% \left(\frac{\rm cad} {\rm 3 days}  \right)^{ 0.7}
                         \left(\frac{\rm sea}  {\rm 4 months}\right)^{-0.3}
                         \left(\frac{\rm camp}{\rm 5 years} \right)^{-0.6} \notag \\
  f_{\rm model} &\approx 30\% \left(\frac{\rm cad} {\rm 3 days}  \right)^{-0.4}
                        \left(\frac{\rm sea}  {\rm 4 months}\right)^{ 0.8}
                        \left(\frac{\rm camp}{\rm 5 years} \right)^{-0.2} \notag
\end{align}

%%%%%%%%%%%%%%%%%%%%%%%%%%%%%%%%%%%%
\begin{table*}
\begin{center}
\capstart
\begin{tabular}{cccccc} \hline\hline
  Rung &  Mean Cadence & Cadence Dispersion & Season   & Campaign & Length   \\
       &  (days)       & (days)             & (months) & (years)  & (epochs) \\ \hline
  0    &    3.0        &   1.0              &   8.0    &    5     & 400      \\
  1    &    3.0        &   1.0              &   4.0    &    10    & 400      \\
  2    &    3.0        &   0.0              &   4.0    &    5     & 200      \\
  3    &    3.0        &   1.0              &   4.0    &    5     & 200      \\
  4    &    6.0        &   1.0              &   4.0    &    10    & 200      \\
\hline\hline
\end{tabular}
\end{center}
\caption{The observing parameters for the five rungs of the Time Delay
Challenge. Reproduced from \citet{LiaoEtal2015}.\label{tab:tdcrungs}}
\end{table*}
%%%%%%%%%%%%%%%%%%%%%%%%%%%%%%%%%%%%

%%%%%%%%%%%%%%%%%%%%%%%%%%%%%%%%%%%
\begin{figure*}[!ht]
  \capstart
  \begin{minipage}[b]{\linewidth}
    \begin{minipage}[b]{0.32\linewidth}
      \centering\includegraphics[width=\linewidth]{figs/Accuracy_season_nca.pdf}
    \end{minipage} \hfill
    \begin{minipage}[b]{0.32\linewidth}
      \centering\includegraphics[width=\linewidth]{figs/Precision_cadence_nca.pdf}
    \end{minipage} \hfill
    \begin{minipage}[b]{0.32\linewidth}
      \centering\includegraphics[width=\linewidth]{figs/Fraction_season_nca.pdf}
    \end{minipage}
  \end{minipage}
\caption{Examples of changes in accuracy $A$ (left), precision $P$ (center) and success fraction $f$ (right) with schedule properties, as seen in the different TDC submissions. The gray
approximate power law model was derived by visual inspection of the
pyCS-SPL results; the signs of the indices were pre-determined according to our expectations. Reproduced from \citet{LiaoEtal2015}.}
\label{fig:tdcresults}
\end{figure*}
%%%%%%%%%%%%%%%%%%%%%%%%%%%%%%%%%%%

All three of these metrics would, in an ideal world, be optimized:
this could be achieved by decreasing the night-to-night cadence (to
better sample the light curves), extending the observing season length
(to maximize the chances of capturing a strong variation and its
echo), and extending the campaign length (to increase the number of
effective time delay measurements). A combined figure of merit should
therefore be readily available. The quantity of greatest scientific
interest is the accuracy in cosmological parameters: efforts to derive
such a figure of merit in terms of the Hubble constant are underway.

% --------------------------------------------------------------------

\subsection{OpSim Analysis}
\label{sec:\secname:analysis}

% OpSim analysis: how good would the default observing strategy be, at
% the time of writing for this science project?

In this section we will present the results of our OpSim analysis,
answering the question ``how good would the current default observing
strategy be for time delay lens cosmography?''

% --------------------------------------------------------------------

\subsection{Discussion}
\label{sec:\secname:discussion}

Discussion: what risks have been identified? What suggestions could be
made to improve this science project's figure of merit, and mitigate
the identified risks?


\navigationbar

% ====================================================================


% --------------------------------------------------------------------
