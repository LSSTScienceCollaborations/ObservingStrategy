% --------------------------------------------------------------------

\chapter[Cosmology]{Keeping It Even: Accurate Cosmological Measurements on the Largest Scales}
\label{chp:cosmology}

\noindent {\it
Eric Gawiser, Peter Kurczynski, Phil Marshall, Ohad Shemmer, Timo Anguita, ...
}

% --------------------------------------------------------------------


\section{Introduction}
\label{sec:cosmology:intro}

% Introduce, with a very broad brush, this chapter's science projects,
% and why it makes sense for them to be considered together.

Cosmology is one of the key science themes for which LSST was designed. Our goal is to measure cosmological parameters, such as the equation of state of dark energy, or departures from General Relativity, with sufficient accuracy to distinguish one model from another, and hence drive our theoretical understanding of how the universe works, as a whole. To do this will necessarily involve a variety of different measurements, that can act as cross-checks of each other, and break parameter degeneracies in any single one.

The  Dark Energy Science Collaboration (DESC) has identified five
different cosmological probes enabled by the LSST: weak lensing (WL),
large scale structure (LSS), type Ia supernovae (SN), strong lensing
(SL), and clusters of galaxies (CL). In all cases, the primary concern
is residual systematic error: the shapes and photometric redshifts of
galaxies, and the properties of supernova and lensed quasar light
curves, will all need to be measured with extraordinary accuracy in order for LSST's high statistical power to be properly harnessed. This accuracy will come from the abundance and heterogeneity of the individual measurements made, and the degree to which they can be modeled and understood. This latter point implies a need for uniformity in the survey, which enables powerful simplifying assumptions to be made when calibrating on the largest, cosmologically most important scales. The need for heterogeneity also implies  uniformity, in the sense that the nuisance parameters that describe the systematic effects need to be sampled over as wide a range as possible (examples include the need to sample a wide range of roll angles to minimize shape error, and observing conditions to understand photometric errors due to the changing atmosphere).

In this chapter we look at some of the key measurements planned by the Dark Energy Science Collaboration, and how they depend on the Observing Strategy.

% Anticipate the results of the chapter: summarize the results of a
% number of investigative sections, where there will be one on each
% science case.


% --------------------------------------------------------------------

% ====================================================================
%+
% SECTION NAME:
%    \secname.tex
%
% CHAPTER:
%    cosmology.tex
%
% ELEVATOR PITCH:
%    Lensed quasars and supernovae provide distance measurements for
%    cosmology. They are a few days to a few weeks in length. To
%    measure them well we need long campaigns (>~3 years) with high
%    night-to-night cadence (better than the standard 5 days if
%    possible, especially as combining all filters might be difficult.)
%
% COMMENTS:
%
%
% BUGS:
%
%
% AUTHORS:
%   Phil Marshall (@drphilmarshall)
%-
% ====================================================================
\clearpage
\section{ Strong Gravitational Lens Time Delays }
\def\secname{lenstimedelays}\label{sec:\secname}

\noindent{\it Phil Marshall} % (Writing team)

% This individual section will need to describe the particular
% discoveries and measurements that are being targeted in this section's
% science case. It will be helpful to think of a ``science case" as a
% ``science project" that the authors {\it actually plan to do}. Then,
% the sections can follow the tried and tested format of an observing
% proposal: a brief description of the investigation, with references,
% followed by a technical feasibility piece. This latter part will need
% to be quantified using the MAF framework, via a set of metrics that
% need to be computed for any given observing strategy to quantify its
% impact on the described science case. Ideally, these metrics would be
% combined in a well-motivated figure of merit. The section can conclude
% with a discussion of any risks that have been identified, and how
% these could be mitigated.

% A short preamble goes here. What's the context for this science
% project? Where does it fit in the big picture?

The multiple images of strongly lensed quasars and supernovae have
delayed arrival times: variability in the first image will be observed
in the second image some time later, as the photons take different
paths around the deflector galaxy, and through different depths of
gravitational potential. If the lens mass distribution can be modeled
independently, using a combination of high resolution imaging of the
distorted quasar/SN host galaxy and stellar dynamics in the lens
galaxy, the measured time delays can be used to infer the``time delay
distance'' in the system. This distance enables a direct measurement
of the Hubble constant, independent of the distance ladder.

% --------------------------------------------------------------------

\subsection{Target measurements and discoveries}
\label{sec:\secname:targets}

% Describe the discoveries and measurements you want to make.

For this cosmological probe to be competitive with LSST's others, the
time delays of several hundred systems (which will be distributed
uniformly over the extragalactic sky) will need to be measured with
bias below the sub-percent level, while the precision required is a
few percent per lens.  In galaxy-scale lenses, the kind that are most
accurately modeled, these time delays are typically between several
days and several weeks long, and so are measurable in monitoring
campaigns having night-to-night cadence of between one and a few days,
and seasons lasting several months or more.

% Now, describe their response to the observing strategy.
% Qualitatively, how will the science project be affected by the
% observing schedule and conditions? In broad terms, how would we
% expect the observing strategy to be optimized for this science?

To obtain accurate as well as precise lensed quasar time delays, several monitoring seasons are required. Lensed supernova time delays have not yet been measured, but their transient nature means that their time delay measurements may be more sensitive to cadence than season or campaign length.

% --------------------------------------------------------------------

\subsection{Metrics}
\label{sec:\secname:metrics}

% Quantifying the response via MAF metrics: definition of the metrics,
% and any derived overall figure of merit.

Anticipating that the time delay accuracy would depend on night-to-night cadence, season length, and campaign length, we carried out a large scale simulation and measurement program that coarsely sampled these schedule properties. In \citet{LiaoEtal2015}, we simulated 5 different light curve datasets, each containing 1000 lenses, and presented them to the strong lensing community in a ``Time Delay Challenge.'' These 5 challenge ``rungs'' differed by their schedule properties, in the ways shown in \autoref{tab:tdcrungs}. Focusing on the best challenge submissions made by the community, we derived a simple power law model for the variation of each of the time delay accuracy, time delay precision, and useable sample fraction, with the schedule properties cadence, season length and campaign length. These models are shown in \autoref{fig:tdcresults}, reproduced from \citet{LiaoEtal2015}, and are given by the following equations:
\begin{align}
|A|_{\rm model} &\approx 0.06\% \left(\frac{\rm cad} {\rm 3 days}  \right)^{0.0}
                          \left(\frac{\rm sea}  {\rm 4 months}\right)^{-1.0}
                          \left(\frac{\rm camp}{\rm 5 years} \right)^{-1.1} \notag \\
  P_{\rm model} &\approx 4.0\% \left(\frac{\rm cad} {\rm 3 days}  \right)^{ 0.7}
                         \left(\frac{\rm sea}  {\rm 4 months}\right)^{-0.3}
                         \left(\frac{\rm camp}{\rm 5 years} \right)^{-0.6} \notag \\
  f_{\rm model} &\approx 30\% \left(\frac{\rm cad} {\rm 3 days}  \right)^{-0.4}
                        \left(\frac{\rm sea}  {\rm 4 months}\right)^{ 0.8}
                        \left(\frac{\rm camp}{\rm 5 years} \right)^{-0.2} \notag
\end{align}

%%%%%%%%%%%%%%%%%%%%%%%%%%%%%%%%%%%%
\begin{table*}
\begin{center}
\capstart
\begin{tabular}{cccccc} \hline\hline
  Rung &  Mean Cadence & Cadence Dispersion & Season   & Campaign & Length   \\
       &  (days)       & (days)             & (months) & (years)  & (epochs) \\ \hline
  0    &    3.0        &   1.0              &   8.0    &    5     & 400      \\
  1    &    3.0        &   1.0              &   4.0    &    10    & 400      \\
  2    &    3.0        &   0.0              &   4.0    &    5     & 200      \\
  3    &    3.0        &   1.0              &   4.0    &    5     & 200      \\
  4    &    6.0        &   1.0              &   4.0    &    10    & 200      \\
\hline\hline
\end{tabular}
\end{center}
\caption{The observing parameters for the five rungs of the Time Delay
Challenge. Reproduced from \citet{LiaoEtal2015}.\label{tab:tdcrungs}}
\end{table*}
%%%%%%%%%%%%%%%%%%%%%%%%%%%%%%%%%%%%

%%%%%%%%%%%%%%%%%%%%%%%%%%%%%%%%%%%
\begin{figure*}[!ht]
  \capstart
  \begin{minipage}[b]{\linewidth}
    \begin{minipage}[b]{0.32\linewidth}
      \centering\includegraphics[width=\linewidth]{figs/Accuracy_season_nca.pdf}
    \end{minipage} \hfill
    \begin{minipage}[b]{0.32\linewidth}
      \centering\includegraphics[width=\linewidth]{figs/Precision_cadence_nca.pdf}
    \end{minipage} \hfill
    \begin{minipage}[b]{0.32\linewidth}
      \centering\includegraphics[width=\linewidth]{figs/Fraction_season_nca.pdf}
    \end{minipage}
  \end{minipage}
\caption{Examples of changes in accuracy $A$ (left), precision $P$ (center) and success fraction $f$ (right) with schedule properties, as seen in the different TDC submissions. The gray
approximate power law model was derived by visual inspection of the
pyCS-SPL results; the signs of the indices were pre-determined according to our expectations. Reproduced from \citet{LiaoEtal2015}.}
\label{fig:tdcresults}
\end{figure*}
%%%%%%%%%%%%%%%%%%%%%%%%%%%%%%%%%%%

All three of these metrics would, in an ideal world, be optimized:
this could be achieved by decreasing the night-to-night cadence (to
better sample the light curves), extending the observing season length
(to maximize the chances of capturing a strong variation and its
echo), and extending the campaign length (to increase the number of
effective time delay measurements). A combined figure of merit should
therefore be readily available. The quantity of greatest scientific
interest is the accuracy in cosmological parameters: efforts to derive
such a figure of merit in terms of the Hubble constant are underway.

% --------------------------------------------------------------------

\subsection{OpSim Analysis}
\label{sec:\secname:analysis}

% OpSim analysis: how good would the default observing strategy be, at
% the time of writing for this science project?

In this section we will present the results of our OpSim analysis,
answering the question ``how good would the current default observing
strategy be for time delay lens cosmography?''

% --------------------------------------------------------------------

\subsection{Discussion}
\label{sec:\secname:discussion}

Discussion: what risks have been identified? What suggestions could be
made to improve this science project's figure of merit, and mitigate
the identified risks?


\navigationbar

% ====================================================================


% --------------------------------------------------------------------

% % ====================================================================
%+
% SECTION NAME:
%    dithering.tex
%
% CHAPTER:
%    cosmology.tex
%
% ELEVATOR PITCH:
%    Large Scale Structure, Weak Lensing, and Clusters all require
% survey uniformity in the static 10-year survey.  A key contributor to 
%this is the pattern of dithers adopted.  
%
% COMMENTS:
%
%
% BUGS:
%
%
% AUTHORS:
%   Eric Gawiser (@egawiser)
%-
% ====================================================================
\clearpage
\section{Dithering Patterns and Timescales}
\def\secname{dithering}\label{sec:\secname}

\noindent{\it Humna Awan, Eric Gawiser, Peter Kurczynski, Lynne Jones} % (Writing team)

% This individual section will need to describe the particular
% discoveries and measurements that are being targeted in this section's
% science case. It will be helpful to think of a ``science case" as a
% ``science project" that the authors {\it actually plan to do}. Then,
% the sections can follow the tried and tested format of an observing
% proposal: a brief description of the investigation, with references,
% followed by a technical feasibility piece. This latter part will need
% to be quantified using the MAF framework, via a set of metrics that
% need to be computed for any given observing strategy to quantify its
% impact on the described science case. Ideally, these metrics would be
% combined in a well-motivated figure of merit. The section can conclude
% with a discussion of any risks that have been identified, and how
% these could be mitigated.

% A short preamble goes here. What's the context for this science
% project? Where does it fit in the big picture?

Three of the key cosmology probes available with LSST represent ``static science'' insensitive to time-domain concerns.  These are Weak Lensing, Large-Scale Structure, and Galaxy Clusters.  Nonetheless, due to the need to track and correct for the survey ``window function'' in all of these probes, cosmology with LSST will benefit greatly from achieving survey depth as uniform as possible over the WFD area.  OpSim tiles the sky in hexagons inscribed within the nearly-circular LSST field-of-view.  It has been shown in \citet{CarrollEtal2014} that the default LSST survey strategy implemented in OpSim runs leads to a strongly non-uniform ``honeycomb'' pattern due to overlapping regions on the edges of these hexagons receiving double the observing time.  A pattern of large dithers proves sufficient to greatly reduce these overlaps, leading to an increase in median survey depth in each filter of 0.08 magnitudes.  

In this section, we report results from an investigation by Awan et al. (in preparation) of several geometrical patterns for dithers performed on timescales varying from once per observing season to once per night to every visit.  

\todo{EG}{Flesh out WL, LSS, and Clusters dependence on survey uniformity to make this section more clearly science-driven.}  

% --------------------------------------------------------------------

\subsection{Dithering Patterns and Timescales}
\label{sec:\secname:strategies}


% --------------------------------------------------------------------

\subsection{Metrics}
\label{sec:\secname:metrics}

% Quantifying the response via MAF metrics: definition of the metrics,
% and any derived overall figure of merit.

Our primary metric is total uncertainty in the derived window function over relevant angular scales, modeled via variations in the angular power spectrum of fake galaxy fluctuations between $gri$ bands.  
Intermediate metrics include the number of galaxies in 
each pixel, fluctuations in this number, total power in the angular power spectrum of a skymap of those fluctuations, and residual power that angular power spectrum after subtracting a smooth fit to it.  



% --------------------------------------------------------------------

\subsection{OpSim Analysis}
\label{sec:\secname:analysis}

% OpSim analysis: how good would the default observing strategy be, at
% the time of writing for this science project?

In this section we present our ongoing \OpSim / MAF
analysis, as we try to
answer the question ``what dithering strategies produce acceptable variations in survey uniformity, and which appears optimal?''

%We used the
%\simsMAFcontrib{SeasonStacker}{mafContrib/seasonStacker.py} to work
%with seasons.

%We used \texttt{ops2\_1075} for most of our tests, but we need to now
%re-run on \opsimdbref{db:enigma}, and others from \autoref{chp:cadence2015}.


%\citeauthor{LiaoEtal2015}). These sky maps show that, over the main

%\autoref{tab:lenstimedelays:results} shows the global (i.e. al-sky)


%--------------------------------------------------------------------

\subsection{Results}
\label{sec:\secname:results}

%%%%%%%%%%%%%%%%%%%%%%%%%%%%%%%%%%%
\begin{figure*}[!ht]
  \capstart
  \begin{minipage}[b]{\linewidth}
    \begin{minipage}[b]{0.32\linewidth}
      \centering\includegraphics[width=\linewidth]{figs/Accuracy_season_nca.pdf}
    \end{minipage} \hfill
    \begin{minipage}[b]{0.32\linewidth}
      \centering\includegraphics[width=\linewidth]{figs/Precision_cadence_nca.pdf}
    \end{minipage} \hfill
    \begin{minipage}[b]{0.32\linewidth}
      \centering\includegraphics[width=\linewidth]{figs/Fraction_season_nca.pdf}
    \end{minipage}
  \end{minipage}
\caption{Examples of changes in accuracy $A$ (left), precision $P$ (center) and success fraction $f$ (right) with schedule properties, as seen in the different TDC submissions. The gray
approximate power law model was derived by visual inspection of the
pyCS-SPL results; the signs of the indices were pre-determined according to our expectations. Reproduced from \citet{LiaoEtal2015}.}
\label{fig:tdcresults}
\end{figure*}
%%%%%%%%%%%%%%%%%%%%%%%%%%%%%%%%%%%


\todo{EG}{Improve figures to originals rather than screen-captures.}

\todo{EG}{Input fuller results and text from Awan et al. draft.}  

%---------------------------------------------------------------------

\subsection{Discussion}
\label{sec:\secname:discussion}



\navigationbar

% ====================================================================


% --------------------------------------------------------------------

% % ====================================================================
%+
%
% SECTION NAME:
%    \secname.tex
%
% CHAPTER:
%    ???.tex
%
%
% COMMENTS:
%
%
% BUGS:
%
%
% AUTHORS:
%   Ohad Shemmer (@ohadshemmer), Timo Anguita (@tanguita), Niel Brandt, Gordon Richards, Scott Anderson(?),
%   Phil Marshall(?) (@drphilmarshall)
%-
% ====================================================================
\clearpage
\section{AGN Science}
\def\secname{agn}\label{sec:\secname}

\noindent{\it Ohad Shemmer, Timo Anguita, Niel Brandt, Gordon Richards, Scott Anderson(?), Phil Marshall(?)}

% This section discusses the potential effects of the LSST observing strategy on AGN science. In short, there appears to be
% a consensus among the AGN and galaxies communities that AGN science will benefit from the most uniform cadence in
% terms of even sampling for each band and uniform sky coverage. It is also expected that any reasonable
% perturbation to the nominal LSST observing strategy will have mostly minor effects on AGN science. This section attempts
% to identify all the areas of AGN science that may be affected by the observing strategy and to point out the metrics that
% can be used to quantify any potential effect. Since the total number of metrics that must be quantified is quite large, and
% the effects are likely small in most cases, the goal of this section is to identify potential ``killers'' that may undermine
% key AGN research areas. For example, certain perturbations may reduce significantly the number of ``interesting'' AGNs,
% such as $z>6$ quasars, lensed quasars, or transient AGNs. Another example is photometric reverberation mapping
% which is one of LSST's greatest advantages for AGN research but is also very sensitive to the cadence; care must
% be taken to ensure that the observing strategy does not undermine the ability to make the best use of this method.

\subsection{AGN Selection and Census}
\label{sec:\secname:selection}

\noindent About $10^7 - 10^8$ AGNs will be selected in the main LSST survey using a combination of criteria, split
broadly into four categories: colors, astrometry, variability, and multiwavelength matching with other surveys.
The LSST observing strategy will affect mostly the first three of these categories.

{\bf Colors:}~The LSST observing strategy will determine the depth in each band, as a function of position on the sky, and will thus affect
the color selection of AGNs. This will eventually determine the AGN $L-z$ distribution and, in particular, may affect the identification
of quasars at $z\gtsim 6$ if, for example, $Y$-band exposures will not be sufficiently deep.

{\bf Variability:} AGNs can be effectively distinguished from (variable) stars, and from quiescent galaxies, by exhibiting certain characteristic variability patterns (e.g., \citet{ButlerandBloom2011}). Non-uniform sampling may ``contaminate'' the variability signal of AGN candidates.

{\bf Astrometry:} AGNs will be selected among sources having zero proper motion, within the uncertainties. The LSST cadence
may affect the level of this uncertainty in each band, and may therefore affect the ability to identify (mostly fainter) AGNs.
%
Differential chromatic refraction (DCR), making use of the astrometric offset a source with emission lines has with respect to
a source with a featureless power-law spectrum, can help in the selection of AGNs and in confirming their photometric redshifts \citep{KaczmarczikEtal2009}. The DCR effect is more pronounced at higher airmasses. AGN selection and photometric redshift confirmation may be affected since the LSST cadence will affect the airmass distribution, in each band, for each AGN candidate.

\subsection{AGN Clustering}
\label{sec:\secname:clustering}

\noindent Measurements of the spatial clustering of AGNs with respect to those of quiescent galaxies can provide clues as to how galaxies
form inside their dark-matter halos and what causes the growth of their supermassive black holes (SMBHs). The impressive inventory
of LSST AGNs will enable the clustering, and thus the host galaxy halo mass, to be determined over the widest range ranges of cosmic
epoch and accretion power.
%
The LSST cadence will not only affect the overall AGN census and its $L-z$ distribution, but also the
depth in each band as a function of sky position that can directly affect the clustering signal.

\subsection{AGNs and the Time Domain}
\label{sec:\secname:time}

{\bf AGN Variability:} A variety of AGN variability studies will be enabled by LSST. These are intended to probe the physical properties of the unresolved inner regions of the central engine. Relations will be sought between variability amplitude and timescale vs. $L$, $z$, $\lambda_{\rm eff}$, color, multiwavelength and spectroscopic properties, if available. The LSST sampling is expected to provide high-quality power spectral density functions for a large number of AGNs; these can be used to constrain the SMBH mass and accretion rate/mode. Furthermore, LSST AGNs exhibiting excess variability over that expected from their luminosities will be further scrutinized as candidates for lensed systems having unresolved images with the excess (extrinsic) variability being attributed mainly to microlensing.

Photometric reverberation mapping (PRM), measuring the time-delayed response of either the flux of the broad emission line region (BELR) lines to the flux of the AGN continuum or between the continuum flux in one (longer wavelength) band to the continuum flux in another (band with shorter wavelength), will be one of the cornerstones of AGN research in the LSST era
(e.g., \citet{Chelouche2013}; \citet{CheloucheandZucker2013}; \citet{CheloucheEtal2014}). For example, LSST is expected to deliver BELR line-continuum time delays in $\sim10^5-10^6$ sources, which is unprecedented when compared to $\sim50-100$ such measurements conducted via the traditional, yet much more expensive (per source) spectroscopic method. Sources in the deep-drilling fields (DDFs) will benefit from the highest quality PRM
time-delay measurements given the factor of $\sim10$ denser sampling. The PRM measurements will probe the size and structure of the accretion disk and BELR, in a statistical sense, and may provide improved SMBH mass estimates for sources that have at least single-epoch spectra.

The PRM method is very sensitive to the sampling in each band, therefore the ability to derive reliable time delays can be affected significantly
by the LSST cadence. The best results will be obtained by having the most uniform sampling equally for each band. Additionally, there is
a trade-off between the number of DDFs and the number of time delays that PRM can obtain \citep{CheloucheEtal2014}. For example,
an increase in the number of DDFs, with similarly dense sampling in each field, can yield a proportionately larger number of high-quality time delays,
down to lower luminosities, but at the expense of far fewer time delays (of relatively high luminosity sources) in the main survey.

{\bf Time Delays in Gravitationally Lensed Quasars:} This aspect is discussed in detail in the
lens time delays section (\autoref{sec:lenstimedelays}).

{\bf AGN Size and Structure with Microlensing:} Microlensing due to stars projected on top of individual lensed quasar images produce additional magnification. Using the fact that the Einstein radii of stars in lensing galaxies closely match the scales of different emission regions in high-redshift AGNs (micro-arcseconds), analyzing microlensing induced flux variations statistically on individual systems allows us to measure ``sizes'' of AGN regions.
%
Assuming a thermal profile for accretion disks, sizes in different emission wavelengths will be probed and as such, constraints on the slope of this thermal profile. Given the sheer number of lensed systems that LSST is expected to discover ($\sim8000$), this will allow us to stack systems for better constraints and hopefully determine the evolution of the size and profile. Due to the typical relative velocities of lenses, microlenses, observers (Earth) and source AGN, the microlensing variation timescales are between months to a few decades.

The quasar microlensing optical depth is $\sim1$, so every lensed quasar should be affected by microlensing at any given point in time. However, measurable variability can occur on longer timescales. \citet{MosqueraandKochanek2011} did a study using all known lensed quasars. They found the median timescale between high magnification events (Einstein crossing time scales) in the observed $I$-band is of the order of $\sim20$~yr (with a distribution between 10 and 40~yr). However, the source crossing time (duration of a high magnification event) is $\sim7.3$~months (with a distribution tail up to 3~yr). This basically means that out of all the lensed quasar {\em images} (microlensing between images is completely uncorrelated) about half of them will be quiescent during the 10~yr baseline of LSST. However, since the typical number of lensed images is either two or four, it means that, statistically, in every system, one (for doubles) or two (for quads) high magnification events should be observed in 10~yr of LSST monitoring.

Note that, the important cadence parameter is the source crossing time, as it is the length of the event to be as uniformly sampled as possible. The 7.3 months crossing time is the median for the observed $i$-band, but this time would be significantly shorter for bluer bands: for a thermal profile with slope $\alpha: R_\lambda \propto \lambda^\alpha$ implies source crossing time $t_{\rm s} \propto \lambda^{1/\alpha} \rightarrow t_u=t_i \times (\lambda_{\rm u} / \lambda_{\rm i})^{1/\alpha}$. For a Shakura-Sunyaev slope of $\alpha=0.75$ this would correspond to $7.3 \times (3600/8140)^{4/3}$ months $\approx 2.5$ months in the $u$-band.

In terms of the cadence, at least three evenly sampled data points per band within 2 to 3 months would be preferred to be able to map the constraining high magnification event(?). Hopefully uniformly spaced. Very tight cadence (e.g., DDFs) would increase the constraints significantly. However, since lensed quasars are not that common, this smaller area would mean only a few ($\sim80$?) suitable systems monitored in the DDFs.
%
Regarding the season length, the ``months'' timescale of high magnification events very likely means that we can/will miss high magnification events in the season gaps, at least in the bluer bands.
%
Killer: observations spread on timescales larger than 3 months(??). This would likely miss the high magnification events. In those cases we could perhaps consider close consecutive photometric bands as equivalent accretion disk regions, however this would mean weaker constraints on the thermal profile.
%
Important Note: all this science needs to be done on lensed quasars with measured or very short time delays to remove the intrinsic variability signal, which might significantly reduce the sample.

{\bf Microlensing Aided Reverberation Mapping:} Given that microlensing mostly affects continuum emission rather than BELR line emission, microlensing may enable disentangling the BELR line $+$ continuum emission in single photometric bands, allowing the use of single broad band PRM measurements \citep{SluseandTewes2014}. As with the two-band PRM method discussed above, the denser (and the longer) the sampling, the more accurate are the constraints that can be obtained for the time delays.

{\bf Transient AGN and TDEs:} This aspect is discussed in detail in the non-periodic variables section (\autoref{sec:variables}).

% --------------------------------------------------------------------

\subsection{Metrics}
\label{sec:\secname:metrics}

% Quantifying the main impacts on AGN science via MAF metrics, including the effects
% of additional cadence facto,rs such as the number of DDFs
% and MC fields, or different dithering patterns,: definition of the metrics,
% and any derived overall figure of merit.

% --------------------------------------------------------------------

\subsection{Discussion}
\label{sec:\secname:discussion}

% Discussion: what risks have been identified? What suggestions could be
% made to improve the figures of merit, and mitigate the identified risks?
% What ``tweaks'', if any, can be proposed to the nominal LSST observing strategy
% in order to help achieve key AGN science goals?

\navigationbar


% --------------------------------------------------------------------
