% --------------------------------------------------------------------

\chapter[Variable Objects]{Variable Objects}
\def\chpname{variables}\label{chp:\chpname}

Chapter editors:
\credit{AshishMahabal},
\credit{lmwalkowicz}.

% \noindent {\it
% Mike Lund, Ashish Mahabal, Stephen Ridgway,
% Lucianne Walkowicz, Rahul Biswas, Michelle Lochner,
% Jeonghee Rho, Eric Bellm...
% }

% --------------------------------------------------------------------


\section{Introduction}

Variable objects are defined as those that exhibit brightness changes, either periodic or non-periodic, detected in quiescence and which are non-destructive to the object itself. Variable objects span a wide range in timescale-of-interest (sometimes even within a single class of objects), and so different science cases benefit from different sampling strategies. These strategies may be significantly different from one another for different science cases, sometimes even mutually exclusive; competing objectives described in this chapter and the next are at the heart of LSST observing strategy and cadence design.

Below we develop a number of key science cases for LSST studies of variable objects, associating them with related metrics that can be used within the Metrics Analysis Framework (MAF) to understand the impact of a given survey strategy realization on the scientific results for that case. The science cases outlined are by no means exhaustive, but rather are motivated by providing key quantitative examples of LSST's performance given any particular deployment of survey strategy. The authors encourage community contribution of similar cases, where the scientific outcome can be quantified using specific metrics. 



%When evaluating a particular observation or series of observations in
%light of how they perform for a specific science case, it may be
%helpful to think of metrics as lying along a continuum between
%discovery and characterization. Discovery requires a minimum amount of
%information to recognize an event or object as a candidate of
%interest, which necessarily involves some level of bare-bones
%characterization (upon which said recognition is based); rich
%characterization, on the other hand, implies that an event may not
%only be recognized as a candidate of interest, but basic properties of
%the event or object may be determined from the observation (e.g.
%including but not limited to classification of the event). The
%interpretation of a given metric along this continuum has implications
%for the subsequent action and analysis required, particularly as
%regards possible follow-up observations with other facilities.

%Target types are here grouped in subsections by variability
%characteristics, but as will be seen, this does not mean that all
%targets in a group require a common cadence, since the times scales
%may vary dramatically.  Acquiring suitable data for a wide range of
%time scales presents a fundamental problem for LSST, since the
%available $~$800 visits to a field over the survey cannot be deployed
%so as to usefully sample all time scales at all times.  This fact
%leads to the concept of a non-uniform survey, in which parts of the
%sky are visited more frequently part of the time.  The merits of such
%options must be traded against the benefits of a more uniform survey
%strategy.

\subsection{Metrics Employed}
\label{sec:keyword:variablemetrics}


\subsection{Metrics}
\label{sec:keyword:metrics}

\begin{center}
\begin{tabular}{| p{5cm} |p{10cm} |}
\hline Metric & Description\\
\hline
Eclipsing binary discovery & Fraction of discoveries vs fractional duration of eclipse\\
Transiting exoplanets (depth dependent) & Fraction of discoveries vs fractional duration of eclipse\\
Phase gap & Histogram vs period of the median and maximum phase gaps achieved in all fields\\
Period determination (period dependent) & Fraction of targets vs survey duration, for which the period can be determined to 5-sigma confidence\\
Period variability (period dependent) & Fraction of targets vs survey duration, for which a period change of 1\% can be determined with 5-sigma confidence\\
  \hline \end{tabular}
 \end{center}

The period metrics can be based on a standard variable curve (e.g.sinusoid) of fiducial amplitude and brightness, and/or a realistic model population of a particular variable type. These metrics can be informative for science programs.  However, it is not clear that the survey strategy can or should attempt to control these metrics, as the requirements are specific to each target, and all targets benefit from a generally uniform distribution of visits.


Lund et al. (2015; \url{http://arxiv.org/pdf/1508.03175.pdf}) discuss three metrics that have been incorporated into the MAF. Two of these metrics deal explicitly with time variable behavior: a) observational triplets, and b) detection of periodic variability. 

\subsubsection{Periodogram purity function (PeriodicMetric)}
This metric calculates the Fourier power spectral window function of each field (Roberts et al. 1987) as a means of quantifying the completeness of phase coverage for a given periodic variable. The periodogram purity is defined as 1 minus the Fourier power spectral window function; in the perfect case, all power is concentrated in a delta function at the correct frequency, and is zero elsewhere. As power ``leaks'' away from the correct frequency as a consequence of discrete, non-ideal data sampling, the periodogram becomes more structured. For the purposes of MAF metrics, which are designed to quantify performance as a single number, the periodogram purity is quantified as the minimum value away from the correct frequency. 

\subsubsection{Phase Gap Metric (PhaseGapMetric)}
Histogram of the median and maximum phase gaps achieved in all fields

\subsubsection{Period Deviation Metric (PeriodDeviationMetric)}

This metric computes the percent deviation of recovered periods from pure sine variability.

\subsection{Proposed Metrics}

The following is a raw list of metric ideas; these need specificity and further description. 

FWHM of the window function (to quantify sampling)

Maximum hour angle difference 

Fraction of discoveries vs fractional duration of eclipse

Fraction of targets vs survey duration, for which the period can be determined to 5-sigma confidence

Fraction of targets vs survey duration, for which a period change of 1$\%$ can be determined with 5-sigma confidence


\begin{center}
\begin{tabular}{| l | p{8cm} |l | l |}
\hline Periodic Variable Type & Examples of target science & Amplitude & Timescale\\
\hline
Periodic binaries & Eclipses, physical properties of stars, distances, ages, evolution, apsidal precession, mass transfer induced period changes, Applegate effect &  small &  hr-day \\
RR Lyrae & Galactic structure, distance ladder, RR Lyrae properties&  large &  day \\
Cepheids & Distance ladder, cepheid properties&  large &  day \\
Long Period Variables & Distance ladder, LPV properties& large  &  weeks \\
Short period pulsators & Instability strip, white dwarf interior properties, evolution&  small & min  \\
Rotational Modulation & Gyrochronology, stellar activity& small  &  days \\
 \hline \end{tabular}
 \end{center}



% --------------------------------------------------------------------


% ====================================================================
%+
% SECTION:
%    periodicpulsators.tex
%
% CHAPTER:
%    variables.tex
%
% ELEVATOR PITCH:
%
%-
% ====================================================================

% \section{Discovery of Periodic Pulsating Variables}
\subsection{Discovery of Periodic Pulsating Variables}
\def\secname{periodicvariables}\label{sec:\secname}

\credit{lmwalkowicz},
\credit{StephenRidgway}

Regular variables, such as Cepheids and RR Lyraes, are valuable tracers
of Galactic structure and cosmic distance. In this case of these and
other strictly (or nearly-strictly) periodic variables, data from
different cycles of observation can be phase-folded to create a more
fully sampled lightcurve as LSST visits will occur effectively at random
phases. In a 10-year survey, most periodic stars of almost any period
will benefit from excellent phase coverage in all filters (only a very
small period range close to the sidereal day will be poorly observed).
Therefore, most implementations of the LSST observing strategy will
provide good sampling of periodic variables.

However, different implementations of the survey may result in different
resulting sample sizes of these periodic variables, and may also affect
the environments in which these stars are discovered. In this section,
we create a framework for understanding how current implementations of
the observing strategy influence (or even bias) the resultant sample
size and environments where these important tracers may be identified.

\subsubsection{Tracing Galactic Structure with RR Lyrae}

RR Lyrae variables are crucial tracers of structure in the Galaxy and beyond
into the Local Group. The incredible sample of RR Lyrae anticipated from LSST
observations will enable discovery of Galactic tidal stream and neighboring
dwarf galaxies throughout much of the Local Group
\citep{IvezicEtal2008}.  LSST also creates the possibility
of detecting and studying RRL variables in the Magellenic Clouds; see Chapter
[CHAPTER] for discussion.

\citet{2012AJ....144....9O} carried out an extensive simulation of period
and lightcurve shape recovery of RR Lyrae variables using an early OpSim run
opsim1$\_$29. Correctly identifying the period aids in building the sample of
interest, whereas fitting the lightcurve shape makes it possible to measure the
metallicity of the star. In their simulation, they employed both a Fourier
analysis and template matching to recover the lightcurve shape, finding that
template matching yielded a more accurate lightcurve shape measurement in the
presence of sparse data. The results of this simulation showed that the vast
majority of RR Lyrae will be discovered by the baseline observing strategy (as
deployed in opsim1$\_$29) within 5 years of survey operations. Half of both
RRLab and RRLc stars will be found out to $\sim$600 kpc and $\sim$250 kpc
(respectively) by the end of the 10-year main survey, and template matching
techniques for lightcurve shape recovery will provide metallicities to
$\sim$0.15dex.

% --------------------------------------------------------------------

\subsubsection{The Cepheid Cosmic Distance Ladder}

Classical cepheids remain an essential step in the cosmic distance
ladder. Their calibration is based largely on LMC cepheids and known
(assumed) distance of the LMC.  The associated errors, while uncertain,
are believed to be of $\>\sim$7\%. (Madore, Barry F.; Freedman, Wendy L.
(2009). "Concerning the Slope of the Cepheid Period?Luminosity
Relation". The Astrophysical Journal 696 (2): 1498. arXiv:0902.3747.
Bibcode:2009ApJ...696.1498M. doi:10.1088/0004-637X/696/2/1498.) New
developments in galactic studies are poised to support substantially
improved descriptive information concerning nearby galactic cepheids,
with possible substantial reductions in this error, by accurately
securing the PL slope and zero point.

Cepheid calibration errors are associated in part with uncertainties in
extinction, both interstellar and in some cases circumstellar, and in
metalicity.  At present, the direct, local calibration of cepheids is limited
by the availability of a few direct distance measurements, obtained with HST,
with errors $\sim$10\%.  The GAIA mission is expected to return $\sim$9000
Galactic cepheids, of all periods, colors and metallicities, with distance
errors less than 10\% (many of them much less) - Windmark, F.; Lindegren, L.;
Hobbs, D., 2011A\&A...530A..76W. It is expected to deliver at least 1000
cepheids in the LMC with expected mean distance error $\sim$7-8\% (Clementini
(2010) - 011EAS....45..267C).  GAIA, as well as other methods, will also
support determination of the 3-d map of galactic interstellar extinction -
including possible variations in the extinction law. These rich data sets will
be supported with direct measurements of cepheid diameters (A. Merand et al,
A\&A in press) and advances in stellar hydrodynamics
\citep{2013MNRAS.435.3191M} which will provide theoretical and empirical basis
for calibrations to reconcile known physics with observational correction
factors.

Galactic cepheids will generally be too bright for LSST, but cepheids in
the local group are sufficiently bright that LSST photometry will be
limited by calibration errors rather than by brightness.  This dataset
will provide superb support for integration of GAIA-based galactic
cepheid studies with extra-galactic cepheid studies.

GAIA will provide similar precision data with the potential to identify
or support distance determinations from many other galactic star types.
LSST photometric catalogs will represent a uniquely extensive and
complete database for such investigations.

% --------------------------------------------------------------------

% \subsection{Metric Analysis}
\subsubsection{Metric Analysis}
\label{sec:\secname:analysis}

Several metrics currently exist in the MAF for evaluating how LSST
survey strategy affect the recovery of periodic sources.
For example,
the PeriodicMetric makes use of the periodogram purity function, which effectively
quantifies aliasing introduced into periodogram analysis from the
sampling of the lightcurve.
%
% and period deviation metric (PeriodDeviationMetric) all return relevant
% information
%
Similarly, the
phase gap metric (PhaseGapMetric),
evaluates the periodicity of the source lightcurve and its coverage
in phase space (the latter being relevant for shape recovery).

Recreating the template matching results of the
\citet{2012AJ....144....9O} simulation requires sampling specific input
lightcurves and comparing with the library of available shapes; this
necessarily requires a step outside of the MAF, but can easily be
enabled using the lightcurve simulation tools under development.
%  [NAME OF FED'S LIGHTCURVE
% TOOL].

Current simulations of the main survey show a broad uniformity of
visits, with thorough randomization of visit phase per period, giving
very good phase coverage with minimum phase gaps.


% % --------------------------------------------------------------------
%
% \subsection{Discussion}
% \label{sec:\secname:discussion}

For periodic variable science, two cadence characteristics should be avoided:
\begin{itemize}
\item an exactly uniform spacing of visits (which is anyway virtually impossible); \
\item a very non-uniform distribution, such as most visits concentrated in a few survey years.
 \end{itemize}

A metric for maximum phase gap will guard against the possibility that a
very unusual cadence might compromise the random sampling of periodic
variables.

In each case, it would help to jump-start science programs if some
fraction of targets had more complete measurements early in the survey.

% ====================================================================
%
% \subsection{Conclusions}
%
% Here we answer the ten questions posed in
% \autoref{sec:intro:evaluation:caseConclusions}:
%
% \begin{description}
%
% \item[Q1:] {\it Does the science case place any constraints on the
% tradeoff between the sky coverage and coadded depth? For example, should
% the sky coverage be maximized (to $\sim$30,000 deg$^2$, as e.g., in
% Pan-STARRS) or the number of detected galaxies (the current baseline but
% with 18,000 deg$^2$)?}
%
% \item[A1:] No strong constraint - but see Q4.
%
% \item[Q2:] {\it Does the science case place any constraints on the
% tradeoff between uniformity of sampling and frequency of  sampling? For
% example, a rolling cadence can provide enhanced sample rates over a part
% of the survey or the entire survey for a designated time at the cost of
% reduced sample rate the rest of the time (while maintaining the nominal
% total visit counts).}
%
% \item[A2:] An enhanced cadence could provide earlier discovery and period confirmation for a subset of targets, but this is
%not a high priority.
%
% \item[Q3:] {\it Does the science case place any constraints on the
% tradeoff between the single-visit depth and the number of visits
% (especially in the $u$-band where longer exposures would minimize the
% impact of the readout noise)?}
%
% \item[A3:] No strong constraint.
%
% \item[Q4:] {\it Does the science case place any constraints on the
% Galactic plane coverage (spatial coverage, temporal sampling, visits per
% band)?}
%
% \item[A4:] Most stars are in the galactic plane. For periodic pulsators, it is desirable to obtain multiple visits
%in all filters, with at least 20-50 epochs in several filters. Obtaining this coverage is more important to periodic 
%pulsator science than greater sky coverage.
%
% \item[Q5:] {\it Does the science case place any constraints on the
% fraction of observing time allocated to each band?}
%
% \item[A5:] No strong constraint, as long as there is a good representation of each, including u.
%
% \item[Q6:] {\it Does the science case place any constraints on the
% cadence for deep drilling fields?}
%
% \item[A6:] For deep drilling on the galaxy or nearby galaxies, the best cadences would cover a range of periods from
%minutes to days - longer periods would be satisfactorily sampled by the main survey.
%
% \item[Q7:] {\it Assuming two visits per night, would the science case
% benefit if they are obtained in the same band or not?}
%
% \item[A7:] Different bands would provide more rapid characterization of targets, but this is not a strong benefit.
%
% \item[Q8:] {\it Will the case science benefit from a special cadence
% prescription during commissioning or early in the survey, such as:
% acquiring a full 10-year count of visits for a small area (either in all
% the bands or in a  selected set); a greatly enhanced cadence for a small
% area?}
%
% \item[A8:] Enhanced cadences could provide earlier science, or somewhat deeper science, depending on details, but 
%periodic variables are not a strong driver for this.
%
% \item[Q9:] {\it Does the science case place any constraints on the
% sampling of observing conditions (e.g., seeing, dark sky, airmass),
% possibly as a function of band, etc.?}
%
% \item[A9:] No constraints that are particular to variable stars.
%
% \item[Q10:] {\it Does the case have science drivers that would require
% real-time exposure time optimization to obtain nearly constant
% single-visit limiting depth?}
%
% \item[A10:] No.
%
% \end{description}

% ====================================================================

\navigationbar


% --------------------------------------------------------------------

% ====================================================================
%+
% NAME:
%    planets.tex
%
% CHAPTER:
%    variables.tex
%
% ELEVATOR PITCH:
%-
% ====================================================================

% \section{Probing Planet Populations with LSST}
\subsection{Probing Planet Populations with LSST}
\def\secname{planets}\label{sec:\secname}

\credit{lundmb},
\credit{shporer},
\credit{stassun}

This section describes the unique discovery space for
extrasolar planets with LSST, namely,
planets in relatively unexplored environments.

% \subsubsection{Planets In Relatively Unexplored Environments}

A large number of exoplanets have been discovered over the past few
decades, with over 1500 exoplanets now confirmed. These discoveries are
primarily the result of two detection methods: The radial velocity (RV)
method where the planet's minimum mass is measured, and the transit
method where the planet radius is measured and RV follow-up allows the
measurement of the planet's mass and hence mean density. Other methods
are currently being developed and use to discover an increasing number of
planets, including the microlensing method and direct imaging. In
addition, the Gaia mission is expected to discover a large number of
planets using astrometry \citep{2014exha.book.....P}.

The {\it Kepler} mission has an additional almost 4000 planet
candidates. While these planet candidates have not been confirmed, the
sample is significant enough that planet characteristics can be studied
statistically, including radius and period distributions and planet
occurrence rates. LSST will extend previous transiting planet searches
by observing stellar populations that have generally not been
well-studied by previous transiting planet searches, including star
clusters, the galactic bulge, red dwarfs, white dwarfs (see below), and
the Magellanic Clouds (see Section~7). Most known exoplanets have been
found relatively nearby, as exoplanet systems with measured distances
have a median distance of around 80~pc, and 80\% of these systems are
within 320~pc (exoplanets.org). LSST is able to recover transiting
exoplanets at much larger distances, including in the galactic bulge and
the Large Magellanic Cloud, allowing for measurements of planet
occurrence rates in these other stellar 
environments \citep{2015AJ....149...16L,2015AJ....150...34J}.
Red dwarfs have often been
underrepresented in searches that have focused on solar-mass stars, however red
dwarfs are plentiful, and better than 1 in 7 are expected to host earth-sized
planets in the habitable zone \citep{2015ApJ...807...45D}.

Another currently unexplored environment where LSST will be able to
probe the exoplanet population is planets orbiting white dwarfs (WDs).
Such systems teach us about the future evolution of planetary systems
with main-sequence primaries, including that of the Solar System. When a
WD is eclipsed by a planet (or any other faint low-mass object,
including a brown dwarf or a small star) the radius and temperature
ratios lead to a very deep eclipse, possibly a complete occultation,
where during eclipse the target can drop below the detection threshold.
The existence of planets orbiting WDs has been suggested
observationally
\citep[e.g.,][]{2009ApJ...694..805F,2009AJ....137.3191J,2010ApJ...722..725Z,2012ApJ...747..148D}.
and theoretically \citep[e.g.,][]{2010MNRAS.408..631N}.
A few brown dwarf companions were already discovered
\citep[e.g.,][]{2006Natur.442..543M,2012ApJ...759L..34C,2006Sci...314.1578L,2014MNRAS.445.2106L},
and \citet{2015Natur.526..546V}
recently discovered a disintegrating planetary body orbiting a WD
\citep[see also][]{2015arXiv151006434C,2016ApJ...818L...7G,2016MNRAS.458.3904R}.

While most of the sky that LSST will survey will be at much lower
cadences than transiting planet searches employ, a sufficient
understanding of the LSST efficiency for detecting planets combined with
the large number of targets may still provide significant results.
Additionally, the multiband nature of LSST provides an extra benefit, as
exoplanet transits are achromatic while many potential astrophysical
false positives, such as binary stars, are not.
Indeed, as demonstrated by \citet{2015AJ....149...16L}, the multi-band LSST light curves 
can likely be combined to create merged light curves with denser sampling and effectively 
higher cadence, enabling detection of transiting exoplanets. The deep-drilling fields in
particular should prove to be a rich trove of transiting exoplanet detections, with 
transit-period recoverability rates as high as $\sim$50\% or more among Hot Jupiters around 
solar-type stars out to distances of many kpc and even the Magellanic Clouds in some cases 
\citep{2015AJ....149...16L,2015AJ....150...34J}.
Yields may be expected perhaps as early as the third year of LSST operations 
(Jacklin et al., in prep).
The ability to detect transiting planets outside of the deep-drilling fields is less certain;
here the details of the cadence among the various passbands will likely be particularly
important to assess carefully.


\subsection{Metrics}
\label{sec:\secname:metrics}
The detection of transiting planets will be dependent on having observations
that will provide sufficient phase coverage for transiting planets, with periods
that can range from less than one day up to tens of days. In order to address
this range of periods, an initial metric that can be used to address the detection
of transiting planets is the Periodogram Purity Function, discussed more thoroughly
in Section~5.2.1.

\subsection{Discussion}
\label{sec:\secname:discussion}
In general, the detection of transiting exoplanets with LSST will rely on
a small subset of potentially detectable planets that can be sufficiently
separated from statistical noise, rather than a clear threshold in a planet's
properties that would distinguish detectable planets vs. nondetectable planets.
This will mean that the best calculation of planet yields will have to come
from simulations of light curves for large numbers of stellar systems in order
to characterize LSST. The computation time involved in this process is sufficiently
prohibitive to prevent a metric being developed based directly on these
simulated light curves, however future work may be able to map relationships
between metric values for individual fields and the corresponding numbers
of planets that can be detected.

%
% % --------------------------------------------------------------------
%
% \subsection{OpSim Analysis}
% \label{sec:\secname:analysis}
%
% % --------------------------------------------------------------------
%
% \subsection{Discussion}
% \label{sec:\secname:discussion}
%
% ====================================================================

\navigationbar


% --------------------------------------------------------------------

% ====================================================================
%+
% NAME:
%    planets.tex
%
% CHAPTER:
%    variables.tex
%
% ELEVATOR PITCH:
%-
% ====================================================================

\section{Age-Mapping the Galaxy Using Gyrochronology}
\def\secname{rotatationalvariables}\label{sec:\secname}

This section describes recovering stellar rotation periods as a means to
mapping ages of stellar populations in the Galaxy.

\subsection{Recovery of Periods from Rotational Modulation}

% --------------------------------------------------------------------

\subsection{Metrics}
\label{sec:\secname:metrics}


% --------------------------------------------------------------------

\subsection{OpSim Analysis}
\label{sec:\secname:analysis}


% --------------------------------------------------------------------

\subsection{Discussion}
\label{sec:\secname:discussion}

% ====================================================================
%
% \subsection{Conclusions}
%
% Here we answer the ten questions posed in
% \autoref{sec:intro:evaluation:caseConclusions}:
%
% \begin{description}
%
% \item[Q1:] {\it Does the science case place any constraints on the
% tradeoff between the sky coverage and coadded depth? For example, should
% the sky coverage be maximized (to $\sim$30,000 deg$^2$, as e.g., in
% Pan-STARRS) or the number of detected galaxies (the current baseline 
% of 18,000 deg$^2$)?}
%
% \item[A1:] ...
%
% \item[Q2:] {\it Does the science case place any constraints on the
% tradeoff between uniformity of sampling and frequency of  sampling? For
% example, a rolling cadence can provide enhanced sample rates over a part
% of the survey or the entire survey for a designated time at the cost of
% reduced sample rate the rest of the time (while maintaining the nominal
% total visit counts).}
%
% \item[A2:] Stellar rotation periods are deduced from photometric changes due to surface patterns. Therefore
%a sufficiently dense sampling is required before the surface pattern (e.g. spots) changes significantly. This
% will be best satisfied with cadences optimized for quasi-periodic fluctuations with time scales of 1-10 days.
%
% \item[Q3:] {\it Does the science case place any constraints on the
% tradeoff between the single-visit depth and the number of visits
% (especially in the $u$-band where longer exposures would minimize the
% impact of the readout noise)?}
%
% \item[A3:] ...
%
% \item[Q4:] {\it Does the science case place any constraints on the
% Galactic plane coverage (spatial coverage, temporal sampling, visits per
% band)?}
%
% \item[A4:] Gyrochronology is primarily of interest for galactic science, and needs better temporal sampling
%than expected from a uniform survey,
%
% \item[Q5:] {\it Does the science case place any constraints on the
% fraction of observing time allocated to each band?}
%
% \item[A5:] ...
%
% \item[Q6:] {\it Does the science case place any constraints on the
% cadence for deep drilling fields?}
%
% \item[A6:] ...
%
% \item[Q7:] {\it Assuming two visits per night, would the science case
% benefit if they are obtained in the same band or not?}
%
% \item[A7:] ...
%
% \item[Q8:] {\it Will the case science benefit from a special cadence
% prescription during commissioning or early in the survey, such as:
% acquiring a full 10-year count of visits for a small area (either in all
% the bands or in a  selected set); a greatly enhanced cadence for a small
% area?}
%
% \item[A8:] ...
%
% \item[Q9:] {\it Does the science case place any constraints on the
% sampling of observing conditions (e.g., seeing, dark sky, airmass),
% possibly as a function of band, etc.?}
%
% \item[A9:] ...
%
% \item[Q10:] {\it Does the case have science drivers that would require
% real-time exposure time optimization to obtain nearly constant
% single-visit limiting depth?}
%
% \item[A10:] ...
%
% \end{description}
% ====================================================================

\navigationbar


% --------------------------------------------------------------------

% ====================================================================
%+
% NAME:
%    youngstars.tex
%
% CHAPTER:
%    variables.tex
%
% ELEVATOR PITCH:
%-
% ====================================================================

\section{Discovery and Characterization of Young Stellar Populations}
\def\secname{youngstars}\label{sec:\secname}

\credit{phartigan},
\credit{CJohnsKrull},
\credit{pmmcgehee}

\subsection{Introduction}

All young stars exhibit some form of
photometric variability, and these variations hold the key
to understanding the diverse physical processes present at starbirth
such as mass accretion events from circumstellar disks, presence of
warps in envelopes, creation of new knots in stellar jets,
evolution of stellar angular momenta, starspot longevity and cycles,
and the frequency and strength of flares.
With the proper cadences and filter choices,
LSST will make a significant impact in our understanding of all
these phenomena simply by providing large enough samples to allow
us to relate these aspects of the young stars and their environments
to stellar properties such as mass, age, binarity,
and their location within their nascent dark clouds in a statistically
significant manner.

Low-mass ($\lessim 1.5 M_{\odot}$) pre-main-sequence stars separate
into two main categories, depending upon whether or not an optically thick
dusty circumstellar disk exists in the system: young stars without dusty
inner disks are known as `weak-lined T Tauri stars' (wTTs), while those
that have inner dusty disks are called `classical T Tauri stars'.
The nature of the variability in young stars changes with evolutionary status.
In cTTs, variability is primarily caused by unsteady mass infall from circumstellar disks
onto their stars, and from periodic extinction events that unfold as dense
clumps or disk warps circle the star in Keplerian orbits.
Once the disks become optically thin, variability in the wTTs phase
is dominated by small-amplitude (typically $\sim$ 0.1 mag)
quasi-periodic variations that arise
from cool star spots, though active regions that generate X-rays in
these objects undoubtedly produce optical flares as well.

At the extreme end of cTTs phenomena, rare massive
outbursts of up to 6 magnitudes with decay times between several months
to over a hundred years in pre-main-sequence stars (EX Ori's \citep{herbig01}
and FU Ori's \citep{hartmann96})
are of great interest to studies of disk accretion because they indicate the onset of
a major disk instability. Only a handful of such systems have been found, and LSST
will easily detect any new ones.  In addition to triggering
follow-up observations, LSST will define the first population
constraints on the duration of high states, particularly for the shorter-lived
versions of these eruptive variables.

Obtaining rotation periods for large numbers of pre-main-sequence stars is the only way to
quantify how angular momenta of young stars vary with age. For both wTTs and cTTs,
phase-coherence in the rotation periods defines the longevity of star spots, while the
amplitude of the periodic component of the lightcurve constrains the spot coverage (and
temperature if multiple filters are used).  As described in section 8.10.2 in the
LSST Science Book (p298--299), irregular flaring in young stars exists across the entire
$ugrizy$~bandpass of LSST. Flaring can last from minutes to years, with
amplitudes from a few tenths to several magnitudes.  In cTTs, $u$-band fluxes rise during
accretion events, which we can distinguish from extinction events if red magnitudes are
also available. A large accretion event is a signal to observe the system
in the future with other instrumentation to look for evidence of a newly-created
jet knot. In wTTs, Flares also occur in wTTs as a consequence of high chromospheric activity.
Flaring in wTTs is also easiest to monitor at $u$, though the rapid decline of
chromospheric flares requires a rapid cadence to capture correctly.

LSST also provides a potential means to discover new young stars by way of
their variability and colors. One of the challenges in this regard will be to
distinguish young stars from other low-amplitude variable stars in the field.
In that regard we expect that machine-learning techniques that incorporate knowledge of
fluxes in other wavebands as well as the LSST lightcurves and colors will
be an ongoing effort. It is possible that X-ray detections will be more
reliable for detection of new young stars, but LSST will at minimum assist
by identifying non-YSO X-ray sources, and should be a means for discovery
for older pre-main-sequence stars (10 $-$ 30 My range) that have had time
to wander away from the well-known sites of star forming activity that are
typical targets for deep, pointed X-ray surveys.

Galactic star formation regions are largely found at low Galactic
latitudes or within the Gould Belt structure. As such study of young
stars with LSST is closely tied to other science goals concerning the
Milky Way Disk and is subject to the concerns of both crowded field
photometry and the observing cadence along the Milky Way. However, recent
DECam observations at the CTIO 4-m that reached depths similar to those
proposed for LSST show negligible crowding in the optical, and $\lessim$
5\%\ crowding at $z$ in Carina. In this case, extinction in the molecular
clouds helps by significantly lowering the frequency of background contamination.

% --------------------------------------------------------------------

\subsection{Analysis}
\label{sec:\secname:analysis}

{\bf Target Regions}

LSST will allow us to survey the outstanding collection of
star formation regions in the Southern hemisphere, including the closest
such regions ($\rho$ Oph, CrA, Cha~I and Lupus), and the most famous intermediate-mass
(Orion) and massive (Carina) examples.
The closest star forming regions have only low-mass molecular clouds, and each contain
only about 100 young stars. More massive molecular clouds produce both higher
mass stars and more low mass stars. In the Orion Nebula Cluster (d = 414 pc), the number of
identified YSOs is $\sim$ 3000, and we expect $\gtrsim$ 30000 pre-main-sequence
stars in the famous southern star formation region in Carina (2.3 kpc).
LSST will make its greatest impact
when observing the more massive star formation regions, where the amount
of young stars is much higher.

Owing to extinction in the dark clouds,
source confusion will generally not be an issue (as evidenced by typical
deep optical images of such regions), though the large fraction
of pre-main-sequence binaries at all separations ensures that many
lightcurves will be composites of the primaries and secondaries.
More distant star-forming regions will suffer from enhanced foreground
contamination, though it should be possible to eliminate most contaminating
variable stars by combining close inspection of their lightcurves with
colors.

{\bf Metrics:}

{\bf A. Magnitude Limits, Filter Choices}

To quantify YSO studies with LSST, we consider V~927 Tau,
a rather faint, moderately-reddened 0.2 M$_\odot$ young star in the Taurus cloud
as a target goal. Extrapolating this star
to the distance of Carina we have $u$=24.0, $g$=23.0, $r$=20.8, $i$=19.4, and $z$=18.0. For reference,
a typical young star in the Carina X-ray catalog has an $i$-magnitude of 18.
Objects that suffer larger extinctions along the line of sight will
be easiest to observe in the red. The universal cadence option of 2$\times$15 sec
exposures will yield $\sigma$ = 0.02 mag for $r$=21.8, a magnitude fainter than
V~927 Tau would be in Carina. We show below that this photometric uncertainty
suffices to recover a typical period from such an object. The mass function
of young stars peaks around 0.3 M$_\odot$, so {\it LSST will
determine periods to near the hydrogen burning limit with nominal r-band measurements
for a region like Carina}. Of course, several additional magnitudes of extinction will
exist towards many embedded sources. For example, if we assume an additional five magnitudes
of extinction at V for the V~927 Tau-like example above,
then $r$=25.2 and $\sigma$ = 0.41 mag per visit with universal cadence,
so no usable lightcurve will be possible at $r$.
However, $z$=20.5 in this case, where $\sigma$ $<$ 0.01 mag and precision
lightcurves are again possible.

{\bf B. Period Recovery for wTTs}

In order to assess how well LSST will recover periods, we created the following
model for wTTs variability.  Based on current surveys of wTTs, the periods are
distributed approximately as a Poisson distribution with a mean of 3.5 days
\citep{Affer2013}.
Amplitudes are typically 0.1 magnitude \citep{Grankin2008},
so we adopt a Poisson distribution
that has a mean amplitude of 0.05 mag, and then add 0.05 mag to ensure that
the mean variability is 0.1 mag. Shapes of T-Tauri lightcurves can be sinusoidal,
but many are `bowl-shaped', influenced by the distribution of large dark starspots
\citep{Alencar2010}.
For the bowl lightcurves we assumed a Gaussian shape with a FWHM in a uniform
distribution of extent between 0 and 0.75 in phase.
Our simulations cover both of these shapes.
Errors for each point were taken to be 0.02 magnitude, corresponding to
about r $\sim$ 21.8 for a universal cadence.

One set of simulations assumed a cadence of
one observation every three days over the course of a year.
If we define a successful period recovery to be better than a 1\%\ error, then
using the standard Scargle method \citep{Horne1986}
we are able to find the correct period in
98\%\ of the the sinusoidal, and 86\%\ of the bowl lightcurves, with the most
difficult challenges being at the short end of the period distribution. If we change
the cadence to once every 7 days, the ability to recover periods drops to
82\%\ and 59\% , respectively, for the two shapes. Interestingly, restricting the
sample to the highest-amplitude sources ($\gtrsim$ 0.1 magnitude) does little
to aid period recovery. The main issue remains the short-period systems where
P $\lesssim$ 2 days.

\begin{figure}[b]
\begin{center}
 \includegraphics[width=5.32in]{figs/starFormation/tts1.pdf}
 \caption{Recovered period vs. true period for a sample of sinusoidal (left),
bowl-shaped (middle) and bowl-shaped with False Alarm Probability $<$ 0.01 (right),
assuming a 3-day cadence and one year of observing. What appears as a solid line are the
individual points with periods that are recovered correctly. The bowl-shaped curves are
more difficult to recover than the sinusoids, but the method is highly successful in both cases.}
   \label{tts1}
\end{center}
\end{figure}

\begin{figure}[b]
\vspace*{1.0 cm}
\begin{center}
 \includegraphics[width=5.32in]{figs/starFormation/tts2.pdf}
 \caption{ Fraction of periods recovered correctly for sinusoidal (left) and bowl light curves (right) for
3-day (solid line) and 7-day (dashed line) cadences over an observing period of one year.
A 3-day cadence is significantly better than a 7-day one. Over 98\%\ of sinusoidal, and 86\%\ of bowl
light curve periods are recovered successfully with the 3-day cadence. The percentages drop to about
82\%\ and 59\%\, respectively, for the 7-day cadence.
}
   \label{tts2}
\end{center}
\end{figure}

\begin{figure}[b]
\vspace*{1.0 cm}
\begin{center}
 \includegraphics[width=5.32in]{figs/starFormation/tts3.pdf}
 \caption{Left and center: Same as Fig 2 but restricting the sample to amplitudes greater than 0.1 mag. The
method is only marginally more successful with the larger amplitude objects than it is with the entire sample.
Right: The narrowest 278 bowls have a significantly higher error rate than the entire sample does.
}
   \label{tts3}
\end{center}
\end{figure}

Overall, standard cadences of once every few days should suffice
to find most periodic T-Tauri stars that have periods $\gtrsim$ 3 days.
A dedicated campaign to observe star-forming regions
at time intervals of an hour or less is required to capture the shorter-period systems.
The r-filter should suffice for most objects, though some benefit will be had
by going to z to allow the more heavily-extincted sources to be observed.

{\bf C. Period Recovery for cTTs}

Complex irregular variations in cTTs lightcurves make it much more difficult,
and in many cases impossible to
recover periods in these systems. While sparse coverage of one observation every
few days is adequate for identifying sudden changes from accretion events, these
events to a large degree overwhelm low-amplitude periodic signatures. Even when
period searches yield a low false-alarm probability, the results are not necessarily
reliable. Results from Palomar Transient Factory surveys in the North American
Nebula \citep{Findeisen2013} and with Spitzer
\citep{Cody2014}
reveal several types of both short- and long-term variations including both bursting and
fading.  These observations emphasize how important it will be to have some
dense phase coverage as a reality check to ensure the reliability of
any periods recovered from sparse data in these objects, as well as to
follow the short-term variations that characterize accreting systems.

{\bf D. Discovery, Accretion and Extinction Events}

As we indicated above, any
cadence will uncover FU Ori and EX Ori events in all filters.
Periodic extinction events follow the same restrictions and
have the same requirements as rotational periods described in subsection B.

In order to assess the ability of LSST to identify and classify
eruptive variables (FUor/EXor), we construct
\autoref{table:pseudoForExor}, which shows a possible Figure of Merit for the
recovery by LSST of the distribution of EXor high-state duration in
outburst.

\begin{table}
\small
\begin{tabular}{c p{12cm}}
& {\it Figure of Merit for recovery of EXor high--state duration distribution}\\
\hline
1.  & Produce ASCII lightcurve for eruptive outburst \\
2.  & Initialise large array to store the maps of fraction detected as a function of duration and amplitude. \\
2.  & for {\it duration T} in range \{min, max\}:  \\
3.  & ~~~~ for {\it amplitude A} in range \{min, max\}: \\
4.  & ~~~~~~~~~~ run {\tt mafContrib/transientAsciiMetric} \\
5.  & ~~~~~~~~~~ store the spatial map of the fraction detected for this (A, T) pair \\
6.  & Initialise master arrays to hold the run of duration distribution measurements.\\
7. & Produce distribution of high--state durations and amplitudes from which the simulations will be drawn. \\
8.  & for {\it iDraw} in range \{1, nDraws\}:\\
9.  & ~~~~ construct model population with input duration distribution \\
10.  & ~~~~ Apply the stored metrics from 2-5 to measure fraction recovered \\
11.  & ~~~~ Characterize the duration distribution for this draw \\
12. & ~~~~ Fill the {\it iDraw}'th entry in the master arrays. \\
13. & {\bf FoM 1:} Compute the median and variance of the upper/lower quintiles. \\
14. & {\bf FoM 2:} Evaluate the bias between recovered and input high-state duration. \\
\hline
\end{tabular}
\caption{Steps for Figure of Merit recovering the distribution
  for the duration of EXor high states.}
\label{table:pseudoForExor}
\end{table}

% --------------------------------------------------------------------

\subsection{Summary and Recommendations}
\label{sec:\secname:discussion}

{\bf Performance for Nominal Cadences}

Nominal cadences that return to a star-forming region every 3-4 days
will suffice to determine rotation periods for $\sim$ 90\%\ of the
young stars within the magnitude limits of LSST.
These cadences are also adequate to detect major episodic
accretion events like FU Ori's and EX Ori's. However, a more focused
annual campaign of about a week duration is necessary to optimize
period recovery and angular momentum studies of young stars.

{\bf The Need for Annual Dense Coverage of a Few Selected Regions}

Occasional dense coverage of targeted regions is the only way to
get quantitative information on short-term accretion and flare
activity.  Dense coverage also removes degeneracies
for periodic variables that have periods less than a day, and is
the only way to provide a sanity check on any periods recovered for
cTTs, which have complex irregular light variations.
Comparing longevities of starspots across the mass ranges of young
stars requires two well-sampled lightcurves separated by large
time intervals.  The embedded and Classical T Tauri stars also undergo
significant and rapid color changes due to both accretion processes
and extinction variations, so it is important
to include multiple filters in any dense coverage campaign.

These goals can be accomplished by having a week every year where
one or more selected fields are observed once every 30 minutes in u, r and z.
A young star with a 2-day period sampled every 30 minutes provides a
data point every 0.01 in phase. For the best-case scenario, observing for 7 nights
and 10 hours per night would yield 140 photometric points in each filter.
Depending on the period aliasing, this coverage should populate the
phases well enough to identify most of the large starspots on the stellar photospheres.

At the beginning of LSST operations we argue that a targeted test field
be observed in this manner to illustrate what can be done with LSST in this mode.
Combining a densely-packed short-interval
dataset with a sparse but long baseline study maxmizes the scientific return
for both methods, and allows LSST to address all of the accretion and
rotational variability associated with young stars.

\subsection{Conclusions}
\begin{description}
\item[Q1:] {\it Does the science case place any constraints on the
tradeoff between the sky coverage and coadded depth? For example, should
the sky coverage be maximized (to $\sim$30,000 deg$^2$, as e.g., in
Pan-STARRS) or the number of detected galaxies (the current baseline 
of 18,000 deg$^2$)?}
%
\item[A1:] 
Most young stars congregate into clusters in specific regions, though there is an older
population that is more distributed. The vast majority are within $\sim$ 25 degrees of the
galactic plane. As long as that swath of sky is covered to the degree possible from Chile,
the survey will provide the young star community with the monitoring capability needed to
identify transient outbursts and to study periodic and aperiodic phenomena in young stars.
While it may be useful to have deep coadded images of some regions, for example, to identify
optical counterparts to X-ray point sources, dust extinction typically limits such efforts
in the optical towards most regions of interest.  Hence, deep coadded frames are a secondary
priority.

\item[Q2:] {\it Does the science case place any constraints on the
tradeoff between uniformity of sampling and frequency of  sampling? For
example, a rolling cadence can provide enhanced sample rates over a part
of the survey or the entire survey for a designated time at the cost of
reduced sample rate the rest of the time (while maintaining the nominal
total visit counts).}

\item[A2:] 
We discussed cadences in some detail above. To obtain the best constraints on periods, a time-intensive
($\sim$ one week/year) campaign on a few selected regions is warranted. Nominal coverage over the
galactic plane will suffice to identify eruptive variables.

\item[Q3:] {\it Does the science case place any constraints on the
tradeoff between the single-visit depth and the number of visits
(especially in the $u$-band where longer exposures would minimize the
impact of the readout noise)?}

\item[A3:] 
This item is discussed in section A above. In young stars, the u-band is particularly useful
as a measure of accretion. At the same time, for period determinations, denser cadences produce
fewer problems with aliasing for the typical young star variable with a rotation period between
a day and two weeks.

\item[Q4:] {\it Does the science case place any constraints on the
Galactic plane coverage (spatial coverage, temporal sampling, visits per
band)?}

\item[A4:] 
A nominal sampling of once every few days will suffice to identify interesting
eruptive variables as long as the galactic plane coverage is good (item A1).
However, a sparse temporal sampling such as this will make it difficult to interpret
the lightcurves of the tens of thousands of T Tauri stars observed by LSST. A single
week's campaign of dense sampling, e.g., 2-3 times per night, of targeted regions
will greatly improve our ability to separate periodic variables from aperiodic ones. 
Knowing the rotation periods of thousands of young stars within a given star forming
region will be a major contribution that LSST makes to this field of research. Such data
will allow us to learn how angular momentum is distributed among newborn
stars, whether it changes with mass and location in the cloud, 
and how it varies as the stars age. 

\item[Q5:] {\it Does the science case place any constraints on the
fraction of observing time allocated to each band?}

\item[A5:] 
In the case we made above for dense sampling, we argued for z, r, and u. The
u-band allows us to follow mass accretion variability, while r and z (for most
objects) will be dominated by photospheric flux. The r-z color is an important
constraint for models of star spots. More colors are always useful, but having
a photospheric color index plus one accretion measure are the science drivers 
for filter choices. Once u, r, and z are observed, having higher cadences
is preferable to having more bands.
The exposure times and magnitudes for typical objects are described in section A
above.

\item[Q6:] {\it Does the science case place any constraints on the
cadence for deep drilling fields?}

\item[A6:] 
Absolutely. We argue in the `Summary and Recommendations' section above
for the advantages of a single week of denser monitoring for specific
regions, with the goal of having several observations per night, separated
in time by at least an hour from one another. The goal here is to provide
some basis in reality for interpreting the irregular lightcurves of young
stars, which also typically have a periodic component. Depending on the longevity of
the star spots, a period might be obvious in high-cadence data taken over
a week, but disappear over a year if some spots vanish and others form. 

\item[Q7:] {\it Assuming two visits per night, would the science case
benefit if they are obtained in the same band or not?}

\item[A7:] 
If the data are taken in the same band
it means better sampling for periods. With different bands it means a lightcurve in two bands,
which is also useful. It's probably a wash, and we can follow whatever the drivers are for the
other science cases of LSST.

\item[Q8:] {\it Will the case science benefit from a special cadence
prescription during commissioning or early in the survey, such as:
acquiring a full 10-year count of visits for a small area (either in all
the bands or in a  selected set); a greatly enhanced cadence for a small
area?}

\item[A8:] 
Yes, definitely. We ought to see what we can get out of a dedicated week-long
LSST cadence on specific regions. If the results are impressive we follow up 
on different regions every year (or even return to the same regions).

\item[Q9:] {\it Does the science case place any constraints on the
sampling of observing conditions (e.g., seeing, dark sky, airmass),
possibly as a function of band, etc.?}

\item[A9:] 

Better seeing helps with unresolved binaries and for regions where contamination comes
into play, for example, in the plane but away from dark clouds. Some of the fainter objects
will be affected if the Moon is very bright and close. However generally these constraints
are probably in a 'typical' category and will not affect the design of the survey.

\item[Q10:] {\it Does the case have science drivers that would require
real-time exposure time optimization to obtain nearly constant
single-visit limiting depth?}

\item[A10:] 
No.

\end{description}

% ====================================================================

% bibtems need pushing into the relevant file!

%Hartmann \& Kenyon 1996, ARA\&A, 34, 207 \\
%Herbig et al. 2001, PASP, 113, 1547 \\
%Herbig 1977, ApJ, 217, 693 \\
%Aspin et al. 2009, ApJ, 692L, 67 \\
%Hodapp et al. 1996, ApJ, 468, 861 \\
%McGehee et al. 2004, ApJ, 616, 1058 \\

%\bibitem[Affer et al. (2013)]{Affer13}
%{Affer, L., Micela, G., Favata, F., Flaccomio, E., \& Bouvier, J.} 2013
%\textit{MNRAS}, 430, 1433

%\bibitem[Alencar al. (2010)]{CoRoT}
%{Alencar, S. H. P. et al.} 2010,
%\textit{A\&A}, 519, A88

%\bibitem[Cody al. (2014)]{cody14}
%Cody, A., et al.
%{Cody, A. et al.} 2014,
%\textit{AJ}, 147, 82

%\bibitem[Grankin al. (2008)]{ROTOR}
%\bibitem[Findeisen al. (2013)]{findeisen13}
%Findeisen, K., Hillenbrand, L., Ofek, E., Levitan, D., Sesar, B., Laher, R., \& Surace, J.
%{Findeisen, K. et al.} 2013,
%\textit{ApJ}, 768, 93

%\bibitem[Grankin al. (2008)]{ROTOR}
%{Grankin, K. N., Bouvier, J., Herbst, W., \& Melnikov, S. Yu.} 2008,
%\textit{A\&A}, 479, 827

%\bibitem[Horne \& Baliunas (1986)]{Scargle}
%{Horne, J. H. \& Baliunas, S.} 1986, \textit{ApJ}, 302, 757

\navigationbar


% --------------------------------------------------------------------
